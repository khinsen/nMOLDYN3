%
% API Documentation for nMOLDYN
% Module nMOLDYN.Analysis.Scattering
%
% Generated by epydoc 3.0.1
% [Thu Oct  8 16:59:58 2009]
%

%%%%%%%%%%%%%%%%%%%%%%%%%%%%%%%%%%%%%%%%%%%%%%%%%%%%%%%%%%%%%%%%%%%%%%%%%%%
%%                          Module Description                           %%
%%%%%%%%%%%%%%%%%%%%%%%%%%%%%%%%%%%%%%%%%%%%%%%%%%%%%%%%%%%%%%%%%%%%%%%%%%%

    \index{nMOLDYN \textit{(package)}!nMOLDYN.Analysis \textit{(package)}!nMOLDYN.Analysis.Scattering \textit{(module)}|(}
\section{Module nMOLDYN.Analysis.Scattering}

    \label{nMOLDYN:Analysis:Scattering}
\begin{alltt}
Collections of classes for the determination of scattering-related properties.

Classes:
    * DynamicCoherentStructureFactor           : sets up a Dynamic Coherent Structure Factor analysis.
    * DynamicCoherentStructureFactorARModel    : sets up a Dynamic Coherent Structure Factor analysis using an Auto Regressive model.
    * DynamicIncoherentStructureFactor         : sets up an Dynamic Incoherent Structure Factor analysis.
    * DynamicIncoherentStructureFactorGaussian : sets up an Dynamic Incoherent Structure Factor analysis using a Gaussian approximation.
    * IncoherentStructureFactorARModel         : sets up an Dynamic Incoherent Structure Factor analysis using an Auto Regressive model.
    * ElasticIncoherentStructureFactor         : sets up an Elastic Incoherent Structure Factor analysis.
    * StaticCoherentStructureFactor            : sets up a Static Coherent Structure Factor analysis.
    
Procedures:
    * DynamicStructureFactor : returns the Dynamic Structure Factor.
\end{alltt}


%%%%%%%%%%%%%%%%%%%%%%%%%%%%%%%%%%%%%%%%%%%%%%%%%%%%%%%%%%%%%%%%%%%%%%%%%%%
%%                               Functions                               %%
%%%%%%%%%%%%%%%%%%%%%%%%%%%%%%%%%%%%%%%%%%%%%%%%%%%%%%%%%%%%%%%%%%%%%%%%%%%

  \subsection{Functions}

    \label{nMOLDYN:Analysis:Scattering:DynamicStructureFactor}
    \index{nMOLDYN \textit{(package)}!nMOLDYN.Analysis \textit{(package)}!nMOLDYN.Analysis.Scattering \textit{(module)}!nMOLDYN.Analysis.Scattering.DynamicStructureFactor \textit{(function)}}

    \vspace{0.5ex}

\hspace{.8\funcindent}\begin{boxedminipage}{\funcwidth}

    \raggedright \textbf{DynamicStructureFactor}(\textit{netcdf}, \textit{alpha})

    \vspace{-1.5ex}

    \rule{\textwidth}{0.5\fboxrule}
\setlength{\parskip}{2ex}
    Computes the dynamic structure factor from an intermediate scattering 
    function.

\setlength{\parskip}{1ex}
      \textbf{Parameters}
      \vspace{-1ex}

      \begin{quote}
        \begin{Ventry}{xxxxxx}

          \item[netcdf]

          the intermediate scattering function from which the dynamic 
          structure factor will be computed..

            {\it (type=string or instance of \_NetCDFFile)}

          \item[alpha]

          the width, in percentage of the trajectory length, of the 
          gaussian used in the smoothing procedure.

            {\it (type=float)}

        \end{Ventry}

      \end{quote}

    \end{boxedminipage}


%%%%%%%%%%%%%%%%%%%%%%%%%%%%%%%%%%%%%%%%%%%%%%%%%%%%%%%%%%%%%%%%%%%%%%%%%%%
%%                           Class Description                           %%
%%%%%%%%%%%%%%%%%%%%%%%%%%%%%%%%%%%%%%%%%%%%%%%%%%%%%%%%%%%%%%%%%%%%%%%%%%%

    \index{nMOLDYN \textit{(package)}!nMOLDYN.Analysis \textit{(package)}!nMOLDYN.Analysis.Scattering \textit{(module)}!nMOLDYN.Analysis.Scattering.DynamicCoherentStructureFactor \textit{(class)}|(}
\subsection{Class DynamicCoherentStructureFactor}

    \label{nMOLDYN:Analysis:Scattering:DynamicCoherentStructureFactor}
\begin{tabular}{cccccc}
% Line for nMOLDYN.Analysis.Analysis.Analysis, linespec=[False]
\multicolumn{2}{r}{\settowidth{\BCL}{nMOLDYN.Analysis.Analysis.Analysis}\multirow{2}{\BCL}{nMOLDYN.Analysis.Analysis.Analysis}}
&&
  \\\cline{3-3}
  &&\multicolumn{1}{c|}{}
&&
  \\
&&\multicolumn{2}{l}{\textbf{nMOLDYN.Analysis.Scattering.DynamicCoherentStructureFactor}}
\end{tabular}

\begin{alltt}
Sets up a Dynamic Coherent Structure Factor analysis.

A Subclass of nMOLDYN.Analysis.Analysis. 

Constructor: DynamicCoherentStructureFactor({\textbar}parameters{\textbar} = None)

Arguments:

    - {\textbar}parameters{\textbar} -- a dictionnary of the input parameters, or 'None' to set up the analysis without parameters.
        * trajectory        -- a trajectory file name or an instance of MMTK.Trajectory.Trajectory class.
        * timeinfo          -- a string of the form 'first:last:step' where 'first' is an integer specifying the first frame 
                               number to consider, 'last' is an integer specifying the last frame number to consider and 
                               'step' is an integer specifying the step number between two frames.
        * qshellvalues      -- a string of the form 'qmin1:qmax1:dq1;qmin2:qmax2:dq2...' where 'qmin1', 'qmin2' ... , 
                               'qmax1', 'qmax2' ... and 'dq1', 'dq2' ... are floats that represents respectively 
                               the q minimum, the q maximum and the q steps for q interval 1, 2 ...
        * qshellwidth       -- a float specifying the width of the q shells.
        * qvectorspershell  -- a float specifying the number of q vectors to generate per q shell.
        * qvectorsgenerator -- a string being one of 'isotropic', 'anisotropic' or 'explicit' specifying the way the q vectors
                               will be generated.
        * qvectorsdirection -- a string of the form 'v1x,v1y,v1z;v2x,v2y,v2z...' where 'v1x', 'v2x' ..., 'v1y', 'v2y' ... and
                               'v1z', 'v2z' ... are floats that represents respectively the x, y and z values of the vectord along 
                               which the q vectors should be generated.
        * fftwindow         -- a float in ]0.0,100.0[ specifying the width of the gaussian, in percentage of the trajectory length
                               that will be used in the smoothing procedure.
        * subset            -- a selection string specifying the atoms to consider for the analysis.
        * deuteration       -- a selection string specifying the hydrogen atoms whose atomic parameters will be those of the deuterium.
        * weights           -- a string equal to 'equal', 'mass', 'coherent' , 'incoherent' or 'atomicNumber' that specifies the weighting
                               scheme to use.
        * dcsf              -- the output NetCDF file name for the intermediate scattering function.
        * pyroserver        -- a string specifying if Pyro will be used and how to run the analysis.
    
Running modes:

    - To run the analysis do: a.runAnalysis() where a is the analysis object.
    - To estimate the analysis do: a.estimateAnalysis() where a is the analysis object.
    - To save the analysis to 'file' file name do: a.saveAnalysis(file) where a is the analysis object.
    
\end{alltt}


%%%%%%%%%%%%%%%%%%%%%%%%%%%%%%%%%%%%%%%%%%%%%%%%%%%%%%%%%%%%%%%%%%%%%%%%%%%
%%                                Methods                                %%
%%%%%%%%%%%%%%%%%%%%%%%%%%%%%%%%%%%%%%%%%%%%%%%%%%%%%%%%%%%%%%%%%%%%%%%%%%%

  \subsubsection{Methods}

    \vspace{0.5ex}

\hspace{.8\funcindent}\begin{boxedminipage}{\funcwidth}

    \raggedright \textbf{\_\_init\_\_}(\textit{self})

    \vspace{-1.5ex}

    \rule{\textwidth}{0.5\fboxrule}
\setlength{\parskip}{2ex}
    The constructor. Insures that the class can not be instanciated 
    directly from here.

\setlength{\parskip}{1ex}
      \textbf{Parameters}
      \vspace{-1ex}

      \begin{quote}
        \begin{Ventry}{xxxxxxxxxx}

          \item[parameters]

          a dictionnary that contains parameters of the selected analysis.

          \item[statusBar]

          if not None, an instance of nMOLDYN.GUI.Widgets.StatusBar. Will 
          attach a status bar to the selected analysis.

        \end{Ventry}

      \end{quote}

      Overrides: nMOLDYN.Analysis.Analysis.Analysis.\_\_init\_\_

    \end{boxedminipage}

    \label{nMOLDYN:Analysis:Scattering:DynamicCoherentStructureFactor:initialize}
    \index{nMOLDYN \textit{(package)}!nMOLDYN.Analysis \textit{(package)}!nMOLDYN.Analysis.Scattering \textit{(module)}!nMOLDYN.Analysis.Scattering.DynamicCoherentStructureFactor \textit{(class)}!nMOLDYN.Analysis.Scattering.DynamicCoherentStructureFactor.initialize \textit{(method)}}

    \vspace{0.5ex}

\hspace{.8\funcindent}\begin{boxedminipage}{\funcwidth}

    \raggedright \textbf{initialize}(\textit{self})

    \vspace{-1.5ex}

    \rule{\textwidth}{0.5\fboxrule}
\setlength{\parskip}{2ex}
    Initializes the analysis (e.g. parses and checks input parameters, set 
    some variables ...).

\setlength{\parskip}{1ex}
    \end{boxedminipage}

    \label{nMOLDYN:Analysis:Scattering:DynamicCoherentStructureFactor:calc}
    \index{nMOLDYN \textit{(package)}!nMOLDYN.Analysis \textit{(package)}!nMOLDYN.Analysis.Scattering \textit{(module)}!nMOLDYN.Analysis.Scattering.DynamicCoherentStructureFactor \textit{(class)}!nMOLDYN.Analysis.Scattering.DynamicCoherentStructureFactor.calc \textit{(method)}}

    \vspace{0.5ex}

\hspace{.8\funcindent}\begin{boxedminipage}{\funcwidth}

    \raggedright \textbf{calc}(\textit{self}, \textit{qIndex}, \textit{trajname})

    \vspace{-1.5ex}

    \rule{\textwidth}{0.5\fboxrule}
\setlength{\parskip}{2ex}
    Calculates the contribution for one Q-shell.

\setlength{\parskip}{1ex}
      \textbf{Parameters}
      \vspace{-1ex}

      \begin{quote}
        \begin{Ventry}{xxxxxxxx}

          \item[qIndex]

          the index of the Q-shell in {\textbar}self.qRadii{\textbar} list.

            {\it (type=integer.)}

          \item[trajname]

          the name of the trajectory file name.

            {\it (type=string)}

        \end{Ventry}

      \end{quote}

    \end{boxedminipage}

    \label{nMOLDYN:Analysis:Scattering:DynamicCoherentStructureFactor:combine}
    \index{nMOLDYN \textit{(package)}!nMOLDYN.Analysis \textit{(package)}!nMOLDYN.Analysis.Scattering \textit{(module)}!nMOLDYN.Analysis.Scattering.DynamicCoherentStructureFactor \textit{(class)}!nMOLDYN.Analysis.Scattering.DynamicCoherentStructureFactor.combine \textit{(method)}}

    \vspace{0.5ex}

\hspace{.8\funcindent}\begin{boxedminipage}{\funcwidth}

    \raggedright \textbf{combine}(\textit{self}, \textit{qIndex}, \textit{x})

\setlength{\parskip}{2ex}
\setlength{\parskip}{1ex}
    \end{boxedminipage}

    \label{nMOLDYN:Analysis:Scattering:DynamicCoherentStructureFactor:finalize}
    \index{nMOLDYN \textit{(package)}!nMOLDYN.Analysis \textit{(package)}!nMOLDYN.Analysis.Scattering \textit{(module)}!nMOLDYN.Analysis.Scattering.DynamicCoherentStructureFactor \textit{(class)}!nMOLDYN.Analysis.Scattering.DynamicCoherentStructureFactor.finalize \textit{(method)}}

    \vspace{0.5ex}

\hspace{.8\funcindent}\begin{boxedminipage}{\funcwidth}

    \raggedright \textbf{finalize}(\textit{self})

    \vspace{-1.5ex}

    \rule{\textwidth}{0.5\fboxrule}
\setlength{\parskip}{2ex}
    Finalizes the calculations (e.g. averaging the total term, output files
    creations ...).

\setlength{\parskip}{1ex}
    \end{boxedminipage}


\large{\textbf{\textit{Inherited from nMOLDYN.Analysis.Analysis.Analysis\textit{(Section \ref{nMOLDYN:Analysis:Analysis:Analysis})}}}}

\begin{quote}
analysisTime(), buildJobInfo(), buildTimeInfo(), deuterationSelection(), groupSelection(), parseInputParameters(), preLoadTrajectory(), runAnalysis(), saveAnalysis(), setInputParameters(), subsetSelection(), updateJobProgress(), weightingScheme()
\end{quote}

%%%%%%%%%%%%%%%%%%%%%%%%%%%%%%%%%%%%%%%%%%%%%%%%%%%%%%%%%%%%%%%%%%%%%%%%%%%
%%                            Class Variables                            %%
%%%%%%%%%%%%%%%%%%%%%%%%%%%%%%%%%%%%%%%%%%%%%%%%%%%%%%%%%%%%%%%%%%%%%%%%%%%

  \subsubsection{Class Variables}

    \vspace{-1cm}
\hspace{\varindent}\begin{longtable}{|p{\varnamewidth}|p{\vardescrwidth}|l}
\cline{1-2}
\cline{1-2} \centering \textbf{Name} & \centering \textbf{Description}& \\
\cline{1-2}
\endhead\cline{1-2}\multicolumn{3}{r}{\small\textit{continued on next page}}\\\endfoot\cline{1-2}
\endlastfoot\raggedright i\-n\-p\-u\-t\-P\-a\-r\-a\-m\-e\-t\-e\-r\-s\-N\-a\-m\-e\-s\- & \raggedright \textbf{Value:} 
{\tt 'trajectory', 'timeinfo', 'qshellvalues', 'qshellwidth', \texttt{...}}&\\
\cline{1-2}
\raggedright d\-e\-f\-a\-u\-l\-t\- & \raggedright \textbf{Value:} 
{\tt \{'weights': 'coherent'\}}&\\
\cline{1-2}
\raggedright s\-h\-o\-r\-t\-N\-a\-m\-e\- & \raggedright \textbf{Value:} 
{\tt 'DCSF'}&\\
\cline{1-2}
\raggedright c\-a\-n\-B\-e\-E\-s\-t\-i\-m\-a\-t\-e\-d\- & \raggedright \textbf{Value:} 
{\tt False}&\\
\cline{1-2}
\end{longtable}

    \index{nMOLDYN \textit{(package)}!nMOLDYN.Analysis \textit{(package)}!nMOLDYN.Analysis.Scattering \textit{(module)}!nMOLDYN.Analysis.Scattering.DynamicCoherentStructureFactor \textit{(class)}|)}

%%%%%%%%%%%%%%%%%%%%%%%%%%%%%%%%%%%%%%%%%%%%%%%%%%%%%%%%%%%%%%%%%%%%%%%%%%%
%%                           Class Description                           %%
%%%%%%%%%%%%%%%%%%%%%%%%%%%%%%%%%%%%%%%%%%%%%%%%%%%%%%%%%%%%%%%%%%%%%%%%%%%

    \index{nMOLDYN \textit{(package)}!nMOLDYN.Analysis \textit{(package)}!nMOLDYN.Analysis.Scattering \textit{(module)}!nMOLDYN.Analysis.Scattering.StaticCoherentStructureFactor \textit{(class)}|(}
\subsection{Class StaticCoherentStructureFactor}

    \label{nMOLDYN:Analysis:Scattering:StaticCoherentStructureFactor}
\begin{tabular}{cccccc}
% Line for nMOLDYN.Analysis.Analysis.Analysis, linespec=[False]
\multicolumn{2}{r}{\settowidth{\BCL}{nMOLDYN.Analysis.Analysis.Analysis}\multirow{2}{\BCL}{nMOLDYN.Analysis.Analysis.Analysis}}
&&
  \\\cline{3-3}
  &&\multicolumn{1}{c|}{}
&&
  \\
&&\multicolumn{2}{l}{\textbf{nMOLDYN.Analysis.Scattering.StaticCoherentStructureFactor}}
\end{tabular}

\begin{alltt}
Sets up a Coherent Structure Factor analysis.

A Subclass of nMOLDYN.Analysis.Analysis. 

Constructor: StaticCoherentStructureFactor({\textbar}parameters{\textbar} = None)

Arguments:

    - {\textbar}parameters{\textbar} -- a dictionnary of the input parameters, or 'None' to set up the analysis without parameters.
        * trajectory        -- a trajectory file name or an instance of MMTK.Trajectory.Trajectory class.
        * timeinfo          -- a string of the form 'first:last:step' where 'first' is an integer specifying the first frame 
                               number to consider, 'last' is an integer specifying the last frame number to consider and 
                               'step' is an integer specifying the step number between two frames.
        * qshellvalues      -- a string of the form 'qmin1:qmax1:dq1;qmin2:qmax2:dq2...' where 'qmin1', 'qmin2' ... , 
                               'qmax1', 'qmax2' ... and 'dq1', 'dq2' ... are floats that represents respectively 
                               the q minimum, the q maximum and the q steps for q interval 1, 2 ...
        * qshellwidth       -- a float specifying the width of the q shells.
        * qvectorspershell  -- a float specifying the number of q vectors to generate per q shell.
        * qvectorsgenerator -- a string being one of 'isotropic', 'anisotropic' or 'explicit' specifying the way the q vectors
                               will be generated.
        * qvectorsdirection -- a string of the form 'v1x,v1y,v1z;v2x,v2y,v2z...' where 'v1x', 'v2x' ..., 'v1y', 'v2y' ... and
                               'v1z', 'v2z' ... are floats that represents respectively the x, y and z values of the vectord along 
                               which the q vectors should be generated.
        * fftwindow         -- a float in ]0.0,100.0[ specifying the width of the gaussian, in percentage of the trajectory length
                               that will be used in the smoothing procedure.
        * subset            -- a selection string specifying the atoms to consider for the analysis.
        * deuteration       -- a selection string specifying the hydrogen atoms whose atomic parameters will be those of the deuterium.
        * weights           -- a string equal to 'equal', 'mass', 'coherent' , 'incoherent' or 'atomicNumber' that specifies the weighting
                               scheme to use.
        * csf               -- the output NetCDF file name for the intermediate scattering function.
        * pyroserver        -- a string specifying if Pyro will be used and how to run the analysis.
    
Running modes:

    - To run the analysis do: a.runAnalysis() where a is the analysis object.
    - To estimate the analysis do: a.estimateAnalysis() where a is the analysis object.
    - To save the analysis to 'file' file name do: a.saveAnalysis(file) where a is the analysis object.
    
\end{alltt}


%%%%%%%%%%%%%%%%%%%%%%%%%%%%%%%%%%%%%%%%%%%%%%%%%%%%%%%%%%%%%%%%%%%%%%%%%%%
%%                                Methods                                %%
%%%%%%%%%%%%%%%%%%%%%%%%%%%%%%%%%%%%%%%%%%%%%%%%%%%%%%%%%%%%%%%%%%%%%%%%%%%

  \subsubsection{Methods}

    \vspace{0.5ex}

\hspace{.8\funcindent}\begin{boxedminipage}{\funcwidth}

    \raggedright \textbf{\_\_init\_\_}(\textit{self})

    \vspace{-1.5ex}

    \rule{\textwidth}{0.5\fboxrule}
\setlength{\parskip}{2ex}
    The constructor. Insures that the class can not be instanciated 
    directly from here.

\setlength{\parskip}{1ex}
      \textbf{Parameters}
      \vspace{-1ex}

      \begin{quote}
        \begin{Ventry}{xxxxxxxxxx}

          \item[parameters]

          a dictionnary that contains parameters of the selected analysis.

          \item[statusBar]

          if not None, an instance of nMOLDYN.GUI.Widgets.StatusBar. Will 
          attach a status bar to the selected analysis.

        \end{Ventry}

      \end{quote}

      Overrides: nMOLDYN.Analysis.Analysis.Analysis.\_\_init\_\_

    \end{boxedminipage}

    \label{nMOLDYN:Analysis:Scattering:StaticCoherentStructureFactor:initialize}
    \index{nMOLDYN \textit{(package)}!nMOLDYN.Analysis \textit{(package)}!nMOLDYN.Analysis.Scattering \textit{(module)}!nMOLDYN.Analysis.Scattering.StaticCoherentStructureFactor \textit{(class)}!nMOLDYN.Analysis.Scattering.StaticCoherentStructureFactor.initialize \textit{(method)}}

    \vspace{0.5ex}

\hspace{.8\funcindent}\begin{boxedminipage}{\funcwidth}

    \raggedright \textbf{initialize}(\textit{self})

    \vspace{-1.5ex}

    \rule{\textwidth}{0.5\fboxrule}
\setlength{\parskip}{2ex}
    Initializes the analysis (e.g. parses and checks input parameters, set 
    some variables ...).

\setlength{\parskip}{1ex}
    \end{boxedminipage}

    \label{nMOLDYN:Analysis:Scattering:StaticCoherentStructureFactor:calc}
    \index{nMOLDYN \textit{(package)}!nMOLDYN.Analysis \textit{(package)}!nMOLDYN.Analysis.Scattering \textit{(module)}!nMOLDYN.Analysis.Scattering.StaticCoherentStructureFactor \textit{(class)}!nMOLDYN.Analysis.Scattering.StaticCoherentStructureFactor.calc \textit{(method)}}

    \vspace{0.5ex}

\hspace{.8\funcindent}\begin{boxedminipage}{\funcwidth}

    \raggedright \textbf{calc}(\textit{self}, \textit{qIndex}, \textit{trajname})

    \vspace{-1.5ex}

    \rule{\textwidth}{0.5\fboxrule}
\setlength{\parskip}{2ex}
    Calculates the contribution for one Q-shell.

\setlength{\parskip}{1ex}
      \textbf{Parameters}
      \vspace{-1ex}

      \begin{quote}
        \begin{Ventry}{xxxxxxxx}

          \item[qIndex]

          the index of the Q-shell in {\textbar}self.qRadii{\textbar} list.

            {\it (type=integer.)}

          \item[trajname]

          the name of the trajectory file name.

            {\it (type=string)}

        \end{Ventry}

      \end{quote}

    \end{boxedminipage}

    \label{nMOLDYN:Analysis:Scattering:StaticCoherentStructureFactor:combine}
    \index{nMOLDYN \textit{(package)}!nMOLDYN.Analysis \textit{(package)}!nMOLDYN.Analysis.Scattering \textit{(module)}!nMOLDYN.Analysis.Scattering.StaticCoherentStructureFactor \textit{(class)}!nMOLDYN.Analysis.Scattering.StaticCoherentStructureFactor.combine \textit{(method)}}

    \vspace{0.5ex}

\hspace{.8\funcindent}\begin{boxedminipage}{\funcwidth}

    \raggedright \textbf{combine}(\textit{self}, \textit{qIndex}, \textit{x})

\setlength{\parskip}{2ex}
\setlength{\parskip}{1ex}
    \end{boxedminipage}

    \label{nMOLDYN:Analysis:Scattering:StaticCoherentStructureFactor:finalize}
    \index{nMOLDYN \textit{(package)}!nMOLDYN.Analysis \textit{(package)}!nMOLDYN.Analysis.Scattering \textit{(module)}!nMOLDYN.Analysis.Scattering.StaticCoherentStructureFactor \textit{(class)}!nMOLDYN.Analysis.Scattering.StaticCoherentStructureFactor.finalize \textit{(method)}}

    \vspace{0.5ex}

\hspace{.8\funcindent}\begin{boxedminipage}{\funcwidth}

    \raggedright \textbf{finalize}(\textit{self})

    \vspace{-1.5ex}

    \rule{\textwidth}{0.5\fboxrule}
\setlength{\parskip}{2ex}
    Finalizes the calculations (e.g. averaging the total term, output files
    creations ...).

\setlength{\parskip}{1ex}
    \end{boxedminipage}


\large{\textbf{\textit{Inherited from nMOLDYN.Analysis.Analysis.Analysis\textit{(Section \ref{nMOLDYN:Analysis:Analysis:Analysis})}}}}

\begin{quote}
analysisTime(), buildJobInfo(), buildTimeInfo(), deuterationSelection(), groupSelection(), parseInputParameters(), preLoadTrajectory(), runAnalysis(), saveAnalysis(), setInputParameters(), subsetSelection(), updateJobProgress(), weightingScheme()
\end{quote}

%%%%%%%%%%%%%%%%%%%%%%%%%%%%%%%%%%%%%%%%%%%%%%%%%%%%%%%%%%%%%%%%%%%%%%%%%%%
%%                            Class Variables                            %%
%%%%%%%%%%%%%%%%%%%%%%%%%%%%%%%%%%%%%%%%%%%%%%%%%%%%%%%%%%%%%%%%%%%%%%%%%%%

  \subsubsection{Class Variables}

    \vspace{-1cm}
\hspace{\varindent}\begin{longtable}{|p{\varnamewidth}|p{\vardescrwidth}|l}
\cline{1-2}
\cline{1-2} \centering \textbf{Name} & \centering \textbf{Description}& \\
\cline{1-2}
\endhead\cline{1-2}\multicolumn{3}{r}{\small\textit{continued on next page}}\\\endfoot\cline{1-2}
\endlastfoot\raggedright i\-n\-p\-u\-t\-P\-a\-r\-a\-m\-e\-t\-e\-r\-s\-N\-a\-m\-e\-s\- & \raggedright \textbf{Value:} 
{\tt 'trajectory', 'timeinfo', 'qshellvalues', 'qshellwidth', \texttt{...}}&\\
\cline{1-2}
\raggedright d\-e\-f\-a\-u\-l\-t\- & \raggedright \textbf{Value:} 
{\tt \{'weights': 'coherent'\}}&\\
\cline{1-2}
\raggedright s\-h\-o\-r\-t\-N\-a\-m\-e\- & \raggedright \textbf{Value:} 
{\tt 'SCSF'}&\\
\cline{1-2}
\raggedright c\-a\-n\-B\-e\-E\-s\-t\-i\-m\-a\-t\-e\-d\- & \raggedright \textbf{Value:} 
{\tt False}&\\
\cline{1-2}
\end{longtable}

    \index{nMOLDYN \textit{(package)}!nMOLDYN.Analysis \textit{(package)}!nMOLDYN.Analysis.Scattering \textit{(module)}!nMOLDYN.Analysis.Scattering.StaticCoherentStructureFactor \textit{(class)}|)}

%%%%%%%%%%%%%%%%%%%%%%%%%%%%%%%%%%%%%%%%%%%%%%%%%%%%%%%%%%%%%%%%%%%%%%%%%%%
%%                           Class Description                           %%
%%%%%%%%%%%%%%%%%%%%%%%%%%%%%%%%%%%%%%%%%%%%%%%%%%%%%%%%%%%%%%%%%%%%%%%%%%%

    \index{nMOLDYN \textit{(package)}!nMOLDYN.Analysis \textit{(package)}!nMOLDYN.Analysis.Scattering \textit{(module)}!nMOLDYN.Analysis.Scattering.DynamicCoherentStructureFactorAR \textit{(class)}|(}
\subsection{Class DynamicCoherentStructureFactorAR}

    \label{nMOLDYN:Analysis:Scattering:DynamicCoherentStructureFactorAR}
\begin{tabular}{cccccc}
% Line for nMOLDYN.Analysis.Analysis.Analysis, linespec=[False]
\multicolumn{2}{r}{\settowidth{\BCL}{nMOLDYN.Analysis.Analysis.Analysis}\multirow{2}{\BCL}{nMOLDYN.Analysis.Analysis.Analysis}}
&&
  \\\cline{3-3}
  &&\multicolumn{1}{c|}{}
&&
  \\
&&\multicolumn{2}{l}{\textbf{nMOLDYN.Analysis.Scattering.DynamicCoherentStructureFactorAR}}
\end{tabular}

\begin{alltt}
Sets up a Dynamic Coherent Structure Factor analysis using an Auto Regressive model.

A Subclass of nMOLDYN.Analysis.Analysis. 

Constructor: DynamicCoherentStructureFactorARModel({\textbar}parameters{\textbar} = None)

Arguments:

    - {\textbar}parameters{\textbar} -- a dictionnary of the input parameters, or 'None' to set up the analysis without parameters.
        * trajectory        -- a trajectory file name or an instance of MMTK.Trajectory.Trajectory class.
        * timeinfo          -- a string of the form 'first:last:step' where 'first' is an integer specifying the first frame 
                               number to consider, 'last' is an integer specifying the last frame number to consider and 
                               'step' is an integer specifying the step number between two frames.
        * armodelorder      -- an integer in [1, len(trajectory)[ specifying the order of the model
        * qshellvalues      -- a string of the form 'qmin1:qmax1:dq1;qmin2:qmax2:dq2...' where 'qmin1', 'qmin2' ... , 
                               'qmax1', 'qmax2' ... and 'dq1', 'dq2' ... are floats that represents respectively 
                               the q minimum, the q maximum and the q steps for q interval 1, 2 ...
        * qshellwidth       -- a float specifying the width of the q shells.
        * qvectorspershell  -- a float specifying the number of q vectors to generate per q shell.
        * qvectorsgenerator -- a string being one of 'isotropic', 'anisotropic' or 'explicit' specifying the way the q vectors
                               will be generated.
        * qvectorsdirection -- a string of the form 'v1x,v1y,v1z;v2x,v2y,v2z...' where 'v1x', 'v2x' ..., 'v1y', 'v2y' ... and
                               'v1z', 'v2z' ... are floats that represents respectively the x, y and z values of the vectord along 
                               which the q vectors should be generated.
        * subset            -- a selection string specifying the atoms to consider for the analysis.
        * deuteration       -- a selection string specifying the hydrogen atoms whose atomic parameters will be those of the deuterium.
        * weights           -- a string equal to 'equal', 'mass', 'coherent' , 'incoherent' or 'atomicNumber' that specifies the weighting
                               scheme to use.
        * dcsfar            -- the output NetCDF file name.
        * pyroserver        -- a string specifying if Pyro will be used and how to run the analysis.
    
Running modes:

    - To run the analysis do: a.runAnalysis() where a is the analysis object.
    - To estimate the analysis do: a.estimateAnalysis() where a is the analysis object.
    - To save the analysis to 'file' file name do: a.saveAnalysis(file) where a is the analysis object.
    
\end{alltt}


%%%%%%%%%%%%%%%%%%%%%%%%%%%%%%%%%%%%%%%%%%%%%%%%%%%%%%%%%%%%%%%%%%%%%%%%%%%
%%                                Methods                                %%
%%%%%%%%%%%%%%%%%%%%%%%%%%%%%%%%%%%%%%%%%%%%%%%%%%%%%%%%%%%%%%%%%%%%%%%%%%%

  \subsubsection{Methods}

    \vspace{0.5ex}

\hspace{.8\funcindent}\begin{boxedminipage}{\funcwidth}

    \raggedright \textbf{\_\_init\_\_}(\textit{self})

    \vspace{-1.5ex}

    \rule{\textwidth}{0.5\fboxrule}
\setlength{\parskip}{2ex}
    The constructor. Insures that the class can not be instanciated 
    directly from here.

\setlength{\parskip}{1ex}
      \textbf{Parameters}
      \vspace{-1ex}

      \begin{quote}
        \begin{Ventry}{xxxxxxxxxx}

          \item[parameters]

          a dictionnary that contains parameters of the selected analysis.

          \item[statusBar]

          if not None, an instance of nMOLDYN.GUI.Widgets.StatusBar. Will 
          attach a status bar to the selected analysis.

        \end{Ventry}

      \end{quote}

      Overrides: nMOLDYN.Analysis.Analysis.Analysis.\_\_init\_\_

    \end{boxedminipage}

    \label{nMOLDYN:Analysis:Scattering:DynamicCoherentStructureFactorAR:initialize}
    \index{nMOLDYN \textit{(package)}!nMOLDYN.Analysis \textit{(package)}!nMOLDYN.Analysis.Scattering \textit{(module)}!nMOLDYN.Analysis.Scattering.DynamicCoherentStructureFactorAR \textit{(class)}!nMOLDYN.Analysis.Scattering.DynamicCoherentStructureFactorAR.initialize \textit{(method)}}

    \vspace{0.5ex}

\hspace{.8\funcindent}\begin{boxedminipage}{\funcwidth}

    \raggedright \textbf{initialize}(\textit{self})

    \vspace{-1.5ex}

    \rule{\textwidth}{0.5\fboxrule}
\setlength{\parskip}{2ex}
    Initializes the analysis (e.g. parses and checks input parameters, set 
    some variables ...).

\setlength{\parskip}{1ex}
    \end{boxedminipage}

    \label{nMOLDYN:Analysis:Scattering:DynamicCoherentStructureFactorAR:calc}
    \index{nMOLDYN \textit{(package)}!nMOLDYN.Analysis \textit{(package)}!nMOLDYN.Analysis.Scattering \textit{(module)}!nMOLDYN.Analysis.Scattering.DynamicCoherentStructureFactorAR \textit{(class)}!nMOLDYN.Analysis.Scattering.DynamicCoherentStructureFactorAR.calc \textit{(method)}}

    \vspace{0.5ex}

\hspace{.8\funcindent}\begin{boxedminipage}{\funcwidth}

    \raggedright \textbf{calc}(\textit{self}, \textit{qIndex}, \textit{trajname})

    \vspace{-1.5ex}

    \rule{\textwidth}{0.5\fboxrule}
\setlength{\parskip}{2ex}
    Calculates the contribution for one Q-shell.

\setlength{\parskip}{1ex}
      \textbf{Parameters}
      \vspace{-1ex}

      \begin{quote}
        \begin{Ventry}{xxxxxxxx}

          \item[qIndex]

          the index of the Q-shell in {\textbar}self.qRadii{\textbar} list.

            {\it (type=integer.)}

          \item[trajname]

          the name of the trajectory file name.

            {\it (type=string)}

        \end{Ventry}

      \end{quote}

    \end{boxedminipage}

    \label{nMOLDYN:Analysis:Scattering:DynamicCoherentStructureFactorAR:combine}
    \index{nMOLDYN \textit{(package)}!nMOLDYN.Analysis \textit{(package)}!nMOLDYN.Analysis.Scattering \textit{(module)}!nMOLDYN.Analysis.Scattering.DynamicCoherentStructureFactorAR \textit{(class)}!nMOLDYN.Analysis.Scattering.DynamicCoherentStructureFactorAR.combine \textit{(method)}}

    \vspace{0.5ex}

\hspace{.8\funcindent}\begin{boxedminipage}{\funcwidth}

    \raggedright \textbf{combine}(\textit{self}, \textit{qIndex}, \textit{x})

\setlength{\parskip}{2ex}
\setlength{\parskip}{1ex}
    \end{boxedminipage}

    \label{nMOLDYN:Analysis:Scattering:DynamicCoherentStructureFactorAR:finalize}
    \index{nMOLDYN \textit{(package)}!nMOLDYN.Analysis \textit{(package)}!nMOLDYN.Analysis.Scattering \textit{(module)}!nMOLDYN.Analysis.Scattering.DynamicCoherentStructureFactorAR \textit{(class)}!nMOLDYN.Analysis.Scattering.DynamicCoherentStructureFactorAR.finalize \textit{(method)}}

    \vspace{0.5ex}

\hspace{.8\funcindent}\begin{boxedminipage}{\funcwidth}

    \raggedright \textbf{finalize}(\textit{self})

    \vspace{-1.5ex}

    \rule{\textwidth}{0.5\fboxrule}
\setlength{\parskip}{2ex}
    Finalizes the calculations (e.g. averaging the total term, output files
    creations ...).

\setlength{\parskip}{1ex}
    \end{boxedminipage}


\large{\textbf{\textit{Inherited from nMOLDYN.Analysis.Analysis.Analysis\textit{(Section \ref{nMOLDYN:Analysis:Analysis:Analysis})}}}}

\begin{quote}
analysisTime(), buildJobInfo(), buildTimeInfo(), deuterationSelection(), groupSelection(), parseInputParameters(), preLoadTrajectory(), runAnalysis(), saveAnalysis(), setInputParameters(), subsetSelection(), updateJobProgress(), weightingScheme()
\end{quote}

%%%%%%%%%%%%%%%%%%%%%%%%%%%%%%%%%%%%%%%%%%%%%%%%%%%%%%%%%%%%%%%%%%%%%%%%%%%
%%                            Class Variables                            %%
%%%%%%%%%%%%%%%%%%%%%%%%%%%%%%%%%%%%%%%%%%%%%%%%%%%%%%%%%%%%%%%%%%%%%%%%%%%

  \subsubsection{Class Variables}

    \vspace{-1cm}
\hspace{\varindent}\begin{longtable}{|p{\varnamewidth}|p{\vardescrwidth}|l}
\cline{1-2}
\cline{1-2} \centering \textbf{Name} & \centering \textbf{Description}& \\
\cline{1-2}
\endhead\cline{1-2}\multicolumn{3}{r}{\small\textit{continued on next page}}\\\endfoot\cline{1-2}
\endlastfoot\raggedright i\-n\-p\-u\-t\-P\-a\-r\-a\-m\-e\-t\-e\-r\-s\-N\-a\-m\-e\-s\- & \raggedright \textbf{Value:} 
{\tt 'trajectory', 'timeinfo', 'armodelorder', 'qshellvalues',\texttt{...}}&\\
\cline{1-2}
\raggedright s\-h\-o\-r\-t\-N\-a\-m\-e\- & \raggedright \textbf{Value:} 
{\tt 'DCSFAR'}&\\
\cline{1-2}
\raggedright c\-a\-n\-B\-e\-E\-s\-t\-i\-m\-a\-t\-e\-d\- & \raggedright \textbf{Value:} 
{\tt False}&\\
\cline{1-2}
\raggedright d\-e\-f\-a\-u\-l\-t\- & \raggedright \textbf{Value:} 
{\tt \{'weights': 'coherent'\}}&\\
\cline{1-2}
\end{longtable}

    \index{nMOLDYN \textit{(package)}!nMOLDYN.Analysis \textit{(package)}!nMOLDYN.Analysis.Scattering \textit{(module)}!nMOLDYN.Analysis.Scattering.DynamicCoherentStructureFactorAR \textit{(class)}|)}

%%%%%%%%%%%%%%%%%%%%%%%%%%%%%%%%%%%%%%%%%%%%%%%%%%%%%%%%%%%%%%%%%%%%%%%%%%%
%%                           Class Description                           %%
%%%%%%%%%%%%%%%%%%%%%%%%%%%%%%%%%%%%%%%%%%%%%%%%%%%%%%%%%%%%%%%%%%%%%%%%%%%

    \index{nMOLDYN \textit{(package)}!nMOLDYN.Analysis \textit{(package)}!nMOLDYN.Analysis.Scattering \textit{(module)}!nMOLDYN.Analysis.Scattering.DynamicIncoherentStructureFactor \textit{(class)}|(}
\subsection{Class DynamicIncoherentStructureFactor}

    \label{nMOLDYN:Analysis:Scattering:DynamicIncoherentStructureFactor}
\begin{tabular}{cccccc}
% Line for nMOLDYN.Analysis.Analysis.Analysis, linespec=[False]
\multicolumn{2}{r}{\settowidth{\BCL}{nMOLDYN.Analysis.Analysis.Analysis}\multirow{2}{\BCL}{nMOLDYN.Analysis.Analysis.Analysis}}
&&
  \\\cline{3-3}
  &&\multicolumn{1}{c|}{}
&&
  \\
&&\multicolumn{2}{l}{\textbf{nMOLDYN.Analysis.Scattering.DynamicIncoherentStructureFactor}}
\end{tabular}

\begin{alltt}
Sets up an Dynamic Incoherent Structure Factor analysis.

A Subclass of nMOLDYN.Analysis.Analysis. 

Constructor: DynamicIncoherentStructureFactorARModel({\textbar}parameters{\textbar} = None)

Arguments:

    - {\textbar}parameters{\textbar} -- a dictionnary of the input parameters, or 'None' to set up the analysis without parameters.
        * trajectory        -- a trajectory file name or an instance of MMTK.Trajectory.Trajectory class.
        * timeinfo          -- a string of the form 'first:last:step' where 'first' is an integer specifying the first frame 
                               number to consider, 'last' is an integer specifying the last frame number to consider and 
                               'step' is an integer specifying the step number between two frames.
        * qshellvalues      -- a string of the form 'qmin1:qmax1:dq1;qmin2:qmax2:dq2...' where 'qmin1', 'qmin2' ... , 
                               'qmax1', 'qmax2' ... and 'dq1', 'dq2' ... are floats that represents respectively 
                               the q minimum, the q maximum and the q steps for q interval 1, 2 ...
        * qshellwidth       -- a float specifying the width of the q shells.
        * qvectorspershell  -- a float specifying the number of q vectors to generate per q shell.
        * qvectorsgenerator -- a string being one of 'isotropic', 'anisotropic' or 'explicit' specifying the way the q vectors
                               will be generated.
        * qvectorsdirection -- a string of the form 'v1x,v1y,v1z;v2x,v2y,v2z...' where 'v1x', 'v2x' ..., 'v1y', 'v2y' ... and
                               'v1z', 'v2z' ... are floats that represents respectively the x, y and z values of the vectord along 
                               which the q vectors should be generated.
        * fftwindow         -- a float in ]0.0,100.0[ specifying the width of the gaussian, in percentage of the trajectory length
                               that will be used in the smoothing procedure.
        * subset            -- a selection string specifying the atoms to consider for the analysis.
        * deuteration       -- a selection string specifying the hydrogen atoms whose atomic parameters will be those of the deuterium.
        * weights           -- a string equal to 'equal', 'mass', 'coherent' , 'incoherent' or 'atomicNumber' that specifies the weighting
                               scheme to use.
        * disf              -- the output NetCDF file name for the intermediate scattering function.
        * pyroserver        -- a string specifying if Pyro will be used and how to run the analysis.
    
Running modes:

    - To run the analysis do: a.runAnalysis() where a is the analysis object.
    - To estimate the analysis do: a.estimateAnalysis() where a is the analysis object.
    - To save the analysis to 'file' file name do: a.saveAnalysis(file) where a is the analysis object.
    
\end{alltt}


%%%%%%%%%%%%%%%%%%%%%%%%%%%%%%%%%%%%%%%%%%%%%%%%%%%%%%%%%%%%%%%%%%%%%%%%%%%
%%                                Methods                                %%
%%%%%%%%%%%%%%%%%%%%%%%%%%%%%%%%%%%%%%%%%%%%%%%%%%%%%%%%%%%%%%%%%%%%%%%%%%%

  \subsubsection{Methods}

    \vspace{0.5ex}

\hspace{.8\funcindent}\begin{boxedminipage}{\funcwidth}

    \raggedright \textbf{\_\_init\_\_}(\textit{self})

    \vspace{-1.5ex}

    \rule{\textwidth}{0.5\fboxrule}
\setlength{\parskip}{2ex}
    The constructor. Insures that the class can not be instanciated 
    directly from here.

\setlength{\parskip}{1ex}
      \textbf{Parameters}
      \vspace{-1ex}

      \begin{quote}
        \begin{Ventry}{xxxxxxxxxx}

          \item[parameters]

          a dictionnary that contains parameters of the selected analysis.

          \item[statusBar]

          if not None, an instance of nMOLDYN.GUI.Widgets.StatusBar. Will 
          attach a status bar to the selected analysis.

        \end{Ventry}

      \end{quote}

      Overrides: nMOLDYN.Analysis.Analysis.Analysis.\_\_init\_\_

    \end{boxedminipage}

    \label{nMOLDYN:Analysis:Scattering:DynamicIncoherentStructureFactor:initialize}
    \index{nMOLDYN \textit{(package)}!nMOLDYN.Analysis \textit{(package)}!nMOLDYN.Analysis.Scattering \textit{(module)}!nMOLDYN.Analysis.Scattering.DynamicIncoherentStructureFactor \textit{(class)}!nMOLDYN.Analysis.Scattering.DynamicIncoherentStructureFactor.initialize \textit{(method)}}

    \vspace{0.5ex}

\hspace{.8\funcindent}\begin{boxedminipage}{\funcwidth}

    \raggedright \textbf{initialize}(\textit{self})

    \vspace{-1.5ex}

    \rule{\textwidth}{0.5\fboxrule}
\setlength{\parskip}{2ex}
    Initializes the analysis (e.g. parses and checks input parameters, set 
    some variables ...).

\setlength{\parskip}{1ex}
    \end{boxedminipage}

    \label{nMOLDYN:Analysis:Scattering:DynamicIncoherentStructureFactor:calc}
    \index{nMOLDYN \textit{(package)}!nMOLDYN.Analysis \textit{(package)}!nMOLDYN.Analysis.Scattering \textit{(module)}!nMOLDYN.Analysis.Scattering.DynamicIncoherentStructureFactor \textit{(class)}!nMOLDYN.Analysis.Scattering.DynamicIncoherentStructureFactor.calc \textit{(method)}}

    \vspace{0.5ex}

\hspace{.8\funcindent}\begin{boxedminipage}{\funcwidth}

    \raggedright \textbf{calc}(\textit{self}, \textit{atom}, \textit{trajname})

    \vspace{-1.5ex}

    \rule{\textwidth}{0.5\fboxrule}
\setlength{\parskip}{2ex}
    Calculates the atomic term.

\setlength{\parskip}{1ex}
      \textbf{Parameters}
      \vspace{-1ex}

      \begin{quote}
        \begin{Ventry}{xxxxxxxx}

          \item[atom]

          the atom on which the atomic term has been calculated.

            {\it (type=an instance of MMTK.Atom class.)}

          \item[trajname]

          the name of the trajectory file name.

            {\it (type=string)}

        \end{Ventry}

      \end{quote}

    \end{boxedminipage}

    \label{nMOLDYN:Analysis:Scattering:DynamicIncoherentStructureFactor:combine}
    \index{nMOLDYN \textit{(package)}!nMOLDYN.Analysis \textit{(package)}!nMOLDYN.Analysis.Scattering \textit{(module)}!nMOLDYN.Analysis.Scattering.DynamicIncoherentStructureFactor \textit{(class)}!nMOLDYN.Analysis.Scattering.DynamicIncoherentStructureFactor.combine \textit{(method)}}

    \vspace{0.5ex}

\hspace{.8\funcindent}\begin{boxedminipage}{\funcwidth}

    \raggedright \textbf{combine}(\textit{self}, \textit{atom}, \textit{x})

\setlength{\parskip}{2ex}
\setlength{\parskip}{1ex}
    \end{boxedminipage}

    \label{nMOLDYN:Analysis:Scattering:DynamicIncoherentStructureFactor:finalize}
    \index{nMOLDYN \textit{(package)}!nMOLDYN.Analysis \textit{(package)}!nMOLDYN.Analysis.Scattering \textit{(module)}!nMOLDYN.Analysis.Scattering.DynamicIncoherentStructureFactor \textit{(class)}!nMOLDYN.Analysis.Scattering.DynamicIncoherentStructureFactor.finalize \textit{(method)}}

    \vspace{0.5ex}

\hspace{.8\funcindent}\begin{boxedminipage}{\funcwidth}

    \raggedright \textbf{finalize}(\textit{self})

    \vspace{-1.5ex}

    \rule{\textwidth}{0.5\fboxrule}
\setlength{\parskip}{2ex}
    Finalizes the calculations (e.g. averaging the total term, output files
    creations ...)

\setlength{\parskip}{1ex}
    \end{boxedminipage}


\large{\textbf{\textit{Inherited from nMOLDYN.Analysis.Analysis.Analysis\textit{(Section \ref{nMOLDYN:Analysis:Analysis:Analysis})}}}}

\begin{quote}
analysisTime(), buildJobInfo(), buildTimeInfo(), deuterationSelection(), groupSelection(), parseInputParameters(), preLoadTrajectory(), runAnalysis(), saveAnalysis(), setInputParameters(), subsetSelection(), updateJobProgress(), weightingScheme()
\end{quote}

%%%%%%%%%%%%%%%%%%%%%%%%%%%%%%%%%%%%%%%%%%%%%%%%%%%%%%%%%%%%%%%%%%%%%%%%%%%
%%                            Class Variables                            %%
%%%%%%%%%%%%%%%%%%%%%%%%%%%%%%%%%%%%%%%%%%%%%%%%%%%%%%%%%%%%%%%%%%%%%%%%%%%

  \subsubsection{Class Variables}

    \vspace{-1cm}
\hspace{\varindent}\begin{longtable}{|p{\varnamewidth}|p{\vardescrwidth}|l}
\cline{1-2}
\cline{1-2} \centering \textbf{Name} & \centering \textbf{Description}& \\
\cline{1-2}
\endhead\cline{1-2}\multicolumn{3}{r}{\small\textit{continued on next page}}\\\endfoot\cline{1-2}
\endlastfoot\raggedright i\-n\-p\-u\-t\-P\-a\-r\-a\-m\-e\-t\-e\-r\-s\-N\-a\-m\-e\-s\- & \raggedright \textbf{Value:} 
{\tt 'trajectory', 'timeinfo', 'qshellvalues', 'qshellwidth', \texttt{...}}&\\
\cline{1-2}
\raggedright d\-e\-f\-a\-u\-l\-t\- & \raggedright \textbf{Value:} 
{\tt \{'weights': 'incoherent'\}}&\\
\cline{1-2}
\raggedright s\-h\-o\-r\-t\-N\-a\-m\-e\- & \raggedright \textbf{Value:} 
{\tt 'DISF'}&\\
\cline{1-2}
\raggedright c\-a\-n\-B\-e\-E\-s\-t\-i\-m\-a\-t\-e\-d\- & \raggedright \textbf{Value:} 
{\tt True}&\\
\cline{1-2}
\end{longtable}

    \index{nMOLDYN \textit{(package)}!nMOLDYN.Analysis \textit{(package)}!nMOLDYN.Analysis.Scattering \textit{(module)}!nMOLDYN.Analysis.Scattering.DynamicIncoherentStructureFactor \textit{(class)}|)}

%%%%%%%%%%%%%%%%%%%%%%%%%%%%%%%%%%%%%%%%%%%%%%%%%%%%%%%%%%%%%%%%%%%%%%%%%%%
%%                           Class Description                           %%
%%%%%%%%%%%%%%%%%%%%%%%%%%%%%%%%%%%%%%%%%%%%%%%%%%%%%%%%%%%%%%%%%%%%%%%%%%%

    \index{nMOLDYN \textit{(package)}!nMOLDYN.Analysis \textit{(package)}!nMOLDYN.Analysis.Scattering \textit{(module)}!nMOLDYN.Analysis.Scattering.DynamicIncoherentStructureFactorAR \textit{(class)}|(}
\subsection{Class DynamicIncoherentStructureFactorAR}

    \label{nMOLDYN:Analysis:Scattering:DynamicIncoherentStructureFactorAR}
\begin{tabular}{cccccc}
% Line for nMOLDYN.Analysis.Analysis.Analysis, linespec=[False]
\multicolumn{2}{r}{\settowidth{\BCL}{nMOLDYN.Analysis.Analysis.Analysis}\multirow{2}{\BCL}{nMOLDYN.Analysis.Analysis.Analysis}}
&&
  \\\cline{3-3}
  &&\multicolumn{1}{c|}{}
&&
  \\
&&\multicolumn{2}{l}{\textbf{nMOLDYN.Analysis.Scattering.DynamicIncoherentStructureFactorAR}}
\end{tabular}

\begin{alltt}
Sets up an Dynamic Incoherent Structure Factor analysis using an Auto Regressive model.

A Subclass of nMOLDYN.Analysis.Analysis. 

Constructor: DynamicIncoherentStructureFactorARModel({\textbar}parameters{\textbar} = None)

Arguments:

    - {\textbar}parameters{\textbar} -- a dictionnary of the input parameters, or 'None' to set up the analysis without parameters.
        * trajectory        -- a trajectory file name or an instance of MMTK.Trajectory.Trajectory class.
        * timeinfo          -- a string of the form 'first:last:step' where 'first' is an integer specifying the first frame 
                               number to consider, 'last' is an integer specifying the last frame number to consider and 
                               'step' is an integer specifying the step number between two frames.
        * armodelorder      -- an integer in [1, len(trajectory)[ specifying the order of the model
        * qshellvalues      -- a string of the form 'qmin1:qmax1:dq1;qmin2:qmax2:dq2...' where 'qmin1', 'qmin2' ... , 
                               'qmax1', 'qmax2' ... and 'dq1', 'dq2' ... are floats that represents respectively 
                               the q minimum, the q maximum and the q steps for q interval 1, 2 ...
        * qshellwidth       -- a float specifying the width of the q shells.
        * qvectorspershell  -- a float specifying the number of q vectors to generate per q shell.
        * qvectorsgenerator -- a string being one of 'isotropic', 'anisotropic' or 'explicit' specifying the way the q vectors
                               will be generated.
        * qvectorsdirection -- a string of the form 'v1x,v1y,v1z;v2x,v2y,v2z...' where 'v1x', 'v2x' ..., 'v1y', 'v2y' ... and
                               'v1z', 'v2z' ... are floats that represents respectively the x, y and z values of the vectord along 
                               which the q vectors should be generated.
        * subset            -- a selection string specifying the atoms to consider for the analysis.
        * deuteration       -- a selection string specifying the hydrogen atoms whose atomic parameters will be those of the deuterium.
        * weights           -- a string equal to 'equal', 'mass', 'coherent' , 'incoherent' or 'atomicNumber' that specifies the weighting
                               scheme to use.
        * disfar            -- the output NetCDF file name for the intermediate scattering function.
        * pyroserver        -- a string specifying if Pyro will be used and how to run the analysis.
    
Running modes:

    - To run the analysis do: a.runAnalysis() where a is the analysis object.
    - To estimate the analysis do: a.estimateAnalysis() where a is the analysis object.
    - To save the analysis to 'file' file name do: a.saveAnalysis(file) where a is the analysis object.
    
\end{alltt}


%%%%%%%%%%%%%%%%%%%%%%%%%%%%%%%%%%%%%%%%%%%%%%%%%%%%%%%%%%%%%%%%%%%%%%%%%%%
%%                                Methods                                %%
%%%%%%%%%%%%%%%%%%%%%%%%%%%%%%%%%%%%%%%%%%%%%%%%%%%%%%%%%%%%%%%%%%%%%%%%%%%

  \subsubsection{Methods}

    \vspace{0.5ex}

\hspace{.8\funcindent}\begin{boxedminipage}{\funcwidth}

    \raggedright \textbf{\_\_init\_\_}(\textit{self})

    \vspace{-1.5ex}

    \rule{\textwidth}{0.5\fboxrule}
\setlength{\parskip}{2ex}
    The constructor. Insures that the class can not be instanciated 
    directly from here.

\setlength{\parskip}{1ex}
      \textbf{Parameters}
      \vspace{-1ex}

      \begin{quote}
        \begin{Ventry}{xxxxxxxxxx}

          \item[parameters]

          a dictionnary that contains parameters of the selected analysis.

          \item[statusBar]

          if not None, an instance of nMOLDYN.GUI.Widgets.StatusBar. Will 
          attach a status bar to the selected analysis.

        \end{Ventry}

      \end{quote}

      Overrides: nMOLDYN.Analysis.Analysis.Analysis.\_\_init\_\_

    \end{boxedminipage}

    \label{nMOLDYN:Analysis:Scattering:DynamicIncoherentStructureFactorAR:initialize}
    \index{nMOLDYN \textit{(package)}!nMOLDYN.Analysis \textit{(package)}!nMOLDYN.Analysis.Scattering \textit{(module)}!nMOLDYN.Analysis.Scattering.DynamicIncoherentStructureFactorAR \textit{(class)}!nMOLDYN.Analysis.Scattering.DynamicIncoherentStructureFactorAR.initialize \textit{(method)}}

    \vspace{0.5ex}

\hspace{.8\funcindent}\begin{boxedminipage}{\funcwidth}

    \raggedright \textbf{initialize}(\textit{self})

    \vspace{-1.5ex}

    \rule{\textwidth}{0.5\fboxrule}
\setlength{\parskip}{2ex}
    Initializes the analysis (e.g. parses and checks input parameters, set 
    some variables ...).

\setlength{\parskip}{1ex}
    \end{boxedminipage}

    \label{nMOLDYN:Analysis:Scattering:DynamicIncoherentStructureFactorAR:calc}
    \index{nMOLDYN \textit{(package)}!nMOLDYN.Analysis \textit{(package)}!nMOLDYN.Analysis.Scattering \textit{(module)}!nMOLDYN.Analysis.Scattering.DynamicIncoherentStructureFactorAR \textit{(class)}!nMOLDYN.Analysis.Scattering.DynamicIncoherentStructureFactorAR.calc \textit{(method)}}

    \vspace{0.5ex}

\hspace{.8\funcindent}\begin{boxedminipage}{\funcwidth}

    \raggedright \textbf{calc}(\textit{self}, \textit{atom}, \textit{trajname})

    \vspace{-1.5ex}

    \rule{\textwidth}{0.5\fboxrule}
\setlength{\parskip}{2ex}
    Calculates the atomic term.

\setlength{\parskip}{1ex}
      \textbf{Parameters}
      \vspace{-1ex}

      \begin{quote}
        \begin{Ventry}{xxxxxxxx}

          \item[atom]

          the atom on which the atomic term has been calculated.

            {\it (type=an instance of MMTK.Atom class.)}

          \item[trajname]

          the name of the trajectory file name.

            {\it (type=string)}

        \end{Ventry}

      \end{quote}

    \end{boxedminipage}

    \label{nMOLDYN:Analysis:Scattering:DynamicIncoherentStructureFactorAR:combine}
    \index{nMOLDYN \textit{(package)}!nMOLDYN.Analysis \textit{(package)}!nMOLDYN.Analysis.Scattering \textit{(module)}!nMOLDYN.Analysis.Scattering.DynamicIncoherentStructureFactorAR \textit{(class)}!nMOLDYN.Analysis.Scattering.DynamicIncoherentStructureFactorAR.combine \textit{(method)}}

    \vspace{0.5ex}

\hspace{.8\funcindent}\begin{boxedminipage}{\funcwidth}

    \raggedright \textbf{combine}(\textit{self}, \textit{atom}, \textit{x})

\setlength{\parskip}{2ex}
\setlength{\parskip}{1ex}
    \end{boxedminipage}

    \label{nMOLDYN:Analysis:Scattering:DynamicIncoherentStructureFactorAR:finalize}
    \index{nMOLDYN \textit{(package)}!nMOLDYN.Analysis \textit{(package)}!nMOLDYN.Analysis.Scattering \textit{(module)}!nMOLDYN.Analysis.Scattering.DynamicIncoherentStructureFactorAR \textit{(class)}!nMOLDYN.Analysis.Scattering.DynamicIncoherentStructureFactorAR.finalize \textit{(method)}}

    \vspace{0.5ex}

\hspace{.8\funcindent}\begin{boxedminipage}{\funcwidth}

    \raggedright \textbf{finalize}(\textit{self})

    \vspace{-1.5ex}

    \rule{\textwidth}{0.5\fboxrule}
\setlength{\parskip}{2ex}
    Finalizes the calculations (e.g. averaging the total term, output files
    creations ...).

\setlength{\parskip}{1ex}
    \end{boxedminipage}


\large{\textbf{\textit{Inherited from nMOLDYN.Analysis.Analysis.Analysis\textit{(Section \ref{nMOLDYN:Analysis:Analysis:Analysis})}}}}

\begin{quote}
analysisTime(), buildJobInfo(), buildTimeInfo(), deuterationSelection(), groupSelection(), parseInputParameters(), preLoadTrajectory(), runAnalysis(), saveAnalysis(), setInputParameters(), subsetSelection(), updateJobProgress(), weightingScheme()
\end{quote}

%%%%%%%%%%%%%%%%%%%%%%%%%%%%%%%%%%%%%%%%%%%%%%%%%%%%%%%%%%%%%%%%%%%%%%%%%%%
%%                            Class Variables                            %%
%%%%%%%%%%%%%%%%%%%%%%%%%%%%%%%%%%%%%%%%%%%%%%%%%%%%%%%%%%%%%%%%%%%%%%%%%%%

  \subsubsection{Class Variables}

    \vspace{-1cm}
\hspace{\varindent}\begin{longtable}{|p{\varnamewidth}|p{\vardescrwidth}|l}
\cline{1-2}
\cline{1-2} \centering \textbf{Name} & \centering \textbf{Description}& \\
\cline{1-2}
\endhead\cline{1-2}\multicolumn{3}{r}{\small\textit{continued on next page}}\\\endfoot\cline{1-2}
\endlastfoot\raggedright i\-n\-p\-u\-t\-P\-a\-r\-a\-m\-e\-t\-e\-r\-s\-N\-a\-m\-e\-s\- & \raggedright \textbf{Value:} 
{\tt 'trajectory', 'timeInfo', 'armodelorder', 'qshellvalues',\texttt{...}}&\\
\cline{1-2}
\raggedright s\-h\-o\-r\-t\-N\-a\-m\-e\- & \raggedright \textbf{Value:} 
{\tt 'DISFAR'}&\\
\cline{1-2}
\raggedright c\-a\-n\-B\-e\-E\-s\-t\-i\-m\-a\-t\-e\-d\- & \raggedright \textbf{Value:} 
{\tt True}&\\
\cline{1-2}
\raggedright d\-e\-f\-a\-u\-l\-t\- & \raggedright \textbf{Value:} 
{\tt \{'weights': 'incoherent'\}}&\\
\cline{1-2}
\end{longtable}

    \index{nMOLDYN \textit{(package)}!nMOLDYN.Analysis \textit{(package)}!nMOLDYN.Analysis.Scattering \textit{(module)}!nMOLDYN.Analysis.Scattering.DynamicIncoherentStructureFactorAR \textit{(class)}|)}

%%%%%%%%%%%%%%%%%%%%%%%%%%%%%%%%%%%%%%%%%%%%%%%%%%%%%%%%%%%%%%%%%%%%%%%%%%%
%%                           Class Description                           %%
%%%%%%%%%%%%%%%%%%%%%%%%%%%%%%%%%%%%%%%%%%%%%%%%%%%%%%%%%%%%%%%%%%%%%%%%%%%

    \index{nMOLDYN \textit{(package)}!nMOLDYN.Analysis \textit{(package)}!nMOLDYN.Analysis.Scattering \textit{(module)}!nMOLDYN.Analysis.Scattering.DynamicIncoherentStructureFactorGaussian \textit{(class)}|(}
\subsection{Class DynamicIncoherentStructureFactorGaussian}

    \label{nMOLDYN:Analysis:Scattering:DynamicIncoherentStructureFactorGaussian}
\begin{tabular}{cccccc}
% Line for nMOLDYN.Analysis.Analysis.Analysis, linespec=[False]
\multicolumn{2}{r}{\settowidth{\BCL}{nMOLDYN.Analysis.Analysis.Analysis}\multirow{2}{\BCL}{nMOLDYN.Analysis.Analysis.Analysis}}
&&
  \\\cline{3-3}
  &&\multicolumn{1}{c|}{}
&&
  \\
&&\multicolumn{2}{l}{\textbf{nMOLDYN.Analysis.Scattering.DynamicIncoherentStructureFactorGaussian}}
\end{tabular}

\begin{alltt}
Sets up an Dynamic Incoherent Structure Factor analysis within Gaussian approximation.

A Subclass of nMOLDYN.Analysis.Analysis. 

Constructor: DynamicIncoherentStructureFactorGaussian({\textbar}parameters{\textbar} = None)

Arguments:

    - {\textbar}parameters{\textbar} -- a dictionnary of the input parameters, or 'None' to set up the analysis without parameters.
        * trajectory   -- a trajectory file name or an instance of MMTK.Trajectory.Trajectory class.
        * timeinfo     -- a string of the form 'first:last:step' where 'first' is an integer specifying the first frame 
                          number to consider, 'last' is an integer specifying the last frame number to consider and 
                          'step' is an integer specifying the step number between two frames.
        * qshellvalues -- a string of the form 'qmin1:qmax1:dq1;qmin2:qmax2:dq2...' where 'qmin1', 'qmin2' ... , 
                          'qmax1', 'qmax2' ... and 'dq1', 'dq2' ... are floats that represents respectively 
                          the q minimum, the q maximum and the q steps for q interval 1, 2 ...
        * fftwindow    -- a float in ]0.0,100.0[ specifying the width of the gaussian, in percentage of the trajectory length
                          that will be used in the smoothing procedure.
        * subset       -- a selection string specifying the atoms to consider for the analysis.
        * deuteration  -- a selection string specifying the hydrogen atoms whose atomic parameters will be those of the deuterium.
        * weights      -- a string equal to 'equal', 'mass', 'coherent' , 'incoherent' or 'atomicNumber' that specifies the weighting
                          scheme to use.
        * disfg        -- the output NetCDF file name for the intermediate scattering function.
        * pyroserver   -- a string specifying if Pyro will be used and how to run the analysis.
    
Running modes:

    - To run the analysis do: a.runAnalysis() where a is the analysis object.
    - To estimate the analysis do: a.estimateAnalysis() where a is the analysis object.
    - To save the analysis to 'file' file name do: a.saveAnalysis(file) where a is the analysis object.
    
\end{alltt}


%%%%%%%%%%%%%%%%%%%%%%%%%%%%%%%%%%%%%%%%%%%%%%%%%%%%%%%%%%%%%%%%%%%%%%%%%%%
%%                                Methods                                %%
%%%%%%%%%%%%%%%%%%%%%%%%%%%%%%%%%%%%%%%%%%%%%%%%%%%%%%%%%%%%%%%%%%%%%%%%%%%

  \subsubsection{Methods}

    \vspace{0.5ex}

\hspace{.8\funcindent}\begin{boxedminipage}{\funcwidth}

    \raggedright \textbf{\_\_init\_\_}(\textit{self})

    \vspace{-1.5ex}

    \rule{\textwidth}{0.5\fboxrule}
\setlength{\parskip}{2ex}
    The constructor. Insures that the class can not be instanciated 
    directly from here.

\setlength{\parskip}{1ex}
      \textbf{Parameters}
      \vspace{-1ex}

      \begin{quote}
        \begin{Ventry}{xxxxxxxxxx}

          \item[parameters]

          a dictionnary that contains parameters of the selected analysis.

          \item[statusBar]

          if not None, an instance of nMOLDYN.GUI.Widgets.StatusBar. Will 
          attach a status bar to the selected analysis.

        \end{Ventry}

      \end{quote}

      Overrides: nMOLDYN.Analysis.Analysis.Analysis.\_\_init\_\_

    \end{boxedminipage}

    \label{nMOLDYN:Analysis:Scattering:DynamicIncoherentStructureFactorGaussian:initialize}
    \index{nMOLDYN \textit{(package)}!nMOLDYN.Analysis \textit{(package)}!nMOLDYN.Analysis.Scattering \textit{(module)}!nMOLDYN.Analysis.Scattering.DynamicIncoherentStructureFactorGaussian \textit{(class)}!nMOLDYN.Analysis.Scattering.DynamicIncoherentStructureFactorGaussian.initialize \textit{(method)}}

    \vspace{0.5ex}

\hspace{.8\funcindent}\begin{boxedminipage}{\funcwidth}

    \raggedright \textbf{initialize}(\textit{self})

    \vspace{-1.5ex}

    \rule{\textwidth}{0.5\fboxrule}
\setlength{\parskip}{2ex}
    Initializes the analysis (e.g. parses and checks input parameters, set 
    some variables ...).

\setlength{\parskip}{1ex}
    \end{boxedminipage}

    \label{nMOLDYN:Analysis:Scattering:DynamicIncoherentStructureFactorGaussian:calc}
    \index{nMOLDYN \textit{(package)}!nMOLDYN.Analysis \textit{(package)}!nMOLDYN.Analysis.Scattering \textit{(module)}!nMOLDYN.Analysis.Scattering.DynamicIncoherentStructureFactorGaussian \textit{(class)}!nMOLDYN.Analysis.Scattering.DynamicIncoherentStructureFactorGaussian.calc \textit{(method)}}

    \vspace{0.5ex}

\hspace{.8\funcindent}\begin{boxedminipage}{\funcwidth}

    \raggedright \textbf{calc}(\textit{self}, \textit{atom}, \textit{trajname})

    \vspace{-1.5ex}

    \rule{\textwidth}{0.5\fboxrule}
\setlength{\parskip}{2ex}
    Calculates the atomic term.

\setlength{\parskip}{1ex}
      \textbf{Parameters}
      \vspace{-1ex}

      \begin{quote}
        \begin{Ventry}{xxxxxxxx}

          \item[atom]

          the atom on which the atomic term has been calculated.

            {\it (type=an instance of MMTK.Atom class.)}

          \item[trajname]

          the name of the trajectory file name.

            {\it (type=string)}

        \end{Ventry}

      \end{quote}

    \end{boxedminipage}

    \label{nMOLDYN:Analysis:Scattering:DynamicIncoherentStructureFactorGaussian:combine}
    \index{nMOLDYN \textit{(package)}!nMOLDYN.Analysis \textit{(package)}!nMOLDYN.Analysis.Scattering \textit{(module)}!nMOLDYN.Analysis.Scattering.DynamicIncoherentStructureFactorGaussian \textit{(class)}!nMOLDYN.Analysis.Scattering.DynamicIncoherentStructureFactorGaussian.combine \textit{(method)}}

    \vspace{0.5ex}

\hspace{.8\funcindent}\begin{boxedminipage}{\funcwidth}

    \raggedright \textbf{combine}(\textit{self}, \textit{atom}, \textit{x})

\setlength{\parskip}{2ex}
\setlength{\parskip}{1ex}
    \end{boxedminipage}

    \label{nMOLDYN:Analysis:Scattering:DynamicIncoherentStructureFactorGaussian:finalize}
    \index{nMOLDYN \textit{(package)}!nMOLDYN.Analysis \textit{(package)}!nMOLDYN.Analysis.Scattering \textit{(module)}!nMOLDYN.Analysis.Scattering.DynamicIncoherentStructureFactorGaussian \textit{(class)}!nMOLDYN.Analysis.Scattering.DynamicIncoherentStructureFactorGaussian.finalize \textit{(method)}}

    \vspace{0.5ex}

\hspace{.8\funcindent}\begin{boxedminipage}{\funcwidth}

    \raggedright \textbf{finalize}(\textit{self})

    \vspace{-1.5ex}

    \rule{\textwidth}{0.5\fboxrule}
\setlength{\parskip}{2ex}
    Finalizes the calculations (e.g. averaging the total term, output files
    creations ...)

\setlength{\parskip}{1ex}
    \end{boxedminipage}

    \label{nMOLDYN:Analysis:Scattering:DynamicIncoherentStructureFactorGaussian:getMSD}
    \index{nMOLDYN \textit{(package)}!nMOLDYN.Analysis \textit{(package)}!nMOLDYN.Analysis.Scattering \textit{(module)}!nMOLDYN.Analysis.Scattering.DynamicIncoherentStructureFactorGaussian \textit{(class)}!nMOLDYN.Analysis.Scattering.DynamicIncoherentStructureFactorGaussian.getMSD \textit{(method)}}

    \vspace{0.5ex}

\hspace{.8\funcindent}\begin{boxedminipage}{\funcwidth}

    \raggedright \textbf{getMSD}(\textit{self}, \textit{series})

    \vspace{-1.5ex}

    \rule{\textwidth}{0.5\fboxrule}
\setlength{\parskip}{2ex}
    Computes the atomic component of the Mean-Square-Displacement. This is 
    the exact copy of the version written in nMOLDYN.Simulations.Dynamics 
    but rewritten here for to keep the module Scattering independant from 
    module Dynamics.

\setlength{\parskip}{1ex}
    \end{boxedminipage}


\large{\textbf{\textit{Inherited from nMOLDYN.Analysis.Analysis.Analysis\textit{(Section \ref{nMOLDYN:Analysis:Analysis:Analysis})}}}}

\begin{quote}
analysisTime(), buildJobInfo(), buildTimeInfo(), deuterationSelection(), groupSelection(), parseInputParameters(), preLoadTrajectory(), runAnalysis(), saveAnalysis(), setInputParameters(), subsetSelection(), updateJobProgress(), weightingScheme()
\end{quote}

%%%%%%%%%%%%%%%%%%%%%%%%%%%%%%%%%%%%%%%%%%%%%%%%%%%%%%%%%%%%%%%%%%%%%%%%%%%
%%                            Class Variables                            %%
%%%%%%%%%%%%%%%%%%%%%%%%%%%%%%%%%%%%%%%%%%%%%%%%%%%%%%%%%%%%%%%%%%%%%%%%%%%

  \subsubsection{Class Variables}

    \vspace{-1cm}
\hspace{\varindent}\begin{longtable}{|p{\varnamewidth}|p{\vardescrwidth}|l}
\cline{1-2}
\cline{1-2} \centering \textbf{Name} & \centering \textbf{Description}& \\
\cline{1-2}
\endhead\cline{1-2}\multicolumn{3}{r}{\small\textit{continued on next page}}\\\endfoot\cline{1-2}
\endlastfoot\raggedright i\-n\-p\-u\-t\-P\-a\-r\-a\-m\-e\-t\-e\-r\-s\-N\-a\-m\-e\-s\- & \raggedright \textbf{Value:} 
{\tt 'trajectory', 'timeinfo', 'qshellvalues', 'fftwindow', 's\texttt{...}}&\\
\cline{1-2}
\raggedright d\-e\-f\-a\-u\-l\-t\- & \raggedright \textbf{Value:} 
{\tt \{'weights': 'incoherent'\}}&\\
\cline{1-2}
\raggedright s\-h\-o\-r\-t\-N\-a\-m\-e\- & \raggedright \textbf{Value:} 
{\tt 'DISFG'}&\\
\cline{1-2}
\raggedright c\-a\-n\-B\-e\-E\-s\-t\-i\-m\-a\-t\-e\-d\- & \raggedright \textbf{Value:} 
{\tt True}&\\
\cline{1-2}
\end{longtable}

    \index{nMOLDYN \textit{(package)}!nMOLDYN.Analysis \textit{(package)}!nMOLDYN.Analysis.Scattering \textit{(module)}!nMOLDYN.Analysis.Scattering.DynamicIncoherentStructureFactorGaussian \textit{(class)}|)}

%%%%%%%%%%%%%%%%%%%%%%%%%%%%%%%%%%%%%%%%%%%%%%%%%%%%%%%%%%%%%%%%%%%%%%%%%%%
%%                           Class Description                           %%
%%%%%%%%%%%%%%%%%%%%%%%%%%%%%%%%%%%%%%%%%%%%%%%%%%%%%%%%%%%%%%%%%%%%%%%%%%%

    \index{nMOLDYN \textit{(package)}!nMOLDYN.Analysis \textit{(package)}!nMOLDYN.Analysis.Scattering \textit{(module)}!nMOLDYN.Analysis.Scattering.ElasticIncoherentStructureFactor \textit{(class)}|(}
\subsection{Class ElasticIncoherentStructureFactor}

    \label{nMOLDYN:Analysis:Scattering:ElasticIncoherentStructureFactor}
\begin{tabular}{cccccc}
% Line for nMOLDYN.Analysis.Analysis.Analysis, linespec=[False]
\multicolumn{2}{r}{\settowidth{\BCL}{nMOLDYN.Analysis.Analysis.Analysis}\multirow{2}{\BCL}{nMOLDYN.Analysis.Analysis.Analysis}}
&&
  \\\cline{3-3}
  &&\multicolumn{1}{c|}{}
&&
  \\
&&\multicolumn{2}{l}{\textbf{nMOLDYN.Analysis.Scattering.ElasticIncoherentStructureFactor}}
\end{tabular}

\begin{alltt}
Sets up an Elastic Incoherent Structure Factor.

A Subclass of nMOLDYN.Analysis.Analysis. 

Constructor: ElasticIncoherentStructureFactor({\textbar}parameters{\textbar} = None)

Arguments:

    - {\textbar}parameters{\textbar} -- a dictionnary of the input parameters, or 'None' to set up the analysis without parameters.
        * trajectory        -- a trajectory file name or an instance of MMTK.Trajectory.Trajectory class.
        * timeinfo          -- a string of the form 'first:last:step' where 'first' is an integer specifying the first frame 
                               number to consider, 'last' is an integer specifying the last frame number to consider and 
                               'step' is an integer specifying the step number between two frames.
        * qshellvalues      -- a string of the form 'qmin1:qmax1:dq1;qmin2:qmax2:dq2...' where 'qmin1', 'qmin2' ... , 
                               'qmax1', 'qmax2' ... and 'dq1', 'dq2' ... are floats that represents respectively 
                               the q minimum, the q maximum and the q steps for q interval 1, 2 ...
        * qshellwidth       -- a float specifying the width of the q shells.
        * qvectorspershell  -- a float specifying the number of q vectors to generate per q shell.
        * qvectorsgenerator -- a string being one of 'isotropic', 'anisotropic' or 'explicit' specifying the way the q vectors
                               will be generated.
        * qvectorsdirection -- a string of the form 'v1x,v1y,v1z;v2x,v2y,v2z...' where 'v1x', 'v2x' ..., 'v1y', 'v2y' ... and
                               'v1z', 'v2z' ... are floats that represents respectively the x, y and z values of the vectord along 
                               which the q vectors should be generated.
        * subset            -- a selection string specifying the atoms to consider for the analysis.
        * deuteration       -- a selection string specifying the hydrogen atoms whose atomic parameters will be those of the deuterium.
        * weights           -- a string equal to 'equal', 'mass', 'coherent' , 'incoherent' or 'atomicNumber' that specifies the weighting
                               scheme to use.
        * eisf              -- the output NetCDF file name. A CDL version of this file will also be generated with the '.cdl' extension
                               instead of the '.nc' extension.
        * pyroserver        -- a string specifying if Pyro will be used and how to run the analysis.
    
Running modes:

    - To run the analysis do: a.runAnalysis() where a is the analysis object.
    - To estimate the analysis do: a.estimateAnalysis() where a is the analysis object.
    - To save the analysis to 'file' file name do: a.saveAnalysis(file) where a is the analysis object.
    
\end{alltt}


%%%%%%%%%%%%%%%%%%%%%%%%%%%%%%%%%%%%%%%%%%%%%%%%%%%%%%%%%%%%%%%%%%%%%%%%%%%
%%                                Methods                                %%
%%%%%%%%%%%%%%%%%%%%%%%%%%%%%%%%%%%%%%%%%%%%%%%%%%%%%%%%%%%%%%%%%%%%%%%%%%%

  \subsubsection{Methods}

    \vspace{0.5ex}

\hspace{.8\funcindent}\begin{boxedminipage}{\funcwidth}

    \raggedright \textbf{\_\_init\_\_}(\textit{self})

    \vspace{-1.5ex}

    \rule{\textwidth}{0.5\fboxrule}
\setlength{\parskip}{2ex}
    The constructor. Insures that the class can not be instanciated 
    directly from here.

\setlength{\parskip}{1ex}
      \textbf{Parameters}
      \vspace{-1ex}

      \begin{quote}
        \begin{Ventry}{xxxxxxxxxx}

          \item[parameters]

          a dictionnary that contains parameters of the selected analysis.

          \item[statusBar]

          if not None, an instance of nMOLDYN.GUI.Widgets.StatusBar. Will 
          attach a status bar to the selected analysis.

        \end{Ventry}

      \end{quote}

      Overrides: nMOLDYN.Analysis.Analysis.Analysis.\_\_init\_\_

    \end{boxedminipage}

    \label{nMOLDYN:Analysis:Scattering:ElasticIncoherentStructureFactor:initialize}
    \index{nMOLDYN \textit{(package)}!nMOLDYN.Analysis \textit{(package)}!nMOLDYN.Analysis.Scattering \textit{(module)}!nMOLDYN.Analysis.Scattering.ElasticIncoherentStructureFactor \textit{(class)}!nMOLDYN.Analysis.Scattering.ElasticIncoherentStructureFactor.initialize \textit{(method)}}

    \vspace{0.5ex}

\hspace{.8\funcindent}\begin{boxedminipage}{\funcwidth}

    \raggedright \textbf{initialize}(\textit{self})

    \vspace{-1.5ex}

    \rule{\textwidth}{0.5\fboxrule}
\setlength{\parskip}{2ex}
    Initializes the analysis (e.g. parses and checks input parameters, set 
    some variables ...).

\setlength{\parskip}{1ex}
    \end{boxedminipage}

    \label{nMOLDYN:Analysis:Scattering:ElasticIncoherentStructureFactor:calc}
    \index{nMOLDYN \textit{(package)}!nMOLDYN.Analysis \textit{(package)}!nMOLDYN.Analysis.Scattering \textit{(module)}!nMOLDYN.Analysis.Scattering.ElasticIncoherentStructureFactor \textit{(class)}!nMOLDYN.Analysis.Scattering.ElasticIncoherentStructureFactor.calc \textit{(method)}}

    \vspace{0.5ex}

\hspace{.8\funcindent}\begin{boxedminipage}{\funcwidth}

    \raggedright \textbf{calc}(\textit{self}, \textit{atom}, \textit{trajname})

    \vspace{-1.5ex}

    \rule{\textwidth}{0.5\fboxrule}
\setlength{\parskip}{2ex}
    Calculates the atomic term.

\setlength{\parskip}{1ex}
      \textbf{Parameters}
      \vspace{-1ex}

      \begin{quote}
        \begin{Ventry}{xxxxxxxx}

          \item[atom]

          the atom on which the atomic term has been calculated.

            {\it (type=an instance of MMTK.Atom class.)}

          \item[trajname]

          the name of the trajectory file name.

            {\it (type=string)}

        \end{Ventry}

      \end{quote}

    \end{boxedminipage}

    \label{nMOLDYN:Analysis:Scattering:ElasticIncoherentStructureFactor:combine}
    \index{nMOLDYN \textit{(package)}!nMOLDYN.Analysis \textit{(package)}!nMOLDYN.Analysis.Scattering \textit{(module)}!nMOLDYN.Analysis.Scattering.ElasticIncoherentStructureFactor \textit{(class)}!nMOLDYN.Analysis.Scattering.ElasticIncoherentStructureFactor.combine \textit{(method)}}

    \vspace{0.5ex}

\hspace{.8\funcindent}\begin{boxedminipage}{\funcwidth}

    \raggedright \textbf{combine}(\textit{self}, \textit{atom}, \textit{x})

\setlength{\parskip}{2ex}
\setlength{\parskip}{1ex}
    \end{boxedminipage}

    \label{nMOLDYN:Analysis:Scattering:ElasticIncoherentStructureFactor:finalize}
    \index{nMOLDYN \textit{(package)}!nMOLDYN.Analysis \textit{(package)}!nMOLDYN.Analysis.Scattering \textit{(module)}!nMOLDYN.Analysis.Scattering.ElasticIncoherentStructureFactor \textit{(class)}!nMOLDYN.Analysis.Scattering.ElasticIncoherentStructureFactor.finalize \textit{(method)}}

    \vspace{0.5ex}

\hspace{.8\funcindent}\begin{boxedminipage}{\funcwidth}

    \raggedright \textbf{finalize}(\textit{self})

    \vspace{-1.5ex}

    \rule{\textwidth}{0.5\fboxrule}
\setlength{\parskip}{2ex}
    Finalizes the calculations (e.g. averaging the total term, output files
    creations ...)

\setlength{\parskip}{1ex}
    \end{boxedminipage}


\large{\textbf{\textit{Inherited from nMOLDYN.Analysis.Analysis.Analysis\textit{(Section \ref{nMOLDYN:Analysis:Analysis:Analysis})}}}}

\begin{quote}
analysisTime(), buildJobInfo(), buildTimeInfo(), deuterationSelection(), groupSelection(), parseInputParameters(), preLoadTrajectory(), runAnalysis(), saveAnalysis(), setInputParameters(), subsetSelection(), updateJobProgress(), weightingScheme()
\end{quote}

%%%%%%%%%%%%%%%%%%%%%%%%%%%%%%%%%%%%%%%%%%%%%%%%%%%%%%%%%%%%%%%%%%%%%%%%%%%
%%                            Class Variables                            %%
%%%%%%%%%%%%%%%%%%%%%%%%%%%%%%%%%%%%%%%%%%%%%%%%%%%%%%%%%%%%%%%%%%%%%%%%%%%

  \subsubsection{Class Variables}

    \vspace{-1cm}
\hspace{\varindent}\begin{longtable}{|p{\varnamewidth}|p{\vardescrwidth}|l}
\cline{1-2}
\cline{1-2} \centering \textbf{Name} & \centering \textbf{Description}& \\
\cline{1-2}
\endhead\cline{1-2}\multicolumn{3}{r}{\small\textit{continued on next page}}\\\endfoot\cline{1-2}
\endlastfoot\raggedright i\-n\-p\-u\-t\-P\-a\-r\-a\-m\-e\-t\-e\-r\-s\-N\-a\-m\-e\-s\- & \raggedright \textbf{Value:} 
{\tt 'trajectory', 'timeinfo', 'qshellvalues', 'qshellwidth', \texttt{...}}&\\
\cline{1-2}
\raggedright d\-e\-f\-a\-u\-l\-t\- & \raggedright \textbf{Value:} 
{\tt \{'weights': 'incoherent'\}}&\\
\cline{1-2}
\raggedright s\-h\-o\-r\-t\-N\-a\-m\-e\- & \raggedright \textbf{Value:} 
{\tt 'EISF'}&\\
\cline{1-2}
\raggedright c\-a\-n\-B\-e\-E\-s\-t\-i\-m\-a\-t\-e\-d\- & \raggedright \textbf{Value:} 
{\tt True}&\\
\cline{1-2}
\end{longtable}

    \index{nMOLDYN \textit{(package)}!nMOLDYN.Analysis \textit{(package)}!nMOLDYN.Analysis.Scattering \textit{(module)}!nMOLDYN.Analysis.Scattering.ElasticIncoherentStructureFactor \textit{(class)}|)}

%%%%%%%%%%%%%%%%%%%%%%%%%%%%%%%%%%%%%%%%%%%%%%%%%%%%%%%%%%%%%%%%%%%%%%%%%%%
%%                           Class Description                           %%
%%%%%%%%%%%%%%%%%%%%%%%%%%%%%%%%%%%%%%%%%%%%%%%%%%%%%%%%%%%%%%%%%%%%%%%%%%%

    \index{nMOLDYN \textit{(package)}!nMOLDYN.Analysis \textit{(package)}!nMOLDYN.Analysis.Scattering \textit{(module)}!nMOLDYN.Analysis.Scattering.SmoothedStaticCoherentStructureFactor \textit{(class)}|(}
\subsection{Class SmoothedStaticCoherentStructureFactor}

    \label{nMOLDYN:Analysis:Scattering:SmoothedStaticCoherentStructureFactor}
\begin{tabular}{cccccc}
% Line for nMOLDYN.Analysis.Analysis.Analysis, linespec=[False]
\multicolumn{2}{r}{\settowidth{\BCL}{nMOLDYN.Analysis.Analysis.Analysis}\multirow{2}{\BCL}{nMOLDYN.Analysis.Analysis.Analysis}}
&&
  \\\cline{3-3}
  &&\multicolumn{1}{c|}{}
&&
  \\
&&\multicolumn{2}{l}{\textbf{nMOLDYN.Analysis.Scattering.SmoothedStaticCoherentStructureFactor}}
\end{tabular}

\begin{alltt}
Sets up an Smoothed Static Coherent Structure Factor.

A Subclass of nMOLDYN.Analysis.Analysis. 

Constructor: SmoothedStaticCoherentStructureFactor({\textbar}parameters{\textbar} = None)

Arguments:

    - {\textbar}parameters{\textbar} -- a dictionnary of the input parameters, or 'None' to set up the analysis without parameters.
        * trajectory        -- a trajectory file name or an instance of MMTK.Trajectory.Trajectory class.
        * timeinfo          -- a string of the form 'first:last:step' where 'first' is an integer specifying the first frame 
                               number to consider, 'last' is an integer specifying the last frame number to consider and 
                               'step' is an integer specifying the step number between two frames.
        * qshellvalues      -- a string of the form 'qmin1:qmax1:dq1;qmin2:qmax2:dq2...' where 'qmin1', 'qmin2' ... , 
                               'qmax1', 'qmax2' ... and 'dq1', 'dq2' ... are floats that represents respectively 
                               the q minimum, the q maximum and the q steps for q interval 1, 2 ...
        * subset            -- a selection string specifying the atoms to consider for the analysis.
        * deuteration       -- a selection string specifying the hydrogen atoms whose atomic parameters will be those of the deuterium.
        * weights           -- a string equal to 'equal', 'mass', 'coherent' , 'incoherent' or 'atomicNumber' that specifies the weighting
                               scheme to use.
        * scsf              -- the output NetCDF file name. A CDL version of this file will also be generated with the '.cdl' extension
                               instead of the '.nc' extension.
        * pyroserver        -- a string specifying if Pyro will be used and how to run the analysis.
    
Running modes:

    - To run the analysis do: a.runAnalysis() where a is the analysis object.
    - To estimate the analysis do: a.estimateAnalysis() where a is the analysis object.
    - To save the analysis to 'file' file name do: a.saveAnalysis(file) where a is the analysis object.
    
Comments:
    
    - The analysis is based on the angular averaged coherent static structure factor formula where the
      summation over the q vectors is replaced by an integral over the q space. The formula used is taken 
      from equation 2.35 of Fischer et al. Rep. Prog. Phys. 69 (2006) 233-299.
    
\end{alltt}


%%%%%%%%%%%%%%%%%%%%%%%%%%%%%%%%%%%%%%%%%%%%%%%%%%%%%%%%%%%%%%%%%%%%%%%%%%%
%%                                Methods                                %%
%%%%%%%%%%%%%%%%%%%%%%%%%%%%%%%%%%%%%%%%%%%%%%%%%%%%%%%%%%%%%%%%%%%%%%%%%%%

  \subsubsection{Methods}

    \vspace{0.5ex}

\hspace{.8\funcindent}\begin{boxedminipage}{\funcwidth}

    \raggedright \textbf{\_\_init\_\_}(\textit{self})

    \vspace{-1.5ex}

    \rule{\textwidth}{0.5\fboxrule}
\setlength{\parskip}{2ex}
    The constructor. Insures that the class can not be instanciated 
    directly from here.

\setlength{\parskip}{1ex}
      \textbf{Parameters}
      \vspace{-1ex}

      \begin{quote}
        \begin{Ventry}{xxxxxxxxxx}

          \item[parameters]

          a dictionnary that contains parameters of the selected analysis.

          \item[statusBar]

          if not None, an instance of nMOLDYN.GUI.Widgets.StatusBar. Will 
          attach a status bar to the selected analysis.

        \end{Ventry}

      \end{quote}

      Overrides: nMOLDYN.Analysis.Analysis.Analysis.\_\_init\_\_

    \end{boxedminipage}

    \label{nMOLDYN:Analysis:Scattering:SmoothedStaticCoherentStructureFactor:initialize}
    \index{nMOLDYN \textit{(package)}!nMOLDYN.Analysis \textit{(package)}!nMOLDYN.Analysis.Scattering \textit{(module)}!nMOLDYN.Analysis.Scattering.SmoothedStaticCoherentStructureFactor \textit{(class)}!nMOLDYN.Analysis.Scattering.SmoothedStaticCoherentStructureFactor.initialize \textit{(method)}}

    \vspace{0.5ex}

\hspace{.8\funcindent}\begin{boxedminipage}{\funcwidth}

    \raggedright \textbf{initialize}(\textit{self})

    \vspace{-1.5ex}

    \rule{\textwidth}{0.5\fboxrule}
\setlength{\parskip}{2ex}
    Initializes the analysis (e.g. parses and checks input parameters, set 
    some variables ...).

\setlength{\parskip}{1ex}
    \end{boxedminipage}

    \label{nMOLDYN:Analysis:Scattering:SmoothedStaticCoherentStructureFactor:calc}
    \index{nMOLDYN \textit{(package)}!nMOLDYN.Analysis \textit{(package)}!nMOLDYN.Analysis.Scattering \textit{(module)}!nMOLDYN.Analysis.Scattering.SmoothedStaticCoherentStructureFactor \textit{(class)}!nMOLDYN.Analysis.Scattering.SmoothedStaticCoherentStructureFactor.calc \textit{(method)}}

    \vspace{0.5ex}

\hspace{.8\funcindent}\begin{boxedminipage}{\funcwidth}

    \raggedright \textbf{calc}(\textit{self}, \textit{frameIndex}, \textit{trajname})

    \vspace{-1.5ex}

    \rule{\textwidth}{0.5\fboxrule}
\setlength{\parskip}{2ex}
    Calculates the contribution for one frame.

\setlength{\parskip}{1ex}
      \textbf{Parameters}
      \vspace{-1ex}

      \begin{quote}
        \begin{Ventry}{xxxxxxxxxx}

          \item[frameIndex]

          the index of the frame in {\textbar}self.frameIndexes{\textbar} 
          array.

            {\it (type=integer.)}

          \item[trajname]

          the name of the trajectory file name.

            {\it (type=string)}

        \end{Ventry}

      \end{quote}

    \end{boxedminipage}

    \label{nMOLDYN:Analysis:Scattering:SmoothedStaticCoherentStructureFactor:combine}
    \index{nMOLDYN \textit{(package)}!nMOLDYN.Analysis \textit{(package)}!nMOLDYN.Analysis.Scattering \textit{(module)}!nMOLDYN.Analysis.Scattering.SmoothedStaticCoherentStructureFactor \textit{(class)}!nMOLDYN.Analysis.Scattering.SmoothedStaticCoherentStructureFactor.combine \textit{(method)}}

    \vspace{0.5ex}

\hspace{.8\funcindent}\begin{boxedminipage}{\funcwidth}

    \raggedright \textbf{combine}(\textit{self}, \textit{frameIndex}, \textit{x})

\setlength{\parskip}{2ex}
\setlength{\parskip}{1ex}
    \end{boxedminipage}

    \label{nMOLDYN:Analysis:Scattering:SmoothedStaticCoherentStructureFactor:finalize}
    \index{nMOLDYN \textit{(package)}!nMOLDYN.Analysis \textit{(package)}!nMOLDYN.Analysis.Scattering \textit{(module)}!nMOLDYN.Analysis.Scattering.SmoothedStaticCoherentStructureFactor \textit{(class)}!nMOLDYN.Analysis.Scattering.SmoothedStaticCoherentStructureFactor.finalize \textit{(method)}}

    \vspace{0.5ex}

\hspace{.8\funcindent}\begin{boxedminipage}{\funcwidth}

    \raggedright \textbf{finalize}(\textit{self})

    \vspace{-1.5ex}

    \rule{\textwidth}{0.5\fboxrule}
\setlength{\parskip}{2ex}
    Finalizes the calculations (e.g. averaging the total term, output files
    creations ...).

\setlength{\parskip}{1ex}
    \end{boxedminipage}


\large{\textbf{\textit{Inherited from nMOLDYN.Analysis.Analysis.Analysis\textit{(Section \ref{nMOLDYN:Analysis:Analysis:Analysis})}}}}

\begin{quote}
analysisTime(), buildJobInfo(), buildTimeInfo(), deuterationSelection(), groupSelection(), parseInputParameters(), preLoadTrajectory(), runAnalysis(), saveAnalysis(), setInputParameters(), subsetSelection(), updateJobProgress(), weightingScheme()
\end{quote}

%%%%%%%%%%%%%%%%%%%%%%%%%%%%%%%%%%%%%%%%%%%%%%%%%%%%%%%%%%%%%%%%%%%%%%%%%%%
%%                            Class Variables                            %%
%%%%%%%%%%%%%%%%%%%%%%%%%%%%%%%%%%%%%%%%%%%%%%%%%%%%%%%%%%%%%%%%%%%%%%%%%%%

  \subsubsection{Class Variables}

    \vspace{-1cm}
\hspace{\varindent}\begin{longtable}{|p{\varnamewidth}|p{\vardescrwidth}|l}
\cline{1-2}
\cline{1-2} \centering \textbf{Name} & \centering \textbf{Description}& \\
\cline{1-2}
\endhead\cline{1-2}\multicolumn{3}{r}{\small\textit{continued on next page}}\\\endfoot\cline{1-2}
\endlastfoot\raggedright i\-n\-p\-u\-t\-P\-a\-r\-a\-m\-e\-t\-e\-r\-s\-N\-a\-m\-e\-s\- & \raggedright \textbf{Value:} 
{\tt 'trajectory', 'timeinfo', 'qshellvalues', 'subset', 'deut\texttt{...}}&\\
\cline{1-2}
\raggedright d\-e\-f\-a\-u\-l\-t\- & \raggedright \textbf{Value:} 
{\tt \{'weights': 'coherent'\}}&\\
\cline{1-2}
\raggedright s\-h\-o\-r\-t\-N\-a\-m\-e\- & \raggedright \textbf{Value:} 
{\tt 'SSCSF'}&\\
\cline{1-2}
\raggedright c\-a\-n\-B\-e\-E\-s\-t\-i\-m\-a\-t\-e\-d\- & \raggedright \textbf{Value:} 
{\tt True}&\\
\cline{1-2}
\end{longtable}

    \index{nMOLDYN \textit{(package)}!nMOLDYN.Analysis \textit{(package)}!nMOLDYN.Analysis.Scattering \textit{(module)}!nMOLDYN.Analysis.Scattering.SmoothedStaticCoherentStructureFactor \textit{(class)}|)}
    \index{nMOLDYN \textit{(package)}!nMOLDYN.Analysis \textit{(package)}!nMOLDYN.Analysis.Scattering \textit{(module)}|)}
