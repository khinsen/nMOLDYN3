%
% API Documentation for nMOLDYN
% Module nMOLDYN.GUI.SelectionDialog
%
% Generated by epydoc 3.0.1
% [Thu Oct  8 17:00:01 2009]
%

%%%%%%%%%%%%%%%%%%%%%%%%%%%%%%%%%%%%%%%%%%%%%%%%%%%%%%%%%%%%%%%%%%%%%%%%%%%
%%                          Module Description                           %%
%%%%%%%%%%%%%%%%%%%%%%%%%%%%%%%%%%%%%%%%%%%%%%%%%%%%%%%%%%%%%%%%%%%%%%%%%%%

    \index{nMOLDYN \textit{(package)}!nMOLDYN.GUI \textit{(package)}!nMOLDYN.GUI.SelectionDialog \textit{(module)}|(}
\section{Module nMOLDYN.GUI.SelectionDialog}

    \label{nMOLDYN:GUI:SelectionDialog}
\begin{alltt}
This modules implements the atom selection dialog used in almost all nMOLDYN analysis.

The atom selection can be performed for various purposes such as selection of atoms for the simulation, selection
of hydrogen atoms to deuterate or selection of several group of atoms on which an analysis will be performed 
collectively.

Classes:
    * SelectionDialog: sets up a selection dialog in the scope of an analysis dialog.
\end{alltt}


%%%%%%%%%%%%%%%%%%%%%%%%%%%%%%%%%%%%%%%%%%%%%%%%%%%%%%%%%%%%%%%%%%%%%%%%%%%
%%                           Class Description                           %%
%%%%%%%%%%%%%%%%%%%%%%%%%%%%%%%%%%%%%%%%%%%%%%%%%%%%%%%%%%%%%%%%%%%%%%%%%%%

    \index{nMOLDYN \textit{(package)}!nMOLDYN.GUI \textit{(package)}!nMOLDYN.GUI.SelectionDialog \textit{(module)}!nMOLDYN.GUI.SelectionDialog.SelectionDialog \textit{(class)}|(}
\subsection{Class SelectionDialog}

    \label{nMOLDYN:GUI:SelectionDialog:SelectionDialog}
\begin{tabular}{cccccccccc}
% Line for Tkinter.Misc, linespec=[False, False, False]
\multicolumn{2}{r}{\settowidth{\BCL}{Tkinter.Misc}\multirow{2}{\BCL}{Tkinter.Misc}}
&&
&&
&&
  \\\cline{3-3}
  &&\multicolumn{1}{c|}{}
&&
&&
&&
  \\
% Line for Tkinter.BaseWidget, linespec=[False, False]
\multicolumn{4}{r}{\settowidth{\BCL}{Tkinter.BaseWidget}\multirow{2}{\BCL}{Tkinter.BaseWidget}}
&&
&&
  \\\cline{5-5}
  &&&&\multicolumn{1}{c|}{}
&&
&&
  \\
% Line for Tkinter.Wm, linespec=[True, False]
\multicolumn{4}{r}{\settowidth{\BCL}{Tkinter.Wm}\multirow{2}{\BCL}{Tkinter.Wm}}
&&\multicolumn{1}{|c}{}
&&
  \\\cline{5-5}
  &&&&\multicolumn{1}{c|}{}
&\multicolumn{1}{|c}{}&
&&
  \\
% Line for Tkinter.Toplevel, linespec=[False]
\multicolumn{6}{r}{\settowidth{\BCL}{Tkinter.Toplevel}\multirow{2}{\BCL}{Tkinter.Toplevel}}
&&
  \\\cline{7-7}
  &&&&&&\multicolumn{1}{c|}{}
&&
  \\
&&&&&&\multicolumn{2}{l}{\textbf{nMOLDYN.GUI.SelectionDialog.SelectionDialog}}
\end{tabular}

Sets up a dialog from which the user can perform an atom selection.


%%%%%%%%%%%%%%%%%%%%%%%%%%%%%%%%%%%%%%%%%%%%%%%%%%%%%%%%%%%%%%%%%%%%%%%%%%%
%%                                Methods                                %%
%%%%%%%%%%%%%%%%%%%%%%%%%%%%%%%%%%%%%%%%%%%%%%%%%%%%%%%%%%%%%%%%%%%%%%%%%%%

  \subsubsection{Methods}

    \vspace{0.5ex}

\hspace{.8\funcindent}\begin{boxedminipage}{\funcwidth}

    \raggedright \textbf{\_\_init\_\_}(\textit{self}, \textit{parent}, \textit{selectionType}, \textit{univContents})

    \vspace{-1.5ex}

    \rule{\textwidth}{0.5\fboxrule}
\setlength{\parskip}{2ex}
    The constructor.

\setlength{\parskip}{1ex}
      \textbf{Parameters}
      \vspace{-1ex}

      \begin{quote}
        \begin{Ventry}{xxxxxxxxxxxxx}

          \item[parent]

          the parent widget.

          \item[selectionType]

          a string being one of 'subset', 'deuteration' or 'group' 
          specifying the atom selection type thatwill be performed.

            {\it (type=string)}

          \item[univContents]

          a dictionnary that contains the universe contents.

            {\it (type=dict)}

        \end{Ventry}

      \end{quote}

      Overrides: Tkinter.BaseWidget.\_\_init\_\_

    \end{boxedminipage}

    \label{nMOLDYN:GUI:SelectionDialog:SelectionDialog:body}
    \index{nMOLDYN \textit{(package)}!nMOLDYN.GUI \textit{(package)}!nMOLDYN.GUI.SelectionDialog \textit{(module)}!nMOLDYN.GUI.SelectionDialog.SelectionDialog \textit{(class)}!nMOLDYN.GUI.SelectionDialog.SelectionDialog.body \textit{(method)}}

    \vspace{0.5ex}

\hspace{.8\funcindent}\begin{boxedminipage}{\funcwidth}

    \raggedright \textbf{body}(\textit{self}, \textit{master})

    \vspace{-1.5ex}

    \rule{\textwidth}{0.5\fboxrule}
\setlength{\parskip}{2ex}
    Create dialog body. Return widget that should have initial focus.

\setlength{\parskip}{1ex}
    \end{boxedminipage}

    \label{nMOLDYN:GUI:SelectionDialog:SelectionDialog:buttonbox}
    \index{nMOLDYN \textit{(package)}!nMOLDYN.GUI \textit{(package)}!nMOLDYN.GUI.SelectionDialog \textit{(module)}!nMOLDYN.GUI.SelectionDialog.SelectionDialog \textit{(class)}!nMOLDYN.GUI.SelectionDialog.SelectionDialog.buttonbox \textit{(method)}}

    \vspace{0.5ex}

\hspace{.8\funcindent}\begin{boxedminipage}{\funcwidth}

    \raggedright \textbf{buttonbox}(\textit{self})

    \vspace{-1.5ex}

    \rule{\textwidth}{0.5\fboxrule}
\setlength{\parskip}{2ex}
    Add standard button box.

\setlength{\parskip}{1ex}
    \end{boxedminipage}

    \label{nMOLDYN:GUI:SelectionDialog:SelectionDialog:ok}
    \index{nMOLDYN \textit{(package)}!nMOLDYN.GUI \textit{(package)}!nMOLDYN.GUI.SelectionDialog \textit{(module)}!nMOLDYN.GUI.SelectionDialog.SelectionDialog \textit{(class)}!nMOLDYN.GUI.SelectionDialog.SelectionDialog.ok \textit{(method)}}

    \vspace{0.5ex}

\hspace{.8\funcindent}\begin{boxedminipage}{\funcwidth}

    \raggedright \textbf{ok}(\textit{self}, \textit{event}={\tt None})

    \vspace{-1.5ex}

    \rule{\textwidth}{0.5\fboxrule}
\setlength{\parskip}{2ex}
    This method is called when the user clicks on the 'OK' button of the 
    selection editor dialog. It closes the selection editor dialog and 
    build the selection string.

\setlength{\parskip}{1ex}
    \end{boxedminipage}

    \label{nMOLDYN:GUI:SelectionDialog:SelectionDialog:cancel}
    \index{nMOLDYN \textit{(package)}!nMOLDYN.GUI \textit{(package)}!nMOLDYN.GUI.SelectionDialog \textit{(module)}!nMOLDYN.GUI.SelectionDialog.SelectionDialog \textit{(class)}!nMOLDYN.GUI.SelectionDialog.SelectionDialog.cancel \textit{(method)}}

    \vspace{0.5ex}

\hspace{.8\funcindent}\begin{boxedminipage}{\funcwidth}

    \raggedright \textbf{cancel}(\textit{self}, \textit{event}={\tt None})

    \vspace{-1.5ex}

    \rule{\textwidth}{0.5\fboxrule}
\setlength{\parskip}{2ex}
    Cancel the selection setting up the selection string to a selection 
    type-dependant value.

\setlength{\parskip}{1ex}
    \end{boxedminipage}

    \label{nMOLDYN:GUI:SelectionDialog:SelectionDialog:validate}
    \index{nMOLDYN \textit{(package)}!nMOLDYN.GUI \textit{(package)}!nMOLDYN.GUI.SelectionDialog \textit{(module)}!nMOLDYN.GUI.SelectionDialog.SelectionDialog \textit{(class)}!nMOLDYN.GUI.SelectionDialog.SelectionDialog.validate \textit{(method)}}

    \vspace{0.5ex}

\hspace{.8\funcindent}\begin{boxedminipage}{\funcwidth}

    \raggedright \textbf{validate}(\textit{self})

\setlength{\parskip}{2ex}
\setlength{\parskip}{1ex}
    \end{boxedminipage}

    \label{nMOLDYN:GUI:SelectionDialog:SelectionDialog:apply}
    \index{nMOLDYN \textit{(package)}!nMOLDYN.GUI \textit{(package)}!nMOLDYN.GUI.SelectionDialog \textit{(module)}!nMOLDYN.GUI.SelectionDialog.SelectionDialog \textit{(class)}!nMOLDYN.GUI.SelectionDialog.SelectionDialog.apply \textit{(method)}}

    \vspace{0.5ex}

\hspace{.8\funcindent}\begin{boxedminipage}{\funcwidth}

    \raggedright \textbf{apply}(\textit{self})

    \vspace{-1.5ex}

    \rule{\textwidth}{0.5\fboxrule}
\setlength{\parskip}{2ex}
    Performs a last checking of the selection string before closing the 
    selection dialog.

\setlength{\parskip}{1ex}
    \end{boxedminipage}

    \label{nMOLDYN:GUI:SelectionDialog:SelectionDialog:getValue}
    \index{nMOLDYN \textit{(package)}!nMOLDYN.GUI \textit{(package)}!nMOLDYN.GUI.SelectionDialog \textit{(module)}!nMOLDYN.GUI.SelectionDialog.SelectionDialog \textit{(class)}!nMOLDYN.GUI.SelectionDialog.SelectionDialog.getValue \textit{(method)}}

    \vspace{0.5ex}

\hspace{.8\funcindent}\begin{boxedminipage}{\funcwidth}

    \raggedright \textbf{getValue}(\textit{self})

    \vspace{-1.5ex}

    \rule{\textwidth}{0.5\fboxrule}
\setlength{\parskip}{2ex}
    This method returns the self.selectionString class attributes.

    Thanks to this method, the selection dialog can be used like any other 
    ComboWidget for which the getValue allows to fetch their contents.

\setlength{\parskip}{1ex}
    \end{boxedminipage}

    \label{nMOLDYN:GUI:SelectionDialog:SelectionDialog:setDefaultSelectionString}
    \index{nMOLDYN \textit{(package)}!nMOLDYN.GUI \textit{(package)}!nMOLDYN.GUI.SelectionDialog \textit{(module)}!nMOLDYN.GUI.SelectionDialog.SelectionDialog \textit{(class)}!nMOLDYN.GUI.SelectionDialog.SelectionDialog.setDefaultSelectionString \textit{(method)}}

    \vspace{0.5ex}

\hspace{.8\funcindent}\begin{boxedminipage}{\funcwidth}

    \raggedright \textbf{setDefaultSelectionString}(\textit{self})

    \vspace{-1.5ex}

    \rule{\textwidth}{0.5\fboxrule}
\setlength{\parskip}{2ex}
    Sets the selection string to its default value. This value depends on 
    the the selection type.

\setlength{\parskip}{1ex}
    \end{boxedminipage}

    \label{nMOLDYN:GUI:SelectionDialog:SelectionDialog:changeSelectionMedia}
    \index{nMOLDYN \textit{(package)}!nMOLDYN.GUI \textit{(package)}!nMOLDYN.GUI.SelectionDialog \textit{(module)}!nMOLDYN.GUI.SelectionDialog.SelectionDialog \textit{(class)}!nMOLDYN.GUI.SelectionDialog.SelectionDialog.changeSelectionMedia \textit{(method)}}

    \vspace{0.5ex}

\hspace{.8\funcindent}\begin{boxedminipage}{\funcwidth}

    \raggedright \textbf{changeSelectionMedia}(\textit{self})

    \vspace{-1.5ex}

    \rule{\textwidth}{0.5\fboxrule}
\setlength{\parskip}{2ex}
    Changes the 'media' from which the selection will be performed.

    It can be either from a selection file, either from the loaded 
    trajectory or from an expression string. When changing selection media,
    the previous selection is cleared.

\setlength{\parskip}{1ex}
    \end{boxedminipage}

    \label{nMOLDYN:GUI:SelectionDialog:SelectionDialog:clear}
    \index{nMOLDYN \textit{(package)}!nMOLDYN.GUI \textit{(package)}!nMOLDYN.GUI.SelectionDialog \textit{(module)}!nMOLDYN.GUI.SelectionDialog.SelectionDialog \textit{(class)}!nMOLDYN.GUI.SelectionDialog.SelectionDialog.clear \textit{(method)}}

    \vspace{0.5ex}

\hspace{.8\funcindent}\begin{boxedminipage}{\funcwidth}

    \raggedright \textbf{clear}(\textit{self})

    \vspace{-1.5ex}

    \rule{\textwidth}{0.5\fboxrule}
\setlength{\parskip}{2ex}
    This methods clears all the listboxes of the 'Selection from the loaded
    trajectory' browser. It resets the selection listboxes and their 
    associated variables, it resets the selection string, and updates the 
    'Selection preview' text widget.

\setlength{\parskip}{1ex}
    \end{boxedminipage}

    \label{nMOLDYN:GUI:SelectionDialog:SelectionDialog:buildSelectionString}
    \index{nMOLDYN \textit{(package)}!nMOLDYN.GUI \textit{(package)}!nMOLDYN.GUI.SelectionDialog \textit{(module)}!nMOLDYN.GUI.SelectionDialog.SelectionDialog \textit{(class)}!nMOLDYN.GUI.SelectionDialog.SelectionDialog.buildSelectionString \textit{(method)}}

    \vspace{0.5ex}

\hspace{.8\funcindent}\begin{boxedminipage}{\funcwidth}

    \raggedright \textbf{buildSelectionString}(\textit{self})

    \vspace{-1.5ex}

    \rule{\textwidth}{0.5\fboxrule}
\setlength{\parskip}{2ex}
    This method actually build the selection string out of the 
    {\textbar}self.selection{\textbar} dictionnary.

\setlength{\parskip}{1ex}
    \end{boxedminipage}

    \label{nMOLDYN:GUI:SelectionDialog:SelectionDialog:displaySelectionString}
    \index{nMOLDYN \textit{(package)}!nMOLDYN.GUI \textit{(package)}!nMOLDYN.GUI.SelectionDialog \textit{(module)}!nMOLDYN.GUI.SelectionDialog.SelectionDialog \textit{(class)}!nMOLDYN.GUI.SelectionDialog.SelectionDialog.displaySelectionString \textit{(method)}}

    \vspace{0.5ex}

\hspace{.8\funcindent}\begin{boxedminipage}{\funcwidth}

    \raggedright \textbf{displaySelectionString}(\textit{self})

    \vspace{-1.5ex}

    \rule{\textwidth}{0.5\fboxrule}
\setlength{\parskip}{2ex}
    Displays in the 'Selection preview' textwidget the selection string 
    under process.

\setlength{\parskip}{1ex}
    \end{boxedminipage}

    \label{nMOLDYN:GUI:SelectionDialog:SelectionDialog:selectFromExpression}
    \index{nMOLDYN \textit{(package)}!nMOLDYN.GUI \textit{(package)}!nMOLDYN.GUI.SelectionDialog \textit{(module)}!nMOLDYN.GUI.SelectionDialog.SelectionDialog \textit{(class)}!nMOLDYN.GUI.SelectionDialog.SelectionDialog.selectFromExpression \textit{(method)}}

    \vspace{0.5ex}

\hspace{.8\funcindent}\begin{boxedminipage}{\funcwidth}

    \raggedright \textbf{selectFromExpression}(\textit{self}, \textit{event})

    \vspace{-1.5ex}

    \rule{\textwidth}{0.5\fboxrule}
\setlength{\parskip}{2ex}
    This callback performs a selection from a expression string by writing 
    the expression in its corresponding text widget.

    The expression must be a set of valid ;-separated python instructions 
    the last one being 'selection = ...' as the selection string parser 
    will search for the selection variables when executing the expression 
    string.

    To refer to the universe just use the variable 'self.universe'.

\setlength{\parskip}{1ex}
    \end{boxedminipage}

    \label{nMOLDYN:GUI:SelectionDialog:SelectionDialog:selectFromFile}
    \index{nMOLDYN \textit{(package)}!nMOLDYN.GUI \textit{(package)}!nMOLDYN.GUI.SelectionDialog \textit{(module)}!nMOLDYN.GUI.SelectionDialog.SelectionDialog \textit{(class)}!nMOLDYN.GUI.SelectionDialog.SelectionDialog.selectFromFile \textit{(method)}}

    \vspace{0.5ex}

\hspace{.8\funcindent}\begin{boxedminipage}{\funcwidth}

    \raggedright \textbf{selectFromFile}(\textit{self}, \textit{event}={\tt None})

    \vspace{-1.5ex}

    \rule{\textwidth}{0.5\fboxrule}
\setlength{\parskip}{2ex}
    This method/callback performs a selection from a file by selection the 
    file from a browser.

\setlength{\parskip}{1ex}
    \end{boxedminipage}

    \label{nMOLDYN:GUI:SelectionDialog:SelectionDialog:selectPrefixName}
    \index{nMOLDYN \textit{(package)}!nMOLDYN.GUI \textit{(package)}!nMOLDYN.GUI.SelectionDialog \textit{(module)}!nMOLDYN.GUI.SelectionDialog.SelectionDialog \textit{(class)}!nMOLDYN.GUI.SelectionDialog.SelectionDialog.selectPrefixName \textit{(method)}}

    \vspace{0.5ex}

\hspace{.8\funcindent}\begin{boxedminipage}{\funcwidth}

    \raggedright \textbf{selectPrefixName}(\textit{self}, \textit{event})

    \vspace{-1.5ex}

    \rule{\textwidth}{0.5\fboxrule}
\setlength{\parskip}{2ex}
    This callback is called when the user clicks on one item of the 'Group 
    number' listbox of the selection editor.

\setlength{\parskip}{1ex}
    \end{boxedminipage}

    \label{nMOLDYN:GUI:SelectionDialog:SelectionDialog:deletePrefixName}
    \index{nMOLDYN \textit{(package)}!nMOLDYN.GUI \textit{(package)}!nMOLDYN.GUI.SelectionDialog \textit{(module)}!nMOLDYN.GUI.SelectionDialog.SelectionDialog \textit{(class)}!nMOLDYN.GUI.SelectionDialog.SelectionDialog.deletePrefixName \textit{(method)}}

    \vspace{0.5ex}

\hspace{.8\funcindent}\begin{boxedminipage}{\funcwidth}

    \raggedright \textbf{deletePrefixName}(\textit{self}, \textit{event})

    \vspace{-1.5ex}

    \rule{\textwidth}{0.5\fboxrule}
\setlength{\parskip}{2ex}
    This callback will remove the selection string associated to the 
    selected prefix name.

\setlength{\parskip}{1ex}
    \end{boxedminipage}

    \label{nMOLDYN:GUI:SelectionDialog:SelectionDialog:createNewGroup}
    \index{nMOLDYN \textit{(package)}!nMOLDYN.GUI \textit{(package)}!nMOLDYN.GUI.SelectionDialog \textit{(module)}!nMOLDYN.GUI.SelectionDialog.SelectionDialog \textit{(class)}!nMOLDYN.GUI.SelectionDialog.SelectionDialog.createNewGroup \textit{(method)}}

    \vspace{0.5ex}

\hspace{.8\funcindent}\begin{boxedminipage}{\funcwidth}

    \raggedright \textbf{createNewGroup}(\textit{self})

    \vspace{-1.5ex}

    \rule{\textwidth}{0.5\fboxrule}
\setlength{\parskip}{2ex}
    This callback will create a new group entry in the 'Group number' 
    listbox.

\setlength{\parskip}{1ex}
    \end{boxedminipage}

    \label{nMOLDYN:GUI:SelectionDialog:SelectionDialog:selectObjectName}
    \index{nMOLDYN \textit{(package)}!nMOLDYN.GUI \textit{(package)}!nMOLDYN.GUI.SelectionDialog \textit{(module)}!nMOLDYN.GUI.SelectionDialog.SelectionDialog \textit{(class)}!nMOLDYN.GUI.SelectionDialog.SelectionDialog.selectObjectName \textit{(method)}}

    \vspace{0.5ex}

\hspace{.8\funcindent}\begin{boxedminipage}{\funcwidth}

    \raggedright \textbf{selectObjectName}(\textit{self}, \textit{event})

    \vspace{-1.5ex}

    \rule{\textwidth}{0.5\fboxrule}
\setlength{\parskip}{2ex}
    This callback is called when the user clicks on one item of the 'Object
    name' listbox of the selection editor. It will display into the 
    'Selection keyword' listbox all the selection keywords corresponding to
    the selected object type.

\setlength{\parskip}{1ex}
    \end{boxedminipage}

    \label{nMOLDYN:GUI:SelectionDialog:SelectionDialog:selectGroupingLevel}
    \index{nMOLDYN \textit{(package)}!nMOLDYN.GUI \textit{(package)}!nMOLDYN.GUI.SelectionDialog \textit{(module)}!nMOLDYN.GUI.SelectionDialog.SelectionDialog \textit{(class)}!nMOLDYN.GUI.SelectionDialog.SelectionDialog.selectGroupingLevel \textit{(method)}}

    \vspace{0.5ex}

\hspace{.8\funcindent}\begin{boxedminipage}{\funcwidth}

    \raggedright \textbf{selectGroupingLevel}(\textit{self}, \textit{event})

\setlength{\parskip}{2ex}
\setlength{\parskip}{1ex}
    \end{boxedminipage}

    \label{nMOLDYN:GUI:SelectionDialog:SelectionDialog:selectKeyword}
    \index{nMOLDYN \textit{(package)}!nMOLDYN.GUI \textit{(package)}!nMOLDYN.GUI.SelectionDialog \textit{(module)}!nMOLDYN.GUI.SelectionDialog.SelectionDialog \textit{(class)}!nMOLDYN.GUI.SelectionDialog.SelectionDialog.selectKeyword \textit{(method)}}

    \vspace{0.5ex}

\hspace{.8\funcindent}\begin{boxedminipage}{\funcwidth}

    \raggedright \textbf{selectKeyword}(\textit{self}, \textit{event})

    \vspace{-1.5ex}

    \rule{\textwidth}{0.5\fboxrule}
\setlength{\parskip}{2ex}
    This callback is called whenever the user clicks on one item of the 
    'Selection keyword' listbox. It will display in the 'Selection value' 
    listbox the selection values available for the selected keyword.

\setlength{\parskip}{1ex}
    \end{boxedminipage}

    \label{nMOLDYN:GUI:SelectionDialog:SelectionDialog:deleteObjectName}
    \index{nMOLDYN \textit{(package)}!nMOLDYN.GUI \textit{(package)}!nMOLDYN.GUI.SelectionDialog \textit{(module)}!nMOLDYN.GUI.SelectionDialog.SelectionDialog \textit{(class)}!nMOLDYN.GUI.SelectionDialog.SelectionDialog.deleteObjectName \textit{(method)}}

    \vspace{0.5ex}

\hspace{.8\funcindent}\begin{boxedminipage}{\funcwidth}

    \raggedright \textbf{deleteObjectName}(\textit{self}, \textit{event})

    \vspace{-1.5ex}

    \rule{\textwidth}{0.5\fboxrule}
\setlength{\parskip}{2ex}
    This method will delete the object name from the selection string if it
    was previously selected.

\setlength{\parskip}{1ex}
    \end{boxedminipage}

    \label{nMOLDYN:GUI:SelectionDialog:SelectionDialog:selectValue}
    \index{nMOLDYN \textit{(package)}!nMOLDYN.GUI \textit{(package)}!nMOLDYN.GUI.SelectionDialog \textit{(module)}!nMOLDYN.GUI.SelectionDialog.SelectionDialog \textit{(class)}!nMOLDYN.GUI.SelectionDialog.SelectionDialog.selectValue \textit{(method)}}

    \vspace{0.5ex}

\hspace{.8\funcindent}\begin{boxedminipage}{\funcwidth}

    \raggedright \textbf{selectValue}(\textit{self}, \textit{event})

    \vspace{-1.5ex}

    \rule{\textwidth}{0.5\fboxrule}
\setlength{\parskip}{2ex}
    This callback is called whenever the user clicks on one entry of the 
    'Selection value' listbox. It will update the current selection.

\setlength{\parskip}{1ex}
    \end{boxedminipage}

    \label{nMOLDYN:GUI:SelectionDialog:SelectionDialog:appendLinker}
    \index{nMOLDYN \textit{(package)}!nMOLDYN.GUI \textit{(package)}!nMOLDYN.GUI.SelectionDialog \textit{(module)}!nMOLDYN.GUI.SelectionDialog.SelectionDialog \textit{(class)}!nMOLDYN.GUI.SelectionDialog.SelectionDialog.appendLinker \textit{(method)}}

    \vspace{0.5ex}

\hspace{.8\funcindent}\begin{boxedminipage}{\funcwidth}

    \raggedright \textbf{appendLinker}(\textit{self}, \textit{linker})

    \vspace{-1.5ex}

    \rule{\textwidth}{0.5\fboxrule}
\setlength{\parskip}{2ex}
    This method is called when the user press the '(', ')', 'AND' or 'OR' 
    buttons of the atom selection dialog. It:

    \begin{itemize}
    \setlength{\parskip}{0.6ex}
      \item checks whether the selected linker can be actually be appended

      \item appends the linker if so.

    \end{itemize}

\setlength{\parskip}{1ex}
    \end{boxedminipage}


\large{\textbf{\textit{Inherited from Tkinter.BaseWidget}}}

\begin{quote}
destroy()
\end{quote}

\large{\textbf{\textit{Inherited from Tkinter.Misc}}}

\begin{quote}
\_\_getitem\_\_(), \_\_setitem\_\_(), \_\_str\_\_(), after(), after\_cancel(), after\_idle(), bbox(), bell(), bind(), bind\_all(), bind\_class(), bindtags(), cget(), clipboard\_append(), clipboard\_clear(), clipboard\_get(), colormodel(), columnconfigure(), config(), configure(), deletecommand(), event\_add(), event\_delete(), event\_generate(), event\_info(), focus(), focus\_displayof(), focus\_force(), focus\_get(), focus\_lastfor(), focus\_set(), getboolean(), getvar(), grab\_current(), grab\_release(), grab\_set(), grab\_set\_global(), grab\_status(), grid\_bbox(), grid\_columnconfigure(), grid\_location(), grid\_propagate(), grid\_rowconfigure(), grid\_size(), grid\_slaves(), image\_names(), image\_types(), keys(), lift(), lower(), mainloop(), nametowidget(), option\_add(), option\_clear(), option\_get(), option\_readfile(), pack\_propagate(), pack\_slaves(), place\_slaves(), propagate(), quit(), register(), rowconfigure(), selection\_clear(), selection\_get(), selection\_handle(), selection\_own(), selection\_own\_get(), send(), setvar(), size(), slaves(), tk\_bisque(), tk\_focusFollowsMouse(), tk\_focusNext(), tk\_focusPrev(), tk\_menuBar(), tk\_setPalette(), tk\_strictMotif(), tkraise(), unbind(), unbind\_all(), unbind\_class(), update(), update\_idletasks(), wait\_variable(), wait\_visibility(), wait\_window(), waitvar(), winfo\_atom(), winfo\_atomname(), winfo\_cells(), winfo\_children(), winfo\_class(), winfo\_colormapfull(), winfo\_containing(), winfo\_depth(), winfo\_exists(), winfo\_fpixels(), winfo\_geometry(), winfo\_height(), winfo\_id(), winfo\_interps(), winfo\_ismapped(), winfo\_manager(), winfo\_name(), winfo\_parent(), winfo\_pathname(), winfo\_pixels(), winfo\_pointerx(), winfo\_pointerxy(), winfo\_pointery(), winfo\_reqheight(), winfo\_reqwidth(), winfo\_rgb(), winfo\_rootx(), winfo\_rooty(), winfo\_screen(), winfo\_screencells(), winfo\_screendepth(), winfo\_screenheight(), winfo\_screenmmheight(), winfo\_screenmmwidth(), winfo\_screenvisual(), winfo\_screenwidth(), winfo\_server(), winfo\_toplevel(), winfo\_viewable(), winfo\_visual(), winfo\_visualid(), winfo\_visualsavailable(), winfo\_vrootheight(), winfo\_vrootwidth(), winfo\_vrootx(), winfo\_vrooty(), winfo\_width(), winfo\_x(), winfo\_y()
\end{quote}

\large{\textbf{\textit{Inherited from Tkinter.Wm}}}

\begin{quote}
aspect(), attributes(), client(), colormapwindows(), command(), deiconify(), focusmodel(), frame(), geometry(), grid(), group(), iconbitmap(), iconify(), iconmask(), iconname(), iconposition(), iconwindow(), maxsize(), minsize(), overrideredirect(), positionfrom(), protocol(), resizable(), sizefrom(), state(), title(), transient(), withdraw(), wm\_aspect(), wm\_attributes(), wm\_client(), wm\_colormapwindows(), wm\_command(), wm\_deiconify(), wm\_focusmodel(), wm\_frame(), wm\_geometry(), wm\_grid(), wm\_group(), wm\_iconbitmap(), wm\_iconify(), wm\_iconmask(), wm\_iconname(), wm\_iconposition(), wm\_iconwindow(), wm\_maxsize(), wm\_minsize(), wm\_overrideredirect(), wm\_positionfrom(), wm\_protocol(), wm\_resizable(), wm\_sizefrom(), wm\_state(), wm\_title(), wm\_transient(), wm\_withdraw()
\end{quote}

%%%%%%%%%%%%%%%%%%%%%%%%%%%%%%%%%%%%%%%%%%%%%%%%%%%%%%%%%%%%%%%%%%%%%%%%%%%
%%                            Class Variables                            %%
%%%%%%%%%%%%%%%%%%%%%%%%%%%%%%%%%%%%%%%%%%%%%%%%%%%%%%%%%%%%%%%%%%%%%%%%%%%

  \subsubsection{Class Variables}

    \vspace{-1cm}
\hspace{\varindent}\begin{longtable}{|p{\varnamewidth}|p{\vardescrwidth}|l}
\cline{1-2}
\cline{1-2} \centering \textbf{Name} & \centering \textbf{Description}& \\
\cline{1-2}
\endhead\cline{1-2}\multicolumn{3}{r}{\small\textit{continued on next page}}\\\endfoot\cline{1-2}
\endlastfoot\multicolumn{2}{|l|}{\textit{Inherited from Tkinter.Misc}}\\
\multicolumn{2}{|p{\varwidth}|}{\raggedright \_noarg\_}\\
\cline{1-2}
\end{longtable}

    \index{nMOLDYN \textit{(package)}!nMOLDYN.GUI \textit{(package)}!nMOLDYN.GUI.SelectionDialog \textit{(module)}!nMOLDYN.GUI.SelectionDialog.SelectionDialog \textit{(class)}|)}
    \index{nMOLDYN \textit{(package)}!nMOLDYN.GUI \textit{(package)}!nMOLDYN.GUI.SelectionDialog \textit{(module)}|)}
