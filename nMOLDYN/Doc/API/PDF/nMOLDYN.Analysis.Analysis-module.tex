%
% API Documentation for nMOLDYN
% Module nMOLDYN.Analysis.Analysis
%
% Generated by epydoc 3.0.1
% [Thu Oct  8 16:59:58 2009]
%

%%%%%%%%%%%%%%%%%%%%%%%%%%%%%%%%%%%%%%%%%%%%%%%%%%%%%%%%%%%%%%%%%%%%%%%%%%%
%%                          Module Description                           %%
%%%%%%%%%%%%%%%%%%%%%%%%%%%%%%%%%%%%%%%%%%%%%%%%%%%%%%%%%%%%%%%%%%%%%%%%%%%

    \index{nMOLDYN \textit{(package)}!nMOLDYN.Analysis \textit{(package)}!nMOLDYN.Analysis.Analysis \textit{(module)}|(}
\section{Module nMOLDYN.Analysis.Analysis}

    \label{nMOLDYN:Analysis:Analysis}
This modules implements the base class for all the analysis available in 
nMOLDYN.


%%%%%%%%%%%%%%%%%%%%%%%%%%%%%%%%%%%%%%%%%%%%%%%%%%%%%%%%%%%%%%%%%%%%%%%%%%%
%%                               Functions                               %%
%%%%%%%%%%%%%%%%%%%%%%%%%%%%%%%%%%%%%%%%%%%%%%%%%%%%%%%%%%%%%%%%%%%%%%%%%%%

  \subsection{Functions}

    \label{nMOLDYN:Analysis:Analysis:setUniverseContents}
    \index{nMOLDYN \textit{(package)}!nMOLDYN.Analysis \textit{(package)}!nMOLDYN.Analysis.Analysis \textit{(module)}!nMOLDYN.Analysis.Analysis.setUniverseContents \textit{(function)}}

    \vspace{0.5ex}

\hspace{.8\funcindent}\begin{boxedminipage}{\funcwidth}

    \raggedright \textbf{setUniverseContents}(\textit{universe})

    \vspace{-1.5ex}

    \rule{\textwidth}{0.5\fboxrule}
\setlength{\parskip}{2ex}
    Sets the contents of each object found in the universe.

\setlength{\parskip}{1ex}
      \textbf{Parameters}
      \vspace{-1ex}

      \begin{quote}
        \begin{Ventry}{xxxxxxxx}

          \item[universe]

          the MMTK universe to look in.

            {\it (type=a instance of MMTK.Universe.)}

        \end{Ventry}

      \end{quote}

    \end{boxedminipage}


%%%%%%%%%%%%%%%%%%%%%%%%%%%%%%%%%%%%%%%%%%%%%%%%%%%%%%%%%%%%%%%%%%%%%%%%%%%
%%                               Variables                               %%
%%%%%%%%%%%%%%%%%%%%%%%%%%%%%%%%%%%%%%%%%%%%%%%%%%%%%%%%%%%%%%%%%%%%%%%%%%%

  \subsection{Variables}

    \vspace{-1cm}
\hspace{\varindent}\begin{longtable}{|p{\varnamewidth}|p{\vardescrwidth}|l}
\cline{1-2}
\cline{1-2} \centering \textbf{Name} & \centering \textbf{Description}& \\
\cline{1-2}
\endhead\cline{1-2}\multicolumn{3}{r}{\small\textit{continued on next page}}\\\endfoot\cline{1-2}
\endlastfoot\raggedright r\-e\-s\-i\-d\-u\-s\-C\-h\-e\-m\-F\-a\-m\-i\-l\-y\- & \raggedright \textbf{Value:} 
{\tt \{'acidic':('Asp', 'Glu'), 'aliphatic':('Ile', 'Leu', 'Val\texttt{...}}&\\
\cline{1-2}
\raggedright n\-m\-o\-l\-d\-y\-n\-\_\-p\-a\-c\-k\-a\-g\-e\-\_\-p\-a\-t\-h\- & \raggedright \textbf{Value:} 
{\tt os.path.dirname(os.path.split(\_\_file\_\_) [0])}&\\
\cline{1-2}
\end{longtable}


%%%%%%%%%%%%%%%%%%%%%%%%%%%%%%%%%%%%%%%%%%%%%%%%%%%%%%%%%%%%%%%%%%%%%%%%%%%
%%                           Class Description                           %%
%%%%%%%%%%%%%%%%%%%%%%%%%%%%%%%%%%%%%%%%%%%%%%%%%%%%%%%%%%%%%%%%%%%%%%%%%%%

    \index{nMOLDYN \textit{(package)}!nMOLDYN.Analysis \textit{(package)}!nMOLDYN.Analysis.Analysis \textit{(module)}!nMOLDYN.Analysis.Analysis.Analysis \textit{(class)}|(}
\subsection{Class Analysis}

    \label{nMOLDYN:Analysis:Analysis:Analysis}
Base class for all analysis defined in nMOLDYN.

The class Analysis is an abstract-base-class that defines attributes and 
methods common to all the analysis available in nMOLDYN. To set up an 
analysis object, use one of its subclass.


%%%%%%%%%%%%%%%%%%%%%%%%%%%%%%%%%%%%%%%%%%%%%%%%%%%%%%%%%%%%%%%%%%%%%%%%%%%
%%                                Methods                                %%
%%%%%%%%%%%%%%%%%%%%%%%%%%%%%%%%%%%%%%%%%%%%%%%%%%%%%%%%%%%%%%%%%%%%%%%%%%%

  \subsubsection{Methods}

    \label{nMOLDYN:Analysis:Analysis:Analysis:__init__}
    \index{nMOLDYN \textit{(package)}!nMOLDYN.Analysis \textit{(package)}!nMOLDYN.Analysis.Analysis \textit{(module)}!nMOLDYN.Analysis.Analysis.Analysis \textit{(class)}!nMOLDYN.Analysis.Analysis.Analysis.\_\_init\_\_ \textit{(method)}}

    \vspace{0.5ex}

\hspace{.8\funcindent}\begin{boxedminipage}{\funcwidth}

    \raggedright \textbf{\_\_init\_\_}(\textit{self}, \textit{parameters}={\tt None}, \textit{statusBar}={\tt None})

    \vspace{-1.5ex}

    \rule{\textwidth}{0.5\fboxrule}
\setlength{\parskip}{2ex}
    The constructor.

\setlength{\parskip}{1ex}
      \textbf{Parameters}
      \vspace{-1ex}

      \begin{quote}
        \begin{Ventry}{xxxxxxxxxx}

          \item[parameters]

          a dictionnary that contains parameters of the selected analysis.

            {\it (type=dict)}

          \item[statusBar]

          if not None, an instance of nMOLDYN.GUI.Widgets.StatusBar. Will 
          attach a status bar to the selected analysis.

            {\it (type=instance of nMOLDYN.GUI.Widgets.StatusBar)}

        \end{Ventry}

      \end{quote}

    \end{boxedminipage}

    \label{nMOLDYN:Analysis:Analysis:Analysis:setInputParameters}
    \index{nMOLDYN \textit{(package)}!nMOLDYN.Analysis \textit{(package)}!nMOLDYN.Analysis.Analysis \textit{(module)}!nMOLDYN.Analysis.Analysis.Analysis \textit{(class)}!nMOLDYN.Analysis.Analysis.Analysis.setInputParameters \textit{(method)}}

    \vspace{0.5ex}

\hspace{.8\funcindent}\begin{boxedminipage}{\funcwidth}

    \raggedright \textbf{setInputParameters}(\textit{self}, \textit{parameters})

    \vspace{-1.5ex}

    \rule{\textwidth}{0.5\fboxrule}
\setlength{\parskip}{2ex}
    Sets the input parameters dictionnary.

\setlength{\parskip}{1ex}
    \end{boxedminipage}

    \label{nMOLDYN:Analysis:Analysis:Analysis:parseInputParameters}
    \index{nMOLDYN \textit{(package)}!nMOLDYN.Analysis \textit{(package)}!nMOLDYN.Analysis.Analysis \textit{(module)}!nMOLDYN.Analysis.Analysis.Analysis \textit{(class)}!nMOLDYN.Analysis.Analysis.Analysis.parseInputParameters \textit{(method)}}

    \vspace{0.5ex}

\hspace{.8\funcindent}\begin{boxedminipage}{\funcwidth}

    \raggedright \textbf{parseInputParameters}(\textit{self})

    \vspace{-1.5ex}

    \rule{\textwidth}{0.5\fboxrule}
\setlength{\parskip}{2ex}
    Parses the input parameters stored in {\textbar}parameters{\textbar} 
    dictionnary.

\setlength{\parskip}{1ex}
      \textbf{Return Value}
    \vspace{-1ex}

      \begin{quote}
      a dictionnary of the parsed parameters.

      {\it (type=dict)}

      \end{quote}

    \end{boxedminipage}

    \label{nMOLDYN:Analysis:Analysis:Analysis:buildTimeInfo}
    \index{nMOLDYN \textit{(package)}!nMOLDYN.Analysis \textit{(package)}!nMOLDYN.Analysis.Analysis \textit{(module)}!nMOLDYN.Analysis.Analysis.Analysis \textit{(class)}!nMOLDYN.Analysis.Analysis.Analysis.buildTimeInfo \textit{(method)}}

    \vspace{0.5ex}

\hspace{.8\funcindent}\begin{boxedminipage}{\funcwidth}

    \raggedright \textbf{buildTimeInfo}(\textit{self})

    \vspace{-1.5ex}

    \rule{\textwidth}{0.5\fboxrule}
\setlength{\parskip}{2ex}
    Builds some attributes related to the frame selection string. They will
    be used to define at which times a given analysis should be run.

\setlength{\parskip}{1ex}
    \end{boxedminipage}

    \label{nMOLDYN:Analysis:Analysis:Analysis:preLoadTrajectory}
    \index{nMOLDYN \textit{(package)}!nMOLDYN.Analysis \textit{(package)}!nMOLDYN.Analysis.Analysis \textit{(module)}!nMOLDYN.Analysis.Analysis.Analysis \textit{(class)}!nMOLDYN.Analysis.Analysis.Analysis.preLoadTrajectory \textit{(method)}}

    \vspace{0.5ex}

\hspace{.8\funcindent}\begin{boxedminipage}{\funcwidth}

    \raggedright \textbf{preLoadTrajectory}(\textit{self}, \textit{structure}, \textit{differentiation}={\tt 1})

\setlength{\parskip}{2ex}
\setlength{\parskip}{1ex}
    \end{boxedminipage}

    \label{nMOLDYN:Analysis:Analysis:Analysis:saveAnalysis}
    \index{nMOLDYN \textit{(package)}!nMOLDYN.Analysis \textit{(package)}!nMOLDYN.Analysis.Analysis \textit{(module)}!nMOLDYN.Analysis.Analysis.Analysis \textit{(class)}!nMOLDYN.Analysis.Analysis.Analysis.saveAnalysis \textit{(method)}}

    \vspace{0.5ex}

\hspace{.8\funcindent}\begin{boxedminipage}{\funcwidth}

    \raggedright \textbf{saveAnalysis}(\textit{self}, \textit{filename})

    \vspace{-1.5ex}

    \rule{\textwidth}{0.5\fboxrule}
\setlength{\parskip}{2ex}
    Saves the settings of an analysis to an output file.

\setlength{\parskip}{1ex}
      \textbf{Parameters}
      \vspace{-1ex}

      \begin{quote}
        \begin{Ventry}{xxxxxxxx}

          \item[filename]

          the name of the output file. If the extension is '.nmi' the 
          output file will be a nMOLDYN input script otherwise the output 
          file will be a nMOLDYN autostart script.

        \end{Ventry}

      \end{quote}

    \end{boxedminipage}

    \label{nMOLDYN:Analysis:Analysis:Analysis:runAnalysis}
    \index{nMOLDYN \textit{(package)}!nMOLDYN.Analysis \textit{(package)}!nMOLDYN.Analysis.Analysis \textit{(module)}!nMOLDYN.Analysis.Analysis.Analysis \textit{(class)}!nMOLDYN.Analysis.Analysis.Analysis.runAnalysis \textit{(method)}}

    \vspace{0.5ex}

\hspace{.8\funcindent}\begin{boxedminipage}{\funcwidth}

    \raggedright \textbf{runAnalysis}(\textit{self})

    \vspace{-1.5ex}

    \rule{\textwidth}{0.5\fboxrule}
\setlength{\parskip}{2ex}
    Runs an analysis.

\setlength{\parskip}{1ex}
      \textbf{Return Value}
    \vspace{-1ex}

      \begin{quote}
      a dictionnary of the form \{'days' : d, 'hours' : h, 'minutes' : m, 
      'seconds' : s\} specifying the time the analysis took in dayx, hours,
      minutes and seconds.

      {\it (type=dict)}

      \end{quote}

    \end{boxedminipage}

    \label{nMOLDYN:Analysis:Analysis:Analysis:updateJobProgress}
    \index{nMOLDYN \textit{(package)}!nMOLDYN.Analysis \textit{(package)}!nMOLDYN.Analysis.Analysis \textit{(module)}!nMOLDYN.Analysis.Analysis.Analysis \textit{(class)}!nMOLDYN.Analysis.Analysis.Analysis.updateJobProgress \textit{(method)}}

    \vspace{0.5ex}

\hspace{.8\funcindent}\begin{boxedminipage}{\funcwidth}

    \raggedright \textbf{updateJobProgress}(\textit{self}, \textit{norm})

    \vspace{-1.5ex}

    \rule{\textwidth}{0.5\fboxrule}
\setlength{\parskip}{2ex}
    Check the progress of the running analysis and displays periodically on
    the console and the logfile how far is the analysis. Called each time a
    step of an analysis loop is achieved.

\setlength{\parskip}{1ex}
      \textbf{Parameters}
      \vspace{-1ex}

      \begin{quote}
        \begin{Ventry}{xxxx}

          \item[norm]

          the maximum number of steps of the analysis.

        \end{Ventry}

      \end{quote}

    \end{boxedminipage}

    \label{nMOLDYN:Analysis:Analysis:Analysis:buildJobInfo}
    \index{nMOLDYN \textit{(package)}!nMOLDYN.Analysis \textit{(package)}!nMOLDYN.Analysis.Analysis \textit{(module)}!nMOLDYN.Analysis.Analysis.Analysis \textit{(class)}!nMOLDYN.Analysis.Analysis.Analysis.buildJobInfo \textit{(method)}}

    \vspace{0.5ex}

\hspace{.8\funcindent}\begin{boxedminipage}{\funcwidth}

    \raggedright \textbf{buildJobInfo}(\textit{self})

    \vspace{-1.5ex}

    \rule{\textwidth}{0.5\fboxrule}
\setlength{\parskip}{2ex}
    Display on the console and in the log file the main ifnormation about 
    the analysis to run.

\setlength{\parskip}{1ex}
    \end{boxedminipage}

    \label{nMOLDYN:Analysis:Analysis:Analysis:analysisTime}
    \index{nMOLDYN \textit{(package)}!nMOLDYN.Analysis \textit{(package)}!nMOLDYN.Analysis.Analysis \textit{(module)}!nMOLDYN.Analysis.Analysis.Analysis \textit{(class)}!nMOLDYN.Analysis.Analysis.Analysis.analysisTime \textit{(method)}}

    \vspace{0.5ex}

\hspace{.8\funcindent}\begin{boxedminipage}{\funcwidth}

    \raggedright \textbf{analysisTime}(\textit{self}, \textit{time})

    \vspace{-1.5ex}

    \rule{\textwidth}{0.5\fboxrule}
\setlength{\parskip}{2ex}
    Converts a time in second in days, hours, minutes and seconds.

\setlength{\parskip}{1ex}
      \textbf{Parameters}
      \vspace{-1ex}

      \begin{quote}
        \begin{Ventry}{xxxx}

          \item[time]

          the time (in seconds) to convert.

            {\it (type=integer.)}

        \end{Ventry}

      \end{quote}

      \textbf{Return Value}
    \vspace{-1ex}

      \begin{quote}
      a dictionnary of the form \{'days' : d, 'hours' : h, 'minutes' : m, 
      'seconds' : s\} where d, h, m and s are integers resulting 
      respectively from the conversion of {\textbar}time{\textbar} in days,
      hours, minutes and seconds.

      {\it (type=dict)}

      \end{quote}

    \end{boxedminipage}

    \label{nMOLDYN:Analysis:Analysis:Analysis:weightingScheme}
    \index{nMOLDYN \textit{(package)}!nMOLDYN.Analysis \textit{(package)}!nMOLDYN.Analysis.Analysis \textit{(module)}!nMOLDYN.Analysis.Analysis.Analysis \textit{(class)}!nMOLDYN.Analysis.Analysis.Analysis.weightingScheme \textit{(method)}}

    \vspace{0.5ex}

\hspace{.8\funcindent}\begin{boxedminipage}{\funcwidth}

    \raggedright \textbf{weightingScheme}(\textit{self}, \textit{universe}, \textit{atoms}, \textit{deuter}, \textit{scheme}={\tt 'equal'})

    \vspace{-1.5ex}

    \rule{\textwidth}{0.5\fboxrule}
\setlength{\parskip}{2ex}
    Returns the weights of {\textbar}atoms{\textbar} MMTK collection of 
    {\textbar}universe{\textbar} MMTK universe using the weighting scheme 
    {\textbar}scheme{\textbar}.

\setlength{\parskip}{1ex}
      \textbf{Parameters}
      \vspace{-1ex}

      \begin{quote}
        \begin{Ventry}{xxxxxxxx}

          \item[universe]

          the MMTK universe.

            {\it (type=instance of MMTK.Universe)}

          \item[atoms]

          the atoms to take into account when defining the weights.

            {\it (type=instance of MMTK.Collections.Collection)}

          \item[deuter]

          the hydrogen atoms that will be parametrized as deuterium atoms.

            {\it (type=instance of MMTK.Collections.Collection)}

          \item[scheme]

          a string equal to 'equal', 'mass', 'coherent' , 'incoherent' or 
          'atomicNumber' that specifies the weighting scheme to use.

            {\it (type=string)}

        \end{Ventry}

      \end{quote}

      \textbf{Return Value}
    \vspace{-1ex}

      \begin{quote}
      the weights of the selected atoms.

      {\it (type=an instance of MMTK.ParticleProperties.ParticledScalar)}

      \end{quote}

    \end{boxedminipage}

    \label{nMOLDYN:Analysis:Analysis:Analysis:subsetSelection}
    \index{nMOLDYN \textit{(package)}!nMOLDYN.Analysis \textit{(package)}!nMOLDYN.Analysis.Analysis \textit{(module)}!nMOLDYN.Analysis.Analysis.Analysis \textit{(class)}!nMOLDYN.Analysis.Analysis.Analysis.subsetSelection \textit{(method)}}

    \vspace{0.5ex}

\hspace{.8\funcindent}\begin{boxedminipage}{\funcwidth}

    \raggedright \textbf{subsetSelection}(\textit{self}, \textit{universe}, \textit{selection})

    \vspace{-1.5ex}

    \rule{\textwidth}{0.5\fboxrule}
\setlength{\parskip}{2ex}
    Returns a MMTK collection of atoms that matches 
    {\textbar}selection{\textbar} selection string. Used to apply an 
    analysis to a subset of atoms.

\setlength{\parskip}{1ex}
      \textbf{Parameters}
      \vspace{-1ex}

      \begin{quote}
        \begin{Ventry}{xxxxxxxxx}

          \item[universe]

          the universe on which the selection will be performed.

            {\it (type=instance of MMTK.Universe)}

          \item[selection]

          the selection string that will define the atoms to select.

            {\it (type=string)}

        \end{Ventry}

      \end{quote}

      \textbf{Return Value}
    \vspace{-1ex}

      \begin{quote}
      a MMTK Collection of the atoms that matches 
      {\textbar}selection{\textbar} selection string.

      {\it (type=instance of MMTK.Collections.Collection)}

      \end{quote}

    \end{boxedminipage}

    \label{nMOLDYN:Analysis:Analysis:Analysis:deuterationSelection}
    \index{nMOLDYN \textit{(package)}!nMOLDYN.Analysis \textit{(package)}!nMOLDYN.Analysis.Analysis \textit{(module)}!nMOLDYN.Analysis.Analysis.Analysis \textit{(class)}!nMOLDYN.Analysis.Analysis.Analysis.deuterationSelection \textit{(method)}}

    \vspace{0.5ex}

\hspace{.8\funcindent}\begin{boxedminipage}{\funcwidth}

    \raggedright \textbf{deuterationSelection}(\textit{self}, \textit{universe}, \textit{selection})

    \vspace{-1.5ex}

    \rule{\textwidth}{0.5\fboxrule}
\setlength{\parskip}{2ex}
    Returns a MMTK collection of atoms that matches 
    {\textbar}selection{\textbar} selection string. Used to switch the 
    parameters of a subset (or all) of hydrogen atoms to the parameters of 
    deuterium in order to simulate deuterated system.

\setlength{\parskip}{1ex}
      \textbf{Parameters}
      \vspace{-1ex}

      \begin{quote}
        \begin{Ventry}{xxxxxxxxx}

          \item[universe]

          the universe on which the selection will be performed.

            {\it (type=instance of MMTK.Universe)}

          \item[selection]

          the selection string that will define the atoms to select.

            {\it (type=string)}

        \end{Ventry}

      \end{quote}

      \textbf{Return Value}
    \vspace{-1ex}

      \begin{quote}
      a MMTK Collection of the atoms that matches 
      {\textbar}selection{\textbar} selection string.

      {\it (type=instance of MMTK.Collections.Collection)}

      \end{quote}

    \end{boxedminipage}

    \label{nMOLDYN:Analysis:Analysis:Analysis:groupSelection}
    \index{nMOLDYN \textit{(package)}!nMOLDYN.Analysis \textit{(package)}!nMOLDYN.Analysis.Analysis \textit{(module)}!nMOLDYN.Analysis.Analysis.Analysis \textit{(class)}!nMOLDYN.Analysis.Analysis.Analysis.groupSelection \textit{(method)}}

    \vspace{0.5ex}

\hspace{.8\funcindent}\begin{boxedminipage}{\funcwidth}

    \raggedright \textbf{groupSelection}(\textit{self}, \textit{universe}, \textit{selection})

    \vspace{-1.5ex}

    \rule{\textwidth}{0.5\fboxrule}
\setlength{\parskip}{2ex}
    Returns a list of MMTK collections where each collection defines a 
    group on which will be applied collectively an analysis.

\setlength{\parskip}{1ex}
      \textbf{Parameters}
      \vspace{-1ex}

      \begin{quote}
        \begin{Ventry}{xxxxxxxxx}

          \item[universe]

          the universe on which the selection will be performed.

            {\it (type=instance of MMTK.Universe)}

          \item[selection]

          the selection string that will define the contents of each group.

            {\it (type=string)}

        \end{Ventry}

      \end{quote}

      \textbf{Return Value}
    \vspace{-1ex}

      \begin{quote}
      a list of MMTK Collection where each collection defines a group..

      {\it (type=list)}

      \end{quote}

    \end{boxedminipage}

    \index{nMOLDYN \textit{(package)}!nMOLDYN.Analysis \textit{(package)}!nMOLDYN.Analysis.Analysis \textit{(module)}!nMOLDYN.Analysis.Analysis.Analysis \textit{(class)}|)}

%%%%%%%%%%%%%%%%%%%%%%%%%%%%%%%%%%%%%%%%%%%%%%%%%%%%%%%%%%%%%%%%%%%%%%%%%%%
%%                           Class Description                           %%
%%%%%%%%%%%%%%%%%%%%%%%%%%%%%%%%%%%%%%%%%%%%%%%%%%%%%%%%%%%%%%%%%%%%%%%%%%%

    \index{nMOLDYN \textit{(package)}!nMOLDYN.Analysis \textit{(package)}!nMOLDYN.Analysis.Analysis \textit{(module)}!nMOLDYN.Analysis.Analysis.QVectors \textit{(class)}|(}
\subsection{Class QVectors}

    \label{nMOLDYN:Analysis:Analysis:QVectors}
Generates a set of QVectors within a given shell.


%%%%%%%%%%%%%%%%%%%%%%%%%%%%%%%%%%%%%%%%%%%%%%%%%%%%%%%%%%%%%%%%%%%%%%%%%%%
%%                                Methods                                %%
%%%%%%%%%%%%%%%%%%%%%%%%%%%%%%%%%%%%%%%%%%%%%%%%%%%%%%%%%%%%%%%%%%%%%%%%%%%

  \subsubsection{Methods}

    \label{nMOLDYN:Analysis:Analysis:QVectors:__init__}
    \index{nMOLDYN \textit{(package)}!nMOLDYN.Analysis \textit{(package)}!nMOLDYN.Analysis.Analysis \textit{(module)}!nMOLDYN.Analysis.Analysis.QVectors \textit{(class)}!nMOLDYN.Analysis.Analysis.QVectors.\_\_init\_\_ \textit{(method)}}

    \vspace{0.5ex}

\hspace{.8\funcindent}\begin{boxedminipage}{\funcwidth}

    \raggedright \textbf{\_\_init\_\_}(\textit{self}, \textit{universe}, \textit{generator}, \textit{qRadii}, \textit{dq}, \textit{qVectorsPerShell}, \textit{qVectorsDirection}={\tt None})

    \vspace{-1.5ex}

    \rule{\textwidth}{0.5\fboxrule}
\setlength{\parskip}{2ex}
    The constructor.

\setlength{\parskip}{1ex}
      \textbf{Parameters}
      \vspace{-1ex}

      \begin{quote}
        \begin{Ventry}{xxxxxxxxxxxxxxxxx}

          \item[universe]

          the MMTK universe used to define the reciprocal space.

            {\it (type=a MMTK.Universe subclass object)}

          \item[generator]

          a string being one of '3d isotropic', '2d isotropic' or 
          'anistropic' the way the q-vectors should be generated.

            {\it (type=string)}

          \item[qRadii]

          a list of floats specifying the radii of the shell in which the q
          vectors have to be generated.

            {\it (type=list)}

          \item[dq]

          a float specifying the width of a qhsell defined as 
          [{\textbar}qRadius{\textbar} - 
          dq/2,{\textbar}qRadius{\textbar}+dq/2].

            {\it (type=float)}

          \item[qVectorsPerShell]

          an integer specifying the number of q-vectors to generate for 
          each shell.

            {\it (type=integer)}

          \item[qVectorsDirection]

          a list of Scientific.Geometry.Vector objects specifying the 
          directions along which the q-vectors should be generated. If 
          None, the q-vectors generation will be isotropic.

            {\it (type=list)}

        \end{Ventry}

      \end{quote}

    \end{boxedminipage}

    \index{nMOLDYN \textit{(package)}!nMOLDYN.Analysis \textit{(package)}!nMOLDYN.Analysis.Analysis \textit{(module)}!nMOLDYN.Analysis.Analysis.QVectors \textit{(class)}|)}
    \index{nMOLDYN \textit{(package)}!nMOLDYN.Analysis \textit{(package)}!nMOLDYN.Analysis.Analysis \textit{(module)}|)}
