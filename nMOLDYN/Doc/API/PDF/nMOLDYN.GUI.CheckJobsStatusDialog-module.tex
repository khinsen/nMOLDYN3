%
% API Documentation for nMOLDYN
% Module nMOLDYN.GUI.CheckJobsStatusDialog
%
% Generated by epydoc 3.0.1
% [Thu Oct  8 17:00:00 2009]
%

%%%%%%%%%%%%%%%%%%%%%%%%%%%%%%%%%%%%%%%%%%%%%%%%%%%%%%%%%%%%%%%%%%%%%%%%%%%
%%                          Module Description                           %%
%%%%%%%%%%%%%%%%%%%%%%%%%%%%%%%%%%%%%%%%%%%%%%%%%%%%%%%%%%%%%%%%%%%%%%%%%%%

    \index{nMOLDYN \textit{(package)}!nMOLDYN.GUI \textit{(package)}!nMOLDYN.GUI.CheckJobsStatusDialog \textit{(module)}|(}
\section{Module nMOLDYN.GUI.CheckJobsStatusDialog}

    \label{nMOLDYN:GUI:CheckJobsStatusDialog}
\begin{alltt}
This modules implements I\{Help--{\textgreater}Check job status\} dialog.

Classes:
    * CheckJobsStatusDialog: creates I\{Help--{\textgreater}Check job status\} dialog used to 
    check the status of the nMOLDYN running jobs.
\end{alltt}


%%%%%%%%%%%%%%%%%%%%%%%%%%%%%%%%%%%%%%%%%%%%%%%%%%%%%%%%%%%%%%%%%%%%%%%%%%%
%%                               Variables                               %%
%%%%%%%%%%%%%%%%%%%%%%%%%%%%%%%%%%%%%%%%%%%%%%%%%%%%%%%%%%%%%%%%%%%%%%%%%%%

  \subsection{Variables}

    \vspace{-1cm}
\hspace{\varindent}\begin{longtable}{|p{\varnamewidth}|p{\vardescrwidth}|l}
\cline{1-2}
\cline{1-2} \centering \textbf{Name} & \centering \textbf{Description}& \\
\cline{1-2}
\endhead\cline{1-2}\multicolumn{3}{r}{\small\textit{continued on next page}}\\\endfoot\cline{1-2}
\endlastfoot\raggedright n\-m\-o\-l\-d\-y\-n\-\_\-p\-a\-c\-k\-a\-g\-e\-\_\-p\-a\-t\-h\- & \raggedright \textbf{Value:} 
{\tt os.path.dirname(os.path.split(\_\_file\_\_) [0])}&\\
\cline{1-2}
\end{longtable}


%%%%%%%%%%%%%%%%%%%%%%%%%%%%%%%%%%%%%%%%%%%%%%%%%%%%%%%%%%%%%%%%%%%%%%%%%%%
%%                           Class Description                           %%
%%%%%%%%%%%%%%%%%%%%%%%%%%%%%%%%%%%%%%%%%%%%%%%%%%%%%%%%%%%%%%%%%%%%%%%%%%%

    \index{nMOLDYN \textit{(package)}!nMOLDYN.GUI \textit{(package)}!nMOLDYN.GUI.CheckJobsStatusDialog \textit{(module)}!nMOLDYN.GUI.CheckJobsStatusDialog.CheckJobsStatusDialog \textit{(class)}|(}
\subsection{Class CheckJobsStatusDialog}

    \label{nMOLDYN:GUI:CheckJobsStatusDialog:CheckJobsStatusDialog}
\begin{tabular}{cccccccccc}
% Line for Tkinter.Misc, linespec=[False, False, False]
\multicolumn{2}{r}{\settowidth{\BCL}{Tkinter.Misc}\multirow{2}{\BCL}{Tkinter.Misc}}
&&
&&
&&
  \\\cline{3-3}
  &&\multicolumn{1}{c|}{}
&&
&&
&&
  \\
% Line for Tkinter.BaseWidget, linespec=[False, False]
\multicolumn{4}{r}{\settowidth{\BCL}{Tkinter.BaseWidget}\multirow{2}{\BCL}{Tkinter.BaseWidget}}
&&
&&
  \\\cline{5-5}
  &&&&\multicolumn{1}{c|}{}
&&
&&
  \\
% Line for Tkinter.Wm, linespec=[True, False]
\multicolumn{4}{r}{\settowidth{\BCL}{Tkinter.Wm}\multirow{2}{\BCL}{Tkinter.Wm}}
&&\multicolumn{1}{|c}{}
&&
  \\\cline{5-5}
  &&&&\multicolumn{1}{c|}{}
&\multicolumn{1}{|c}{}&
&&
  \\
% Line for Tkinter.Toplevel, linespec=[False]
\multicolumn{6}{r}{\settowidth{\BCL}{Tkinter.Toplevel}\multirow{2}{\BCL}{Tkinter.Toplevel}}
&&
  \\\cline{7-7}
  &&&&&&\multicolumn{1}{c|}{}
&&
  \\
&&&&&&\multicolumn{2}{l}{\textbf{nMOLDYN.GUI.CheckJobsStatusDialog.CheckJobsStatusDialog}}
\end{tabular}

Sets up a dialog used to check the status of the nMOLDYN running jobs.


%%%%%%%%%%%%%%%%%%%%%%%%%%%%%%%%%%%%%%%%%%%%%%%%%%%%%%%%%%%%%%%%%%%%%%%%%%%
%%                                Methods                                %%
%%%%%%%%%%%%%%%%%%%%%%%%%%%%%%%%%%%%%%%%%%%%%%%%%%%%%%%%%%%%%%%%%%%%%%%%%%%

  \subsubsection{Methods}

    \vspace{0.5ex}

\hspace{.8\funcindent}\begin{boxedminipage}{\funcwidth}

    \raggedright \textbf{\_\_init\_\_}(\textit{self}, \textit{parent}, \textit{title}={\tt None})

    \vspace{-1.5ex}

    \rule{\textwidth}{0.5\fboxrule}
\setlength{\parskip}{2ex}
    The constructor.

\setlength{\parskip}{1ex}
      \textbf{Parameters}
      \vspace{-1ex}

      \begin{quote}
        \begin{Ventry}{xxxxxx}

          \item[parent]

          the parent widget.

          \item[title]

          a string specifying the title of the dialog.

            {\it (type=string)}

        \end{Ventry}

      \end{quote}

      Overrides: Tkinter.BaseWidget.\_\_init\_\_

    \end{boxedminipage}

    \label{nMOLDYN:GUI:CheckJobsStatusDialog:CheckJobsStatusDialog:body}
    \index{nMOLDYN \textit{(package)}!nMOLDYN.GUI \textit{(package)}!nMOLDYN.GUI.CheckJobsStatusDialog \textit{(module)}!nMOLDYN.GUI.CheckJobsStatusDialog.CheckJobsStatusDialog \textit{(class)}!nMOLDYN.GUI.CheckJobsStatusDialog.CheckJobsStatusDialog.body \textit{(method)}}

    \vspace{0.5ex}

\hspace{.8\funcindent}\begin{boxedminipage}{\funcwidth}

    \raggedright \textbf{body}(\textit{self}, \textit{master})

    \vspace{-1.5ex}

    \rule{\textwidth}{0.5\fboxrule}
\setlength{\parskip}{2ex}
    Create dialog body. Return widget that should have initial focus.

\setlength{\parskip}{1ex}
    \end{boxedminipage}

    \label{nMOLDYN:GUI:CheckJobsStatusDialog:CheckJobsStatusDialog:buttonbox}
    \index{nMOLDYN \textit{(package)}!nMOLDYN.GUI \textit{(package)}!nMOLDYN.GUI.CheckJobsStatusDialog \textit{(module)}!nMOLDYN.GUI.CheckJobsStatusDialog.CheckJobsStatusDialog \textit{(class)}!nMOLDYN.GUI.CheckJobsStatusDialog.CheckJobsStatusDialog.buttonbox \textit{(method)}}

    \vspace{0.5ex}

\hspace{.8\funcindent}\begin{boxedminipage}{\funcwidth}

    \raggedright \textbf{buttonbox}(\textit{self})

    \vspace{-1.5ex}

    \rule{\textwidth}{0.5\fboxrule}
\setlength{\parskip}{2ex}
    Add standard button box.

\setlength{\parskip}{1ex}
    \end{boxedminipage}

    \label{nMOLDYN:GUI:CheckJobsStatusDialog:CheckJobsStatusDialog:ok}
    \index{nMOLDYN \textit{(package)}!nMOLDYN.GUI \textit{(package)}!nMOLDYN.GUI.CheckJobsStatusDialog \textit{(module)}!nMOLDYN.GUI.CheckJobsStatusDialog.CheckJobsStatusDialog \textit{(class)}!nMOLDYN.GUI.CheckJobsStatusDialog.CheckJobsStatusDialog.ok \textit{(method)}}

    \vspace{0.5ex}

\hspace{.8\funcindent}\begin{boxedminipage}{\funcwidth}

    \raggedright \textbf{ok}(\textit{self}, \textit{event}={\tt None})

\setlength{\parskip}{2ex}
\setlength{\parskip}{1ex}
    \end{boxedminipage}

    \label{nMOLDYN:GUI:CheckJobsStatusDialog:CheckJobsStatusDialog:cancel}
    \index{nMOLDYN \textit{(package)}!nMOLDYN.GUI \textit{(package)}!nMOLDYN.GUI.CheckJobsStatusDialog \textit{(module)}!nMOLDYN.GUI.CheckJobsStatusDialog.CheckJobsStatusDialog \textit{(class)}!nMOLDYN.GUI.CheckJobsStatusDialog.CheckJobsStatusDialog.cancel \textit{(method)}}

    \vspace{0.5ex}

\hspace{.8\funcindent}\begin{boxedminipage}{\funcwidth}

    \raggedright \textbf{cancel}(\textit{self}, \textit{event}={\tt None})

\setlength{\parskip}{2ex}
\setlength{\parskip}{1ex}
    \end{boxedminipage}

    \label{nMOLDYN:GUI:CheckJobsStatusDialog:CheckJobsStatusDialog:validate}
    \index{nMOLDYN \textit{(package)}!nMOLDYN.GUI \textit{(package)}!nMOLDYN.GUI.CheckJobsStatusDialog \textit{(module)}!nMOLDYN.GUI.CheckJobsStatusDialog.CheckJobsStatusDialog \textit{(class)}!nMOLDYN.GUI.CheckJobsStatusDialog.CheckJobsStatusDialog.validate \textit{(method)}}

    \vspace{0.5ex}

\hspace{.8\funcindent}\begin{boxedminipage}{\funcwidth}

    \raggedright \textbf{validate}(\textit{self})

\setlength{\parskip}{2ex}
\setlength{\parskip}{1ex}
    \end{boxedminipage}

    \label{nMOLDYN:GUI:CheckJobsStatusDialog:CheckJobsStatusDialog:apply}
    \index{nMOLDYN \textit{(package)}!nMOLDYN.GUI \textit{(package)}!nMOLDYN.GUI.CheckJobsStatusDialog \textit{(module)}!nMOLDYN.GUI.CheckJobsStatusDialog.CheckJobsStatusDialog \textit{(class)}!nMOLDYN.GUI.CheckJobsStatusDialog.CheckJobsStatusDialog.apply \textit{(method)}}

    \vspace{0.5ex}

\hspace{.8\funcindent}\begin{boxedminipage}{\funcwidth}

    \raggedright \textbf{apply}(\textit{self})

\setlength{\parskip}{2ex}
\setlength{\parskip}{1ex}
    \end{boxedminipage}

    \label{nMOLDYN:GUI:CheckJobsStatusDialog:CheckJobsStatusDialog:refresh}
    \index{nMOLDYN \textit{(package)}!nMOLDYN.GUI \textit{(package)}!nMOLDYN.GUI.CheckJobsStatusDialog \textit{(module)}!nMOLDYN.GUI.CheckJobsStatusDialog.CheckJobsStatusDialog \textit{(class)}!nMOLDYN.GUI.CheckJobsStatusDialog.CheckJobsStatusDialog.refresh \textit{(method)}}

    \vspace{0.5ex}

\hspace{.8\funcindent}\begin{boxedminipage}{\funcwidth}

    \raggedright \textbf{refresh}(\textit{self})

    \vspace{-1.5ex}

    \rule{\textwidth}{0.5\fboxrule}
\setlength{\parskip}{2ex}
    Refreshes the nMOLDYN running jobs list and its associated frame.

\setlength{\parskip}{1ex}
    \end{boxedminipage}

    \label{nMOLDYN:GUI:CheckJobsStatusDialog:CheckJobsStatusDialog:findJobs}
    \index{nMOLDYN \textit{(package)}!nMOLDYN.GUI \textit{(package)}!nMOLDYN.GUI.CheckJobsStatusDialog \textit{(module)}!nMOLDYN.GUI.CheckJobsStatusDialog.CheckJobsStatusDialog \textit{(class)}!nMOLDYN.GUI.CheckJobsStatusDialog.CheckJobsStatusDialog.findJobs \textit{(method)}}

    \vspace{0.5ex}

\hspace{.8\funcindent}\begin{boxedminipage}{\funcwidth}

    \raggedright \textbf{findJobs}(\textit{self})

    \vspace{-1.5ex}

    \rule{\textwidth}{0.5\fboxrule}
\setlength{\parskip}{2ex}
\begin{alltt}

This method find the nMOLDYN active and inactive jobs.
Output:
    -a list of the temporary nMOLDYN running job logfiles. 
\end{alltt}

\setlength{\parskip}{1ex}
    \end{boxedminipage}

    \label{nMOLDYN:GUI:CheckJobsStatusDialog:CheckJobsStatusDialog:killJobs}
    \index{nMOLDYN \textit{(package)}!nMOLDYN.GUI \textit{(package)}!nMOLDYN.GUI.CheckJobsStatusDialog \textit{(module)}!nMOLDYN.GUI.CheckJobsStatusDialog.CheckJobsStatusDialog \textit{(class)}!nMOLDYN.GUI.CheckJobsStatusDialog.CheckJobsStatusDialog.killJobs \textit{(method)}}

    \vspace{0.5ex}

\hspace{.8\funcindent}\begin{boxedminipage}{\funcwidth}

    \raggedright \textbf{killJobs}(\textit{self}, \textit{pid})

    \vspace{-1.5ex}

    \rule{\textwidth}{0.5\fboxrule}
\setlength{\parskip}{2ex}
    This method is called when the user press the button 'Kill' of the 
    dialog. It loops over all the running jobs and for those which have 
    been selected to be killed asks the user to confirm that (s)he really 
    wants to kill them. If so, kills them and updates the dialog.

    Arguments:

    \begin{itemize}
    \setlength{\parskip}{0.6ex}
      \item pid: the pid of the job to kill.

    \end{itemize}

\setlength{\parskip}{1ex}
    \end{boxedminipage}


\large{\textbf{\textit{Inherited from Tkinter.BaseWidget}}}

\begin{quote}
destroy()
\end{quote}

\large{\textbf{\textit{Inherited from Tkinter.Misc}}}

\begin{quote}
\_\_getitem\_\_(), \_\_setitem\_\_(), \_\_str\_\_(), after(), after\_cancel(), after\_idle(), bbox(), bell(), bind(), bind\_all(), bind\_class(), bindtags(), cget(), clipboard\_append(), clipboard\_clear(), clipboard\_get(), colormodel(), columnconfigure(), config(), configure(), deletecommand(), event\_add(), event\_delete(), event\_generate(), event\_info(), focus(), focus\_displayof(), focus\_force(), focus\_get(), focus\_lastfor(), focus\_set(), getboolean(), getvar(), grab\_current(), grab\_release(), grab\_set(), grab\_set\_global(), grab\_status(), grid\_bbox(), grid\_columnconfigure(), grid\_location(), grid\_propagate(), grid\_rowconfigure(), grid\_size(), grid\_slaves(), image\_names(), image\_types(), keys(), lift(), lower(), mainloop(), nametowidget(), option\_add(), option\_clear(), option\_get(), option\_readfile(), pack\_propagate(), pack\_slaves(), place\_slaves(), propagate(), quit(), register(), rowconfigure(), selection\_clear(), selection\_get(), selection\_handle(), selection\_own(), selection\_own\_get(), send(), setvar(), size(), slaves(), tk\_bisque(), tk\_focusFollowsMouse(), tk\_focusNext(), tk\_focusPrev(), tk\_menuBar(), tk\_setPalette(), tk\_strictMotif(), tkraise(), unbind(), unbind\_all(), unbind\_class(), update(), update\_idletasks(), wait\_variable(), wait\_visibility(), wait\_window(), waitvar(), winfo\_atom(), winfo\_atomname(), winfo\_cells(), winfo\_children(), winfo\_class(), winfo\_colormapfull(), winfo\_containing(), winfo\_depth(), winfo\_exists(), winfo\_fpixels(), winfo\_geometry(), winfo\_height(), winfo\_id(), winfo\_interps(), winfo\_ismapped(), winfo\_manager(), winfo\_name(), winfo\_parent(), winfo\_pathname(), winfo\_pixels(), winfo\_pointerx(), winfo\_pointerxy(), winfo\_pointery(), winfo\_reqheight(), winfo\_reqwidth(), winfo\_rgb(), winfo\_rootx(), winfo\_rooty(), winfo\_screen(), winfo\_screencells(), winfo\_screendepth(), winfo\_screenheight(), winfo\_screenmmheight(), winfo\_screenmmwidth(), winfo\_screenvisual(), winfo\_screenwidth(), winfo\_server(), winfo\_toplevel(), winfo\_viewable(), winfo\_visual(), winfo\_visualid(), winfo\_visualsavailable(), winfo\_vrootheight(), winfo\_vrootwidth(), winfo\_vrootx(), winfo\_vrooty(), winfo\_width(), winfo\_x(), winfo\_y()
\end{quote}

\large{\textbf{\textit{Inherited from Tkinter.Wm}}}

\begin{quote}
aspect(), attributes(), client(), colormapwindows(), command(), deiconify(), focusmodel(), frame(), geometry(), grid(), group(), iconbitmap(), iconify(), iconmask(), iconname(), iconposition(), iconwindow(), maxsize(), minsize(), overrideredirect(), positionfrom(), protocol(), resizable(), sizefrom(), state(), title(), transient(), withdraw(), wm\_aspect(), wm\_attributes(), wm\_client(), wm\_colormapwindows(), wm\_command(), wm\_deiconify(), wm\_focusmodel(), wm\_frame(), wm\_geometry(), wm\_grid(), wm\_group(), wm\_iconbitmap(), wm\_iconify(), wm\_iconmask(), wm\_iconname(), wm\_iconposition(), wm\_iconwindow(), wm\_maxsize(), wm\_minsize(), wm\_overrideredirect(), wm\_positionfrom(), wm\_protocol(), wm\_resizable(), wm\_sizefrom(), wm\_state(), wm\_title(), wm\_transient(), wm\_withdraw()
\end{quote}

%%%%%%%%%%%%%%%%%%%%%%%%%%%%%%%%%%%%%%%%%%%%%%%%%%%%%%%%%%%%%%%%%%%%%%%%%%%
%%                            Class Variables                            %%
%%%%%%%%%%%%%%%%%%%%%%%%%%%%%%%%%%%%%%%%%%%%%%%%%%%%%%%%%%%%%%%%%%%%%%%%%%%

  \subsubsection{Class Variables}

    \vspace{-1cm}
\hspace{\varindent}\begin{longtable}{|p{\varnamewidth}|p{\vardescrwidth}|l}
\cline{1-2}
\cline{1-2} \centering \textbf{Name} & \centering \textbf{Description}& \\
\cline{1-2}
\endhead\cline{1-2}\multicolumn{3}{r}{\small\textit{continued on next page}}\\\endfoot\cline{1-2}
\endlastfoot\multicolumn{2}{|l|}{\textit{Inherited from Tkinter.Misc}}\\
\multicolumn{2}{|p{\varwidth}|}{\raggedright \_noarg\_}\\
\cline{1-2}
\end{longtable}

    \index{nMOLDYN \textit{(package)}!nMOLDYN.GUI \textit{(package)}!nMOLDYN.GUI.CheckJobsStatusDialog \textit{(module)}!nMOLDYN.GUI.CheckJobsStatusDialog.CheckJobsStatusDialog \textit{(class)}|)}
    \index{nMOLDYN \textit{(package)}!nMOLDYN.GUI \textit{(package)}!nMOLDYN.GUI.CheckJobsStatusDialog \textit{(module)}|)}
