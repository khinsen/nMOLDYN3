%
% API Documentation for nMOLDYN
% Module nMOLDYN.GUI.ASCIIToNetCDFConversionDialog
%
% Generated by epydoc 3.0.1
% [Thu Oct  8 17:00:00 2009]
%

%%%%%%%%%%%%%%%%%%%%%%%%%%%%%%%%%%%%%%%%%%%%%%%%%%%%%%%%%%%%%%%%%%%%%%%%%%%
%%                          Module Description                           %%
%%%%%%%%%%%%%%%%%%%%%%%%%%%%%%%%%%%%%%%%%%%%%%%%%%%%%%%%%%%%%%%%%%%%%%%%%%%

    \index{nMOLDYN \textit{(package)}!nMOLDYN.GUI \textit{(package)}!nMOLDYN.GUI.ASCIIToNetCDFConversionDialog \textit{(module)}|(}
\section{Module nMOLDYN.GUI.ASCIIToNetCDFConversionDialog}

    \label{nMOLDYN:GUI:ASCIIToNetCDFConversionDialog}
\begin{alltt}
This modules implements I\{File--{\textgreater}Convert ASCII to NetCDF\} dialog.

Classes:
    * ASCIIToNetCDFConversionDialog: creates I\{File--{\textgreater}Convert ASCII to NetCDF\} dialog used to 
      convert a file in ASCII format to a file in NetCDF format.
\end{alltt}


%%%%%%%%%%%%%%%%%%%%%%%%%%%%%%%%%%%%%%%%%%%%%%%%%%%%%%%%%%%%%%%%%%%%%%%%%%%
%%                           Class Description                           %%
%%%%%%%%%%%%%%%%%%%%%%%%%%%%%%%%%%%%%%%%%%%%%%%%%%%%%%%%%%%%%%%%%%%%%%%%%%%

    \index{nMOLDYN \textit{(package)}!nMOLDYN.GUI \textit{(package)}!nMOLDYN.GUI.ASCIIToNetCDFConversionDialog \textit{(module)}!nMOLDYN.GUI.ASCIIToNetCDFConversionDialog.ASCIIToNetCDFConversionDialog \textit{(class)}|(}
\subsection{Class ASCIIToNetCDFConversionDialog}

    \label{nMOLDYN:GUI:ASCIIToNetCDFConversionDialog:ASCIIToNetCDFConversionDialog}
\begin{tabular}{cccccc}
% Line for nMOLDYN.GUI.Widgets.Toplevel, linespec=[False]
\multicolumn{2}{r}{\settowidth{\BCL}{nMOLDYN.GUI.Widgets.Toplevel}\multirow{2}{\BCL}{nMOLDYN.GUI.Widgets.Toplevel}}
&&
  \\\cline{3-3}
  &&\multicolumn{1}{c|}{}
&&
  \\
&&\multicolumn{2}{l}{\textbf{nMOLDYN.GUI.ASCIIToNetCDFConversionDialog.ASCIIToNetCDFConversionDialog}}
\end{tabular}

Sets up a dialog from where the user can convert a file with numeric data 
in ASCII or CDL format to NetCDF format.

The ASCII file may contain some comments introduced with the \# character. 
These comments will also be written in the NetCDF output file 
({\textbar}comment{\textbar} attribute). The numeric datas have to be 
organized by column. The only restriction is that all the columns should 
have the same length.


%%%%%%%%%%%%%%%%%%%%%%%%%%%%%%%%%%%%%%%%%%%%%%%%%%%%%%%%%%%%%%%%%%%%%%%%%%%
%%                                Methods                                %%
%%%%%%%%%%%%%%%%%%%%%%%%%%%%%%%%%%%%%%%%%%%%%%%%%%%%%%%%%%%%%%%%%%%%%%%%%%%

  \subsubsection{Methods}

    \label{nMOLDYN:GUI:ASCIIToNetCDFConversionDialog:ASCIIToNetCDFConversionDialog:__init__}
    \index{nMOLDYN \textit{(package)}!nMOLDYN.GUI \textit{(package)}!nMOLDYN.GUI.ASCIIToNetCDFConversionDialog \textit{(module)}!nMOLDYN.GUI.ASCIIToNetCDFConversionDialog.ASCIIToNetCDFConversionDialog \textit{(class)}!nMOLDYN.GUI.ASCIIToNetCDFConversionDialog.ASCIIToNetCDFConversionDialog.\_\_init\_\_ \textit{(method)}}

    \vspace{0.5ex}

\hspace{.8\funcindent}\begin{boxedminipage}{\funcwidth}

    \raggedright \textbf{\_\_init\_\_}(\textit{self}, \textit{parent}, \textit{title}={\tt None})

    \vspace{-1.5ex}

    \rule{\textwidth}{0.5\fboxrule}
\setlength{\parskip}{2ex}
    The constructor.

\setlength{\parskip}{1ex}
      \textbf{Parameters}
      \vspace{-1ex}

      \begin{quote}
        \begin{Ventry}{xxxxxx}

          \item[parent]

          the parent widget.

          \item[title]

          a string specifying the title of the dialog.

            {\it (type=string)}

        \end{Ventry}

      \end{quote}

    \end{boxedminipage}

    \label{nMOLDYN:GUI:ASCIIToNetCDFConversionDialog:ASCIIToNetCDFConversionDialog:body}
    \index{nMOLDYN \textit{(package)}!nMOLDYN.GUI \textit{(package)}!nMOLDYN.GUI.ASCIIToNetCDFConversionDialog \textit{(module)}!nMOLDYN.GUI.ASCIIToNetCDFConversionDialog.ASCIIToNetCDFConversionDialog \textit{(class)}!nMOLDYN.GUI.ASCIIToNetCDFConversionDialog.ASCIIToNetCDFConversionDialog.body \textit{(method)}}

    \vspace{0.5ex}

\hspace{.8\funcindent}\begin{boxedminipage}{\funcwidth}

    \raggedright \textbf{body}(\textit{self}, \textit{master})

    \vspace{-1.5ex}

    \rule{\textwidth}{0.5\fboxrule}
\setlength{\parskip}{2ex}
    Create dialog body. Return widget that should have initial focus.

\setlength{\parskip}{1ex}
    \end{boxedminipage}

    \label{nMOLDYN:GUI:ASCIIToNetCDFConversionDialog:ASCIIToNetCDFConversionDialog:buttonbox}
    \index{nMOLDYN \textit{(package)}!nMOLDYN.GUI \textit{(package)}!nMOLDYN.GUI.ASCIIToNetCDFConversionDialog \textit{(module)}!nMOLDYN.GUI.ASCIIToNetCDFConversionDialog.ASCIIToNetCDFConversionDialog \textit{(class)}!nMOLDYN.GUI.ASCIIToNetCDFConversionDialog.ASCIIToNetCDFConversionDialog.buttonbox \textit{(method)}}

    \vspace{0.5ex}

\hspace{.8\funcindent}\begin{boxedminipage}{\funcwidth}

    \raggedright \textbf{buttonbox}(\textit{self})

    \vspace{-1.5ex}

    \rule{\textwidth}{0.5\fboxrule}
\setlength{\parskip}{2ex}
    Add standard button box.

\setlength{\parskip}{1ex}
    \end{boxedminipage}

    \label{nMOLDYN:GUI:ASCIIToNetCDFConversionDialog:ASCIIToNetCDFConversionDialog:ok}
    \index{nMOLDYN \textit{(package)}!nMOLDYN.GUI \textit{(package)}!nMOLDYN.GUI.ASCIIToNetCDFConversionDialog \textit{(module)}!nMOLDYN.GUI.ASCIIToNetCDFConversionDialog.ASCIIToNetCDFConversionDialog \textit{(class)}!nMOLDYN.GUI.ASCIIToNetCDFConversionDialog.ASCIIToNetCDFConversionDialog.ok \textit{(method)}}

    \vspace{0.5ex}

\hspace{.8\funcindent}\begin{boxedminipage}{\funcwidth}

    \raggedright \textbf{ok}(\textit{self}, \textit{event}={\tt None})

\setlength{\parskip}{2ex}
\setlength{\parskip}{1ex}
    \end{boxedminipage}

    \label{nMOLDYN:GUI:ASCIIToNetCDFConversionDialog:ASCIIToNetCDFConversionDialog:cancel}
    \index{nMOLDYN \textit{(package)}!nMOLDYN.GUI \textit{(package)}!nMOLDYN.GUI.ASCIIToNetCDFConversionDialog \textit{(module)}!nMOLDYN.GUI.ASCIIToNetCDFConversionDialog.ASCIIToNetCDFConversionDialog \textit{(class)}!nMOLDYN.GUI.ASCIIToNetCDFConversionDialog.ASCIIToNetCDFConversionDialog.cancel \textit{(method)}}

    \vspace{0.5ex}

\hspace{.8\funcindent}\begin{boxedminipage}{\funcwidth}

    \raggedright \textbf{cancel}(\textit{self}, \textit{event}={\tt None})

\setlength{\parskip}{2ex}
\setlength{\parskip}{1ex}
    \end{boxedminipage}

    \label{nMOLDYN:GUI:ASCIIToNetCDFConversionDialog:ASCIIToNetCDFConversionDialog:validate}
    \index{nMOLDYN \textit{(package)}!nMOLDYN.GUI \textit{(package)}!nMOLDYN.GUI.ASCIIToNetCDFConversionDialog \textit{(module)}!nMOLDYN.GUI.ASCIIToNetCDFConversionDialog.ASCIIToNetCDFConversionDialog \textit{(class)}!nMOLDYN.GUI.ASCIIToNetCDFConversionDialog.ASCIIToNetCDFConversionDialog.validate \textit{(method)}}

    \vspace{0.5ex}

\hspace{.8\funcindent}\begin{boxedminipage}{\funcwidth}

    \raggedright \textbf{validate}(\textit{self})

\setlength{\parskip}{2ex}
\setlength{\parskip}{1ex}
    \end{boxedminipage}

    \label{nMOLDYN:GUI:ASCIIToNetCDFConversionDialog:ASCIIToNetCDFConversionDialog:apply}
    \index{nMOLDYN \textit{(package)}!nMOLDYN.GUI \textit{(package)}!nMOLDYN.GUI.ASCIIToNetCDFConversionDialog \textit{(module)}!nMOLDYN.GUI.ASCIIToNetCDFConversionDialog.ASCIIToNetCDFConversionDialog \textit{(class)}!nMOLDYN.GUI.ASCIIToNetCDFConversionDialog.ASCIIToNetCDFConversionDialog.apply \textit{(method)}}

    \vspace{0.5ex}

\hspace{.8\funcindent}\begin{boxedminipage}{\funcwidth}

    \raggedright \textbf{apply}(\textit{self})

    \vspace{-1.5ex}

    \rule{\textwidth}{0.5\fboxrule}
\setlength{\parskip}{2ex}
    This method is called when the user clicks on the OK button of the 
    conversion dialog. It performs the conversion from the loaded NetCDF 
    file to the selected ASCII file.

\setlength{\parskip}{1ex}
    \end{boxedminipage}

    \label{nMOLDYN:GUI:ASCIIToNetCDFConversionDialog:ASCIIToNetCDFConversionDialog:openASCIIFile}
    \index{nMOLDYN \textit{(package)}!nMOLDYN.GUI \textit{(package)}!nMOLDYN.GUI.ASCIIToNetCDFConversionDialog \textit{(module)}!nMOLDYN.GUI.ASCIIToNetCDFConversionDialog.ASCIIToNetCDFConversionDialog \textit{(class)}!nMOLDYN.GUI.ASCIIToNetCDFConversionDialog.ASCIIToNetCDFConversionDialog.openASCIIFile \textit{(method)}}

    \vspace{0.5ex}

\hspace{.8\funcindent}\begin{boxedminipage}{\funcwidth}

    \raggedright \textbf{openASCIIFile}(\textit{self}, \textit{event}={\tt None})

    \vspace{-1.5ex}

    \rule{\textwidth}{0.5\fboxrule}
\setlength{\parskip}{2ex}
    This method/callback is called when the user press Return on the entry 
    of the input file browser or browse directlry from the file browser. It
    will set the filebrowser entry to the name of the browsed file and 
    propose and set a name for the output file based on the basename of the
    browsed file.

\setlength{\parskip}{1ex}
    \end{boxedminipage}

    \index{nMOLDYN \textit{(package)}!nMOLDYN.GUI \textit{(package)}!nMOLDYN.GUI.ASCIIToNetCDFConversionDialog \textit{(module)}!nMOLDYN.GUI.ASCIIToNetCDFConversionDialog.ASCIIToNetCDFConversionDialog \textit{(class)}|)}
    \index{nMOLDYN \textit{(package)}!nMOLDYN.GUI \textit{(package)}!nMOLDYN.GUI.ASCIIToNetCDFConversionDialog \textit{(module)}|)}
