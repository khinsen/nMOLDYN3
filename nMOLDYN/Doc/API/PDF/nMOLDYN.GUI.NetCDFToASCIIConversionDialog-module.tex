%
% API Documentation for nMOLDYN
% Module nMOLDYN.GUI.NetCDFToASCIIConversionDialog
%
% Generated by epydoc 3.0.1
% [Thu Oct  8 17:00:01 2009]
%

%%%%%%%%%%%%%%%%%%%%%%%%%%%%%%%%%%%%%%%%%%%%%%%%%%%%%%%%%%%%%%%%%%%%%%%%%%%
%%                          Module Description                           %%
%%%%%%%%%%%%%%%%%%%%%%%%%%%%%%%%%%%%%%%%%%%%%%%%%%%%%%%%%%%%%%%%%%%%%%%%%%%

    \index{nMOLDYN \textit{(package)}!nMOLDYN.GUI \textit{(package)}!nMOLDYN.GUI.NetCDFToASCIIConversionDialog \textit{(module)}|(}
\section{Module nMOLDYN.GUI.NetCDFToASCIIConversionDialog}

    \label{nMOLDYN:GUI:NetCDFToASCIIConversionDialog}
\begin{alltt}
This modules implements I\{File--{\textgreater}Convert NetCDF to ASCII\} dialog.

Classes:
    * NetCDFToASCIIConversionDialog: creates I\{File--{\textgreater}Convert NetCDF to ASCII\} dialog used to 
      convert a file in NetCDF format to a file in ASCII format.
\end{alltt}


%%%%%%%%%%%%%%%%%%%%%%%%%%%%%%%%%%%%%%%%%%%%%%%%%%%%%%%%%%%%%%%%%%%%%%%%%%%
%%                           Class Description                           %%
%%%%%%%%%%%%%%%%%%%%%%%%%%%%%%%%%%%%%%%%%%%%%%%%%%%%%%%%%%%%%%%%%%%%%%%%%%%

    \index{nMOLDYN \textit{(package)}!nMOLDYN.GUI \textit{(package)}!nMOLDYN.GUI.NetCDFToASCIIConversionDialog \textit{(module)}!nMOLDYN.GUI.NetCDFToASCIIConversionDialog.NetCDFToASCIIConversionDialog \textit{(class)}|(}
\subsection{Class NetCDFToASCIIConversionDialog}

    \label{nMOLDYN:GUI:NetCDFToASCIIConversionDialog:NetCDFToASCIIConversionDialog}
\begin{tabular}{cccccc}
% Line for nMOLDYN.GUI.Widgets.Toplevel, linespec=[False]
\multicolumn{2}{r}{\settowidth{\BCL}{nMOLDYN.GUI.Widgets.Toplevel}\multirow{2}{\BCL}{nMOLDYN.GUI.Widgets.Toplevel}}
&&
  \\\cline{3-3}
  &&\multicolumn{1}{c|}{}
&&
  \\
&&\multicolumn{2}{l}{\textbf{nMOLDYN.GUI.NetCDFToASCIIConversionDialog.NetCDFToASCIIConversionDialog}}
\end{tabular}

Sets up a dialog that allows the conversion of any numeric variables 
present in a NetCDF file into an ASCII file.


%%%%%%%%%%%%%%%%%%%%%%%%%%%%%%%%%%%%%%%%%%%%%%%%%%%%%%%%%%%%%%%%%%%%%%%%%%%
%%                                Methods                                %%
%%%%%%%%%%%%%%%%%%%%%%%%%%%%%%%%%%%%%%%%%%%%%%%%%%%%%%%%%%%%%%%%%%%%%%%%%%%

  \subsubsection{Methods}

    \label{nMOLDYN:GUI:NetCDFToASCIIConversionDialog:NetCDFToASCIIConversionDialog:__init__}
    \index{nMOLDYN \textit{(package)}!nMOLDYN.GUI \textit{(package)}!nMOLDYN.GUI.NetCDFToASCIIConversionDialog \textit{(module)}!nMOLDYN.GUI.NetCDFToASCIIConversionDialog.NetCDFToASCIIConversionDialog \textit{(class)}!nMOLDYN.GUI.NetCDFToASCIIConversionDialog.NetCDFToASCIIConversionDialog.\_\_init\_\_ \textit{(method)}}

    \vspace{0.5ex}

\hspace{.8\funcindent}\begin{boxedminipage}{\funcwidth}

    \raggedright \textbf{\_\_init\_\_}(\textit{self}, \textit{parent}, \textit{title}={\tt None}, \textit{netcdf}={\tt None})

    \vspace{-1.5ex}

    \rule{\textwidth}{0.5\fboxrule}
\setlength{\parskip}{2ex}
    The constructor.

\setlength{\parskip}{1ex}
      \textbf{Parameters}
      \vspace{-1ex}

      \begin{quote}
        \begin{Ventry}{xxxxxx}

          \item[parent]

          the parent widget.

          \item[title]

          a string specifying the title of the dialog.

            {\it (type=string)}

        \end{Ventry}

      \end{quote}

    \end{boxedminipage}

    \label{nMOLDYN:GUI:NetCDFToASCIIConversionDialog:NetCDFToASCIIConversionDialog:body}
    \index{nMOLDYN \textit{(package)}!nMOLDYN.GUI \textit{(package)}!nMOLDYN.GUI.NetCDFToASCIIConversionDialog \textit{(module)}!nMOLDYN.GUI.NetCDFToASCIIConversionDialog.NetCDFToASCIIConversionDialog \textit{(class)}!nMOLDYN.GUI.NetCDFToASCIIConversionDialog.NetCDFToASCIIConversionDialog.body \textit{(method)}}

    \vspace{0.5ex}

\hspace{.8\funcindent}\begin{boxedminipage}{\funcwidth}

    \raggedright \textbf{body}(\textit{self}, \textit{master})

    \vspace{-1.5ex}

    \rule{\textwidth}{0.5\fboxrule}
\setlength{\parskip}{2ex}
    Create dialog body. Return widget that should have initial focus.

\setlength{\parskip}{1ex}
    \end{boxedminipage}

    \label{nMOLDYN:GUI:NetCDFToASCIIConversionDialog:NetCDFToASCIIConversionDialog:buttonbox}
    \index{nMOLDYN \textit{(package)}!nMOLDYN.GUI \textit{(package)}!nMOLDYN.GUI.NetCDFToASCIIConversionDialog \textit{(module)}!nMOLDYN.GUI.NetCDFToASCIIConversionDialog.NetCDFToASCIIConversionDialog \textit{(class)}!nMOLDYN.GUI.NetCDFToASCIIConversionDialog.NetCDFToASCIIConversionDialog.buttonbox \textit{(method)}}

    \vspace{0.5ex}

\hspace{.8\funcindent}\begin{boxedminipage}{\funcwidth}

    \raggedright \textbf{buttonbox}(\textit{self})

    \vspace{-1.5ex}

    \rule{\textwidth}{0.5\fboxrule}
\setlength{\parskip}{2ex}
    Add standard button box.

\setlength{\parskip}{1ex}
    \end{boxedminipage}

    \label{nMOLDYN:GUI:NetCDFToASCIIConversionDialog:NetCDFToASCIIConversionDialog:ok}
    \index{nMOLDYN \textit{(package)}!nMOLDYN.GUI \textit{(package)}!nMOLDYN.GUI.NetCDFToASCIIConversionDialog \textit{(module)}!nMOLDYN.GUI.NetCDFToASCIIConversionDialog.NetCDFToASCIIConversionDialog \textit{(class)}!nMOLDYN.GUI.NetCDFToASCIIConversionDialog.NetCDFToASCIIConversionDialog.ok \textit{(method)}}

    \vspace{0.5ex}

\hspace{.8\funcindent}\begin{boxedminipage}{\funcwidth}

    \raggedright \textbf{ok}(\textit{self}, \textit{event}={\tt None})

\setlength{\parskip}{2ex}
\setlength{\parskip}{1ex}
    \end{boxedminipage}

    \label{nMOLDYN:GUI:NetCDFToASCIIConversionDialog:NetCDFToASCIIConversionDialog:cancel}
    \index{nMOLDYN \textit{(package)}!nMOLDYN.GUI \textit{(package)}!nMOLDYN.GUI.NetCDFToASCIIConversionDialog \textit{(module)}!nMOLDYN.GUI.NetCDFToASCIIConversionDialog.NetCDFToASCIIConversionDialog \textit{(class)}!nMOLDYN.GUI.NetCDFToASCIIConversionDialog.NetCDFToASCIIConversionDialog.cancel \textit{(method)}}

    \vspace{0.5ex}

\hspace{.8\funcindent}\begin{boxedminipage}{\funcwidth}

    \raggedright \textbf{cancel}(\textit{self}, \textit{event}={\tt None})

\setlength{\parskip}{2ex}
\setlength{\parskip}{1ex}
    \end{boxedminipage}

    \label{nMOLDYN:GUI:NetCDFToASCIIConversionDialog:NetCDFToASCIIConversionDialog:validate}
    \index{nMOLDYN \textit{(package)}!nMOLDYN.GUI \textit{(package)}!nMOLDYN.GUI.NetCDFToASCIIConversionDialog \textit{(module)}!nMOLDYN.GUI.NetCDFToASCIIConversionDialog.NetCDFToASCIIConversionDialog \textit{(class)}!nMOLDYN.GUI.NetCDFToASCIIConversionDialog.NetCDFToASCIIConversionDialog.validate \textit{(method)}}

    \vspace{0.5ex}

\hspace{.8\funcindent}\begin{boxedminipage}{\funcwidth}

    \raggedright \textbf{validate}(\textit{self})

\setlength{\parskip}{2ex}
\setlength{\parskip}{1ex}
    \end{boxedminipage}

    \label{nMOLDYN:GUI:NetCDFToASCIIConversionDialog:NetCDFToASCIIConversionDialog:apply}
    \index{nMOLDYN \textit{(package)}!nMOLDYN.GUI \textit{(package)}!nMOLDYN.GUI.NetCDFToASCIIConversionDialog \textit{(module)}!nMOLDYN.GUI.NetCDFToASCIIConversionDialog.NetCDFToASCIIConversionDialog \textit{(class)}!nMOLDYN.GUI.NetCDFToASCIIConversionDialog.NetCDFToASCIIConversionDialog.apply \textit{(method)}}

    \vspace{0.5ex}

\hspace{.8\funcindent}\begin{boxedminipage}{\funcwidth}

    \raggedright \textbf{apply}(\textit{self})

    \vspace{-1.5ex}

    \rule{\textwidth}{0.5\fboxrule}
\setlength{\parskip}{2ex}
    This method is called when the user clicks on the OK button of the 
    conversion dialog. It performs the conversion from the loaded NetCDF 
    file to the selected ASCII/CDL file.

\setlength{\parskip}{1ex}
    \end{boxedminipage}

    \label{nMOLDYN:GUI:NetCDFToASCIIConversionDialog:NetCDFToASCIIConversionDialog:openNetCDFFile}
    \index{nMOLDYN \textit{(package)}!nMOLDYN.GUI \textit{(package)}!nMOLDYN.GUI.NetCDFToASCIIConversionDialog \textit{(module)}!nMOLDYN.GUI.NetCDFToASCIIConversionDialog.NetCDFToASCIIConversionDialog \textit{(class)}!nMOLDYN.GUI.NetCDFToASCIIConversionDialog.NetCDFToASCIIConversionDialog.openNetCDFFile \textit{(method)}}

    \vspace{0.5ex}

\hspace{.8\funcindent}\begin{boxedminipage}{\funcwidth}

    \raggedright \textbf{openNetCDFFile}(\textit{self}, \textit{event}={\tt None})

    \vspace{-1.5ex}

    \rule{\textwidth}{0.5\fboxrule}
\setlength{\parskip}{2ex}
\begin{alltt}

This method opens a NetCDF file and updates the dialog with the data read from that file.
Arguments:
    -event: Tkinter event.
\end{alltt}

\setlength{\parskip}{1ex}
    \end{boxedminipage}

    \label{nMOLDYN:GUI:NetCDFToASCIIConversionDialog:NetCDFToASCIIConversionDialog:selectVariable}
    \index{nMOLDYN \textit{(package)}!nMOLDYN.GUI \textit{(package)}!nMOLDYN.GUI.NetCDFToASCIIConversionDialog \textit{(module)}!nMOLDYN.GUI.NetCDFToASCIIConversionDialog.NetCDFToASCIIConversionDialog \textit{(class)}!nMOLDYN.GUI.NetCDFToASCIIConversionDialog.NetCDFToASCIIConversionDialog.selectVariable \textit{(method)}}

    \vspace{0.5ex}

\hspace{.8\funcindent}\begin{boxedminipage}{\funcwidth}

    \raggedright \textbf{selectVariable}(\textit{self})

\setlength{\parskip}{2ex}
\setlength{\parskip}{1ex}
    \end{boxedminipage}

    \label{nMOLDYN:GUI:NetCDFToASCIIConversionDialog:NetCDFToASCIIConversionDialog:displayNetCDFContents}
    \index{nMOLDYN \textit{(package)}!nMOLDYN.GUI \textit{(package)}!nMOLDYN.GUI.NetCDFToASCIIConversionDialog \textit{(module)}!nMOLDYN.GUI.NetCDFToASCIIConversionDialog.NetCDFToASCIIConversionDialog \textit{(class)}!nMOLDYN.GUI.NetCDFToASCIIConversionDialog.NetCDFToASCIIConversionDialog.displayNetCDFContents \textit{(method)}}

    \vspace{0.5ex}

\hspace{.8\funcindent}\begin{boxedminipage}{\funcwidth}

    \raggedright \textbf{displayNetCDFContents}(\textit{self})

    \vspace{-1.5ex}

    \rule{\textwidth}{0.5\fboxrule}
\setlength{\parskip}{2ex}
    This method display the variables found in the NetCDF file.

\setlength{\parskip}{1ex}
    \end{boxedminipage}

    \index{nMOLDYN \textit{(package)}!nMOLDYN.GUI \textit{(package)}!nMOLDYN.GUI.NetCDFToASCIIConversionDialog \textit{(module)}!nMOLDYN.GUI.NetCDFToASCIIConversionDialog.NetCDFToASCIIConversionDialog \textit{(class)}|)}
    \index{nMOLDYN \textit{(package)}!nMOLDYN.GUI \textit{(package)}!nMOLDYN.GUI.NetCDFToASCIIConversionDialog \textit{(module)}|)}
