%
% API Documentation for nMOLDYN
% Module nMOLDYN.GUI.MainDialog
%
% Generated by epydoc 3.0.1
% [Thu Oct  8 17:00:01 2009]
%

%%%%%%%%%%%%%%%%%%%%%%%%%%%%%%%%%%%%%%%%%%%%%%%%%%%%%%%%%%%%%%%%%%%%%%%%%%%
%%                          Module Description                           %%
%%%%%%%%%%%%%%%%%%%%%%%%%%%%%%%%%%%%%%%%%%%%%%%%%%%%%%%%%%%%%%%%%%%%%%%%%%%

    \index{nMOLDYN \textit{(package)}!nMOLDYN.GUI \textit{(package)}!nMOLDYN.GUI.MainDialog \textit{(module)}|(}
\section{Module nMOLDYN.GUI.MainDialog}

    \label{nMOLDYN:GUI:MainDialog}
\begin{alltt}
This is where the main window of nMOLDYN is defined.

Classes:
    * MainDialog: The class that defines the nMOLDYN GUI main window and its associated actions.
\end{alltt}


%%%%%%%%%%%%%%%%%%%%%%%%%%%%%%%%%%%%%%%%%%%%%%%%%%%%%%%%%%%%%%%%%%%%%%%%%%%
%%                               Variables                               %%
%%%%%%%%%%%%%%%%%%%%%%%%%%%%%%%%%%%%%%%%%%%%%%%%%%%%%%%%%%%%%%%%%%%%%%%%%%%

  \subsection{Variables}

    \vspace{-1cm}
\hspace{\varindent}\begin{longtable}{|p{\varnamewidth}|p{\vardescrwidth}|l}
\cline{1-2}
\cline{1-2} \centering \textbf{Name} & \centering \textbf{Description}& \\
\cline{1-2}
\endhead\cline{1-2}\multicolumn{3}{r}{\small\textit{continued on next page}}\\\endfoot\cline{1-2}
\endlastfoot\raggedright n\-m\-o\-l\-d\-y\-n\-\_\-p\-a\-c\-k\-a\-g\-e\-\_\-p\-a\-t\-h\- & \raggedright \textbf{Value:} 
{\tt os.path.dirname(os.path.split(\_\_file\_\_) [0])}&\\
\cline{1-2}
\end{longtable}


%%%%%%%%%%%%%%%%%%%%%%%%%%%%%%%%%%%%%%%%%%%%%%%%%%%%%%%%%%%%%%%%%%%%%%%%%%%
%%                           Class Description                           %%
%%%%%%%%%%%%%%%%%%%%%%%%%%%%%%%%%%%%%%%%%%%%%%%%%%%%%%%%%%%%%%%%%%%%%%%%%%%

    \index{nMOLDYN \textit{(package)}!nMOLDYN.GUI \textit{(package)}!nMOLDYN.GUI.MainDialog \textit{(module)}!nMOLDYN.GUI.MainDialog.MainDialog \textit{(class)}|(}
\subsection{Class MainDialog}

    \label{nMOLDYN:GUI:MainDialog:MainDialog}
\begin{tabular}{cccccccc}
% Line for Tkinter.Misc, linespec=[False, False]
\multicolumn{2}{r}{\settowidth{\BCL}{Tkinter.Misc}\multirow{2}{\BCL}{Tkinter.Misc}}
&&
&&
  \\\cline{3-3}
  &&\multicolumn{1}{c|}{}
&&
&&
  \\
% Line for Tkinter.Wm, linespec=[True, False]
\multicolumn{2}{r}{\settowidth{\BCL}{Tkinter.Wm}\multirow{2}{\BCL}{Tkinter.Wm}}
&&\multicolumn{1}{|c}{}
&&
  \\\cline{3-3}
  &&\multicolumn{1}{c|}{}
&\multicolumn{1}{|c}{}&
&&
  \\
% Line for Tkinter.Tk, linespec=[False]
\multicolumn{4}{r}{\settowidth{\BCL}{Tkinter.Tk}\multirow{2}{\BCL}{Tkinter.Tk}}
&&
  \\\cline{5-5}
  &&&&\multicolumn{1}{c|}{}
&&
  \\
&&&&\multicolumn{2}{l}{\textbf{nMOLDYN.GUI.MainDialog.MainDialog}}
\end{tabular}

This is the base class for the nMoldyn GUI. It launches the main window of 
nMoldyn from which different menus can be accessed.


%%%%%%%%%%%%%%%%%%%%%%%%%%%%%%%%%%%%%%%%%%%%%%%%%%%%%%%%%%%%%%%%%%%%%%%%%%%
%%                                Methods                                %%
%%%%%%%%%%%%%%%%%%%%%%%%%%%%%%%%%%%%%%%%%%%%%%%%%%%%%%%%%%%%%%%%%%%%%%%%%%%

  \subsubsection{Methods}

    \vspace{0.5ex}

\hspace{.8\funcindent}\begin{boxedminipage}{\funcwidth}

    \raggedright \textbf{\_\_init\_\_}(\textit{self}, \textit{netcdf\_filename}={\tt None})

    \vspace{-1.5ex}

    \rule{\textwidth}{0.5\fboxrule}
\setlength{\parskip}{2ex}
    The constructor. Displays the main window.

\setlength{\parskip}{1ex}
      \textbf{Parameters}
      \vspace{-1ex}

      \begin{quote}
        \begin{Ventry}{xxxxxxxxxxxxxxx}

          \item[netcdf\_filename]

          a string specifying a NetCDF file, nMOLDYN should be started 
          with.

        \end{Ventry}

      \end{quote}

      \textbf{Return Value}
    \vspace{-1ex}

      \begin{quote}
      string

      \end{quote}

      Overrides: Tkinter.Tk.\_\_init\_\_

    \end{boxedminipage}

    \label{nMOLDYN:GUI:MainDialog:MainDialog:body}
    \index{nMOLDYN \textit{(package)}!nMOLDYN.GUI \textit{(package)}!nMOLDYN.GUI.MainDialog \textit{(module)}!nMOLDYN.GUI.MainDialog.MainDialog \textit{(class)}!nMOLDYN.GUI.MainDialog.MainDialog.body \textit{(method)}}

    \vspace{0.5ex}

\hspace{.8\funcindent}\begin{boxedminipage}{\funcwidth}

    \raggedright \textbf{body}(\textit{self}, \textit{master})

\setlength{\parskip}{2ex}
\setlength{\parskip}{1ex}
    \end{boxedminipage}

    \label{nMOLDYN:GUI:MainDialog:MainDialog:cancel}
    \index{nMOLDYN \textit{(package)}!nMOLDYN.GUI \textit{(package)}!nMOLDYN.GUI.MainDialog \textit{(module)}!nMOLDYN.GUI.MainDialog.MainDialog \textit{(class)}!nMOLDYN.GUI.MainDialog.MainDialog.cancel \textit{(method)}}

    \vspace{0.5ex}

\hspace{.8\funcindent}\begin{boxedminipage}{\funcwidth}

    \raggedright \textbf{cancel}(\textit{self}, \textit{event}={\tt None})

\setlength{\parskip}{2ex}
\setlength{\parskip}{1ex}
    \end{boxedminipage}

    \label{nMOLDYN:GUI:MainDialog:MainDialog:loadNetCDF}
    \index{nMOLDYN \textit{(package)}!nMOLDYN.GUI \textit{(package)}!nMOLDYN.GUI.MainDialog \textit{(module)}!nMOLDYN.GUI.MainDialog.MainDialog \textit{(class)}!nMOLDYN.GUI.MainDialog.MainDialog.loadNetCDF \textit{(method)}}

    \vspace{0.5ex}

\hspace{.8\funcindent}\begin{boxedminipage}{\funcwidth}

    \raggedright \textbf{loadNetCDF}(\textit{self}, \textit{event}={\tt None}, \textit{filename}={\tt None})

    \vspace{-1.5ex}

    \rule{\textwidth}{0.5\fboxrule}
\setlength{\parskip}{2ex}
    This method is launched when the user clicks on the {\textbar}Load 
    NetCDF{\textbar} of the {\textbar}File{\textbar} menu. It loads the 
    NetCDF file and displays its main informations in the information 
    window.

\setlength{\parskip}{1ex}
    \end{boxedminipage}

    \label{nMOLDYN:GUI:MainDialog:MainDialog:extractTrajectoryFrame}
    \index{nMOLDYN \textit{(package)}!nMOLDYN.GUI \textit{(package)}!nMOLDYN.GUI.MainDialog \textit{(module)}!nMOLDYN.GUI.MainDialog.MainDialog \textit{(class)}!nMOLDYN.GUI.MainDialog.MainDialog.extractTrajectoryFrame \textit{(method)}}

    \vspace{0.5ex}

\hspace{.8\funcindent}\begin{boxedminipage}{\funcwidth}

    \raggedright \textbf{extractTrajectoryFrame}(\textit{self}, \textit{event}={\tt None})

    \vspace{-1.5ex}

    \rule{\textwidth}{0.5\fboxrule}
\setlength{\parskip}{2ex}
    This method pops up a dialog from where the user can extract a PDB file
    from a NetCDF trajectory frame.

\setlength{\parskip}{1ex}
    \end{boxedminipage}

    \label{nMOLDYN:GUI:MainDialog:MainDialog:convertNetCDFToASCII}
    \index{nMOLDYN \textit{(package)}!nMOLDYN.GUI \textit{(package)}!nMOLDYN.GUI.MainDialog \textit{(module)}!nMOLDYN.GUI.MainDialog.MainDialog \textit{(class)}!nMOLDYN.GUI.MainDialog.MainDialog.convertNetCDFToASCII \textit{(method)}}

    \vspace{0.5ex}

\hspace{.8\funcindent}\begin{boxedminipage}{\funcwidth}

    \raggedright \textbf{convertNetCDFToASCII}(\textit{self}, \textit{event}={\tt None})

    \vspace{-1.5ex}

    \rule{\textwidth}{0.5\fboxrule}
\setlength{\parskip}{2ex}
    This method pops up a dialog where the user can proceed to a conversion
    from a file in NetCDF format to ASCII format.

\setlength{\parskip}{1ex}
    \end{boxedminipage}

    \label{nMOLDYN:GUI:MainDialog:MainDialog:convertASCIIToNetCDF}
    \index{nMOLDYN \textit{(package)}!nMOLDYN.GUI \textit{(package)}!nMOLDYN.GUI.MainDialog \textit{(module)}!nMOLDYN.GUI.MainDialog.MainDialog \textit{(class)}!nMOLDYN.GUI.MainDialog.MainDialog.convertASCIIToNetCDF \textit{(method)}}

    \vspace{0.5ex}

\hspace{.8\funcindent}\begin{boxedminipage}{\funcwidth}

    \raggedright \textbf{convertASCIIToNetCDF}(\textit{self}, \textit{event}={\tt None})

    \vspace{-1.5ex}

    \rule{\textwidth}{0.5\fboxrule}
\setlength{\parskip}{2ex}
    This method pops up a dialog where the user can proceed to a conversion
    from a file in ASCII format to NetCDF format.

\setlength{\parskip}{1ex}
    \end{boxedminipage}

    \label{nMOLDYN:GUI:MainDialog:MainDialog:checkConfiguration}
    \index{nMOLDYN \textit{(package)}!nMOLDYN.GUI \textit{(package)}!nMOLDYN.GUI.MainDialog \textit{(module)}!nMOLDYN.GUI.MainDialog.MainDialog \textit{(class)}!nMOLDYN.GUI.MainDialog.MainDialog.checkConfiguration \textit{(method)}}

    \vspace{0.5ex}

\hspace{.8\funcindent}\begin{boxedminipage}{\funcwidth}

    \raggedright \textbf{checkConfiguration}(\textit{self})

    \vspace{-1.5ex}

    \rule{\textwidth}{0.5\fboxrule}
\setlength{\parskip}{2ex}
    This method checks for missing external programs and display some 
    warning if it found some.

\setlength{\parskip}{1ex}
    \end{boxedminipage}

    \label{nMOLDYN:GUI:MainDialog:MainDialog:setPreferences}
    \index{nMOLDYN \textit{(package)}!nMOLDYN.GUI \textit{(package)}!nMOLDYN.GUI.MainDialog \textit{(module)}!nMOLDYN.GUI.MainDialog.MainDialog \textit{(class)}!nMOLDYN.GUI.MainDialog.MainDialog.setPreferences \textit{(method)}}

    \vspace{0.5ex}

\hspace{.8\funcindent}\begin{boxedminipage}{\funcwidth}

    \raggedright \textbf{setPreferences}(\textit{self}, \textit{event}={\tt None})

    \vspace{-1.5ex}

    \rule{\textwidth}{0.5\fboxrule}
\setlength{\parskip}{2ex}
    This method pops up a dialog from where the user can edit the nMOLDYN 
    configuration file.

\setlength{\parskip}{1ex}
    \end{boxedminipage}

    \label{nMOLDYN:GUI:MainDialog:MainDialog:analysisDialog}
    \index{nMOLDYN \textit{(package)}!nMOLDYN.GUI \textit{(package)}!nMOLDYN.GUI.MainDialog \textit{(module)}!nMOLDYN.GUI.MainDialog.MainDialog \textit{(class)}!nMOLDYN.GUI.MainDialog.MainDialog.analysisDialog \textit{(method)}}

    \vspace{0.5ex}

\hspace{.8\funcindent}\begin{boxedminipage}{\funcwidth}

    \raggedright \textbf{analysisDialog}(\textit{self}, \textit{analysis})

\setlength{\parskip}{2ex}
\setlength{\parskip}{1ex}
    \end{boxedminipage}

    \label{nMOLDYN:GUI:MainDialog:MainDialog:plotNetCDF}
    \index{nMOLDYN \textit{(package)}!nMOLDYN.GUI \textit{(package)}!nMOLDYN.GUI.MainDialog \textit{(module)}!nMOLDYN.GUI.MainDialog.MainDialog \textit{(class)}!nMOLDYN.GUI.MainDialog.MainDialog.plotNetCDF \textit{(method)}}

    \vspace{0.5ex}

\hspace{.8\funcindent}\begin{boxedminipage}{\funcwidth}

    \raggedright \textbf{plotNetCDF}(\textit{self}, \textit{event}={\tt None})

    \vspace{-1.5ex}

    \rule{\textwidth}{0.5\fboxrule}
\setlength{\parskip}{2ex}
    This method pops up a dialog from where the user can display any 
    numeric 2D or 3D NetCDF variables.

\setlength{\parskip}{1ex}
    \end{boxedminipage}

    \label{nMOLDYN:GUI:MainDialog:MainDialog:animateTrajectory}
    \index{nMOLDYN \textit{(package)}!nMOLDYN.GUI \textit{(package)}!nMOLDYN.GUI.MainDialog \textit{(module)}!nMOLDYN.GUI.MainDialog.MainDialog \textit{(class)}!nMOLDYN.GUI.MainDialog.MainDialog.animateTrajectory \textit{(method)}}

    \vspace{0.5ex}

\hspace{.8\funcindent}\begin{boxedminipage}{\funcwidth}

    \raggedright \textbf{animateTrajectory}(\textit{self}, \textit{event}={\tt None})

    \vspace{-1.5ex}

    \rule{\textwidth}{0.5\fboxrule}
\setlength{\parskip}{2ex}
    This method pops up a dialog from where the user can animate a 
    trajectory. If a trajectory has been loaded for analysis this will be 
    the default one. Otherwise the user can still browse one from the 
    dialog. The animation requires VMD.

\setlength{\parskip}{1ex}
    \end{boxedminipage}

    \label{nMOLDYN:GUI:MainDialog:MainDialog:viewEffectiveMode}
    \index{nMOLDYN \textit{(package)}!nMOLDYN.GUI \textit{(package)}!nMOLDYN.GUI.MainDialog \textit{(module)}!nMOLDYN.GUI.MainDialog.MainDialog \textit{(class)}!nMOLDYN.GUI.MainDialog.MainDialog.viewEffectiveMode \textit{(method)}}

    \vspace{0.5ex}

\hspace{.8\funcindent}\begin{boxedminipage}{\funcwidth}

    \raggedright \textbf{viewEffectiveMode}(\textit{self}, \textit{event}={\tt None})

    \vspace{-1.5ex}

    \rule{\textwidth}{0.5\fboxrule}
\setlength{\parskip}{2ex}
    This method pops up a dialog from where the user can animate an 
    effective mode coming from a QHA analysis. The animation require VMD.

\setlength{\parskip}{1ex}
    \end{boxedminipage}

    \label{nMOLDYN:GUI:MainDialog:MainDialog:traceAnalysis}
    \index{nMOLDYN \textit{(package)}!nMOLDYN.GUI \textit{(package)}!nMOLDYN.GUI.MainDialog \textit{(module)}!nMOLDYN.GUI.MainDialog.MainDialog \textit{(class)}!nMOLDYN.GUI.MainDialog.MainDialog.traceAnalysis \textit{(method)}}

    \vspace{0.5ex}

\hspace{.8\funcindent}\begin{boxedminipage}{\funcwidth}

    \raggedright \textbf{traceAnalysis}(\textit{self}, \textit{event}={\tt None})

    \vspace{-1.5ex}

    \rule{\textwidth}{0.5\fboxrule}
\setlength{\parskip}{2ex}
    This method pops up a dialog from where the user can check the march of
    the running jobs. The dialog can be updated dynamically by pressing its
    button 'refresh'.

\setlength{\parskip}{1ex}
    \end{boxedminipage}

    \label{nMOLDYN:GUI:MainDialog:MainDialog:analysisBenchmark}
    \index{nMOLDYN \textit{(package)}!nMOLDYN.GUI \textit{(package)}!nMOLDYN.GUI.MainDialog \textit{(module)}!nMOLDYN.GUI.MainDialog.MainDialog \textit{(class)}!nMOLDYN.GUI.MainDialog.MainDialog.analysisBenchmark \textit{(method)}}

    \vspace{0.5ex}

\hspace{.8\funcindent}\begin{boxedminipage}{\funcwidth}

    \raggedright \textbf{analysisBenchmark}(\textit{self})

    \vspace{-1.5ex}

    \rule{\textwidth}{0.5\fboxrule}
\setlength{\parskip}{2ex}
    This method pops up a dialog from where the user can perform some 
    analysis benchmark. The benchmark is done between the current version 
    and a reference version that is the version 2.2.5 the last official 
    release of nMOLDYN.

\setlength{\parskip}{1ex}
    \end{boxedminipage}

    \label{nMOLDYN:GUI:MainDialog:MainDialog:displayDocumentation}
    \index{nMOLDYN \textit{(package)}!nMOLDYN.GUI \textit{(package)}!nMOLDYN.GUI.MainDialog \textit{(module)}!nMOLDYN.GUI.MainDialog.MainDialog \textit{(class)}!nMOLDYN.GUI.MainDialog.MainDialog.displayDocumentation \textit{(method)}}

    \vspace{0.5ex}

\hspace{.8\funcindent}\begin{boxedminipage}{\funcwidth}

    \raggedright \textbf{displayDocumentation}(\textit{self}, \textit{event}={\tt None})

    \vspace{-1.5ex}

    \rule{\textwidth}{0.5\fboxrule}
\setlength{\parskip}{2ex}
    This methode opens the nMOLDYN pdf users guide. The users guide was 
    written by E. Pellegrini, V. Calandrini, P. Calligari, K. Hinsen and 
    G.R. Kneller.

\setlength{\parskip}{1ex}
    \end{boxedminipage}

    \label{nMOLDYN:GUI:MainDialog:MainDialog:displayMailingList}
    \index{nMOLDYN \textit{(package)}!nMOLDYN.GUI \textit{(package)}!nMOLDYN.GUI.MainDialog \textit{(module)}!nMOLDYN.GUI.MainDialog.MainDialog \textit{(class)}!nMOLDYN.GUI.MainDialog.MainDialog.displayMailingList \textit{(method)}}

    \vspace{0.5ex}

\hspace{.8\funcindent}\begin{boxedminipage}{\funcwidth}

    \raggedright \textbf{displayMailingList}(\textit{self})

    \vspace{-1.5ex}

    \rule{\textwidth}{0.5\fboxrule}
\setlength{\parskip}{2ex}
    This methode opens the nMOLDYN mailing list.

\setlength{\parskip}{1ex}
    \end{boxedminipage}

    \label{nMOLDYN:GUI:MainDialog:MainDialog:displayAPI}
    \index{nMOLDYN \textit{(package)}!nMOLDYN.GUI \textit{(package)}!nMOLDYN.GUI.MainDialog \textit{(module)}!nMOLDYN.GUI.MainDialog.MainDialog \textit{(class)}!nMOLDYN.GUI.MainDialog.MainDialog.displayAPI \textit{(method)}}

    \vspace{0.5ex}

\hspace{.8\funcindent}\begin{boxedminipage}{\funcwidth}

    \raggedright \textbf{displayAPI}(\textit{self}, \textit{event}={\tt None})

\setlength{\parskip}{2ex}
\setlength{\parskip}{1ex}
    \end{boxedminipage}

    \label{nMOLDYN:GUI:MainDialog:MainDialog:aboutNMOLDYN}
    \index{nMOLDYN \textit{(package)}!nMOLDYN.GUI \textit{(package)}!nMOLDYN.GUI.MainDialog \textit{(module)}!nMOLDYN.GUI.MainDialog.MainDialog \textit{(class)}!nMOLDYN.GUI.MainDialog.MainDialog.aboutNMOLDYN \textit{(method)}}

    \vspace{0.5ex}

\hspace{.8\funcindent}\begin{boxedminipage}{\funcwidth}

    \raggedright \textbf{aboutNMOLDYN}(\textit{self}, \textit{event}={\tt None})

    \vspace{-1.5ex}

    \rule{\textwidth}{0.5\fboxrule}
\setlength{\parskip}{2ex}
    This method displays general informations about the program such as the
    developper, the main versions ...

\setlength{\parskip}{1ex}
    \end{boxedminipage}


\large{\textbf{\textit{Inherited from Tkinter.Tk}}}

\begin{quote}
\_\_getattr\_\_(), destroy(), loadtk(), readprofile(), report\_callback\_exception()
\end{quote}

\large{\textbf{\textit{Inherited from Tkinter.Misc}}}

\begin{quote}
\_\_getitem\_\_(), \_\_setitem\_\_(), \_\_str\_\_(), after(), after\_cancel(), after\_idle(), bbox(), bell(), bind(), bind\_all(), bind\_class(), bindtags(), cget(), clipboard\_append(), clipboard\_clear(), clipboard\_get(), colormodel(), columnconfigure(), config(), configure(), deletecommand(), event\_add(), event\_delete(), event\_generate(), event\_info(), focus(), focus\_displayof(), focus\_force(), focus\_get(), focus\_lastfor(), focus\_set(), getboolean(), getvar(), grab\_current(), grab\_release(), grab\_set(), grab\_set\_global(), grab\_status(), grid\_bbox(), grid\_columnconfigure(), grid\_location(), grid\_propagate(), grid\_rowconfigure(), grid\_size(), grid\_slaves(), image\_names(), image\_types(), keys(), lift(), lower(), mainloop(), nametowidget(), option\_add(), option\_clear(), option\_get(), option\_readfile(), pack\_propagate(), pack\_slaves(), place\_slaves(), propagate(), quit(), register(), rowconfigure(), selection\_clear(), selection\_get(), selection\_handle(), selection\_own(), selection\_own\_get(), send(), setvar(), size(), slaves(), tk\_bisque(), tk\_focusFollowsMouse(), tk\_focusNext(), tk\_focusPrev(), tk\_menuBar(), tk\_setPalette(), tk\_strictMotif(), tkraise(), unbind(), unbind\_all(), unbind\_class(), update(), update\_idletasks(), wait\_variable(), wait\_visibility(), wait\_window(), waitvar(), winfo\_atom(), winfo\_atomname(), winfo\_cells(), winfo\_children(), winfo\_class(), winfo\_colormapfull(), winfo\_containing(), winfo\_depth(), winfo\_exists(), winfo\_fpixels(), winfo\_geometry(), winfo\_height(), winfo\_id(), winfo\_interps(), winfo\_ismapped(), winfo\_manager(), winfo\_name(), winfo\_parent(), winfo\_pathname(), winfo\_pixels(), winfo\_pointerx(), winfo\_pointerxy(), winfo\_pointery(), winfo\_reqheight(), winfo\_reqwidth(), winfo\_rgb(), winfo\_rootx(), winfo\_rooty(), winfo\_screen(), winfo\_screencells(), winfo\_screendepth(), winfo\_screenheight(), winfo\_screenmmheight(), winfo\_screenmmwidth(), winfo\_screenvisual(), winfo\_screenwidth(), winfo\_server(), winfo\_toplevel(), winfo\_viewable(), winfo\_visual(), winfo\_visualid(), winfo\_visualsavailable(), winfo\_vrootheight(), winfo\_vrootwidth(), winfo\_vrootx(), winfo\_vrooty(), winfo\_width(), winfo\_x(), winfo\_y()
\end{quote}

\large{\textbf{\textit{Inherited from Tkinter.Wm}}}

\begin{quote}
aspect(), attributes(), client(), colormapwindows(), command(), deiconify(), focusmodel(), frame(), geometry(), grid(), group(), iconbitmap(), iconify(), iconmask(), iconname(), iconposition(), iconwindow(), maxsize(), minsize(), overrideredirect(), positionfrom(), protocol(), resizable(), sizefrom(), state(), title(), transient(), withdraw(), wm\_aspect(), wm\_attributes(), wm\_client(), wm\_colormapwindows(), wm\_command(), wm\_deiconify(), wm\_focusmodel(), wm\_frame(), wm\_geometry(), wm\_grid(), wm\_group(), wm\_iconbitmap(), wm\_iconify(), wm\_iconmask(), wm\_iconname(), wm\_iconposition(), wm\_iconwindow(), wm\_maxsize(), wm\_minsize(), wm\_overrideredirect(), wm\_positionfrom(), wm\_protocol(), wm\_resizable(), wm\_sizefrom(), wm\_state(), wm\_title(), wm\_transient(), wm\_withdraw()
\end{quote}

%%%%%%%%%%%%%%%%%%%%%%%%%%%%%%%%%%%%%%%%%%%%%%%%%%%%%%%%%%%%%%%%%%%%%%%%%%%
%%                            Class Variables                            %%
%%%%%%%%%%%%%%%%%%%%%%%%%%%%%%%%%%%%%%%%%%%%%%%%%%%%%%%%%%%%%%%%%%%%%%%%%%%

  \subsubsection{Class Variables}

    \vspace{-1cm}
\hspace{\varindent}\begin{longtable}{|p{\varnamewidth}|p{\vardescrwidth}|l}
\cline{1-2}
\cline{1-2} \centering \textbf{Name} & \centering \textbf{Description}& \\
\cline{1-2}
\endhead\cline{1-2}\multicolumn{3}{r}{\small\textit{continued on next page}}\\\endfoot\cline{1-2}
\endlastfoot\multicolumn{2}{|l|}{\textit{Inherited from Tkinter.Misc}}\\
\multicolumn{2}{|p{\varwidth}|}{\raggedright \_noarg\_}\\
\cline{1-2}
\end{longtable}

    \index{nMOLDYN \textit{(package)}!nMOLDYN.GUI \textit{(package)}!nMOLDYN.GUI.MainDialog \textit{(module)}!nMOLDYN.GUI.MainDialog.MainDialog \textit{(class)}|)}
    \index{nMOLDYN \textit{(package)}!nMOLDYN.GUI \textit{(package)}!nMOLDYN.GUI.MainDialog \textit{(module)}|)}
