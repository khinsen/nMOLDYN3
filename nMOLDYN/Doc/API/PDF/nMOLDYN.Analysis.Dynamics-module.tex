%
% API Documentation for nMOLDYN
% Module nMOLDYN.Analysis.Dynamics
%
% Generated by epydoc 3.0.1
% [Thu Oct  8 16:59:58 2009]
%

%%%%%%%%%%%%%%%%%%%%%%%%%%%%%%%%%%%%%%%%%%%%%%%%%%%%%%%%%%%%%%%%%%%%%%%%%%%
%%                          Module Description                           %%
%%%%%%%%%%%%%%%%%%%%%%%%%%%%%%%%%%%%%%%%%%%%%%%%%%%%%%%%%%%%%%%%%%%%%%%%%%%

    \index{nMOLDYN \textit{(package)}!nMOLDYN.Analysis \textit{(package)}!nMOLDYN.Analysis.Dynamics \textit{(module)}|(}
\section{Module nMOLDYN.Analysis.Dynamics}

    \label{nMOLDYN:Analysis:Dynamics}
\begin{alltt}
Collections of classes for the determination of dynamics-related properties.

Classes:

    * MeanSquareDisplacement                   : sets up a Mean-Square-Displacement analysis.
    * RootMeanSquareDeviation                  : sets up a Root Mean-Square-Deviation analysis.
    * GyrationRadius                           : sets up a Gyration Radius analysis.
    * AngularCorrelation                       : sets up an Angular Correlation analysis.
    * CartesianVelocityAutoCorrelationFunction : sets up a Cartesian Velocity AutoCorrelation analysis. 
    * DensityOfStates                          : sets up a Density Of States analysis.
    * AutoRegressiveAnalysis                   : sets up an Auto-Regressive analysis.
    * QuasiHarmonicAnalysis                    : sets up a Quasi-Harmonic analysis.
    * PassBandTrajectoryFilter                 : sets up a Pass-Band Trajectory Filter.
    * GlobalMotionTrajectoryFilter             : sets up a Global Motion Trajectory Filter.
    * CenterOfMassTrajectory                   : sets up a Center Of Mass Trajectory.
    * RigidBodyTrajectory                      : sets up a Rigid-Body Trajectory.
    * AngularVelocityAutoCorrelationFunction   : sets up an Angular Velocity AutoCorrelation Function.
    * AngularDensityOfStates                   : sets up an Angular Density Of States.
\end{alltt}


%%%%%%%%%%%%%%%%%%%%%%%%%%%%%%%%%%%%%%%%%%%%%%%%%%%%%%%%%%%%%%%%%%%%%%%%%%%
%%                           Class Description                           %%
%%%%%%%%%%%%%%%%%%%%%%%%%%%%%%%%%%%%%%%%%%%%%%%%%%%%%%%%%%%%%%%%%%%%%%%%%%%

    \index{nMOLDYN \textit{(package)}!nMOLDYN.Analysis \textit{(package)}!nMOLDYN.Analysis.Dynamics \textit{(module)}!nMOLDYN.Analysis.Dynamics.MeanSquareDisplacement \textit{(class)}|(}
\subsection{Class MeanSquareDisplacement}

    \label{nMOLDYN:Analysis:Dynamics:MeanSquareDisplacement}
\begin{tabular}{cccccc}
% Line for nMOLDYN.Analysis.Analysis.Analysis, linespec=[False]
\multicolumn{2}{r}{\settowidth{\BCL}{nMOLDYN.Analysis.Analysis.Analysis}\multirow{2}{\BCL}{nMOLDYN.Analysis.Analysis.Analysis}}
&&
  \\\cline{3-3}
  &&\multicolumn{1}{c|}{}
&&
  \\
&&\multicolumn{2}{l}{\textbf{nMOLDYN.Analysis.Dynamics.MeanSquareDisplacement}}
\end{tabular}

\begin{alltt}
Sets up a Mean Square Displacement analysis.

A Subclass of nMOLDYN.Analysis.Analysis. 

Constructor: MeanSquareDisplacement({\textbar}parameters{\textbar} = None)

Arguments:

    - {\textbar}parameters{\textbar} -- a dictionnary of the input parameters, or 'None' to set up the analysis without parameters.
        * trajectory  -- a trajectory file name or an instance of MMTK.Trajectory.Trajectory class.
        * timeinfo    -- a string of the form 'first:last:step' where 'first' is an integer specifying the first frame 
                         number to consider, 'last' is an integer specifying the last frame number to consider and 
                         'step' is an integer specifying the step number between two frames.
        * projection  -- a string of the form 'vx,vy,vz' specifying the vector along which the analysis
                         will be computed. 'vx', 'vy', and 'vz' are floats specifying respectively the x, y and z value 
                         of that vector.
        * subset      -- a selection string specifying the atoms to consider for the analysis.
        * deuteration -- a selection string specifying the hydrogen atoms whose atomic parameters will be those of the deuterium.
        * weights     -- a string equal to 'equal', 'mass', 'coherent' , 'incoherent' or 'atomicNumber' that specifies the weighting
                         scheme to use.
        * msd         -- the output NetCDF file name. A CDL version of this file will also be generated with the '.cdl' extension
                         instead of the '.nc' extension.
        * pyroserver  -- a string specifying if Pyro will be used and how to run the analysis.
    
Running modes:

    - To run the analysis do: a.runAnalysis() where a is the analysis object.
    - To estimate the analysis do: a.estimateAnalysis() where a is the analysis object.
    - To save the analysis to 'file' file name do: a.saveAnalysis(file) where a is the analysis object.
    
Comments:

    - The algorithm is based on the Fast Correlation Algorithm (FCA) algorithm
\end{alltt}


%%%%%%%%%%%%%%%%%%%%%%%%%%%%%%%%%%%%%%%%%%%%%%%%%%%%%%%%%%%%%%%%%%%%%%%%%%%
%%                                Methods                                %%
%%%%%%%%%%%%%%%%%%%%%%%%%%%%%%%%%%%%%%%%%%%%%%%%%%%%%%%%%%%%%%%%%%%%%%%%%%%

  \subsubsection{Methods}

    \vspace{0.5ex}

\hspace{.8\funcindent}\begin{boxedminipage}{\funcwidth}

    \raggedright \textbf{\_\_init\_\_}(\textit{self})

    \vspace{-1.5ex}

    \rule{\textwidth}{0.5\fboxrule}
\setlength{\parskip}{2ex}
    The constructor. Insures that the class can not be instanciated 
    directly from here.

\setlength{\parskip}{1ex}
      \textbf{Parameters}
      \vspace{-1ex}

      \begin{quote}
        \begin{Ventry}{xxxxxxxxxx}

          \item[parameters]

          a dictionnary that contains parameters of the selected analysis.

          \item[statusBar]

          if not None, an instance of nMOLDYN.GUI.Widgets.StatusBar. Will 
          attach a status bar to the selected analysis.

        \end{Ventry}

      \end{quote}

      Overrides: nMOLDYN.Analysis.Analysis.Analysis.\_\_init\_\_

    \end{boxedminipage}

    \label{nMOLDYN:Analysis:Dynamics:MeanSquareDisplacement:initialize}
    \index{nMOLDYN \textit{(package)}!nMOLDYN.Analysis \textit{(package)}!nMOLDYN.Analysis.Dynamics \textit{(module)}!nMOLDYN.Analysis.Dynamics.MeanSquareDisplacement \textit{(class)}!nMOLDYN.Analysis.Dynamics.MeanSquareDisplacement.initialize \textit{(method)}}

    \vspace{0.5ex}

\hspace{.8\funcindent}\begin{boxedminipage}{\funcwidth}

    \raggedright \textbf{initialize}(\textit{self})

    \vspace{-1.5ex}

    \rule{\textwidth}{0.5\fboxrule}
\setlength{\parskip}{2ex}
    Initializes the analysis (e.g. parses and checks input parameters, set 
    some variables ...).

\setlength{\parskip}{1ex}
    \end{boxedminipage}

    \label{nMOLDYN:Analysis:Dynamics:MeanSquareDisplacement:calc}
    \index{nMOLDYN \textit{(package)}!nMOLDYN.Analysis \textit{(package)}!nMOLDYN.Analysis.Dynamics \textit{(module)}!nMOLDYN.Analysis.Dynamics.MeanSquareDisplacement \textit{(class)}!nMOLDYN.Analysis.Dynamics.MeanSquareDisplacement.calc \textit{(method)}}

    \vspace{0.5ex}

\hspace{.8\funcindent}\begin{boxedminipage}{\funcwidth}

    \raggedright \textbf{calc}(\textit{self}, \textit{atom}, \textit{trajname})

    \vspace{-1.5ex}

    \rule{\textwidth}{0.5\fboxrule}
\setlength{\parskip}{2ex}
    Calculates the atomic term.

\setlength{\parskip}{1ex}
      \textbf{Parameters}
      \vspace{-1ex}

      \begin{quote}
        \begin{Ventry}{xxxxxxxx}

          \item[atom]

          the atom on which the atomic term has been calculated.

            {\it (type=an instance of MMTK.Atom class.)}

          \item[trajname]

          the name of the trajectory file name.

            {\it (type=string)}

        \end{Ventry}

      \end{quote}

    \end{boxedminipage}

    \label{nMOLDYN:Analysis:Dynamics:MeanSquareDisplacement:combine}
    \index{nMOLDYN \textit{(package)}!nMOLDYN.Analysis \textit{(package)}!nMOLDYN.Analysis.Dynamics \textit{(module)}!nMOLDYN.Analysis.Dynamics.MeanSquareDisplacement \textit{(class)}!nMOLDYN.Analysis.Dynamics.MeanSquareDisplacement.combine \textit{(method)}}

    \vspace{0.5ex}

\hspace{.8\funcindent}\begin{boxedminipage}{\funcwidth}

    \raggedright \textbf{combine}(\textit{self}, \textit{atom}, \textit{x})

\setlength{\parskip}{2ex}
\setlength{\parskip}{1ex}
    \end{boxedminipage}

    \label{nMOLDYN:Analysis:Dynamics:MeanSquareDisplacement:finalize}
    \index{nMOLDYN \textit{(package)}!nMOLDYN.Analysis \textit{(package)}!nMOLDYN.Analysis.Dynamics \textit{(module)}!nMOLDYN.Analysis.Dynamics.MeanSquareDisplacement \textit{(class)}!nMOLDYN.Analysis.Dynamics.MeanSquareDisplacement.finalize \textit{(method)}}

    \vspace{0.5ex}

\hspace{.8\funcindent}\begin{boxedminipage}{\funcwidth}

    \raggedright \textbf{finalize}(\textit{self})

    \vspace{-1.5ex}

    \rule{\textwidth}{0.5\fboxrule}
\setlength{\parskip}{2ex}
    Finalizes the calculations (e.g. averaging the total term, output files
    creations ...).

\setlength{\parskip}{1ex}
    \end{boxedminipage}

    \label{nMOLDYN:Analysis:Dynamics:MeanSquareDisplacement:atomicMSD}
    \index{nMOLDYN \textit{(package)}!nMOLDYN.Analysis \textit{(package)}!nMOLDYN.Analysis.Dynamics \textit{(module)}!nMOLDYN.Analysis.Dynamics.MeanSquareDisplacement \textit{(class)}!nMOLDYN.Analysis.Dynamics.MeanSquareDisplacement.atomicMSD \textit{(method)}}

    \vspace{0.5ex}

\hspace{.8\funcindent}\begin{boxedminipage}{\funcwidth}

    \raggedright \textbf{atomicMSD}(\textit{self}, \textit{atom}, \textit{series})

    \vspace{-1.5ex}

    \rule{\textwidth}{0.5\fboxrule}
\setlength{\parskip}{2ex}
    Returns the atomic Mean-Square-Displacement.

\setlength{\parskip}{1ex}
      \textbf{Parameters}
      \vspace{-1ex}

      \begin{quote}
        \begin{Ventry}{xxxxxx}

          \item[atom]

          the atom on which the atomic MSD has been calculated.

            {\it (type=an instance of MMTK.Atom class.)}

          \item[series]

          a array of dimension (self.nFrames,3) specifying the coordinates 
          of atom {\textbar}atom{\textbar} for the selected frames.

            {\it (type=NumPy array)}

        \end{Ventry}

      \end{quote}

      \textbf{Return Value}
    \vspace{-1ex}

      \begin{quote}
      the MSD computed for atom {\textbar}atom{\textbar} with trajectory 
      {\textbar}series{\textbar}.

      {\it (type=Numpy array)}

      \end{quote}

    \end{boxedminipage}


\large{\textbf{\textit{Inherited from nMOLDYN.Analysis.Analysis.Analysis\textit{(Section \ref{nMOLDYN:Analysis:Analysis:Analysis})}}}}

\begin{quote}
analysisTime(), buildJobInfo(), buildTimeInfo(), deuterationSelection(), groupSelection(), parseInputParameters(), preLoadTrajectory(), runAnalysis(), saveAnalysis(), setInputParameters(), subsetSelection(), updateJobProgress(), weightingScheme()
\end{quote}

%%%%%%%%%%%%%%%%%%%%%%%%%%%%%%%%%%%%%%%%%%%%%%%%%%%%%%%%%%%%%%%%%%%%%%%%%%%
%%                            Class Variables                            %%
%%%%%%%%%%%%%%%%%%%%%%%%%%%%%%%%%%%%%%%%%%%%%%%%%%%%%%%%%%%%%%%%%%%%%%%%%%%

  \subsubsection{Class Variables}

    \vspace{-1cm}
\hspace{\varindent}\begin{longtable}{|p{\varnamewidth}|p{\vardescrwidth}|l}
\cline{1-2}
\cline{1-2} \centering \textbf{Name} & \centering \textbf{Description}& \\
\cline{1-2}
\endhead\cline{1-2}\multicolumn{3}{r}{\small\textit{continued on next page}}\\\endfoot\cline{1-2}
\endlastfoot\raggedright i\-n\-p\-u\-t\-P\-a\-r\-a\-m\-e\-t\-e\-r\-s\-N\-a\-m\-e\-s\- & \raggedright \textbf{Value:} 
{\tt 'trajectory', 'timeinfo', 'projection', 'subset', 'deuter\texttt{...}}&\\
\cline{1-2}
\raggedright s\-h\-o\-r\-t\-N\-a\-m\-e\- & \raggedright \textbf{Value:} 
{\tt 'MSD'}&\\
\cline{1-2}
\raggedright c\-a\-n\-B\-e\-E\-s\-t\-i\-m\-a\-t\-e\-d\- & \raggedright \textbf{Value:} 
{\tt True}&\\
\cline{1-2}
\end{longtable}

    \index{nMOLDYN \textit{(package)}!nMOLDYN.Analysis \textit{(package)}!nMOLDYN.Analysis.Dynamics \textit{(module)}!nMOLDYN.Analysis.Dynamics.MeanSquareDisplacement \textit{(class)}|)}

%%%%%%%%%%%%%%%%%%%%%%%%%%%%%%%%%%%%%%%%%%%%%%%%%%%%%%%%%%%%%%%%%%%%%%%%%%%
%%                           Class Description                           %%
%%%%%%%%%%%%%%%%%%%%%%%%%%%%%%%%%%%%%%%%%%%%%%%%%%%%%%%%%%%%%%%%%%%%%%%%%%%

    \index{nMOLDYN \textit{(package)}!nMOLDYN.Analysis \textit{(package)}!nMOLDYN.Analysis.Dynamics \textit{(module)}!nMOLDYN.Analysis.Dynamics.RootMeanSquareDeviation \textit{(class)}|(}
\subsection{Class RootMeanSquareDeviation}

    \label{nMOLDYN:Analysis:Dynamics:RootMeanSquareDeviation}
\begin{tabular}{cccccc}
% Line for nMOLDYN.Analysis.Analysis.Analysis, linespec=[False]
\multicolumn{2}{r}{\settowidth{\BCL}{nMOLDYN.Analysis.Analysis.Analysis}\multirow{2}{\BCL}{nMOLDYN.Analysis.Analysis.Analysis}}
&&
  \\\cline{3-3}
  &&\multicolumn{1}{c|}{}
&&
  \\
&&\multicolumn{2}{l}{\textbf{nMOLDYN.Analysis.Dynamics.RootMeanSquareDeviation}}
\end{tabular}

\begin{alltt}
Sets up a Root Mean Square Deviation analysis.

A Subclass of nMOLDYN.Analysis.Analysis. 

Constructor: RootMeanSquareDeviation({\textbar}parameters{\textbar} = None)

Arguments:

    - {\textbar}parameters{\textbar} -- a dictionnary of the input parameters, or 'None' to set up the analysis without parameters.
        * trajectory     -- a trajectory file name or an instance of MMTK.Trajectory.Trajectory class.
        * timeinfo       -- a string of the form 'first:last:step' where 'first' is an integer specifying the first frame 
                            number to consider, 'last' is an integer specifying the last frame number to consider and 
                            'step' is an integer specifying the step number between two frames.
        * referenceframe -- an integer in [1,len(trajectory)] specifying which frame should be the reference.
        * subset         -- a selection string specifying the atoms to consider for the analysis.
        * deuteration    -- a selection string specifying the hydrogen atoms whose atomic parameters will be those of the deuterium.
        * weights        -- a string equal to 'equal', 'mass', 'coherent' , 'incoherent' or 'atomicNumber' that specifies the weighting
                            scheme to use.
        * rmsd           -- the output NetCDF file name. A CDL version of this file will also be generated with the '.cdl' extension
                            instead of the '.nc' extension.
        * pyroserver     -- a string specifying if Pyro will be used and how to run the analysis.
    
Running modes:

    - To run the analysis do: a.runAnalysis() where a is the analysis object.
    - To estimate the analysis do: a.estimateAnalysis() where a is the analysis object.
    - To save the analysis to 'file' file name do: a.saveAnalysis(file) where a is the analysis object.
\end{alltt}


%%%%%%%%%%%%%%%%%%%%%%%%%%%%%%%%%%%%%%%%%%%%%%%%%%%%%%%%%%%%%%%%%%%%%%%%%%%
%%                                Methods                                %%
%%%%%%%%%%%%%%%%%%%%%%%%%%%%%%%%%%%%%%%%%%%%%%%%%%%%%%%%%%%%%%%%%%%%%%%%%%%

  \subsubsection{Methods}

    \vspace{0.5ex}

\hspace{.8\funcindent}\begin{boxedminipage}{\funcwidth}

    \raggedright \textbf{\_\_init\_\_}(\textit{self})

    \vspace{-1.5ex}

    \rule{\textwidth}{0.5\fboxrule}
\setlength{\parskip}{2ex}
    The constructor. Insures that the class can not be instanciated 
    directly from here.

\setlength{\parskip}{1ex}
      \textbf{Parameters}
      \vspace{-1ex}

      \begin{quote}
        \begin{Ventry}{xxxxxxxxxx}

          \item[parameters]

          a dictionnary that contains parameters of the selected analysis.

          \item[statusBar]

          if not None, an instance of nMOLDYN.GUI.Widgets.StatusBar. Will 
          attach a status bar to the selected analysis.

        \end{Ventry}

      \end{quote}

      Overrides: nMOLDYN.Analysis.Analysis.Analysis.\_\_init\_\_

    \end{boxedminipage}

    \label{nMOLDYN:Analysis:Dynamics:RootMeanSquareDeviation:initialize}
    \index{nMOLDYN \textit{(package)}!nMOLDYN.Analysis \textit{(package)}!nMOLDYN.Analysis.Dynamics \textit{(module)}!nMOLDYN.Analysis.Dynamics.RootMeanSquareDeviation \textit{(class)}!nMOLDYN.Analysis.Dynamics.RootMeanSquareDeviation.initialize \textit{(method)}}

    \vspace{0.5ex}

\hspace{.8\funcindent}\begin{boxedminipage}{\funcwidth}

    \raggedright \textbf{initialize}(\textit{self})

    \vspace{-1.5ex}

    \rule{\textwidth}{0.5\fboxrule}
\setlength{\parskip}{2ex}
    Initializes the analysis (e.g. parses and checks input parameters, set 
    some variables ...).

\setlength{\parskip}{1ex}
    \end{boxedminipage}

    \label{nMOLDYN:Analysis:Dynamics:RootMeanSquareDeviation:calc}
    \index{nMOLDYN \textit{(package)}!nMOLDYN.Analysis \textit{(package)}!nMOLDYN.Analysis.Dynamics \textit{(module)}!nMOLDYN.Analysis.Dynamics.RootMeanSquareDeviation \textit{(class)}!nMOLDYN.Analysis.Dynamics.RootMeanSquareDeviation.calc \textit{(method)}}

    \vspace{0.5ex}

\hspace{.8\funcindent}\begin{boxedminipage}{\funcwidth}

    \raggedright \textbf{calc}(\textit{self}, \textit{atom}, \textit{trajname})

    \vspace{-1.5ex}

    \rule{\textwidth}{0.5\fboxrule}
\setlength{\parskip}{2ex}
    Calculates the atomic term.

\setlength{\parskip}{1ex}
      \textbf{Parameters}
      \vspace{-1ex}

      \begin{quote}
        \begin{Ventry}{xxxxxxxx}

          \item[atom]

          the atom on which the atomic term has been calculated.

            {\it (type=an instance of MMTK.Atom class.)}

          \item[trajname]

          the name of the trajectory file name.

            {\it (type=string)}

        \end{Ventry}

      \end{quote}

\textbf{Note:} an atom-by-atom implementation was prefered than a frame-by-frame 
implementation of the type: msd = t.configuration[frame] - 
t.configuration[self.referenceFrame] msd = self.weights * msd * msd 
self.RMSD[frameIndex] = N.sqrt(N.add.reduce(msd))



    \end{boxedminipage}

    \label{nMOLDYN:Analysis:Dynamics:RootMeanSquareDeviation:combine}
    \index{nMOLDYN \textit{(package)}!nMOLDYN.Analysis \textit{(package)}!nMOLDYN.Analysis.Dynamics \textit{(module)}!nMOLDYN.Analysis.Dynamics.RootMeanSquareDeviation \textit{(class)}!nMOLDYN.Analysis.Dynamics.RootMeanSquareDeviation.combine \textit{(method)}}

    \vspace{0.5ex}

\hspace{.8\funcindent}\begin{boxedminipage}{\funcwidth}

    \raggedright \textbf{combine}(\textit{self}, \textit{atom}, \textit{x})

\setlength{\parskip}{2ex}
\setlength{\parskip}{1ex}
    \end{boxedminipage}

    \label{nMOLDYN:Analysis:Dynamics:RootMeanSquareDeviation:finalize}
    \index{nMOLDYN \textit{(package)}!nMOLDYN.Analysis \textit{(package)}!nMOLDYN.Analysis.Dynamics \textit{(module)}!nMOLDYN.Analysis.Dynamics.RootMeanSquareDeviation \textit{(class)}!nMOLDYN.Analysis.Dynamics.RootMeanSquareDeviation.finalize \textit{(method)}}

    \vspace{0.5ex}

\hspace{.8\funcindent}\begin{boxedminipage}{\funcwidth}

    \raggedright \textbf{finalize}(\textit{self})

    \vspace{-1.5ex}

    \rule{\textwidth}{0.5\fboxrule}
\setlength{\parskip}{2ex}
    Finalizes the calculations (e.g. averaging the total term, output files
    creations ...)

\setlength{\parskip}{1ex}
    \end{boxedminipage}


\large{\textbf{\textit{Inherited from nMOLDYN.Analysis.Analysis.Analysis\textit{(Section \ref{nMOLDYN:Analysis:Analysis:Analysis})}}}}

\begin{quote}
analysisTime(), buildJobInfo(), buildTimeInfo(), deuterationSelection(), groupSelection(), parseInputParameters(), preLoadTrajectory(), runAnalysis(), saveAnalysis(), setInputParameters(), subsetSelection(), updateJobProgress(), weightingScheme()
\end{quote}

%%%%%%%%%%%%%%%%%%%%%%%%%%%%%%%%%%%%%%%%%%%%%%%%%%%%%%%%%%%%%%%%%%%%%%%%%%%
%%                            Class Variables                            %%
%%%%%%%%%%%%%%%%%%%%%%%%%%%%%%%%%%%%%%%%%%%%%%%%%%%%%%%%%%%%%%%%%%%%%%%%%%%

  \subsubsection{Class Variables}

    \vspace{-1cm}
\hspace{\varindent}\begin{longtable}{|p{\varnamewidth}|p{\vardescrwidth}|l}
\cline{1-2}
\cline{1-2} \centering \textbf{Name} & \centering \textbf{Description}& \\
\cline{1-2}
\endhead\cline{1-2}\multicolumn{3}{r}{\small\textit{continued on next page}}\\\endfoot\cline{1-2}
\endlastfoot\raggedright i\-n\-p\-u\-t\-P\-a\-r\-a\-m\-e\-t\-e\-r\-s\-N\-a\-m\-e\-s\- & \raggedright \textbf{Value:} 
{\tt 'trajectory', 'timeinfo', 'referenceframe', 'subset', 'de\texttt{...}}&\\
\cline{1-2}
\raggedright s\-h\-o\-r\-t\-N\-a\-m\-e\- & \raggedright \textbf{Value:} 
{\tt 'RMSD'}&\\
\cline{1-2}
\raggedright c\-a\-n\-B\-e\-E\-s\-t\-i\-m\-a\-t\-e\-d\- & \raggedright \textbf{Value:} 
{\tt True}&\\
\cline{1-2}
\end{longtable}

    \index{nMOLDYN \textit{(package)}!nMOLDYN.Analysis \textit{(package)}!nMOLDYN.Analysis.Dynamics \textit{(module)}!nMOLDYN.Analysis.Dynamics.RootMeanSquareDeviation \textit{(class)}|)}

%%%%%%%%%%%%%%%%%%%%%%%%%%%%%%%%%%%%%%%%%%%%%%%%%%%%%%%%%%%%%%%%%%%%%%%%%%%
%%                           Class Description                           %%
%%%%%%%%%%%%%%%%%%%%%%%%%%%%%%%%%%%%%%%%%%%%%%%%%%%%%%%%%%%%%%%%%%%%%%%%%%%

    \index{nMOLDYN \textit{(package)}!nMOLDYN.Analysis \textit{(package)}!nMOLDYN.Analysis.Dynamics \textit{(module)}!nMOLDYN.Analysis.Dynamics.CartesianVelocityAutoCorrelationFunction \textit{(class)}|(}
\subsection{Class CartesianVelocityAutoCorrelationFunction}

    \label{nMOLDYN:Analysis:Dynamics:CartesianVelocityAutoCorrelationFunction}
\begin{tabular}{cccccc}
% Line for nMOLDYN.Analysis.Analysis.Analysis, linespec=[False]
\multicolumn{2}{r}{\settowidth{\BCL}{nMOLDYN.Analysis.Analysis.Analysis}\multirow{2}{\BCL}{nMOLDYN.Analysis.Analysis.Analysis}}
&&
  \\\cline{3-3}
  &&\multicolumn{1}{c|}{}
&&
  \\
&&\multicolumn{2}{l}{\textbf{nMOLDYN.Analysis.Dynamics.CartesianVelocityAutoCorrelationFunction}}
\end{tabular}

\begin{alltt}
Sets up a Cartesian Velocity AutoCorrelation analysis.

A Subclass of nMOLDYN.Analysis.Analysis. 

Constructor: CartesianVelocityAutoCorrelationFunction({\textbar}parameters{\textbar} = None)

Arguments:

    - {\textbar}parameters{\textbar} -- a dictionnary of the input parameters, or 'None' to set up the analysis without parameters.
        * trajectory      -- a trajectory file name or an instance of MMTK.Trajectory.Trajectory class.
        * timeinfo        -- a string of the form 'first:last:step' where 'first' is an integer specifying the first frame 
                             number to consider, 'last' is an integer specifying the last frame number to consider and 
                             'step' is an integer specifying the step number between two frames.
        * differentiation -- an integer in [0,5] specifying the order of the differentiation used to get the velocities
                             out of the coordinates. 0 means that the velocities are already present in the trajectory loaded
                             for analysis.
        * projection      -- a string of the form 'vx,vy,vz' specifying the vector along which the analysis
                             will be computed. 'vx', 'vy', and 'vz' are floats specifying respectively the x, y and z value 
                             of that vector.
        * normalize       -- a string being one of 'Yes' or 'No' specifying whether the analysis should be normalized to 1
                             at t = 0 ('Yes') or not ('No').
        * subset          -- a selection string specifying the atoms to consider for the analysis.
        * deuteration     -- a selection string specifying the hydrogen atoms whose atomic parameters will be those of the deuterium.
        * weights         -- a string equal to 'equal', 'mass', 'coherent' , 'incoherent' or 'atomicNumber' that specifies the weighting
                             scheme to use.
        * vacf            -- the output NetCDF file name. A CDL version of this file will also be generated with the '.cdl' extension
                             instead of the '.nc' extension.
        * pyroserver      -- a string specifying if Pyro will be used and how to run the analysis.
    
Running modes:

    - To run the analysis do: a.runAnalysis() where a is the analysis object.
    - To estimate the analysis do: a.estimateAnalysis() where a is the analysis object.
    - To save the analysis to 'file' file name do: a.saveAnalysis(file) where a is the analysis object.
\end{alltt}


%%%%%%%%%%%%%%%%%%%%%%%%%%%%%%%%%%%%%%%%%%%%%%%%%%%%%%%%%%%%%%%%%%%%%%%%%%%
%%                                Methods                                %%
%%%%%%%%%%%%%%%%%%%%%%%%%%%%%%%%%%%%%%%%%%%%%%%%%%%%%%%%%%%%%%%%%%%%%%%%%%%

  \subsubsection{Methods}

    \vspace{0.5ex}

\hspace{.8\funcindent}\begin{boxedminipage}{\funcwidth}

    \raggedright \textbf{\_\_init\_\_}(\textit{self})

    \vspace{-1.5ex}

    \rule{\textwidth}{0.5\fboxrule}
\setlength{\parskip}{2ex}
    The constructor. Insures that the class can not be instanciated 
    directly from here.

\setlength{\parskip}{1ex}
      \textbf{Parameters}
      \vspace{-1ex}

      \begin{quote}
        \begin{Ventry}{xxxxxxxxxx}

          \item[parameters]

          a dictionnary that contains parameters of the selected analysis.

          \item[statusBar]

          if not None, an instance of nMOLDYN.GUI.Widgets.StatusBar. Will 
          attach a status bar to the selected analysis.

        \end{Ventry}

      \end{quote}

      Overrides: nMOLDYN.Analysis.Analysis.Analysis.\_\_init\_\_

    \end{boxedminipage}

    \label{nMOLDYN:Analysis:Dynamics:CartesianVelocityAutoCorrelationFunction:initialize}
    \index{nMOLDYN \textit{(package)}!nMOLDYN.Analysis \textit{(package)}!nMOLDYN.Analysis.Dynamics \textit{(module)}!nMOLDYN.Analysis.Dynamics.CartesianVelocityAutoCorrelationFunction \textit{(class)}!nMOLDYN.Analysis.Dynamics.CartesianVelocityAutoCorrelationFunction.initialize \textit{(method)}}

    \vspace{0.5ex}

\hspace{.8\funcindent}\begin{boxedminipage}{\funcwidth}

    \raggedright \textbf{initialize}(\textit{self})

    \vspace{-1.5ex}

    \rule{\textwidth}{0.5\fboxrule}
\setlength{\parskip}{2ex}
    Initializes the analysis (e.g. parses and checks input parameters, set 
    some variables ...).

\setlength{\parskip}{1ex}
    \end{boxedminipage}

    \label{nMOLDYN:Analysis:Dynamics:CartesianVelocityAutoCorrelationFunction:calc}
    \index{nMOLDYN \textit{(package)}!nMOLDYN.Analysis \textit{(package)}!nMOLDYN.Analysis.Dynamics \textit{(module)}!nMOLDYN.Analysis.Dynamics.CartesianVelocityAutoCorrelationFunction \textit{(class)}!nMOLDYN.Analysis.Dynamics.CartesianVelocityAutoCorrelationFunction.calc \textit{(method)}}

    \vspace{0.5ex}

\hspace{.8\funcindent}\begin{boxedminipage}{\funcwidth}

    \raggedright \textbf{calc}(\textit{self}, \textit{atom}, \textit{trajname})

    \vspace{-1.5ex}

    \rule{\textwidth}{0.5\fboxrule}
\setlength{\parskip}{2ex}
    Calculates the atomic term.

\setlength{\parskip}{1ex}
      \textbf{Parameters}
      \vspace{-1ex}

      \begin{quote}
        \begin{Ventry}{xxxxxxxx}

          \item[atom]

          the atom on which the atomic term has been calculated.

            {\it (type=an instance of MMTK.Atom class.)}

          \item[trajname]

          the name of the trajectory file name.

            {\it (type=string)}

        \end{Ventry}

      \end{quote}

    \end{boxedminipage}

    \label{nMOLDYN:Analysis:Dynamics:CartesianVelocityAutoCorrelationFunction:combine}
    \index{nMOLDYN \textit{(package)}!nMOLDYN.Analysis \textit{(package)}!nMOLDYN.Analysis.Dynamics \textit{(module)}!nMOLDYN.Analysis.Dynamics.CartesianVelocityAutoCorrelationFunction \textit{(class)}!nMOLDYN.Analysis.Dynamics.CartesianVelocityAutoCorrelationFunction.combine \textit{(method)}}

    \vspace{0.5ex}

\hspace{.8\funcindent}\begin{boxedminipage}{\funcwidth}

    \raggedright \textbf{combine}(\textit{self}, \textit{atom}, \textit{x})

\setlength{\parskip}{2ex}
\setlength{\parskip}{1ex}
    \end{boxedminipage}

    \label{nMOLDYN:Analysis:Dynamics:CartesianVelocityAutoCorrelationFunction:finalize}
    \index{nMOLDYN \textit{(package)}!nMOLDYN.Analysis \textit{(package)}!nMOLDYN.Analysis.Dynamics \textit{(module)}!nMOLDYN.Analysis.Dynamics.CartesianVelocityAutoCorrelationFunction \textit{(class)}!nMOLDYN.Analysis.Dynamics.CartesianVelocityAutoCorrelationFunction.finalize \textit{(method)}}

    \vspace{0.5ex}

\hspace{.8\funcindent}\begin{boxedminipage}{\funcwidth}

    \raggedright \textbf{finalize}(\textit{self})

    \vspace{-1.5ex}

    \rule{\textwidth}{0.5\fboxrule}
\setlength{\parskip}{2ex}
    Finalizes the calculations (e.g. averaging the total term, output files
    creations ...)

\setlength{\parskip}{1ex}
    \end{boxedminipage}


\large{\textbf{\textit{Inherited from nMOLDYN.Analysis.Analysis.Analysis\textit{(Section \ref{nMOLDYN:Analysis:Analysis:Analysis})}}}}

\begin{quote}
analysisTime(), buildJobInfo(), buildTimeInfo(), deuterationSelection(), groupSelection(), parseInputParameters(), preLoadTrajectory(), runAnalysis(), saveAnalysis(), setInputParameters(), subsetSelection(), updateJobProgress(), weightingScheme()
\end{quote}

%%%%%%%%%%%%%%%%%%%%%%%%%%%%%%%%%%%%%%%%%%%%%%%%%%%%%%%%%%%%%%%%%%%%%%%%%%%
%%                            Class Variables                            %%
%%%%%%%%%%%%%%%%%%%%%%%%%%%%%%%%%%%%%%%%%%%%%%%%%%%%%%%%%%%%%%%%%%%%%%%%%%%

  \subsubsection{Class Variables}

    \vspace{-1cm}
\hspace{\varindent}\begin{longtable}{|p{\varnamewidth}|p{\vardescrwidth}|l}
\cline{1-2}
\cline{1-2} \centering \textbf{Name} & \centering \textbf{Description}& \\
\cline{1-2}
\endhead\cline{1-2}\multicolumn{3}{r}{\small\textit{continued on next page}}\\\endfoot\cline{1-2}
\endlastfoot\raggedright i\-n\-p\-u\-t\-P\-a\-r\-a\-m\-e\-t\-e\-r\-s\-N\-a\-m\-e\-s\- & \raggedright \textbf{Value:} 
{\tt 'trajectory', 'timeinfo', 'differentiation', 'projection'\texttt{...}}&\\
\cline{1-2}
\raggedright s\-h\-o\-r\-t\-N\-a\-m\-e\- & \raggedright \textbf{Value:} 
{\tt 'VACF'}&\\
\cline{1-2}
\raggedright c\-a\-n\-B\-e\-E\-s\-t\-i\-m\-a\-t\-e\-d\- & \raggedright \textbf{Value:} 
{\tt True}&\\
\cline{1-2}
\end{longtable}

    \index{nMOLDYN \textit{(package)}!nMOLDYN.Analysis \textit{(package)}!nMOLDYN.Analysis.Dynamics \textit{(module)}!nMOLDYN.Analysis.Dynamics.CartesianVelocityAutoCorrelationFunction \textit{(class)}|)}

%%%%%%%%%%%%%%%%%%%%%%%%%%%%%%%%%%%%%%%%%%%%%%%%%%%%%%%%%%%%%%%%%%%%%%%%%%%
%%                           Class Description                           %%
%%%%%%%%%%%%%%%%%%%%%%%%%%%%%%%%%%%%%%%%%%%%%%%%%%%%%%%%%%%%%%%%%%%%%%%%%%%

    \index{nMOLDYN \textit{(package)}!nMOLDYN.Analysis \textit{(package)}!nMOLDYN.Analysis.Dynamics \textit{(module)}!nMOLDYN.Analysis.Dynamics.CartesianDensityOfStates \textit{(class)}|(}
\subsection{Class CartesianDensityOfStates}

    \label{nMOLDYN:Analysis:Dynamics:CartesianDensityOfStates}
\begin{tabular}{cccccc}
% Line for nMOLDYN.Analysis.Analysis.Analysis, linespec=[False]
\multicolumn{2}{r}{\settowidth{\BCL}{nMOLDYN.Analysis.Analysis.Analysis}\multirow{2}{\BCL}{nMOLDYN.Analysis.Analysis.Analysis}}
&&
  \\\cline{3-3}
  &&\multicolumn{1}{c|}{}
&&
  \\
&&\multicolumn{2}{l}{\textbf{nMOLDYN.Analysis.Dynamics.CartesianDensityOfStates}}
\end{tabular}

\begin{alltt}
Sets up a Cartesian Density Of States analysis.

A Subclass of nMOLDYN.Analysis.Analysis. 

Constructor: CartesianDensityOfStates({\textbar}parameters{\textbar} = None)

Arguments:

    - {\textbar}parameters{\textbar} -- a dictionnary of the input parameters, or 'None' to set up the analysis without parameters.
        * trajectory      -- a trajectory file name or an instance of MMTK.Trajectory.Trajectory class.
        * timeinfo        -- a string of the form 'first:last:step' where 'first' is an integer specifying the first frame 
                             number to consider, 'last' is an integer specifying the last frame number to consider and 
                             'step' is an integer specifying the step number between two frames.
        * differentiation -- an integer in [0,5] specifying the order of the differentiation used to get the velocities
                             out of the coordinates. 0 means that the velocities are already present in the trajectory loaded
                             for analysis.
        * projection      -- a string of the form 'vx,vy,vz' specifying the vector along which the analysis
                             will be computed. 'vx', 'vy', and 'vz' are floats specifying respectively the x, y and z value 
                             of that vector.
        * fftwindow       -- a float in ]0.0,100.0[ specifying the width of the gaussian, in percentage of the trajectory length
                             that will be used in the smoothing procedure.
        * subset          -- a selection string specifying the atoms to consider for the analysis.
        * deuteration     -- a selection string specifying the hydrogen atoms whose atomic parameters will be those of the deuterium.
        * weights         -- a string equal to 'equal', 'mass', 'coherent' , 'incoherent' or 'atomicNumber' that specifies the weighting
                             scheme to use.
        * dos             -- the output NetCDF file name. A CDL version of this file will also be generated with the '.cdl' extension
                             instead of the '.nc' extension.
        * pyroserver      -- a string specifying if Pyro will be used and how to run the analysis.
    
Running modes:

    - To run the analysis do: a.runAnalysis() where a is the analysis object.
    - To estimate the analysis do: a.estimateAnalysis() where a is the analysis object.
    - To save the analysis to 'file' file name do: a.saveAnalysis(file) where a is the analysis object.
\end{alltt}


%%%%%%%%%%%%%%%%%%%%%%%%%%%%%%%%%%%%%%%%%%%%%%%%%%%%%%%%%%%%%%%%%%%%%%%%%%%
%%                                Methods                                %%
%%%%%%%%%%%%%%%%%%%%%%%%%%%%%%%%%%%%%%%%%%%%%%%%%%%%%%%%%%%%%%%%%%%%%%%%%%%

  \subsubsection{Methods}

    \vspace{0.5ex}

\hspace{.8\funcindent}\begin{boxedminipage}{\funcwidth}

    \raggedright \textbf{\_\_init\_\_}(\textit{self})

    \vspace{-1.5ex}

    \rule{\textwidth}{0.5\fboxrule}
\setlength{\parskip}{2ex}
    The constructor. Insures that the class can not be instanciated 
    directly from here.

\setlength{\parskip}{1ex}
      \textbf{Parameters}
      \vspace{-1ex}

      \begin{quote}
        \begin{Ventry}{xxxxxxxxxx}

          \item[parameters]

          a dictionnary that contains parameters of the selected analysis.

          \item[statusBar]

          if not None, an instance of nMOLDYN.GUI.Widgets.StatusBar. Will 
          attach a status bar to the selected analysis.

        \end{Ventry}

      \end{quote}

      Overrides: nMOLDYN.Analysis.Analysis.Analysis.\_\_init\_\_

    \end{boxedminipage}

    \label{nMOLDYN:Analysis:Dynamics:CartesianDensityOfStates:initialize}
    \index{nMOLDYN \textit{(package)}!nMOLDYN.Analysis \textit{(package)}!nMOLDYN.Analysis.Dynamics \textit{(module)}!nMOLDYN.Analysis.Dynamics.CartesianDensityOfStates \textit{(class)}!nMOLDYN.Analysis.Dynamics.CartesianDensityOfStates.initialize \textit{(method)}}

    \vspace{0.5ex}

\hspace{.8\funcindent}\begin{boxedminipage}{\funcwidth}

    \raggedright \textbf{initialize}(\textit{self})

    \vspace{-1.5ex}

    \rule{\textwidth}{0.5\fboxrule}
\setlength{\parskip}{2ex}
    Initializes the analysis (e.g. parses and checks input parameters, set 
    some variables ...).

\setlength{\parskip}{1ex}
    \end{boxedminipage}

    \label{nMOLDYN:Analysis:Dynamics:CartesianDensityOfStates:calc}
    \index{nMOLDYN \textit{(package)}!nMOLDYN.Analysis \textit{(package)}!nMOLDYN.Analysis.Dynamics \textit{(module)}!nMOLDYN.Analysis.Dynamics.CartesianDensityOfStates \textit{(class)}!nMOLDYN.Analysis.Dynamics.CartesianDensityOfStates.calc \textit{(method)}}

    \vspace{0.5ex}

\hspace{.8\funcindent}\begin{boxedminipage}{\funcwidth}

    \raggedright \textbf{calc}(\textit{self}, \textit{atom}, \textit{trajname})

    \vspace{-1.5ex}

    \rule{\textwidth}{0.5\fboxrule}
\setlength{\parskip}{2ex}
    Calculates the atomic term.

\setlength{\parskip}{1ex}
      \textbf{Parameters}
      \vspace{-1ex}

      \begin{quote}
        \begin{Ventry}{xxxxxxxx}

          \item[atom]

          the atom on which the atomic term has been calculated.

            {\it (type=an instance of MMTK.Atom class.)}

          \item[trajname]

          the name of the trajectory file name.

            {\it (type=string)}

        \end{Ventry}

      \end{quote}

    \end{boxedminipage}

    \label{nMOLDYN:Analysis:Dynamics:CartesianDensityOfStates:combine}
    \index{nMOLDYN \textit{(package)}!nMOLDYN.Analysis \textit{(package)}!nMOLDYN.Analysis.Dynamics \textit{(module)}!nMOLDYN.Analysis.Dynamics.CartesianDensityOfStates \textit{(class)}!nMOLDYN.Analysis.Dynamics.CartesianDensityOfStates.combine \textit{(method)}}

    \vspace{0.5ex}

\hspace{.8\funcindent}\begin{boxedminipage}{\funcwidth}

    \raggedright \textbf{combine}(\textit{self}, \textit{atom}, \textit{x})

\setlength{\parskip}{2ex}
\setlength{\parskip}{1ex}
    \end{boxedminipage}

    \label{nMOLDYN:Analysis:Dynamics:CartesianDensityOfStates:finalize}
    \index{nMOLDYN \textit{(package)}!nMOLDYN.Analysis \textit{(package)}!nMOLDYN.Analysis.Dynamics \textit{(module)}!nMOLDYN.Analysis.Dynamics.CartesianDensityOfStates \textit{(class)}!nMOLDYN.Analysis.Dynamics.CartesianDensityOfStates.finalize \textit{(method)}}

    \vspace{0.5ex}

\hspace{.8\funcindent}\begin{boxedminipage}{\funcwidth}

    \raggedright \textbf{finalize}(\textit{self})

    \vspace{-1.5ex}

    \rule{\textwidth}{0.5\fboxrule}
\setlength{\parskip}{2ex}
    Finalizes the calculations (e.g. averaging the total term, output files
    creations ...)

\setlength{\parskip}{1ex}
    \end{boxedminipage}


\large{\textbf{\textit{Inherited from nMOLDYN.Analysis.Analysis.Analysis\textit{(Section \ref{nMOLDYN:Analysis:Analysis:Analysis})}}}}

\begin{quote}
analysisTime(), buildJobInfo(), buildTimeInfo(), deuterationSelection(), groupSelection(), parseInputParameters(), preLoadTrajectory(), runAnalysis(), saveAnalysis(), setInputParameters(), subsetSelection(), updateJobProgress(), weightingScheme()
\end{quote}

%%%%%%%%%%%%%%%%%%%%%%%%%%%%%%%%%%%%%%%%%%%%%%%%%%%%%%%%%%%%%%%%%%%%%%%%%%%
%%                            Class Variables                            %%
%%%%%%%%%%%%%%%%%%%%%%%%%%%%%%%%%%%%%%%%%%%%%%%%%%%%%%%%%%%%%%%%%%%%%%%%%%%

  \subsubsection{Class Variables}

    \vspace{-1cm}
\hspace{\varindent}\begin{longtable}{|p{\varnamewidth}|p{\vardescrwidth}|l}
\cline{1-2}
\cline{1-2} \centering \textbf{Name} & \centering \textbf{Description}& \\
\cline{1-2}
\endhead\cline{1-2}\multicolumn{3}{r}{\small\textit{continued on next page}}\\\endfoot\cline{1-2}
\endlastfoot\raggedright i\-n\-p\-u\-t\-P\-a\-r\-a\-m\-e\-t\-e\-r\-s\-N\-a\-m\-e\-s\- & \raggedright \textbf{Value:} 
{\tt 'trajectory', 'timeinfo', 'differentiation', 'projection'\texttt{...}}&\\
\cline{1-2}
\raggedright s\-h\-o\-r\-t\-N\-a\-m\-e\- & \raggedright \textbf{Value:} 
{\tt 'DOS'}&\\
\cline{1-2}
\raggedright c\-a\-n\-B\-e\-E\-s\-t\-i\-m\-a\-t\-e\-d\- & \raggedright \textbf{Value:} 
{\tt True}&\\
\cline{1-2}
\end{longtable}

    \index{nMOLDYN \textit{(package)}!nMOLDYN.Analysis \textit{(package)}!nMOLDYN.Analysis.Dynamics \textit{(module)}!nMOLDYN.Analysis.Dynamics.CartesianDensityOfStates \textit{(class)}|)}

%%%%%%%%%%%%%%%%%%%%%%%%%%%%%%%%%%%%%%%%%%%%%%%%%%%%%%%%%%%%%%%%%%%%%%%%%%%
%%                           Class Description                           %%
%%%%%%%%%%%%%%%%%%%%%%%%%%%%%%%%%%%%%%%%%%%%%%%%%%%%%%%%%%%%%%%%%%%%%%%%%%%

    \index{nMOLDYN \textit{(package)}!nMOLDYN.Analysis \textit{(package)}!nMOLDYN.Analysis.Dynamics \textit{(module)}!nMOLDYN.Analysis.Dynamics.AutoRegressiveAnalysis \textit{(class)}|(}
\subsection{Class AutoRegressiveAnalysis}

    \label{nMOLDYN:Analysis:Dynamics:AutoRegressiveAnalysis}
\begin{tabular}{cccccc}
% Line for nMOLDYN.Analysis.Analysis.Analysis, linespec=[False]
\multicolumn{2}{r}{\settowidth{\BCL}{nMOLDYN.Analysis.Analysis.Analysis}\multirow{2}{\BCL}{nMOLDYN.Analysis.Analysis.Analysis}}
&&
  \\\cline{3-3}
  &&\multicolumn{1}{c|}{}
&&
  \\
&&\multicolumn{2}{l}{\textbf{nMOLDYN.Analysis.Dynamics.AutoRegressiveAnalysis}}
\end{tabular}

\begin{alltt}
Sets up an AutoRegressive Analysis analysis.

A Subclass of nMOLDYN.Analysis.Analysis. 

Constructor: AutoRegressiveAnalysis({\textbar}parameters{\textbar} = None)

Arguments:

    - {\textbar}parameters{\textbar} -- a dictionnary of the input parameters, or 'None' to set up the analysis without parameters.
        * trajectory      -- a trajectory file name or an instance of MMTK.Trajectory.Trajectory class.
        * timeinfo        -- a string of the form 'first:last:step' where 'first' is an integer specifying the first frame 
                             number to consider, 'last' is an integer specifying the last frame number to consider and 
                             'step' is an integer specifying the step number between two frames.
        * differentiation -- an integer in [0,5] specifying the order of the differentiation used to get the velocities
                             out of the coordinates. 0 means that the velocities are already present in the trajectory loaded
                             for analysis.
        * projection      -- a string of the form 'vx,vy,vz' specifying the vector along which the analysis
                             will be computed. 'vx', 'vy', and 'vz' are floats specifying respectively the x, y and z value 
                             of that vector.
        * armodelorder    -- an integer in [1, len(trajectory)[ specifying the order of the model
        * subset          -- a selection string specifying the atoms to consider for the analysis.
        * deuteration     -- a selection string specifying the hydrogen atoms whose atomic parameters will be those of the deuterium.
        * weights         -- a string equal to 'equal', 'mass', 'coherent' , 'incoherent' or 'atomicNumber' that specifies the weighting
                             scheme to use.
        * ara             -- the output NetCDF file name. A CDL version of this file will also be generated with the '.cdl' extension
                             instead of the '.nc' extension.
        * pyroserver      -- a string specifying if Pyro will be used and how to run the analysis.
    
Running modes:

    - To run the analysis do: a.runAnalysis() where a is the analysis object.
    - To estimate the analysis do: a.estimateAnalysis() where a is the analysis object.
    - To save the analysis to 'file' file name do: a.saveAnalysis(file) where a is the analysis object.
\end{alltt}


%%%%%%%%%%%%%%%%%%%%%%%%%%%%%%%%%%%%%%%%%%%%%%%%%%%%%%%%%%%%%%%%%%%%%%%%%%%
%%                                Methods                                %%
%%%%%%%%%%%%%%%%%%%%%%%%%%%%%%%%%%%%%%%%%%%%%%%%%%%%%%%%%%%%%%%%%%%%%%%%%%%

  \subsubsection{Methods}

    \vspace{0.5ex}

\hspace{.8\funcindent}\begin{boxedminipage}{\funcwidth}

    \raggedright \textbf{\_\_init\_\_}(\textit{self})

    \vspace{-1.5ex}

    \rule{\textwidth}{0.5\fboxrule}
\setlength{\parskip}{2ex}
    The constructor. Insures that the class can not be instanciated 
    directly from here.

\setlength{\parskip}{1ex}
      \textbf{Parameters}
      \vspace{-1ex}

      \begin{quote}
        \begin{Ventry}{xxxxxxxxxx}

          \item[parameters]

          a dictionnary that contains parameters of the selected analysis.

          \item[statusBar]

          if not None, an instance of nMOLDYN.GUI.Widgets.StatusBar. Will 
          attach a status bar to the selected analysis.

        \end{Ventry}

      \end{quote}

      Overrides: nMOLDYN.Analysis.Analysis.Analysis.\_\_init\_\_

    \end{boxedminipage}

    \label{nMOLDYN:Analysis:Dynamics:AutoRegressiveAnalysis:initialize}
    \index{nMOLDYN \textit{(package)}!nMOLDYN.Analysis \textit{(package)}!nMOLDYN.Analysis.Dynamics \textit{(module)}!nMOLDYN.Analysis.Dynamics.AutoRegressiveAnalysis \textit{(class)}!nMOLDYN.Analysis.Dynamics.AutoRegressiveAnalysis.initialize \textit{(method)}}

    \vspace{0.5ex}

\hspace{.8\funcindent}\begin{boxedminipage}{\funcwidth}

    \raggedright \textbf{initialize}(\textit{self})

    \vspace{-1.5ex}

    \rule{\textwidth}{0.5\fboxrule}
\setlength{\parskip}{2ex}
    Initializes the analysis (e.g. parses and checks input parameters, set 
    some variables ...).

\setlength{\parskip}{1ex}
    \end{boxedminipage}

    \label{nMOLDYN:Analysis:Dynamics:AutoRegressiveAnalysis:calc}
    \index{nMOLDYN \textit{(package)}!nMOLDYN.Analysis \textit{(package)}!nMOLDYN.Analysis.Dynamics \textit{(module)}!nMOLDYN.Analysis.Dynamics.AutoRegressiveAnalysis \textit{(class)}!nMOLDYN.Analysis.Dynamics.AutoRegressiveAnalysis.calc \textit{(method)}}

    \vspace{0.5ex}

\hspace{.8\funcindent}\begin{boxedminipage}{\funcwidth}

    \raggedright \textbf{calc}(\textit{self}, \textit{atom}, \textit{trajname})

    \vspace{-1.5ex}

    \rule{\textwidth}{0.5\fboxrule}
\setlength{\parskip}{2ex}
    Calculates the atomic term.

\setlength{\parskip}{1ex}
      \textbf{Parameters}
      \vspace{-1ex}

      \begin{quote}
        \begin{Ventry}{xxxxxxxx}

          \item[atom]

          the atom on which the atomic term has been calculated.

            {\it (type=an instance of MMTK.Atom class.)}

          \item[trajname]

          the name of the trajectory file name.

            {\it (type=string)}

        \end{Ventry}

      \end{quote}

    \end{boxedminipage}

    \label{nMOLDYN:Analysis:Dynamics:AutoRegressiveAnalysis:combine}
    \index{nMOLDYN \textit{(package)}!nMOLDYN.Analysis \textit{(package)}!nMOLDYN.Analysis.Dynamics \textit{(module)}!nMOLDYN.Analysis.Dynamics.AutoRegressiveAnalysis \textit{(class)}!nMOLDYN.Analysis.Dynamics.AutoRegressiveAnalysis.combine \textit{(method)}}

    \vspace{0.5ex}

\hspace{.8\funcindent}\begin{boxedminipage}{\funcwidth}

    \raggedright \textbf{combine}(\textit{self}, \textit{atom}, \textit{x})

\setlength{\parskip}{2ex}
\setlength{\parskip}{1ex}
    \end{boxedminipage}

    \label{nMOLDYN:Analysis:Dynamics:AutoRegressiveAnalysis:finalize}
    \index{nMOLDYN \textit{(package)}!nMOLDYN.Analysis \textit{(package)}!nMOLDYN.Analysis.Dynamics \textit{(module)}!nMOLDYN.Analysis.Dynamics.AutoRegressiveAnalysis \textit{(class)}!nMOLDYN.Analysis.Dynamics.AutoRegressiveAnalysis.finalize \textit{(method)}}

    \vspace{0.5ex}

\hspace{.8\funcindent}\begin{boxedminipage}{\funcwidth}

    \raggedright \textbf{finalize}(\textit{self})

    \vspace{-1.5ex}

    \rule{\textwidth}{0.5\fboxrule}
\setlength{\parskip}{2ex}
    Finalizes the calculations (e.g. averaging the total term, output files
    creations ...)

\setlength{\parskip}{1ex}
    \end{boxedminipage}


\large{\textbf{\textit{Inherited from nMOLDYN.Analysis.Analysis.Analysis\textit{(Section \ref{nMOLDYN:Analysis:Analysis:Analysis})}}}}

\begin{quote}
analysisTime(), buildJobInfo(), buildTimeInfo(), deuterationSelection(), groupSelection(), parseInputParameters(), preLoadTrajectory(), runAnalysis(), saveAnalysis(), setInputParameters(), subsetSelection(), updateJobProgress(), weightingScheme()
\end{quote}

%%%%%%%%%%%%%%%%%%%%%%%%%%%%%%%%%%%%%%%%%%%%%%%%%%%%%%%%%%%%%%%%%%%%%%%%%%%
%%                            Class Variables                            %%
%%%%%%%%%%%%%%%%%%%%%%%%%%%%%%%%%%%%%%%%%%%%%%%%%%%%%%%%%%%%%%%%%%%%%%%%%%%

  \subsubsection{Class Variables}

    \vspace{-1cm}
\hspace{\varindent}\begin{longtable}{|p{\varnamewidth}|p{\vardescrwidth}|l}
\cline{1-2}
\cline{1-2} \centering \textbf{Name} & \centering \textbf{Description}& \\
\cline{1-2}
\endhead\cline{1-2}\multicolumn{3}{r}{\small\textit{continued on next page}}\\\endfoot\cline{1-2}
\endlastfoot\raggedright i\-n\-p\-u\-t\-P\-a\-r\-a\-m\-e\-t\-e\-r\-s\-N\-a\-m\-e\-s\- & \raggedright \textbf{Value:} 
{\tt 'trajectory', 'timeinfo', 'differentiation', 'projection'\texttt{...}}&\\
\cline{1-2}
\raggedright s\-h\-o\-r\-t\-N\-a\-m\-e\- & \raggedright \textbf{Value:} 
{\tt 'ARA'}&\\
\cline{1-2}
\raggedright c\-a\-n\-B\-e\-E\-s\-t\-i\-m\-a\-t\-e\-d\- & \raggedright \textbf{Value:} 
{\tt True}&\\
\cline{1-2}
\end{longtable}

    \index{nMOLDYN \textit{(package)}!nMOLDYN.Analysis \textit{(package)}!nMOLDYN.Analysis.Dynamics \textit{(module)}!nMOLDYN.Analysis.Dynamics.AutoRegressiveAnalysis \textit{(class)}|)}

%%%%%%%%%%%%%%%%%%%%%%%%%%%%%%%%%%%%%%%%%%%%%%%%%%%%%%%%%%%%%%%%%%%%%%%%%%%
%%                           Class Description                           %%
%%%%%%%%%%%%%%%%%%%%%%%%%%%%%%%%%%%%%%%%%%%%%%%%%%%%%%%%%%%%%%%%%%%%%%%%%%%

    \index{nMOLDYN \textit{(package)}!nMOLDYN.Analysis \textit{(package)}!nMOLDYN.Analysis.Dynamics \textit{(module)}!nMOLDYN.Analysis.Dynamics.PassBandFilteredTrajectory \textit{(class)}|(}
\subsection{Class PassBandFilteredTrajectory}

    \label{nMOLDYN:Analysis:Dynamics:PassBandFilteredTrajectory}
\begin{tabular}{cccccc}
% Line for nMOLDYN.Analysis.Analysis.Analysis, linespec=[False]
\multicolumn{2}{r}{\settowidth{\BCL}{nMOLDYN.Analysis.Analysis.Analysis}\multirow{2}{\BCL}{nMOLDYN.Analysis.Analysis.Analysis}}
&&
  \\\cline{3-3}
  &&\multicolumn{1}{c|}{}
&&
  \\
&&\multicolumn{2}{l}{\textbf{nMOLDYN.Analysis.Dynamics.PassBandFilteredTrajectory}}
\end{tabular}

\begin{alltt}
Sets up a Pass-Band Trajectory Filter analysis.

A Subclass of nMOLDYN.Analysis.Analysis. 

Constructor: PassBandFilteredTrajectory({\textbar}parameters{\textbar} = None)

Arguments:

    - {\textbar}parameters{\textbar} -- a dictionnary of the input parameters, or 'None' to set up the analysis without parameters.
        * trajectory -- a trajectory file name or an instance of MMTK.Trajectory.Trajectory class.
        * timeinfo   -- a string of the form 'first:last:step' where 'first' is an integer specifying the first frame 
                        number to consider, 'last' is an integer specifying the last frame number to consider and 
                        'step' is an integer specifying the step number between two frames.
        * filter     -- a string of the form 'low:high' where 'low' and 'high' are floats specifying respectively 
                        the lower and the upper bounds of the pass-band filter.
        * subset     -- a selection string specifying the atoms to consider for the analysis.
        * pbft       -- the output NetCDF file name.
        * pyroserver -- a string specifying if Pyro will be used and how to run the analysis.
    
Running modes:

    - To run the analysis do: a.runAnalysis() where a is the analysis object.
    - To estimate the analysis do: a.estimateAnalysis() where a is the analysis object.
    - To save the analysis to 'file' file name do: a.saveAnalysis(file) where a is the analysis object.
\end{alltt}


%%%%%%%%%%%%%%%%%%%%%%%%%%%%%%%%%%%%%%%%%%%%%%%%%%%%%%%%%%%%%%%%%%%%%%%%%%%
%%                                Methods                                %%
%%%%%%%%%%%%%%%%%%%%%%%%%%%%%%%%%%%%%%%%%%%%%%%%%%%%%%%%%%%%%%%%%%%%%%%%%%%

  \subsubsection{Methods}

    \vspace{0.5ex}

\hspace{.8\funcindent}\begin{boxedminipage}{\funcwidth}

    \raggedright \textbf{\_\_init\_\_}(\textit{self})

    \vspace{-1.5ex}

    \rule{\textwidth}{0.5\fboxrule}
\setlength{\parskip}{2ex}
    The constructor. Insures that the class can not be instanciated 
    directly from here.

\setlength{\parskip}{1ex}
      \textbf{Parameters}
      \vspace{-1ex}

      \begin{quote}
        \begin{Ventry}{xxxxxxxxxx}

          \item[parameters]

          a dictionnary that contains parameters of the selected analysis.

          \item[statusBar]

          if not None, an instance of nMOLDYN.GUI.Widgets.StatusBar. Will 
          attach a status bar to the selected analysis.

        \end{Ventry}

      \end{quote}

      Overrides: nMOLDYN.Analysis.Analysis.Analysis.\_\_init\_\_

    \end{boxedminipage}

    \label{nMOLDYN:Analysis:Dynamics:PassBandFilteredTrajectory:initialize}
    \index{nMOLDYN \textit{(package)}!nMOLDYN.Analysis \textit{(package)}!nMOLDYN.Analysis.Dynamics \textit{(module)}!nMOLDYN.Analysis.Dynamics.PassBandFilteredTrajectory \textit{(class)}!nMOLDYN.Analysis.Dynamics.PassBandFilteredTrajectory.initialize \textit{(method)}}

    \vspace{0.5ex}

\hspace{.8\funcindent}\begin{boxedminipage}{\funcwidth}

    \raggedright \textbf{initialize}(\textit{self})

    \vspace{-1.5ex}

    \rule{\textwidth}{0.5\fboxrule}
\setlength{\parskip}{2ex}
    Initializes the analysis (e.g. parses and checks input parameters, set 
    some variables ...).

\setlength{\parskip}{1ex}
    \end{boxedminipage}

    \label{nMOLDYN:Analysis:Dynamics:PassBandFilteredTrajectory:calc}
    \index{nMOLDYN \textit{(package)}!nMOLDYN.Analysis \textit{(package)}!nMOLDYN.Analysis.Dynamics \textit{(module)}!nMOLDYN.Analysis.Dynamics.PassBandFilteredTrajectory \textit{(class)}!nMOLDYN.Analysis.Dynamics.PassBandFilteredTrajectory.calc \textit{(method)}}

    \vspace{0.5ex}

\hspace{.8\funcindent}\begin{boxedminipage}{\funcwidth}

    \raggedright \textbf{calc}(\textit{self}, \textit{atom}, \textit{trajname})

    \vspace{-1.5ex}

    \rule{\textwidth}{0.5\fboxrule}
\setlength{\parskip}{2ex}
    Calculates the atomic term.

\setlength{\parskip}{1ex}
      \textbf{Parameters}
      \vspace{-1ex}

      \begin{quote}
        \begin{Ventry}{xxxxxxxx}

          \item[atom]

          the atom on which the atomic term has been calculated.

            {\it (type=an instance of MMTK.Atom class.)}

          \item[trajname]

          the name of the trajectory file name.

            {\it (type=string)}

        \end{Ventry}

      \end{quote}

    \end{boxedminipage}

    \label{nMOLDYN:Analysis:Dynamics:PassBandFilteredTrajectory:combine}
    \index{nMOLDYN \textit{(package)}!nMOLDYN.Analysis \textit{(package)}!nMOLDYN.Analysis.Dynamics \textit{(module)}!nMOLDYN.Analysis.Dynamics.PassBandFilteredTrajectory \textit{(class)}!nMOLDYN.Analysis.Dynamics.PassBandFilteredTrajectory.combine \textit{(method)}}

    \vspace{0.5ex}

\hspace{.8\funcindent}\begin{boxedminipage}{\funcwidth}

    \raggedright \textbf{combine}(\textit{self}, \textit{atom}, \textit{x})

\setlength{\parskip}{2ex}
\setlength{\parskip}{1ex}
    \end{boxedminipage}

    \label{nMOLDYN:Analysis:Dynamics:PassBandFilteredTrajectory:finalize}
    \index{nMOLDYN \textit{(package)}!nMOLDYN.Analysis \textit{(package)}!nMOLDYN.Analysis.Dynamics \textit{(module)}!nMOLDYN.Analysis.Dynamics.PassBandFilteredTrajectory \textit{(class)}!nMOLDYN.Analysis.Dynamics.PassBandFilteredTrajectory.finalize \textit{(method)}}

    \vspace{0.5ex}

\hspace{.8\funcindent}\begin{boxedminipage}{\funcwidth}

    \raggedright \textbf{finalize}(\textit{self})

    \vspace{-1.5ex}

    \rule{\textwidth}{0.5\fboxrule}
\setlength{\parskip}{2ex}
    Finalizes the calculations (e.g. averaging the total term, output files
    creations ...)

\setlength{\parskip}{1ex}
    \end{boxedminipage}


\large{\textbf{\textit{Inherited from nMOLDYN.Analysis.Analysis.Analysis\textit{(Section \ref{nMOLDYN:Analysis:Analysis:Analysis})}}}}

\begin{quote}
analysisTime(), buildJobInfo(), buildTimeInfo(), deuterationSelection(), groupSelection(), parseInputParameters(), preLoadTrajectory(), runAnalysis(), saveAnalysis(), setInputParameters(), subsetSelection(), updateJobProgress(), weightingScheme()
\end{quote}

%%%%%%%%%%%%%%%%%%%%%%%%%%%%%%%%%%%%%%%%%%%%%%%%%%%%%%%%%%%%%%%%%%%%%%%%%%%
%%                            Class Variables                            %%
%%%%%%%%%%%%%%%%%%%%%%%%%%%%%%%%%%%%%%%%%%%%%%%%%%%%%%%%%%%%%%%%%%%%%%%%%%%

  \subsubsection{Class Variables}

    \vspace{-1cm}
\hspace{\varindent}\begin{longtable}{|p{\varnamewidth}|p{\vardescrwidth}|l}
\cline{1-2}
\cline{1-2} \centering \textbf{Name} & \centering \textbf{Description}& \\
\cline{1-2}
\endhead\cline{1-2}\multicolumn{3}{r}{\small\textit{continued on next page}}\\\endfoot\cline{1-2}
\endlastfoot\raggedright i\-n\-p\-u\-t\-P\-a\-r\-a\-m\-e\-t\-e\-r\-s\-N\-a\-m\-e\-s\- & \raggedright \textbf{Value:} 
{\tt 'trajectory', 'timeinfo', 'filter', 'subset', 'pbft', 'py\texttt{...}}&\\
\cline{1-2}
\raggedright s\-h\-o\-r\-t\-N\-a\-m\-e\- & \raggedright \textbf{Value:} 
{\tt 'PBFT'}&\\
\cline{1-2}
\raggedright c\-a\-n\-B\-e\-E\-s\-t\-i\-m\-a\-t\-e\-d\- & \raggedright \textbf{Value:} 
{\tt True}&\\
\cline{1-2}
\end{longtable}

    \index{nMOLDYN \textit{(package)}!nMOLDYN.Analysis \textit{(package)}!nMOLDYN.Analysis.Dynamics \textit{(module)}!nMOLDYN.Analysis.Dynamics.PassBandFilteredTrajectory \textit{(class)}|)}

%%%%%%%%%%%%%%%%%%%%%%%%%%%%%%%%%%%%%%%%%%%%%%%%%%%%%%%%%%%%%%%%%%%%%%%%%%%
%%                           Class Description                           %%
%%%%%%%%%%%%%%%%%%%%%%%%%%%%%%%%%%%%%%%%%%%%%%%%%%%%%%%%%%%%%%%%%%%%%%%%%%%

    \index{nMOLDYN \textit{(package)}!nMOLDYN.Analysis \textit{(package)}!nMOLDYN.Analysis.Dynamics \textit{(module)}!nMOLDYN.Analysis.Dynamics.RadiusOfGyration \textit{(class)}|(}
\subsection{Class RadiusOfGyration}

    \label{nMOLDYN:Analysis:Dynamics:RadiusOfGyration}
\begin{tabular}{cccccc}
% Line for nMOLDYN.Analysis.Analysis.Analysis, linespec=[False]
\multicolumn{2}{r}{\settowidth{\BCL}{nMOLDYN.Analysis.Analysis.Analysis}\multirow{2}{\BCL}{nMOLDYN.Analysis.Analysis.Analysis}}
&&
  \\\cline{3-3}
  &&\multicolumn{1}{c|}{}
&&
  \\
&&\multicolumn{2}{l}{\textbf{nMOLDYN.Analysis.Dynamics.RadiusOfGyration}}
\end{tabular}

\begin{alltt}
Sets up a Radius Of Gyration analysis.

A Subclass of nMOLDYN.Analysis.Analysis. 

Constructor: RadiusOfGyration({\textbar}parameters{\textbar} = None)

Arguments:

    - {\textbar}parameters{\textbar} -- a dictionnary of the input parameters, or 'None' to set up the analysis without parameters.
        * trajectory     -- a trajectory file name or an instance of MMTK.Trajectory.Trajectory class.
        * timeinfo       -- a string of the form 'first:last:step' where 'first' is an integer specifying the first frame 
                            number to consider, 'last' is an integer specifying the last frame number to consider and 
                            'step' is an integer specifying the step number between two frames.
        * subset         -- a selection string specifying the atoms to consider for the analysis.
        * rog            -- the output NetCDF file name. A CDL version of this file will also be generated with the '.cdl' extension
                            instead of the '.nc' extension.
        * pyroserver     -- a string specifying if Pyro will be used and how to run the analysis.
    
Running modes:

    - To run the analysis do: a.runAnalysis() where a is the analysis object.
    - To estimate the analysis do: a.estimateAnalysis() where a is the analysis object.
    - To save the analysis to 'file' file name do: a.saveAnalysis(file) where a is the analysis object.
\end{alltt}


%%%%%%%%%%%%%%%%%%%%%%%%%%%%%%%%%%%%%%%%%%%%%%%%%%%%%%%%%%%%%%%%%%%%%%%%%%%
%%                                Methods                                %%
%%%%%%%%%%%%%%%%%%%%%%%%%%%%%%%%%%%%%%%%%%%%%%%%%%%%%%%%%%%%%%%%%%%%%%%%%%%

  \subsubsection{Methods}

    \vspace{0.5ex}

\hspace{.8\funcindent}\begin{boxedminipage}{\funcwidth}

    \raggedright \textbf{\_\_init\_\_}(\textit{self})

    \vspace{-1.5ex}

    \rule{\textwidth}{0.5\fboxrule}
\setlength{\parskip}{2ex}
    The constructor. Insures that the class can not be instanciated 
    directly from here.

\setlength{\parskip}{1ex}
      \textbf{Parameters}
      \vspace{-1ex}

      \begin{quote}
        \begin{Ventry}{xxxxxxxxxx}

          \item[parameters]

          a dictionnary that contains parameters of the selected analysis.

          \item[statusBar]

          if not None, an instance of nMOLDYN.GUI.Widgets.StatusBar. Will 
          attach a status bar to the selected analysis.

        \end{Ventry}

      \end{quote}

      Overrides: nMOLDYN.Analysis.Analysis.Analysis.\_\_init\_\_

    \end{boxedminipage}

    \label{nMOLDYN:Analysis:Dynamics:RadiusOfGyration:initialize}
    \index{nMOLDYN \textit{(package)}!nMOLDYN.Analysis \textit{(package)}!nMOLDYN.Analysis.Dynamics \textit{(module)}!nMOLDYN.Analysis.Dynamics.RadiusOfGyration \textit{(class)}!nMOLDYN.Analysis.Dynamics.RadiusOfGyration.initialize \textit{(method)}}

    \vspace{0.5ex}

\hspace{.8\funcindent}\begin{boxedminipage}{\funcwidth}

    \raggedright \textbf{initialize}(\textit{self})

    \vspace{-1.5ex}

    \rule{\textwidth}{0.5\fboxrule}
\setlength{\parskip}{2ex}
    Initializes the analysis (e.g. parses and checks input parameters, set 
    some variables ...).

\setlength{\parskip}{1ex}
    \end{boxedminipage}

    \label{nMOLDYN:Analysis:Dynamics:RadiusOfGyration:calc}
    \index{nMOLDYN \textit{(package)}!nMOLDYN.Analysis \textit{(package)}!nMOLDYN.Analysis.Dynamics \textit{(module)}!nMOLDYN.Analysis.Dynamics.RadiusOfGyration \textit{(class)}!nMOLDYN.Analysis.Dynamics.RadiusOfGyration.calc \textit{(method)}}

    \vspace{0.5ex}

\hspace{.8\funcindent}\begin{boxedminipage}{\funcwidth}

    \raggedright \textbf{calc}(\textit{self}, \textit{atom}, \textit{trajname})

    \vspace{-1.5ex}

    \rule{\textwidth}{0.5\fboxrule}
\setlength{\parskip}{2ex}
    Calculates the atomic term.

\setlength{\parskip}{1ex}
      \textbf{Parameters}
      \vspace{-1ex}

      \begin{quote}
        \begin{Ventry}{xxxxxxxx}

          \item[atom]

          the atom on which the atomic term has been calculated.

            {\it (type=an instance of MMTK.Atom class.)}

          \item[trajname]

          the name of the trajectory file name.

            {\it (type=string)}

        \end{Ventry}

      \end{quote}

    \end{boxedminipage}

    \label{nMOLDYN:Analysis:Dynamics:RadiusOfGyration:combine}
    \index{nMOLDYN \textit{(package)}!nMOLDYN.Analysis \textit{(package)}!nMOLDYN.Analysis.Dynamics \textit{(module)}!nMOLDYN.Analysis.Dynamics.RadiusOfGyration \textit{(class)}!nMOLDYN.Analysis.Dynamics.RadiusOfGyration.combine \textit{(method)}}

    \vspace{0.5ex}

\hspace{.8\funcindent}\begin{boxedminipage}{\funcwidth}

    \raggedright \textbf{combine}(\textit{self}, \textit{atom}, \textit{x})

\setlength{\parskip}{2ex}
\setlength{\parskip}{1ex}
    \end{boxedminipage}

    \label{nMOLDYN:Analysis:Dynamics:RadiusOfGyration:finalize}
    \index{nMOLDYN \textit{(package)}!nMOLDYN.Analysis \textit{(package)}!nMOLDYN.Analysis.Dynamics \textit{(module)}!nMOLDYN.Analysis.Dynamics.RadiusOfGyration \textit{(class)}!nMOLDYN.Analysis.Dynamics.RadiusOfGyration.finalize \textit{(method)}}

    \vspace{0.5ex}

\hspace{.8\funcindent}\begin{boxedminipage}{\funcwidth}

    \raggedright \textbf{finalize}(\textit{self})

    \vspace{-1.5ex}

    \rule{\textwidth}{0.5\fboxrule}
\setlength{\parskip}{2ex}
    Finalizes the calculations (e.g. averaging the total term, output files
    creations ...).

\setlength{\parskip}{1ex}
    \end{boxedminipage}


\large{\textbf{\textit{Inherited from nMOLDYN.Analysis.Analysis.Analysis\textit{(Section \ref{nMOLDYN:Analysis:Analysis:Analysis})}}}}

\begin{quote}
analysisTime(), buildJobInfo(), buildTimeInfo(), deuterationSelection(), groupSelection(), parseInputParameters(), preLoadTrajectory(), runAnalysis(), saveAnalysis(), setInputParameters(), subsetSelection(), updateJobProgress(), weightingScheme()
\end{quote}

%%%%%%%%%%%%%%%%%%%%%%%%%%%%%%%%%%%%%%%%%%%%%%%%%%%%%%%%%%%%%%%%%%%%%%%%%%%
%%                            Class Variables                            %%
%%%%%%%%%%%%%%%%%%%%%%%%%%%%%%%%%%%%%%%%%%%%%%%%%%%%%%%%%%%%%%%%%%%%%%%%%%%

  \subsubsection{Class Variables}

    \vspace{-1cm}
\hspace{\varindent}\begin{longtable}{|p{\varnamewidth}|p{\vardescrwidth}|l}
\cline{1-2}
\cline{1-2} \centering \textbf{Name} & \centering \textbf{Description}& \\
\cline{1-2}
\endhead\cline{1-2}\multicolumn{3}{r}{\small\textit{continued on next page}}\\\endfoot\cline{1-2}
\endlastfoot\raggedright i\-n\-p\-u\-t\-P\-a\-r\-a\-m\-e\-t\-e\-r\-s\-N\-a\-m\-e\-s\- & \raggedright \textbf{Value:} 
{\tt 'trajectory', 'timeinfo', 'subset', 'rog', 'pyroserver',}&\\
\cline{1-2}
\raggedright s\-h\-o\-r\-t\-N\-a\-m\-e\- & \raggedright \textbf{Value:} 
{\tt 'ROG'}&\\
\cline{1-2}
\raggedright c\-a\-n\-B\-e\-E\-s\-t\-i\-m\-a\-t\-e\-d\- & \raggedright \textbf{Value:} 
{\tt True}&\\
\cline{1-2}
\end{longtable}

    \index{nMOLDYN \textit{(package)}!nMOLDYN.Analysis \textit{(package)}!nMOLDYN.Analysis.Dynamics \textit{(module)}!nMOLDYN.Analysis.Dynamics.RadiusOfGyration \textit{(class)}|)}

%%%%%%%%%%%%%%%%%%%%%%%%%%%%%%%%%%%%%%%%%%%%%%%%%%%%%%%%%%%%%%%%%%%%%%%%%%%
%%                           Class Description                           %%
%%%%%%%%%%%%%%%%%%%%%%%%%%%%%%%%%%%%%%%%%%%%%%%%%%%%%%%%%%%%%%%%%%%%%%%%%%%

    \index{nMOLDYN \textit{(package)}!nMOLDYN.Analysis \textit{(package)}!nMOLDYN.Analysis.Dynamics \textit{(module)}!nMOLDYN.Analysis.Dynamics.GlobalMotionFilteredTrajectory \textit{(class)}|(}
\subsection{Class GlobalMotionFilteredTrajectory}

    \label{nMOLDYN:Analysis:Dynamics:GlobalMotionFilteredTrajectory}
\begin{tabular}{cccccc}
% Line for nMOLDYN.Analysis.Analysis.Analysis, linespec=[False]
\multicolumn{2}{r}{\settowidth{\BCL}{nMOLDYN.Analysis.Analysis.Analysis}\multirow{2}{\BCL}{nMOLDYN.Analysis.Analysis.Analysis}}
&&
  \\\cline{3-3}
  &&\multicolumn{1}{c|}{}
&&
  \\
&&\multicolumn{2}{l}{\textbf{nMOLDYN.Analysis.Dynamics.GlobalMotionFilteredTrajectory}}
\end{tabular}

\begin{alltt}
Sets up a Global Motion Trajectory Filter analysis.

A Subclass of nMOLDYN.Analysis.Analysis. 

Constructor: GlobalMotionFilteredTrajectory({\textbar}parameters{\textbar} = None)

Arguments:

    - {\textbar}parameters{\textbar} -- a dictionnary of the input parameters, or 'None' to set up the analysis without parameters.
        * trajectory -- a trajectory file name or an instance of MMTK.Trajectory.Trajectory class.
        * timeinfo   -- a string of the form 'first:last:step' where 'first' is an integer specifying the first frame 
                        number to consider, 'last' is an integer specifying the last frame number to consider and 
                        'step' is an integer specifying the step number between two frames.
        * subset     -- a selection string specifying the atoms to consider for the analysis.
        * gmft       -- the output NetCDF file name.
        * pyroserver -- a string specifying if Pyro will be used and how to run the analysis.
    
Running modes:

    - To run the analysis do: a.runAnalysis() where a is the analysis object.
    - To estimate the analysis do: a.estimateAnalysis() where a is the analysis object.
    - To save the analysis to 'file' file name do: a.saveAnalysis(file) where a is the analysis object.
\end{alltt}


%%%%%%%%%%%%%%%%%%%%%%%%%%%%%%%%%%%%%%%%%%%%%%%%%%%%%%%%%%%%%%%%%%%%%%%%%%%
%%                                Methods                                %%
%%%%%%%%%%%%%%%%%%%%%%%%%%%%%%%%%%%%%%%%%%%%%%%%%%%%%%%%%%%%%%%%%%%%%%%%%%%

  \subsubsection{Methods}

    \vspace{0.5ex}

\hspace{.8\funcindent}\begin{boxedminipage}{\funcwidth}

    \raggedright \textbf{\_\_init\_\_}(\textit{self})

    \vspace{-1.5ex}

    \rule{\textwidth}{0.5\fboxrule}
\setlength{\parskip}{2ex}
    The constructor. Insures that the class can not be instanciated 
    directly from here.

\setlength{\parskip}{1ex}
      \textbf{Parameters}
      \vspace{-1ex}

      \begin{quote}
        \begin{Ventry}{xxxxxxxxxx}

          \item[parameters]

          a dictionnary that contains parameters of the selected analysis.

          \item[statusBar]

          if not None, an instance of nMOLDYN.GUI.Widgets.StatusBar. Will 
          attach a status bar to the selected analysis.

        \end{Ventry}

      \end{quote}

      Overrides: nMOLDYN.Analysis.Analysis.Analysis.\_\_init\_\_

    \end{boxedminipage}

    \label{nMOLDYN:Analysis:Dynamics:GlobalMotionFilteredTrajectory:initialize}
    \index{nMOLDYN \textit{(package)}!nMOLDYN.Analysis \textit{(package)}!nMOLDYN.Analysis.Dynamics \textit{(module)}!nMOLDYN.Analysis.Dynamics.GlobalMotionFilteredTrajectory \textit{(class)}!nMOLDYN.Analysis.Dynamics.GlobalMotionFilteredTrajectory.initialize \textit{(method)}}

    \vspace{0.5ex}

\hspace{.8\funcindent}\begin{boxedminipage}{\funcwidth}

    \raggedright \textbf{initialize}(\textit{self})

    \vspace{-1.5ex}

    \rule{\textwidth}{0.5\fboxrule}
\setlength{\parskip}{2ex}
    Initializes the analysis (e.g. parses and checks input parameters, set 
    some variables ...).

\setlength{\parskip}{1ex}
    \end{boxedminipage}

    \label{nMOLDYN:Analysis:Dynamics:GlobalMotionFilteredTrajectory:calc}
    \index{nMOLDYN \textit{(package)}!nMOLDYN.Analysis \textit{(package)}!nMOLDYN.Analysis.Dynamics \textit{(module)}!nMOLDYN.Analysis.Dynamics.GlobalMotionFilteredTrajectory \textit{(class)}!nMOLDYN.Analysis.Dynamics.GlobalMotionFilteredTrajectory.calc \textit{(method)}}

    \vspace{0.5ex}

\hspace{.8\funcindent}\begin{boxedminipage}{\funcwidth}

    \raggedright \textbf{calc}(\textit{self}, \textit{frameIndex}, \textit{trajname})

    \vspace{-1.5ex}

    \rule{\textwidth}{0.5\fboxrule}
\setlength{\parskip}{2ex}
    Calculates the contribution for one frame.

\setlength{\parskip}{1ex}
      \textbf{Parameters}
      \vspace{-1ex}

      \begin{quote}
        \begin{Ventry}{xxxxxxxxxx}

          \item[frameIndex]

          the index of the frame in {\textbar}self.frameIndexes{\textbar} 
          array.

            {\it (type=integer.)}

          \item[trajname]

          the name of the trajectory file name.

            {\it (type=string)}

        \end{Ventry}

      \end{quote}

    \end{boxedminipage}

    \label{nMOLDYN:Analysis:Dynamics:GlobalMotionFilteredTrajectory:combine}
    \index{nMOLDYN \textit{(package)}!nMOLDYN.Analysis \textit{(package)}!nMOLDYN.Analysis.Dynamics \textit{(module)}!nMOLDYN.Analysis.Dynamics.GlobalMotionFilteredTrajectory \textit{(class)}!nMOLDYN.Analysis.Dynamics.GlobalMotionFilteredTrajectory.combine \textit{(method)}}

    \vspace{0.5ex}

\hspace{.8\funcindent}\begin{boxedminipage}{\funcwidth}

    \raggedright \textbf{combine}(\textit{self}, \textit{frameIndex}, \textit{x})

\setlength{\parskip}{2ex}
\setlength{\parskip}{1ex}
    \end{boxedminipage}

    \label{nMOLDYN:Analysis:Dynamics:GlobalMotionFilteredTrajectory:finalize}
    \index{nMOLDYN \textit{(package)}!nMOLDYN.Analysis \textit{(package)}!nMOLDYN.Analysis.Dynamics \textit{(module)}!nMOLDYN.Analysis.Dynamics.GlobalMotionFilteredTrajectory \textit{(class)}!nMOLDYN.Analysis.Dynamics.GlobalMotionFilteredTrajectory.finalize \textit{(method)}}

    \vspace{0.5ex}

\hspace{.8\funcindent}\begin{boxedminipage}{\funcwidth}

    \raggedright \textbf{finalize}(\textit{self})

    \vspace{-1.5ex}

    \rule{\textwidth}{0.5\fboxrule}
\setlength{\parskip}{2ex}
    Finalizes the calculations (e.g. averaging the total term, output files
    creations ...).

\setlength{\parskip}{1ex}
    \end{boxedminipage}


\large{\textbf{\textit{Inherited from nMOLDYN.Analysis.Analysis.Analysis\textit{(Section \ref{nMOLDYN:Analysis:Analysis:Analysis})}}}}

\begin{quote}
analysisTime(), buildJobInfo(), buildTimeInfo(), deuterationSelection(), groupSelection(), parseInputParameters(), preLoadTrajectory(), runAnalysis(), saveAnalysis(), setInputParameters(), subsetSelection(), updateJobProgress(), weightingScheme()
\end{quote}

%%%%%%%%%%%%%%%%%%%%%%%%%%%%%%%%%%%%%%%%%%%%%%%%%%%%%%%%%%%%%%%%%%%%%%%%%%%
%%                            Class Variables                            %%
%%%%%%%%%%%%%%%%%%%%%%%%%%%%%%%%%%%%%%%%%%%%%%%%%%%%%%%%%%%%%%%%%%%%%%%%%%%

  \subsubsection{Class Variables}

    \vspace{-1cm}
\hspace{\varindent}\begin{longtable}{|p{\varnamewidth}|p{\vardescrwidth}|l}
\cline{1-2}
\cline{1-2} \centering \textbf{Name} & \centering \textbf{Description}& \\
\cline{1-2}
\endhead\cline{1-2}\multicolumn{3}{r}{\small\textit{continued on next page}}\\\endfoot\cline{1-2}
\endlastfoot\raggedright i\-n\-p\-u\-t\-P\-a\-r\-a\-m\-e\-t\-e\-r\-s\-N\-a\-m\-e\-s\- & \raggedright \textbf{Value:} 
{\tt 'trajectory', 'timeinfo', 'subset', 'gmft', 'pyroserver',}&\\
\cline{1-2}
\raggedright s\-h\-o\-r\-t\-N\-a\-m\-e\- & \raggedright \textbf{Value:} 
{\tt 'GMFT'}&\\
\cline{1-2}
\raggedright c\-a\-n\-B\-e\-E\-s\-t\-i\-m\-a\-t\-e\-d\- & \raggedright \textbf{Value:} 
{\tt True}&\\
\cline{1-2}
\end{longtable}

    \index{nMOLDYN \textit{(package)}!nMOLDYN.Analysis \textit{(package)}!nMOLDYN.Analysis.Dynamics \textit{(module)}!nMOLDYN.Analysis.Dynamics.GlobalMotionFilteredTrajectory \textit{(class)}|)}

%%%%%%%%%%%%%%%%%%%%%%%%%%%%%%%%%%%%%%%%%%%%%%%%%%%%%%%%%%%%%%%%%%%%%%%%%%%
%%                           Class Description                           %%
%%%%%%%%%%%%%%%%%%%%%%%%%%%%%%%%%%%%%%%%%%%%%%%%%%%%%%%%%%%%%%%%%%%%%%%%%%%

    \index{nMOLDYN \textit{(package)}!nMOLDYN.Analysis \textit{(package)}!nMOLDYN.Analysis.Dynamics \textit{(module)}!nMOLDYN.Analysis.Dynamics.CenterOfMassTrajectory \textit{(class)}|(}
\subsection{Class CenterOfMassTrajectory}

    \label{nMOLDYN:Analysis:Dynamics:CenterOfMassTrajectory}
\begin{tabular}{cccccc}
% Line for nMOLDYN.Analysis.Analysis.Analysis, linespec=[False]
\multicolumn{2}{r}{\settowidth{\BCL}{nMOLDYN.Analysis.Analysis.Analysis}\multirow{2}{\BCL}{nMOLDYN.Analysis.Analysis.Analysis}}
&&
  \\\cline{3-3}
  &&\multicolumn{1}{c|}{}
&&
  \\
&&\multicolumn{2}{l}{\textbf{nMOLDYN.Analysis.Dynamics.CenterOfMassTrajectory}}
\end{tabular}

\begin{alltt}
Sets up a Center Of Mass Trajectory analysis.

A Subclass of nMOLDYN.Analysis.Analysis. 

Constructor: CenterOfMassTrajectory({\textbar}parameters{\textbar} = None)

Arguments:

    - {\textbar}parameters{\textbar} -- a dictionnary of the input parameters, or 'None' to set up the analysis without parameters.
        * trajectory -- a trajectory file name or an instance of MMTK.Trajectory.Trajectory class.
        * timeinfo   -- a string of the form 'first:last:step' where 'first' is an integer specifying the first frame 
                        number to consider, 'last' is an integer specifying the last frame number to consider and 
                        'step' is an integer specifying the step number between two frames.
        * group      -- a selection string specifying the groups of atoms on which the center of mass will be defined
                        (one center of mass per group).
        * comt       -- the output NetCDF file name.
        * pyroserver -- a string specifying if Pyro will be used and how to run the analysis.
    
Running modes:

    - To run the analysis do: a.runAnalysis() where a is the analysis object.
    - To estimate the analysis do: a.estimateAnalysis() where a is the analysis object.
    - To save the analysis to 'file' file name do: a.saveAnalysis(file) where a is the analysis object.
\end{alltt}


%%%%%%%%%%%%%%%%%%%%%%%%%%%%%%%%%%%%%%%%%%%%%%%%%%%%%%%%%%%%%%%%%%%%%%%%%%%
%%                                Methods                                %%
%%%%%%%%%%%%%%%%%%%%%%%%%%%%%%%%%%%%%%%%%%%%%%%%%%%%%%%%%%%%%%%%%%%%%%%%%%%

  \subsubsection{Methods}

    \vspace{0.5ex}

\hspace{.8\funcindent}\begin{boxedminipage}{\funcwidth}

    \raggedright \textbf{\_\_init\_\_}(\textit{self})

    \vspace{-1.5ex}

    \rule{\textwidth}{0.5\fboxrule}
\setlength{\parskip}{2ex}
    The constructor. Insures that the class can not be instanciated 
    directly from here.

\setlength{\parskip}{1ex}
      \textbf{Parameters}
      \vspace{-1ex}

      \begin{quote}
        \begin{Ventry}{xxxxxxxxxx}

          \item[parameters]

          a dictionnary that contains parameters of the selected analysis.

          \item[statusBar]

          if not None, an instance of nMOLDYN.GUI.Widgets.StatusBar. Will 
          attach a status bar to the selected analysis.

        \end{Ventry}

      \end{quote}

      Overrides: nMOLDYN.Analysis.Analysis.Analysis.\_\_init\_\_

    \end{boxedminipage}

    \label{nMOLDYN:Analysis:Dynamics:CenterOfMassTrajectory:initialize}
    \index{nMOLDYN \textit{(package)}!nMOLDYN.Analysis \textit{(package)}!nMOLDYN.Analysis.Dynamics \textit{(module)}!nMOLDYN.Analysis.Dynamics.CenterOfMassTrajectory \textit{(class)}!nMOLDYN.Analysis.Dynamics.CenterOfMassTrajectory.initialize \textit{(method)}}

    \vspace{0.5ex}

\hspace{.8\funcindent}\begin{boxedminipage}{\funcwidth}

    \raggedright \textbf{initialize}(\textit{self})

    \vspace{-1.5ex}

    \rule{\textwidth}{0.5\fboxrule}
\setlength{\parskip}{2ex}
    Initializes the analysis (e.g. parses and checks input parameters, set 
    some variables ...).

\setlength{\parskip}{1ex}
    \end{boxedminipage}

    \label{nMOLDYN:Analysis:Dynamics:CenterOfMassTrajectory:calc}
    \index{nMOLDYN \textit{(package)}!nMOLDYN.Analysis \textit{(package)}!nMOLDYN.Analysis.Dynamics \textit{(module)}!nMOLDYN.Analysis.Dynamics.CenterOfMassTrajectory \textit{(class)}!nMOLDYN.Analysis.Dynamics.CenterOfMassTrajectory.calc \textit{(method)}}

    \vspace{0.5ex}

\hspace{.8\funcindent}\begin{boxedminipage}{\funcwidth}

    \raggedright \textbf{calc}(\textit{self}, \textit{frameIndex}, \textit{trajname})

    \vspace{-1.5ex}

    \rule{\textwidth}{0.5\fboxrule}
\setlength{\parskip}{2ex}
    Calculates the contribution for one frame.

\setlength{\parskip}{1ex}
      \textbf{Parameters}
      \vspace{-1ex}

      \begin{quote}
        \begin{Ventry}{xxxxxxxxxx}

          \item[frameIndex]

          the index of the frame in {\textbar}self.frameIndexes{\textbar} 
          array.

            {\it (type=integer.)}

          \item[trajname]

          the name of the trajectory file name.

            {\it (type=string)}

        \end{Ventry}

      \end{quote}

    \end{boxedminipage}

    \label{nMOLDYN:Analysis:Dynamics:CenterOfMassTrajectory:combine}
    \index{nMOLDYN \textit{(package)}!nMOLDYN.Analysis \textit{(package)}!nMOLDYN.Analysis.Dynamics \textit{(module)}!nMOLDYN.Analysis.Dynamics.CenterOfMassTrajectory \textit{(class)}!nMOLDYN.Analysis.Dynamics.CenterOfMassTrajectory.combine \textit{(method)}}

    \vspace{0.5ex}

\hspace{.8\funcindent}\begin{boxedminipage}{\funcwidth}

    \raggedright \textbf{combine}(\textit{frameIndex}, \textit{x})

\setlength{\parskip}{2ex}
\setlength{\parskip}{1ex}
    \end{boxedminipage}

    \label{nMOLDYN:Analysis:Dynamics:CenterOfMassTrajectory:finalize}
    \index{nMOLDYN \textit{(package)}!nMOLDYN.Analysis \textit{(package)}!nMOLDYN.Analysis.Dynamics \textit{(module)}!nMOLDYN.Analysis.Dynamics.CenterOfMassTrajectory \textit{(class)}!nMOLDYN.Analysis.Dynamics.CenterOfMassTrajectory.finalize \textit{(method)}}

    \vspace{0.5ex}

\hspace{.8\funcindent}\begin{boxedminipage}{\funcwidth}

    \raggedright \textbf{finalize}(\textit{self})

    \vspace{-1.5ex}

    \rule{\textwidth}{0.5\fboxrule}
\setlength{\parskip}{2ex}
    Finalizes the calculations (e.g. averaging the total term, output files
    creations ...).

\setlength{\parskip}{1ex}
    \end{boxedminipage}


\large{\textbf{\textit{Inherited from nMOLDYN.Analysis.Analysis.Analysis\textit{(Section \ref{nMOLDYN:Analysis:Analysis:Analysis})}}}}

\begin{quote}
analysisTime(), buildJobInfo(), buildTimeInfo(), deuterationSelection(), groupSelection(), parseInputParameters(), preLoadTrajectory(), runAnalysis(), saveAnalysis(), setInputParameters(), subsetSelection(), updateJobProgress(), weightingScheme()
\end{quote}

%%%%%%%%%%%%%%%%%%%%%%%%%%%%%%%%%%%%%%%%%%%%%%%%%%%%%%%%%%%%%%%%%%%%%%%%%%%
%%                            Class Variables                            %%
%%%%%%%%%%%%%%%%%%%%%%%%%%%%%%%%%%%%%%%%%%%%%%%%%%%%%%%%%%%%%%%%%%%%%%%%%%%

  \subsubsection{Class Variables}

    \vspace{-1cm}
\hspace{\varindent}\begin{longtable}{|p{\varnamewidth}|p{\vardescrwidth}|l}
\cline{1-2}
\cline{1-2} \centering \textbf{Name} & \centering \textbf{Description}& \\
\cline{1-2}
\endhead\cline{1-2}\multicolumn{3}{r}{\small\textit{continued on next page}}\\\endfoot\cline{1-2}
\endlastfoot\raggedright i\-n\-p\-u\-t\-P\-a\-r\-a\-m\-e\-t\-e\-r\-s\-N\-a\-m\-e\-s\- & \raggedright \textbf{Value:} 
{\tt 'trajectory', 'timeinfo', 'group', 'comt', 'pyroserver',}&\\
\cline{1-2}
\raggedright s\-h\-o\-r\-t\-N\-a\-m\-e\- & \raggedright \textbf{Value:} 
{\tt 'COMT'}&\\
\cline{1-2}
\raggedright c\-a\-n\-B\-e\-E\-s\-t\-i\-m\-a\-t\-e\-d\- & \raggedright \textbf{Value:} 
{\tt True}&\\
\cline{1-2}
\end{longtable}

    \index{nMOLDYN \textit{(package)}!nMOLDYN.Analysis \textit{(package)}!nMOLDYN.Analysis.Dynamics \textit{(module)}!nMOLDYN.Analysis.Dynamics.CenterOfMassTrajectory \textit{(class)}|)}

%%%%%%%%%%%%%%%%%%%%%%%%%%%%%%%%%%%%%%%%%%%%%%%%%%%%%%%%%%%%%%%%%%%%%%%%%%%
%%                           Class Description                           %%
%%%%%%%%%%%%%%%%%%%%%%%%%%%%%%%%%%%%%%%%%%%%%%%%%%%%%%%%%%%%%%%%%%%%%%%%%%%

    \index{nMOLDYN \textit{(package)}!nMOLDYN.Analysis \textit{(package)}!nMOLDYN.Analysis.Dynamics \textit{(module)}!nMOLDYN.Analysis.Dynamics.QuasiHarmonicAnalysis \textit{(class)}|(}
\subsection{Class QuasiHarmonicAnalysis}

    \label{nMOLDYN:Analysis:Dynamics:QuasiHarmonicAnalysis}
\begin{tabular}{cccccc}
% Line for nMOLDYN.Analysis.Analysis.Analysis, linespec=[False]
\multicolumn{2}{r}{\settowidth{\BCL}{nMOLDYN.Analysis.Analysis.Analysis}\multirow{2}{\BCL}{nMOLDYN.Analysis.Analysis.Analysis}}
&&
  \\\cline{3-3}
  &&\multicolumn{1}{c|}{}
&&
  \\
&&\multicolumn{2}{l}{\textbf{nMOLDYN.Analysis.Dynamics.QuasiHarmonicAnalysis}}
\end{tabular}

\begin{alltt}
Sets up a Quasi Harmonic Analysis analysis.

A Subclass of nMOLDYN.Analysis.Analysis. 

Constructor: QuasiHarmonicAnalysis({\textbar}parameters{\textbar} = None)

Arguments:

    - {\textbar}parameters{\textbar} -- a dictionnary of the input parameters, or 'None' to set up the analysis without parameters.
        * trajectory  -- a trajectory file name or an instance of MMTK.Trajectory.Trajectory class.
        * timeinfo    -- a string of the form 'first:last:step' where 'first' is an integer specifying the first frame 
                         number to consider, 'last' is an integer specifying the last frame number to consider and 
                         'step' is an integer specifying the step number between two frames.
        * temperature -- the temperature at which the MD was performed.
        * subset      -- a selection string specifying the atoms to consider for the analysis.
        * qha         -- the output NetCDF file name.
    
Running modes:

    - To run the analysis do: a.runAnalysis() where a is the analysis object.
    - To estimate the analysis do: a.estimateAnalysis() where a is the analysis object.
    - To save the analysis to 'file' file name do: a.saveAnalysis(file) where a is the analysis object.
    
Comments:

    - This analysis is used to get effective modes of vibration from fluctuations calculated by an MD simulation. 
      The results of such an analysis can be seen by generating pseudo-trajectories reproducing the vibrations along
      a vibration mode.
    - For more details: Brooks et al., J. Comp. Chem. 1995, 16, 1522-1542.
\end{alltt}


%%%%%%%%%%%%%%%%%%%%%%%%%%%%%%%%%%%%%%%%%%%%%%%%%%%%%%%%%%%%%%%%%%%%%%%%%%%
%%                                Methods                                %%
%%%%%%%%%%%%%%%%%%%%%%%%%%%%%%%%%%%%%%%%%%%%%%%%%%%%%%%%%%%%%%%%%%%%%%%%%%%

  \subsubsection{Methods}

    \vspace{0.5ex}

\hspace{.8\funcindent}\begin{boxedminipage}{\funcwidth}

    \raggedright \textbf{\_\_init\_\_}(\textit{self})

    \vspace{-1.5ex}

    \rule{\textwidth}{0.5\fboxrule}
\setlength{\parskip}{2ex}
    The constructor. Insures that the class can not be instanciated 
    directly from here.

\setlength{\parskip}{1ex}
      \textbf{Parameters}
      \vspace{-1ex}

      \begin{quote}
        \begin{Ventry}{xxxxxxxxxx}

          \item[parameters]

          a dictionnary that contains parameters of the selected analysis.

          \item[statusBar]

          if not None, an instance of nMOLDYN.GUI.Widgets.StatusBar. Will 
          attach a status bar to the selected analysis.

        \end{Ventry}

      \end{quote}

      Overrides: nMOLDYN.Analysis.Analysis.Analysis.\_\_init\_\_

    \end{boxedminipage}

    \label{nMOLDYN:Analysis:Dynamics:QuasiHarmonicAnalysis:initialize}
    \index{nMOLDYN \textit{(package)}!nMOLDYN.Analysis \textit{(package)}!nMOLDYN.Analysis.Dynamics \textit{(module)}!nMOLDYN.Analysis.Dynamics.QuasiHarmonicAnalysis \textit{(class)}!nMOLDYN.Analysis.Dynamics.QuasiHarmonicAnalysis.initialize \textit{(method)}}

    \vspace{0.5ex}

\hspace{.8\funcindent}\begin{boxedminipage}{\funcwidth}

    \raggedright \textbf{initialize}(\textit{self})

    \vspace{-1.5ex}

    \rule{\textwidth}{0.5\fboxrule}
\setlength{\parskip}{2ex}
    Initializes the analysis (e.g. parses and checks input parameters, set 
    some variables ...).

\setlength{\parskip}{1ex}
    \end{boxedminipage}

    \label{nMOLDYN:Analysis:Dynamics:QuasiHarmonicAnalysis:internalRun}
    \index{nMOLDYN \textit{(package)}!nMOLDYN.Analysis \textit{(package)}!nMOLDYN.Analysis.Dynamics \textit{(module)}!nMOLDYN.Analysis.Dynamics.QuasiHarmonicAnalysis \textit{(class)}!nMOLDYN.Analysis.Dynamics.QuasiHarmonicAnalysis.internalRun \textit{(method)}}

    \vspace{0.5ex}

\hspace{.8\funcindent}\begin{boxedminipage}{\funcwidth}

    \raggedright \textbf{internalRun}(\textit{self})

    \vspace{-1.5ex}

    \rule{\textwidth}{0.5\fboxrule}
\setlength{\parskip}{2ex}
    Runs the analysis.

\setlength{\parskip}{1ex}
    \end{boxedminipage}


\large{\textbf{\textit{Inherited from nMOLDYN.Analysis.Analysis.Analysis\textit{(Section \ref{nMOLDYN:Analysis:Analysis:Analysis})}}}}

\begin{quote}
analysisTime(), buildJobInfo(), buildTimeInfo(), deuterationSelection(), groupSelection(), parseInputParameters(), preLoadTrajectory(), runAnalysis(), saveAnalysis(), setInputParameters(), subsetSelection(), updateJobProgress(), weightingScheme()
\end{quote}

%%%%%%%%%%%%%%%%%%%%%%%%%%%%%%%%%%%%%%%%%%%%%%%%%%%%%%%%%%%%%%%%%%%%%%%%%%%
%%                            Class Variables                            %%
%%%%%%%%%%%%%%%%%%%%%%%%%%%%%%%%%%%%%%%%%%%%%%%%%%%%%%%%%%%%%%%%%%%%%%%%%%%

  \subsubsection{Class Variables}

    \vspace{-1cm}
\hspace{\varindent}\begin{longtable}{|p{\varnamewidth}|p{\vardescrwidth}|l}
\cline{1-2}
\cline{1-2} \centering \textbf{Name} & \centering \textbf{Description}& \\
\cline{1-2}
\endhead\cline{1-2}\multicolumn{3}{r}{\small\textit{continued on next page}}\\\endfoot\cline{1-2}
\endlastfoot\raggedright i\-n\-p\-u\-t\-P\-a\-r\-a\-m\-e\-t\-e\-r\-s\-N\-a\-m\-e\-s\- & \raggedright \textbf{Value:} 
{\tt 'trajectory', 'timeinfo', 'temperature', 'subset', 'qha'}&\\
\cline{1-2}
\raggedright s\-h\-o\-r\-t\-N\-a\-m\-e\- & \raggedright \textbf{Value:} 
{\tt 'QHA'}&\\
\cline{1-2}
\raggedright c\-a\-n\-B\-e\-E\-s\-t\-i\-m\-a\-t\-e\-d\- & \raggedright \textbf{Value:} 
{\tt False}&\\
\cline{1-2}
\end{longtable}

    \index{nMOLDYN \textit{(package)}!nMOLDYN.Analysis \textit{(package)}!nMOLDYN.Analysis.Dynamics \textit{(module)}!nMOLDYN.Analysis.Dynamics.QuasiHarmonicAnalysis \textit{(class)}|)}

%%%%%%%%%%%%%%%%%%%%%%%%%%%%%%%%%%%%%%%%%%%%%%%%%%%%%%%%%%%%%%%%%%%%%%%%%%%
%%                           Class Description                           %%
%%%%%%%%%%%%%%%%%%%%%%%%%%%%%%%%%%%%%%%%%%%%%%%%%%%%%%%%%%%%%%%%%%%%%%%%%%%

    \index{nMOLDYN \textit{(package)}!nMOLDYN.Analysis \textit{(package)}!nMOLDYN.Analysis.Dynamics \textit{(module)}!nMOLDYN.Analysis.Dynamics.AngularCorrelation \textit{(class)}|(}
\subsection{Class AngularCorrelation}

    \label{nMOLDYN:Analysis:Dynamics:AngularCorrelation}
\begin{tabular}{cccccc}
% Line for nMOLDYN.Analysis.Analysis.Analysis, linespec=[False]
\multicolumn{2}{r}{\settowidth{\BCL}{nMOLDYN.Analysis.Analysis.Analysis}\multirow{2}{\BCL}{nMOLDYN.Analysis.Analysis.Analysis}}
&&
  \\\cline{3-3}
  &&\multicolumn{1}{c|}{}
&&
  \\
&&\multicolumn{2}{l}{\textbf{nMOLDYN.Analysis.Dynamics.AngularCorrelation}}
\end{tabular}

\begin{alltt}
Sets up an Angular Correlation analysis.

A Subclass of nMOLDYN.Analysis.Analysis. 

Constructor: AngularCorrelation({\textbar}parameters{\textbar} = None)

Arguments:

    - {\textbar}parameters{\textbar} -- a dictionnary of the input parameters, or 'None' to set up the analysis without parameters.
        * trajectory      -- a trajectory file name or an instance of MMTK.Trajectory.Trajectory class.
        * timeinfo        -- a string of the form 'first:last:step' where 'first' is an integer specifying the first frame 
                             number to consider, 'last' is an integer specifying the last frame number to consider and 
                             'step' is an integer specifying the step number between two frames.
        * triplet         -- a selection string specifying the groups of three atoms that will define the plane on 
                             which the angular correlation will be computed.
        * atomorder       -- a string of the form 'atom1,atom2,atom3' where 'atom1', 'atom2' and 'atom3' are 
                             respectively the MMTK atom names of the atoms in the way they should be ordered.
        * ac              -- the output NetCDF file name. A CDL version of this file will also be generated with the '.cdl' extension
                             instead of the '.nc' extension.
        * pyroserver      -- a string specifying if Pyro will be used and how to run the analysis.
    
Running modes:

    - To run the analysis do: a.runAnalysis() where a is the analysis object.
    - To estimate the analysis do: a.estimateAnalysis() where a is the analysis object.
    - To save the analysis to 'file' file name do: a.saveAnalysis(file) where a is the analysis object.
\end{alltt}


%%%%%%%%%%%%%%%%%%%%%%%%%%%%%%%%%%%%%%%%%%%%%%%%%%%%%%%%%%%%%%%%%%%%%%%%%%%
%%                                Methods                                %%
%%%%%%%%%%%%%%%%%%%%%%%%%%%%%%%%%%%%%%%%%%%%%%%%%%%%%%%%%%%%%%%%%%%%%%%%%%%

  \subsubsection{Methods}

    \vspace{0.5ex}

\hspace{.8\funcindent}\begin{boxedminipage}{\funcwidth}

    \raggedright \textbf{\_\_init\_\_}(\textit{self})

    \vspace{-1.5ex}

    \rule{\textwidth}{0.5\fboxrule}
\setlength{\parskip}{2ex}
    The constructor. Insures that the class can not be instanciated 
    directly from here.

\setlength{\parskip}{1ex}
      \textbf{Parameters}
      \vspace{-1ex}

      \begin{quote}
        \begin{Ventry}{xxxxxxxxxx}

          \item[parameters]

          a dictionnary that contains parameters of the selected analysis.

          \item[statusBar]

          if not None, an instance of nMOLDYN.GUI.Widgets.StatusBar. Will 
          attach a status bar to the selected analysis.

        \end{Ventry}

      \end{quote}

      Overrides: nMOLDYN.Analysis.Analysis.Analysis.\_\_init\_\_

    \end{boxedminipage}

    \label{nMOLDYN:Analysis:Dynamics:AngularCorrelation:initialize}
    \index{nMOLDYN \textit{(package)}!nMOLDYN.Analysis \textit{(package)}!nMOLDYN.Analysis.Dynamics \textit{(module)}!nMOLDYN.Analysis.Dynamics.AngularCorrelation \textit{(class)}!nMOLDYN.Analysis.Dynamics.AngularCorrelation.initialize \textit{(method)}}

    \vspace{0.5ex}

\hspace{.8\funcindent}\begin{boxedminipage}{\funcwidth}

    \raggedright \textbf{initialize}(\textit{self})

    \vspace{-1.5ex}

    \rule{\textwidth}{0.5\fboxrule}
\setlength{\parskip}{2ex}
    Initializes the analysis (e.g. parses and checks input parameters, set 
    some variables ...).

\setlength{\parskip}{1ex}
    \end{boxedminipage}

    \label{nMOLDYN:Analysis:Dynamics:AngularCorrelation:calc}
    \index{nMOLDYN \textit{(package)}!nMOLDYN.Analysis \textit{(package)}!nMOLDYN.Analysis.Dynamics \textit{(module)}!nMOLDYN.Analysis.Dynamics.AngularCorrelation \textit{(class)}!nMOLDYN.Analysis.Dynamics.AngularCorrelation.calc \textit{(method)}}

    \vspace{0.5ex}

\hspace{.8\funcindent}\begin{boxedminipage}{\funcwidth}

    \raggedright \textbf{calc}(\textit{self}, \textit{tripletIndex}, \textit{trajname})

    \vspace{-1.5ex}

    \rule{\textwidth}{0.5\fboxrule}
\setlength{\parskip}{2ex}
    Calculates the contribution for one group.

\setlength{\parskip}{1ex}
      \textbf{Parameters}
      \vspace{-1ex}

      \begin{quote}
        \begin{Ventry}{xxxxxxxxxxxx}

          \item[tripletIndex]

          the index of the triplet in {\textbar}self.triplet{\textbar} 
          list.

            {\it (type=integer.)}

          \item[trajname]

          the name of the trajectory file name.

            {\it (type=string)}

        \end{Ventry}

      \end{quote}

    \end{boxedminipage}

    \label{nMOLDYN:Analysis:Dynamics:AngularCorrelation:combine}
    \index{nMOLDYN \textit{(package)}!nMOLDYN.Analysis \textit{(package)}!nMOLDYN.Analysis.Dynamics \textit{(module)}!nMOLDYN.Analysis.Dynamics.AngularCorrelation \textit{(class)}!nMOLDYN.Analysis.Dynamics.AngularCorrelation.combine \textit{(method)}}

    \vspace{0.5ex}

\hspace{.8\funcindent}\begin{boxedminipage}{\funcwidth}

    \raggedright \textbf{combine}(\textit{self}, \textit{tripletIndex}, \textit{x})

\setlength{\parskip}{2ex}
\setlength{\parskip}{1ex}
    \end{boxedminipage}

    \label{nMOLDYN:Analysis:Dynamics:AngularCorrelation:finalize}
    \index{nMOLDYN \textit{(package)}!nMOLDYN.Analysis \textit{(package)}!nMOLDYN.Analysis.Dynamics \textit{(module)}!nMOLDYN.Analysis.Dynamics.AngularCorrelation \textit{(class)}!nMOLDYN.Analysis.Dynamics.AngularCorrelation.finalize \textit{(method)}}

    \vspace{0.5ex}

\hspace{.8\funcindent}\begin{boxedminipage}{\funcwidth}

    \raggedright \textbf{finalize}(\textit{self})

    \vspace{-1.5ex}

    \rule{\textwidth}{0.5\fboxrule}
\setlength{\parskip}{2ex}
    Finalizes the calculations (e.g. averaging the total term, output files
    creations ...).

\setlength{\parskip}{1ex}
    \end{boxedminipage}


\large{\textbf{\textit{Inherited from nMOLDYN.Analysis.Analysis.Analysis\textit{(Section \ref{nMOLDYN:Analysis:Analysis:Analysis})}}}}

\begin{quote}
analysisTime(), buildJobInfo(), buildTimeInfo(), deuterationSelection(), groupSelection(), parseInputParameters(), preLoadTrajectory(), runAnalysis(), saveAnalysis(), setInputParameters(), subsetSelection(), updateJobProgress(), weightingScheme()
\end{quote}

%%%%%%%%%%%%%%%%%%%%%%%%%%%%%%%%%%%%%%%%%%%%%%%%%%%%%%%%%%%%%%%%%%%%%%%%%%%
%%                            Class Variables                            %%
%%%%%%%%%%%%%%%%%%%%%%%%%%%%%%%%%%%%%%%%%%%%%%%%%%%%%%%%%%%%%%%%%%%%%%%%%%%

  \subsubsection{Class Variables}

    \vspace{-1cm}
\hspace{\varindent}\begin{longtable}{|p{\varnamewidth}|p{\vardescrwidth}|l}
\cline{1-2}
\cline{1-2} \centering \textbf{Name} & \centering \textbf{Description}& \\
\cline{1-2}
\endhead\cline{1-2}\multicolumn{3}{r}{\small\textit{continued on next page}}\\\endfoot\cline{1-2}
\endlastfoot\raggedright i\-n\-p\-u\-t\-P\-a\-r\-a\-m\-e\-t\-e\-r\-s\-N\-a\-m\-e\-s\- & \raggedright \textbf{Value:} 
{\tt 'trajectory', 'timeinfo', 'triplet', 'atomorder', 'ac', '\texttt{...}}&\\
\cline{1-2}
\raggedright s\-h\-o\-r\-t\-N\-a\-m\-e\- & \raggedright \textbf{Value:} 
{\tt 'AC'}&\\
\cline{1-2}
\raggedright c\-a\-n\-B\-e\-E\-s\-t\-i\-m\-a\-t\-e\-d\- & \raggedright \textbf{Value:} 
{\tt True}&\\
\cline{1-2}
\end{longtable}

    \index{nMOLDYN \textit{(package)}!nMOLDYN.Analysis \textit{(package)}!nMOLDYN.Analysis.Dynamics \textit{(module)}!nMOLDYN.Analysis.Dynamics.AngularCorrelation \textit{(class)}|)}

%%%%%%%%%%%%%%%%%%%%%%%%%%%%%%%%%%%%%%%%%%%%%%%%%%%%%%%%%%%%%%%%%%%%%%%%%%%
%%                           Class Description                           %%
%%%%%%%%%%%%%%%%%%%%%%%%%%%%%%%%%%%%%%%%%%%%%%%%%%%%%%%%%%%%%%%%%%%%%%%%%%%

    \index{nMOLDYN \textit{(package)}!nMOLDYN.Analysis \textit{(package)}!nMOLDYN.Analysis.Dynamics \textit{(module)}!nMOLDYN.Analysis.Dynamics.RigidBodyTrajectory \textit{(class)}|(}
\subsection{Class RigidBodyTrajectory}

    \label{nMOLDYN:Analysis:Dynamics:RigidBodyTrajectory}
\begin{tabular}{cccccc}
% Line for nMOLDYN.Analysis.Analysis.Analysis, linespec=[False]
\multicolumn{2}{r}{\settowidth{\BCL}{nMOLDYN.Analysis.Analysis.Analysis}\multirow{2}{\BCL}{nMOLDYN.Analysis.Analysis.Analysis}}
&&
  \\\cline{3-3}
  &&\multicolumn{1}{c|}{}
&&
  \\
&&\multicolumn{2}{l}{\textbf{nMOLDYN.Analysis.Dynamics.RigidBodyTrajectory}}
\end{tabular}

\begin{alltt}
Sets up a Rigid Body Trajectory analysis.

A Subclass of nMOLDYN.Analysis.Analysis. 

Constructor: RigidBodyTrajectory({\textbar}parameters{\textbar} = None)

Arguments:

    - {\textbar}parameters{\textbar} -- a dictionnary of the input parameters, or 'None' to set up the analysis without parameters.
        * trajectory     -- a trajectory file name or an instance of MMTK.Trajectory.Trajectory class.
        * timeinfo       -- a string of the form 'first:last:step' where 'first' is an integer specifying the first frame 
                            number to consider, 'last' is an integer specifying the last frame number to consider and 
                            'step' is an integer specifying the step number between two frames.
        * referenceframe -- an integer in [1,len(trajectory)] specifying which frame should be the reference.
        * stepwiserbt    -- a string being one of 'Yes' or 'No' specifying whether the reference frame for frame i should be 
                            the frame i - 1 ('Yes') or should be a fixed frame defined with {\textbar}referenceframe{\textbar} ('No').
        * group          -- a selection string specifying the groups of atoms on which the rigid body trajectory will be defined.
                            (each group being a rigid body).
        * rbt            -- the output NetCDF file name.
        * pyroserver     -- a string specifying if Pyro will be used and how to run the analysis.
    
Running modes:

    - To run the analysis do: a.runAnalysis() where a is the analysis object.
    - To estimate the analysis do: a.estimateAnalysis() where a is the analysis object.
    - To save the analysis to 'file' file name do: a.saveAnalysis(file) where a is the analysis object.
\end{alltt}


%%%%%%%%%%%%%%%%%%%%%%%%%%%%%%%%%%%%%%%%%%%%%%%%%%%%%%%%%%%%%%%%%%%%%%%%%%%
%%                                Methods                                %%
%%%%%%%%%%%%%%%%%%%%%%%%%%%%%%%%%%%%%%%%%%%%%%%%%%%%%%%%%%%%%%%%%%%%%%%%%%%

  \subsubsection{Methods}

    \vspace{0.5ex}

\hspace{.8\funcindent}\begin{boxedminipage}{\funcwidth}

    \raggedright \textbf{\_\_init\_\_}(\textit{self})

    \vspace{-1.5ex}

    \rule{\textwidth}{0.5\fboxrule}
\setlength{\parskip}{2ex}
    The constructor. Insures that the class can not be instanciated 
    directly from here.

\setlength{\parskip}{1ex}
      \textbf{Parameters}
      \vspace{-1ex}

      \begin{quote}
        \begin{Ventry}{xxxxxxxxxx}

          \item[parameters]

          a dictionnary that contains parameters of the selected analysis.

          \item[statusBar]

          if not None, an instance of nMOLDYN.GUI.Widgets.StatusBar. Will 
          attach a status bar to the selected analysis.

        \end{Ventry}

      \end{quote}

      Overrides: nMOLDYN.Analysis.Analysis.Analysis.\_\_init\_\_

    \end{boxedminipage}

    \label{nMOLDYN:Analysis:Dynamics:RigidBodyTrajectory:initialize}
    \index{nMOLDYN \textit{(package)}!nMOLDYN.Analysis \textit{(package)}!nMOLDYN.Analysis.Dynamics \textit{(module)}!nMOLDYN.Analysis.Dynamics.RigidBodyTrajectory \textit{(class)}!nMOLDYN.Analysis.Dynamics.RigidBodyTrajectory.initialize \textit{(method)}}

    \vspace{0.5ex}

\hspace{.8\funcindent}\begin{boxedminipage}{\funcwidth}

    \raggedright \textbf{initialize}(\textit{self})

    \vspace{-1.5ex}

    \rule{\textwidth}{0.5\fboxrule}
\setlength{\parskip}{2ex}
    Initializes the analysis (e.g. parses and checks input parameters, set 
    some variables ...).

\setlength{\parskip}{1ex}
    \end{boxedminipage}

    \label{nMOLDYN:Analysis:Dynamics:RigidBodyTrajectory:calc}
    \index{nMOLDYN \textit{(package)}!nMOLDYN.Analysis \textit{(package)}!nMOLDYN.Analysis.Dynamics \textit{(module)}!nMOLDYN.Analysis.Dynamics.RigidBodyTrajectory \textit{(class)}!nMOLDYN.Analysis.Dynamics.RigidBodyTrajectory.calc \textit{(method)}}

    \vspace{0.5ex}

\hspace{.8\funcindent}\begin{boxedminipage}{\funcwidth}

    \raggedright \textbf{calc}(\textit{self}, \textit{groupIndex}, \textit{trajname})

    \vspace{-1.5ex}

    \rule{\textwidth}{0.5\fboxrule}
\setlength{\parskip}{2ex}
    Calculates the contribution for one group.

\setlength{\parskip}{1ex}
      \textbf{Parameters}
      \vspace{-1ex}

      \begin{quote}
        \begin{Ventry}{xxxxxxxxxx}

          \item[groupIndex]

          the index of the group in {\textbar}self.group{\textbar} list.

            {\it (type=integer.)}

          \item[trajname]

          the name of the trajectory file name.

            {\it (type=string)}

        \end{Ventry}

      \end{quote}

    \end{boxedminipage}

    \label{nMOLDYN:Analysis:Dynamics:RigidBodyTrajectory:combine}
    \index{nMOLDYN \textit{(package)}!nMOLDYN.Analysis \textit{(package)}!nMOLDYN.Analysis.Dynamics \textit{(module)}!nMOLDYN.Analysis.Dynamics.RigidBodyTrajectory \textit{(class)}!nMOLDYN.Analysis.Dynamics.RigidBodyTrajectory.combine \textit{(method)}}

    \vspace{0.5ex}

\hspace{.8\funcindent}\begin{boxedminipage}{\funcwidth}

    \raggedright \textbf{combine}(\textit{self}, \textit{groupIndex}, \textit{x})

\setlength{\parskip}{2ex}
\setlength{\parskip}{1ex}
    \end{boxedminipage}

    \label{nMOLDYN:Analysis:Dynamics:RigidBodyTrajectory:finalize}
    \index{nMOLDYN \textit{(package)}!nMOLDYN.Analysis \textit{(package)}!nMOLDYN.Analysis.Dynamics \textit{(module)}!nMOLDYN.Analysis.Dynamics.RigidBodyTrajectory \textit{(class)}!nMOLDYN.Analysis.Dynamics.RigidBodyTrajectory.finalize \textit{(method)}}

    \vspace{0.5ex}

\hspace{.8\funcindent}\begin{boxedminipage}{\funcwidth}

    \raggedright \textbf{finalize}(\textit{self})

    \vspace{-1.5ex}

    \rule{\textwidth}{0.5\fboxrule}
\setlength{\parskip}{2ex}
    Finalizes the calculations (e.g. averaging the total term, output files
    creations ...).

\setlength{\parskip}{1ex}
    \end{boxedminipage}


\large{\textbf{\textit{Inherited from nMOLDYN.Analysis.Analysis.Analysis\textit{(Section \ref{nMOLDYN:Analysis:Analysis:Analysis})}}}}

\begin{quote}
analysisTime(), buildJobInfo(), buildTimeInfo(), deuterationSelection(), groupSelection(), parseInputParameters(), preLoadTrajectory(), runAnalysis(), saveAnalysis(), setInputParameters(), subsetSelection(), updateJobProgress(), weightingScheme()
\end{quote}

%%%%%%%%%%%%%%%%%%%%%%%%%%%%%%%%%%%%%%%%%%%%%%%%%%%%%%%%%%%%%%%%%%%%%%%%%%%
%%                            Class Variables                            %%
%%%%%%%%%%%%%%%%%%%%%%%%%%%%%%%%%%%%%%%%%%%%%%%%%%%%%%%%%%%%%%%%%%%%%%%%%%%

  \subsubsection{Class Variables}

    \vspace{-1cm}
\hspace{\varindent}\begin{longtable}{|p{\varnamewidth}|p{\vardescrwidth}|l}
\cline{1-2}
\cline{1-2} \centering \textbf{Name} & \centering \textbf{Description}& \\
\cline{1-2}
\endhead\cline{1-2}\multicolumn{3}{r}{\small\textit{continued on next page}}\\\endfoot\cline{1-2}
\endlastfoot\raggedright i\-n\-p\-u\-t\-P\-a\-r\-a\-m\-e\-t\-e\-r\-s\-N\-a\-m\-e\-s\- & \raggedright \textbf{Value:} 
{\tt 'trajectory', 'timeinfo', 'referenceframe', 'removetransl\texttt{...}}&\\
\cline{1-2}
\raggedright s\-h\-o\-r\-t\-N\-a\-m\-e\- & \raggedright \textbf{Value:} 
{\tt 'RBT'}&\\
\cline{1-2}
\raggedright c\-a\-n\-B\-e\-E\-s\-t\-i\-m\-a\-t\-e\-d\- & \raggedright \textbf{Value:} 
{\tt True}&\\
\cline{1-2}
\end{longtable}

    \index{nMOLDYN \textit{(package)}!nMOLDYN.Analysis \textit{(package)}!nMOLDYN.Analysis.Dynamics \textit{(module)}!nMOLDYN.Analysis.Dynamics.RigidBodyTrajectory \textit{(class)}|)}

%%%%%%%%%%%%%%%%%%%%%%%%%%%%%%%%%%%%%%%%%%%%%%%%%%%%%%%%%%%%%%%%%%%%%%%%%%%
%%                           Class Description                           %%
%%%%%%%%%%%%%%%%%%%%%%%%%%%%%%%%%%%%%%%%%%%%%%%%%%%%%%%%%%%%%%%%%%%%%%%%%%%

    \index{nMOLDYN \textit{(package)}!nMOLDYN.Analysis \textit{(package)}!nMOLDYN.Analysis.Dynamics \textit{(module)}!nMOLDYN.Analysis.Dynamics.ReorientationalCorrelationFunction \textit{(class)}|(}
\subsection{Class ReorientationalCorrelationFunction}

    \label{nMOLDYN:Analysis:Dynamics:ReorientationalCorrelationFunction}
\begin{tabular}{cccccc}
% Line for nMOLDYN.Analysis.Analysis.Analysis, linespec=[False]
\multicolumn{2}{r}{\settowidth{\BCL}{nMOLDYN.Analysis.Analysis.Analysis}\multirow{2}{\BCL}{nMOLDYN.Analysis.Analysis.Analysis}}
&&
  \\\cline{3-3}
  &&\multicolumn{1}{c|}{}
&&
  \\
&&\multicolumn{2}{l}{\textbf{nMOLDYN.Analysis.Dynamics.ReorientationalCorrelationFunction}}
\end{tabular}

\begin{alltt}
Sets up a Reorientational Correlation Function analysis.

A Subclass of nMOLDYN.Analysis.Analysis. 

Constructor: ReorientationalCorrelationFunction({\textbar}parameters{\textbar} = None)

Arguments:

    - {\textbar}parameters{\textbar} -- a dictionnary of the input parameters, or 'None' to set up the analysis without parameters.
        * trajectory     -- a trajectory file name or an instance of MMTK.Trajectory.Trajectory class.
        * timeinfo       -- a string of the form 'first:last:step' where 'first' is an integer specifying the first frame 
                            number to consider, 'last' is an integer specifying the last frame number to consider and 
                            'step' is an integer specifying the step number between two frames.
        * referenceframe -- an integer in [1,len(trajectory)] specifying which frame should be the reference.
        * stepwiserbt    -- a string being one of 'Yes' or 'No' specifying whether the reference frame for frame i should be 
                            the frame i - 1 ('Yes') or should be a fixed frame defined with {\textbar}referenceframe{\textbar} ('No').
        * wignerindexes  -- a string of the form 'j,m,n' where 'j', 'm' and 'n' are respectively the j, m and n indexes of the
                            Wigner function Djmn.
        * group          -- a selection string specifying the groups of atoms on which the rigid body trajectory will be defined.
                            (each group being a rigid body).
        * rcf            -- the output NetCDF file name. A CDL version of this file will also be generated with the '.cdl' extension
                             instead of the '.nc' extension.
        * pyroserver     -- a string specifying if Pyro will be used and how to run the analysis.
    
Running modes:

    - To run the analysis do: a.runAnalysis() where a is the analysis object.
    - To estimate the analysis do: a.estimateAnalysis() where a is the analysis object.
    - To save the analysis to 'file' file name do: a.saveAnalysis(file) where a is the analysis object.
\end{alltt}


%%%%%%%%%%%%%%%%%%%%%%%%%%%%%%%%%%%%%%%%%%%%%%%%%%%%%%%%%%%%%%%%%%%%%%%%%%%
%%                                Methods                                %%
%%%%%%%%%%%%%%%%%%%%%%%%%%%%%%%%%%%%%%%%%%%%%%%%%%%%%%%%%%%%%%%%%%%%%%%%%%%

  \subsubsection{Methods}

    \vspace{0.5ex}

\hspace{.8\funcindent}\begin{boxedminipage}{\funcwidth}

    \raggedright \textbf{\_\_init\_\_}(\textit{self})

    \vspace{-1.5ex}

    \rule{\textwidth}{0.5\fboxrule}
\setlength{\parskip}{2ex}
    The constructor. Insures that the class can not be instanciated 
    directly from here.

\setlength{\parskip}{1ex}
      \textbf{Parameters}
      \vspace{-1ex}

      \begin{quote}
        \begin{Ventry}{xxxxxxxxxx}

          \item[parameters]

          a dictionnary that contains parameters of the selected analysis.

          \item[statusBar]

          if not None, an instance of nMOLDYN.GUI.Widgets.StatusBar. Will 
          attach a status bar to the selected analysis.

        \end{Ventry}

      \end{quote}

      Overrides: nMOLDYN.Analysis.Analysis.Analysis.\_\_init\_\_

    \end{boxedminipage}

    \label{nMOLDYN:Analysis:Dynamics:ReorientationalCorrelationFunction:initialize}
    \index{nMOLDYN \textit{(package)}!nMOLDYN.Analysis \textit{(package)}!nMOLDYN.Analysis.Dynamics \textit{(module)}!nMOLDYN.Analysis.Dynamics.ReorientationalCorrelationFunction \textit{(class)}!nMOLDYN.Analysis.Dynamics.ReorientationalCorrelationFunction.initialize \textit{(method)}}

    \vspace{0.5ex}

\hspace{.8\funcindent}\begin{boxedminipage}{\funcwidth}

    \raggedright \textbf{initialize}(\textit{self})

    \vspace{-1.5ex}

    \rule{\textwidth}{0.5\fboxrule}
\setlength{\parskip}{2ex}
    Initializes the analysis (e.g. parses and checks input parameters, set 
    some variables ...).

\setlength{\parskip}{1ex}
    \end{boxedminipage}

    \label{nMOLDYN:Analysis:Dynamics:ReorientationalCorrelationFunction:calc}
    \index{nMOLDYN \textit{(package)}!nMOLDYN.Analysis \textit{(package)}!nMOLDYN.Analysis.Dynamics \textit{(module)}!nMOLDYN.Analysis.Dynamics.ReorientationalCorrelationFunction \textit{(class)}!nMOLDYN.Analysis.Dynamics.ReorientationalCorrelationFunction.calc \textit{(method)}}

    \vspace{0.5ex}

\hspace{.8\funcindent}\begin{boxedminipage}{\funcwidth}

    \raggedright \textbf{calc}(\textit{self}, \textit{groupIndex}, \textit{trajname})

    \vspace{-1.5ex}

    \rule{\textwidth}{0.5\fboxrule}
\setlength{\parskip}{2ex}
    Calculates the contribution for one group.

\setlength{\parskip}{1ex}
      \textbf{Parameters}
      \vspace{-1ex}

      \begin{quote}
        \begin{Ventry}{xxxxxxxxxx}

          \item[groupIndex]

          the index of the group in {\textbar}self.group{\textbar} list.

            {\it (type=integer.)}

          \item[trajname]

          the name of the trajectory file name.

            {\it (type=string)}

        \end{Ventry}

      \end{quote}

    \end{boxedminipage}

    \label{nMOLDYN:Analysis:Dynamics:ReorientationalCorrelationFunction:combine}
    \index{nMOLDYN \textit{(package)}!nMOLDYN.Analysis \textit{(package)}!nMOLDYN.Analysis.Dynamics \textit{(module)}!nMOLDYN.Analysis.Dynamics.ReorientationalCorrelationFunction \textit{(class)}!nMOLDYN.Analysis.Dynamics.ReorientationalCorrelationFunction.combine \textit{(method)}}

    \vspace{0.5ex}

\hspace{.8\funcindent}\begin{boxedminipage}{\funcwidth}

    \raggedright \textbf{combine}(\textit{self}, \textit{groupIndex}, \textit{x})

\setlength{\parskip}{2ex}
\setlength{\parskip}{1ex}
    \end{boxedminipage}

    \label{nMOLDYN:Analysis:Dynamics:ReorientationalCorrelationFunction:finalize}
    \index{nMOLDYN \textit{(package)}!nMOLDYN.Analysis \textit{(package)}!nMOLDYN.Analysis.Dynamics \textit{(module)}!nMOLDYN.Analysis.Dynamics.ReorientationalCorrelationFunction \textit{(class)}!nMOLDYN.Analysis.Dynamics.ReorientationalCorrelationFunction.finalize \textit{(method)}}

    \vspace{0.5ex}

\hspace{.8\funcindent}\begin{boxedminipage}{\funcwidth}

    \raggedright \textbf{finalize}(\textit{self})

    \vspace{-1.5ex}

    \rule{\textwidth}{0.5\fboxrule}
\setlength{\parskip}{2ex}
    Finalizes the calculations (e.g. averaging the total term, output files
    creations ...).

\setlength{\parskip}{1ex}
    \end{boxedminipage}


\large{\textbf{\textit{Inherited from nMOLDYN.Analysis.Analysis.Analysis\textit{(Section \ref{nMOLDYN:Analysis:Analysis:Analysis})}}}}

\begin{quote}
analysisTime(), buildJobInfo(), buildTimeInfo(), deuterationSelection(), groupSelection(), parseInputParameters(), preLoadTrajectory(), runAnalysis(), saveAnalysis(), setInputParameters(), subsetSelection(), updateJobProgress(), weightingScheme()
\end{quote}

%%%%%%%%%%%%%%%%%%%%%%%%%%%%%%%%%%%%%%%%%%%%%%%%%%%%%%%%%%%%%%%%%%%%%%%%%%%
%%                            Class Variables                            %%
%%%%%%%%%%%%%%%%%%%%%%%%%%%%%%%%%%%%%%%%%%%%%%%%%%%%%%%%%%%%%%%%%%%%%%%%%%%

  \subsubsection{Class Variables}

    \vspace{-1cm}
\hspace{\varindent}\begin{longtable}{|p{\varnamewidth}|p{\vardescrwidth}|l}
\cline{1-2}
\cline{1-2} \centering \textbf{Name} & \centering \textbf{Description}& \\
\cline{1-2}
\endhead\cline{1-2}\multicolumn{3}{r}{\small\textit{continued on next page}}\\\endfoot\cline{1-2}
\endlastfoot\raggedright i\-n\-p\-u\-t\-P\-a\-r\-a\-m\-e\-t\-e\-r\-s\-N\-a\-m\-e\-s\- & \raggedright \textbf{Value:} 
{\tt 'trajectory', 'timeinfo', 'referenceframe', 'stepwiserbt'\texttt{...}}&\\
\cline{1-2}
\raggedright s\-h\-o\-r\-t\-N\-a\-m\-e\- & \raggedright \textbf{Value:} 
{\tt 'RCF'}&\\
\cline{1-2}
\raggedright c\-a\-n\-B\-e\-E\-s\-t\-i\-m\-a\-t\-e\-d\- & \raggedright \textbf{Value:} 
{\tt True}&\\
\cline{1-2}
\end{longtable}

    \index{nMOLDYN \textit{(package)}!nMOLDYN.Analysis \textit{(package)}!nMOLDYN.Analysis.Dynamics \textit{(module)}!nMOLDYN.Analysis.Dynamics.ReorientationalCorrelationFunction \textit{(class)}|)}

%%%%%%%%%%%%%%%%%%%%%%%%%%%%%%%%%%%%%%%%%%%%%%%%%%%%%%%%%%%%%%%%%%%%%%%%%%%
%%                           Class Description                           %%
%%%%%%%%%%%%%%%%%%%%%%%%%%%%%%%%%%%%%%%%%%%%%%%%%%%%%%%%%%%%%%%%%%%%%%%%%%%

    \index{nMOLDYN \textit{(package)}!nMOLDYN.Analysis \textit{(package)}!nMOLDYN.Analysis.Dynamics \textit{(module)}!nMOLDYN.Analysis.Dynamics.AngularVelocity \textit{(class)}|(}
\subsection{Class AngularVelocity}

    \label{nMOLDYN:Analysis:Dynamics:AngularVelocity}
An intermediate class used by 
{\textbar}AngularVelocityAutoCorrelationFunction{\textbar} and 
{\textbar}AngularDensityOfStates{\textbar} classes.


%%%%%%%%%%%%%%%%%%%%%%%%%%%%%%%%%%%%%%%%%%%%%%%%%%%%%%%%%%%%%%%%%%%%%%%%%%%
%%                                Methods                                %%
%%%%%%%%%%%%%%%%%%%%%%%%%%%%%%%%%%%%%%%%%%%%%%%%%%%%%%%%%%%%%%%%%%%%%%%%%%%

  \subsubsection{Methods}

    \label{nMOLDYN:Analysis:Dynamics:AngularVelocity:__init__}
    \index{nMOLDYN \textit{(package)}!nMOLDYN.Analysis \textit{(package)}!nMOLDYN.Analysis.Dynamics \textit{(module)}!nMOLDYN.Analysis.Dynamics.AngularVelocity \textit{(class)}!nMOLDYN.Analysis.Dynamics.AngularVelocity.\_\_init\_\_ \textit{(method)}}

    \vspace{0.5ex}

\hspace{.8\funcindent}\begin{boxedminipage}{\funcwidth}

    \raggedright \textbf{\_\_init\_\_}(\textit{self})

\setlength{\parskip}{2ex}
\setlength{\parskip}{1ex}
    \end{boxedminipage}

    \label{nMOLDYN:Analysis:Dynamics:AngularVelocity:qMatrix}
    \index{nMOLDYN \textit{(package)}!nMOLDYN.Analysis \textit{(package)}!nMOLDYN.Analysis.Dynamics \textit{(module)}!nMOLDYN.Analysis.Dynamics.AngularVelocity \textit{(class)}!nMOLDYN.Analysis.Dynamics.AngularVelocity.qMatrix \textit{(method)}}

    \vspace{0.5ex}

\hspace{.8\funcindent}\begin{boxedminipage}{\funcwidth}

    \raggedright \textbf{qMatrix}(\textit{self}, \textit{data})

\setlength{\parskip}{2ex}
\setlength{\parskip}{1ex}
    \end{boxedminipage}

    \label{nMOLDYN:Analysis:Dynamics:AngularVelocity:getAngularVelocity}
    \index{nMOLDYN \textit{(package)}!nMOLDYN.Analysis \textit{(package)}!nMOLDYN.Analysis.Dynamics \textit{(module)}!nMOLDYN.Analysis.Dynamics.AngularVelocity \textit{(class)}!nMOLDYN.Analysis.Dynamics.AngularVelocity.getAngularVelocity \textit{(method)}}

    \vspace{0.5ex}

\hspace{.8\funcindent}\begin{boxedminipage}{\funcwidth}

    \raggedright \textbf{getAngularVelocity}(\textit{self}, \textit{t}, \textit{g})

    \vspace{-1.5ex}

    \rule{\textwidth}{0.5\fboxrule}
\setlength{\parskip}{2ex}
    Computes the Angular Velocity Function for a group 
    {\textbar}g{\textbar} (a MMTK Collection).

\setlength{\parskip}{1ex}
    \end{boxedminipage}

    \index{nMOLDYN \textit{(package)}!nMOLDYN.Analysis \textit{(package)}!nMOLDYN.Analysis.Dynamics \textit{(module)}!nMOLDYN.Analysis.Dynamics.AngularVelocity \textit{(class)}|)}

%%%%%%%%%%%%%%%%%%%%%%%%%%%%%%%%%%%%%%%%%%%%%%%%%%%%%%%%%%%%%%%%%%%%%%%%%%%
%%                           Class Description                           %%
%%%%%%%%%%%%%%%%%%%%%%%%%%%%%%%%%%%%%%%%%%%%%%%%%%%%%%%%%%%%%%%%%%%%%%%%%%%

    \index{nMOLDYN \textit{(package)}!nMOLDYN.Analysis \textit{(package)}!nMOLDYN.Analysis.Dynamics \textit{(module)}!nMOLDYN.Analysis.Dynamics.AngularVelocityAutoCorrelationFunction \textit{(class)}|(}
\subsection{Class AngularVelocityAutoCorrelationFunction}

    \label{nMOLDYN:Analysis:Dynamics:AngularVelocityAutoCorrelationFunction}
\begin{tabular}{cccccc}
% Line for nMOLDYN.Analysis.Analysis.Analysis, linespec=[False]
\multicolumn{2}{r}{\settowidth{\BCL}{nMOLDYN.Analysis.Analysis.Analysis}\multirow{2}{\BCL}{nMOLDYN.Analysis.Analysis.Analysis}}
&&
  \\\cline{3-3}
  &&\multicolumn{1}{c|}{}
&&
  \\
% Line for nMOLDYN.Analysis.Dynamics.AngularVelocity, linespec=[True]
\multicolumn{2}{r}{\settowidth{\BCL}{nMOLDYN.Analysis.Dynamics.AngularVelocity}\multirow{2}{\BCL}{nMOLDYN.Analysis.Dynamics.AngularVelocity}}
&&\multicolumn{1}{|c}{}
  \\\cline{3-3}
  &&\multicolumn{1}{c|}{}
&\multicolumn{1}{|c}{}&
  \\
&&\multicolumn{2}{l}{\textbf{nMOLDYN.Analysis.Dynamics.AngularVelocityAutoCorrelationFunction}}
\end{tabular}

\begin{alltt}
Sets up an Angular Velocity AutoCorrelation Function analysis.

A Subclass of nMOLDYN.Analysis.Analysis. 

Constructor: AngularVelocityAutoCorrelationFunction({\textbar}parameters{\textbar} = None)

Arguments:

    - {\textbar}parameters{\textbar} -- a dictionnary of the input parameters, or 'None' to set up the analysis without parameters.
        * trajectory      -- a trajectory file name or an instance of MMTK.Trajectory.Trajectory class.
        * timeinfo        -- a string of the form 'first:last:step' where 'first' is an integer specifying the first frame 
                             number to consider, 'last' is an integer specifying the last frame number to consider and 
                             'step' is an integer specifying the step number between two frames.
        * differentiation -- an integer in [0,5] specifying the order of the differentiation used to get the velocities
                             out of the coordinates. 0 means that the velocities are already present in the trajectory loaded
                             for analysis.
        * projection      -- a string of the form 'vx,vy,vz' specifying the vector along which the analysis
                             will be computed. 'vx', 'vy', and 'vz' are floats specifying respectively the x, y and z value 
                             of that vector.
        * referenceframe  -- an integer in [1,len(trajectory)] specifying which frame should be the reference.
        * stepwiserbt     -- a string being one of 'Yes' or 'No' specifying whether the reference frame for frame i should be 
                             the frame i - 1 ('Yes') or should be a fixed frame defined with {\textbar}referenceframe{\textbar} ('No').
        * group           -- a selection string specifying the groups of atoms on which the rigid body trajectory will be defined.
                             (each group being a rigid body).
        * avacf           -- the output NetCDF file name. A CDL version of this file will also be generated with the '.cdl' extension
                             instead of the '.nc' extension.
        * pyroserver      -- a string specifying if Pyro will be used and how to run the analysis.
    
Running modes:

    - To run the analysis do: a.runAnalysis() where a is the analysis object.
    - To estimate the analysis do: a.estimateAnalysis() where a is the analysis object.
    - To save the analysis to 'file' file name do: a.saveAnalysis(file) where a is the analysis object.
\end{alltt}


%%%%%%%%%%%%%%%%%%%%%%%%%%%%%%%%%%%%%%%%%%%%%%%%%%%%%%%%%%%%%%%%%%%%%%%%%%%
%%                                Methods                                %%
%%%%%%%%%%%%%%%%%%%%%%%%%%%%%%%%%%%%%%%%%%%%%%%%%%%%%%%%%%%%%%%%%%%%%%%%%%%

  \subsubsection{Methods}

    \vspace{0.5ex}

\hspace{.8\funcindent}\begin{boxedminipage}{\funcwidth}

    \raggedright \textbf{\_\_init\_\_}(\textit{self})

    \vspace{-1.5ex}

    \rule{\textwidth}{0.5\fboxrule}
\setlength{\parskip}{2ex}
    The constructor. Insures that the class can not be instanciated 
    directly from here.

\setlength{\parskip}{1ex}
      \textbf{Parameters}
      \vspace{-1ex}

      \begin{quote}
        \begin{Ventry}{xxxxxxxxxx}

          \item[parameters]

          a dictionnary that contains parameters of the selected analysis.

          \item[statusBar]

          if not None, an instance of nMOLDYN.GUI.Widgets.StatusBar. Will 
          attach a status bar to the selected analysis.

        \end{Ventry}

      \end{quote}

      Overrides: nMOLDYN.Analysis.Dynamics.AngularVelocity.\_\_init\_\_

    \end{boxedminipage}

    \label{nMOLDYN:Analysis:Dynamics:AngularVelocityAutoCorrelationFunction:initialize}
    \index{nMOLDYN \textit{(package)}!nMOLDYN.Analysis \textit{(package)}!nMOLDYN.Analysis.Dynamics \textit{(module)}!nMOLDYN.Analysis.Dynamics.AngularVelocityAutoCorrelationFunction \textit{(class)}!nMOLDYN.Analysis.Dynamics.AngularVelocityAutoCorrelationFunction.initialize \textit{(method)}}

    \vspace{0.5ex}

\hspace{.8\funcindent}\begin{boxedminipage}{\funcwidth}

    \raggedright \textbf{initialize}(\textit{self})

    \vspace{-1.5ex}

    \rule{\textwidth}{0.5\fboxrule}
\setlength{\parskip}{2ex}
    Initializes the analysis (e.g. parses and checks input parameters, set 
    some variables ...).

\setlength{\parskip}{1ex}
    \end{boxedminipage}

    \label{nMOLDYN:Analysis:Dynamics:AngularVelocityAutoCorrelationFunction:calc}
    \index{nMOLDYN \textit{(package)}!nMOLDYN.Analysis \textit{(package)}!nMOLDYN.Analysis.Dynamics \textit{(module)}!nMOLDYN.Analysis.Dynamics.AngularVelocityAutoCorrelationFunction \textit{(class)}!nMOLDYN.Analysis.Dynamics.AngularVelocityAutoCorrelationFunction.calc \textit{(method)}}

    \vspace{0.5ex}

\hspace{.8\funcindent}\begin{boxedminipage}{\funcwidth}

    \raggedright \textbf{calc}(\textit{self}, \textit{groupIndex}, \textit{trajname})

    \vspace{-1.5ex}

    \rule{\textwidth}{0.5\fboxrule}
\setlength{\parskip}{2ex}
    Calculates the contribution for one group.

\setlength{\parskip}{1ex}
      \textbf{Parameters}
      \vspace{-1ex}

      \begin{quote}
        \begin{Ventry}{xxxxxxxxxx}

          \item[groupIndex]

          the index of the group in {\textbar}self.group{\textbar} list.

            {\it (type=integer.)}

          \item[trajname]

          the name of the trajectory file name.

            {\it (type=string)}

        \end{Ventry}

      \end{quote}

    \end{boxedminipage}

    \label{nMOLDYN:Analysis:Dynamics:AngularVelocityAutoCorrelationFunction:combine}
    \index{nMOLDYN \textit{(package)}!nMOLDYN.Analysis \textit{(package)}!nMOLDYN.Analysis.Dynamics \textit{(module)}!nMOLDYN.Analysis.Dynamics.AngularVelocityAutoCorrelationFunction \textit{(class)}!nMOLDYN.Analysis.Dynamics.AngularVelocityAutoCorrelationFunction.combine \textit{(method)}}

    \vspace{0.5ex}

\hspace{.8\funcindent}\begin{boxedminipage}{\funcwidth}

    \raggedright \textbf{combine}(\textit{self}, \textit{groupIndex}, \textit{x})

\setlength{\parskip}{2ex}
\setlength{\parskip}{1ex}
    \end{boxedminipage}

    \label{nMOLDYN:Analysis:Dynamics:AngularVelocityAutoCorrelationFunction:finalize}
    \index{nMOLDYN \textit{(package)}!nMOLDYN.Analysis \textit{(package)}!nMOLDYN.Analysis.Dynamics \textit{(module)}!nMOLDYN.Analysis.Dynamics.AngularVelocityAutoCorrelationFunction \textit{(class)}!nMOLDYN.Analysis.Dynamics.AngularVelocityAutoCorrelationFunction.finalize \textit{(method)}}

    \vspace{0.5ex}

\hspace{.8\funcindent}\begin{boxedminipage}{\funcwidth}

    \raggedright \textbf{finalize}(\textit{self})

    \vspace{-1.5ex}

    \rule{\textwidth}{0.5\fboxrule}
\setlength{\parskip}{2ex}
    Finalizes the calculations (e.g. averaging the total term, output files
    creations ...).

\setlength{\parskip}{1ex}
    \end{boxedminipage}


\large{\textbf{\textit{Inherited from nMOLDYN.Analysis.Analysis.Analysis\textit{(Section \ref{nMOLDYN:Analysis:Analysis:Analysis})}}}}

\begin{quote}
analysisTime(), buildJobInfo(), buildTimeInfo(), deuterationSelection(), groupSelection(), parseInputParameters(), preLoadTrajectory(), runAnalysis(), saveAnalysis(), setInputParameters(), subsetSelection(), updateJobProgress(), weightingScheme()
\end{quote}

\large{\textbf{\textit{Inherited from nMOLDYN.Analysis.Dynamics.AngularVelocity\textit{(Section \ref{nMOLDYN:Analysis:Dynamics:AngularVelocity})}}}}

\begin{quote}
getAngularVelocity(), qMatrix()
\end{quote}

%%%%%%%%%%%%%%%%%%%%%%%%%%%%%%%%%%%%%%%%%%%%%%%%%%%%%%%%%%%%%%%%%%%%%%%%%%%
%%                            Class Variables                            %%
%%%%%%%%%%%%%%%%%%%%%%%%%%%%%%%%%%%%%%%%%%%%%%%%%%%%%%%%%%%%%%%%%%%%%%%%%%%

  \subsubsection{Class Variables}

    \vspace{-1cm}
\hspace{\varindent}\begin{longtable}{|p{\varnamewidth}|p{\vardescrwidth}|l}
\cline{1-2}
\cline{1-2} \centering \textbf{Name} & \centering \textbf{Description}& \\
\cline{1-2}
\endhead\cline{1-2}\multicolumn{3}{r}{\small\textit{continued on next page}}\\\endfoot\cline{1-2}
\endlastfoot\raggedright i\-n\-p\-u\-t\-P\-a\-r\-a\-m\-e\-t\-e\-r\-s\-N\-a\-m\-e\-s\- & \raggedright \textbf{Value:} 
{\tt 'trajectory', 'timeinfo', 'differentiation', 'projection'\texttt{...}}&\\
\cline{1-2}
\raggedright s\-h\-o\-r\-t\-N\-a\-m\-e\- & \raggedright \textbf{Value:} 
{\tt 'AVACF'}&\\
\cline{1-2}
\raggedright c\-a\-n\-B\-e\-E\-s\-t\-i\-m\-a\-t\-e\-d\- & \raggedright \textbf{Value:} 
{\tt True}&\\
\cline{1-2}
\end{longtable}

    \index{nMOLDYN \textit{(package)}!nMOLDYN.Analysis \textit{(package)}!nMOLDYN.Analysis.Dynamics \textit{(module)}!nMOLDYN.Analysis.Dynamics.AngularVelocityAutoCorrelationFunction \textit{(class)}|)}

%%%%%%%%%%%%%%%%%%%%%%%%%%%%%%%%%%%%%%%%%%%%%%%%%%%%%%%%%%%%%%%%%%%%%%%%%%%
%%                           Class Description                           %%
%%%%%%%%%%%%%%%%%%%%%%%%%%%%%%%%%%%%%%%%%%%%%%%%%%%%%%%%%%%%%%%%%%%%%%%%%%%

    \index{nMOLDYN \textit{(package)}!nMOLDYN.Analysis \textit{(package)}!nMOLDYN.Analysis.Dynamics \textit{(module)}!nMOLDYN.Analysis.Dynamics.AngularDensityOfStates \textit{(class)}|(}
\subsection{Class AngularDensityOfStates}

    \label{nMOLDYN:Analysis:Dynamics:AngularDensityOfStates}
\begin{tabular}{cccccc}
% Line for nMOLDYN.Analysis.Analysis.Analysis, linespec=[False]
\multicolumn{2}{r}{\settowidth{\BCL}{nMOLDYN.Analysis.Analysis.Analysis}\multirow{2}{\BCL}{nMOLDYN.Analysis.Analysis.Analysis}}
&&
  \\\cline{3-3}
  &&\multicolumn{1}{c|}{}
&&
  \\
% Line for nMOLDYN.Analysis.Dynamics.AngularVelocity, linespec=[True]
\multicolumn{2}{r}{\settowidth{\BCL}{nMOLDYN.Analysis.Dynamics.AngularVelocity}\multirow{2}{\BCL}{nMOLDYN.Analysis.Dynamics.AngularVelocity}}
&&\multicolumn{1}{|c}{}
  \\\cline{3-3}
  &&\multicolumn{1}{c|}{}
&\multicolumn{1}{|c}{}&
  \\
&&\multicolumn{2}{l}{\textbf{nMOLDYN.Analysis.Dynamics.AngularDensityOfStates}}
\end{tabular}

\begin{alltt}
Sets up an Angular Density Of States analysis.

A Subclass of nMOLDYN.Analysis.Analysis. 

Constructor: AngularDensityOfStates({\textbar}parameters{\textbar} = None)

Arguments:

    - {\textbar}parameters{\textbar} -- a dictionnary of the input parameters, or 'None' to set up the analysis without parameters.
        * trajectory      -- a trajectory file name or an instance of MMTK.Trajectory.Trajectory class.
        * timeinfo        -- a string of the form 'first:last:step' where 'first' is an integer specifying the first frame 
                             number to consider, 'last' is an integer specifying the last frame number to consider and 
                             'step' is an integer specifying the step number between two frames.
        * differentiation -- an integer in [0,5] specifying the order of the differentiation used to get the velocities
                             out of the coordinates. 0 means that the velocities are already present in the trajectory loaded
                             for analysis.
        * projection      -- a string of the form 'vx,vy,vz' specifying the vector along which the analysis
                             will be computed. 'vx', 'vy', and 'vz' are floats specifying respectively the x, y and z value 
                             of that vector.
        * referenceframe  -- an integer in [1,len(trajectory)] specifying which frame should be the reference.
        * stepwiserbt     -- a string being one of 'Yes' or 'No' specifying whether the reference frame for frame i should be 
                             the frame i - 1 ('Yes') or should be a fixed frame defined with {\textbar}referenceframe{\textbar} ('No').
        * fftwindow       -- a float in ]0.0,100.0[ specifying the width of the gaussian, in percentage of the trajectory length
                             that will be used in the smoothing procedure.
        * group           -- a selection string specifying the groups of atoms on which the rigid body trajectory will be defined.
                             (each group being a rigid body).
        * ados            -- the output NetCDF file name. A CDL version of this file will also be generated with the '.cdl' extension
                             instead of the '.nc' extension.
        * pyroserver      -- a string specifying if Pyro will be used and how to run the analysis.
    
Running modes:

    - To run the analysis do: a.runAnalysis() where a is the analysis object.
    - To estimate the analysis do: a.estimateAnalysis() where a is the analysis object.
    - To save the analysis to 'file' file name do: a.saveAnalysis(file) where a is the analysis object.
\end{alltt}


%%%%%%%%%%%%%%%%%%%%%%%%%%%%%%%%%%%%%%%%%%%%%%%%%%%%%%%%%%%%%%%%%%%%%%%%%%%
%%                                Methods                                %%
%%%%%%%%%%%%%%%%%%%%%%%%%%%%%%%%%%%%%%%%%%%%%%%%%%%%%%%%%%%%%%%%%%%%%%%%%%%

  \subsubsection{Methods}

    \vspace{0.5ex}

\hspace{.8\funcindent}\begin{boxedminipage}{\funcwidth}

    \raggedright \textbf{\_\_init\_\_}(\textit{self})

    \vspace{-1.5ex}

    \rule{\textwidth}{0.5\fboxrule}
\setlength{\parskip}{2ex}
    The constructor. Insures that the class can not be instanciated 
    directly from here.

\setlength{\parskip}{1ex}
      \textbf{Parameters}
      \vspace{-1ex}

      \begin{quote}
        \begin{Ventry}{xxxxxxxxxx}

          \item[parameters]

          a dictionnary that contains parameters of the selected analysis.

          \item[statusBar]

          if not None, an instance of nMOLDYN.GUI.Widgets.StatusBar. Will 
          attach a status bar to the selected analysis.

        \end{Ventry}

      \end{quote}

      Overrides: nMOLDYN.Analysis.Dynamics.AngularVelocity.\_\_init\_\_

    \end{boxedminipage}

    \label{nMOLDYN:Analysis:Dynamics:AngularDensityOfStates:initialize}
    \index{nMOLDYN \textit{(package)}!nMOLDYN.Analysis \textit{(package)}!nMOLDYN.Analysis.Dynamics \textit{(module)}!nMOLDYN.Analysis.Dynamics.AngularDensityOfStates \textit{(class)}!nMOLDYN.Analysis.Dynamics.AngularDensityOfStates.initialize \textit{(method)}}

    \vspace{0.5ex}

\hspace{.8\funcindent}\begin{boxedminipage}{\funcwidth}

    \raggedright \textbf{initialize}(\textit{self})

    \vspace{-1.5ex}

    \rule{\textwidth}{0.5\fboxrule}
\setlength{\parskip}{2ex}
    Initializes the analysis (e.g. parses and checks input parameters, set 
    some variables ...).

\setlength{\parskip}{1ex}
    \end{boxedminipage}

    \label{nMOLDYN:Analysis:Dynamics:AngularDensityOfStates:calc}
    \index{nMOLDYN \textit{(package)}!nMOLDYN.Analysis \textit{(package)}!nMOLDYN.Analysis.Dynamics \textit{(module)}!nMOLDYN.Analysis.Dynamics.AngularDensityOfStates \textit{(class)}!nMOLDYN.Analysis.Dynamics.AngularDensityOfStates.calc \textit{(method)}}

    \vspace{0.5ex}

\hspace{.8\funcindent}\begin{boxedminipage}{\funcwidth}

    \raggedright \textbf{calc}(\textit{self}, \textit{groupIndex}, \textit{trajname})

    \vspace{-1.5ex}

    \rule{\textwidth}{0.5\fboxrule}
\setlength{\parskip}{2ex}
    Calculates the contribution for one group.

\setlength{\parskip}{1ex}
      \textbf{Parameters}
      \vspace{-1ex}

      \begin{quote}
        \begin{Ventry}{xxxxxxxxxx}

          \item[groupIndex]

          the index of the group in {\textbar}self.group{\textbar} list.

            {\it (type=integer.)}

          \item[trajname]

          the name of the trajectory file name.

            {\it (type=string)}

        \end{Ventry}

      \end{quote}

    \end{boxedminipage}

    \label{nMOLDYN:Analysis:Dynamics:AngularDensityOfStates:combine}
    \index{nMOLDYN \textit{(package)}!nMOLDYN.Analysis \textit{(package)}!nMOLDYN.Analysis.Dynamics \textit{(module)}!nMOLDYN.Analysis.Dynamics.AngularDensityOfStates \textit{(class)}!nMOLDYN.Analysis.Dynamics.AngularDensityOfStates.combine \textit{(method)}}

    \vspace{0.5ex}

\hspace{.8\funcindent}\begin{boxedminipage}{\funcwidth}

    \raggedright \textbf{combine}(\textit{self}, \textit{groupIndex}, \textit{x})

\setlength{\parskip}{2ex}
\setlength{\parskip}{1ex}
    \end{boxedminipage}

    \label{nMOLDYN:Analysis:Dynamics:AngularDensityOfStates:finalize}
    \index{nMOLDYN \textit{(package)}!nMOLDYN.Analysis \textit{(package)}!nMOLDYN.Analysis.Dynamics \textit{(module)}!nMOLDYN.Analysis.Dynamics.AngularDensityOfStates \textit{(class)}!nMOLDYN.Analysis.Dynamics.AngularDensityOfStates.finalize \textit{(method)}}

    \vspace{0.5ex}

\hspace{.8\funcindent}\begin{boxedminipage}{\funcwidth}

    \raggedright \textbf{finalize}(\textit{self})

    \vspace{-1.5ex}

    \rule{\textwidth}{0.5\fboxrule}
\setlength{\parskip}{2ex}
    Finalizes the calculations (e.g. averaging the total term, output files
    creations ...).

\setlength{\parskip}{1ex}
    \end{boxedminipage}


\large{\textbf{\textit{Inherited from nMOLDYN.Analysis.Analysis.Analysis\textit{(Section \ref{nMOLDYN:Analysis:Analysis:Analysis})}}}}

\begin{quote}
analysisTime(), buildJobInfo(), buildTimeInfo(), deuterationSelection(), groupSelection(), parseInputParameters(), preLoadTrajectory(), runAnalysis(), saveAnalysis(), setInputParameters(), subsetSelection(), updateJobProgress(), weightingScheme()
\end{quote}

\large{\textbf{\textit{Inherited from nMOLDYN.Analysis.Dynamics.AngularVelocity\textit{(Section \ref{nMOLDYN:Analysis:Dynamics:AngularVelocity})}}}}

\begin{quote}
getAngularVelocity(), qMatrix()
\end{quote}

%%%%%%%%%%%%%%%%%%%%%%%%%%%%%%%%%%%%%%%%%%%%%%%%%%%%%%%%%%%%%%%%%%%%%%%%%%%
%%                            Class Variables                            %%
%%%%%%%%%%%%%%%%%%%%%%%%%%%%%%%%%%%%%%%%%%%%%%%%%%%%%%%%%%%%%%%%%%%%%%%%%%%

  \subsubsection{Class Variables}

    \vspace{-1cm}
\hspace{\varindent}\begin{longtable}{|p{\varnamewidth}|p{\vardescrwidth}|l}
\cline{1-2}
\cline{1-2} \centering \textbf{Name} & \centering \textbf{Description}& \\
\cline{1-2}
\endhead\cline{1-2}\multicolumn{3}{r}{\small\textit{continued on next page}}\\\endfoot\cline{1-2}
\endlastfoot\raggedright i\-n\-p\-u\-t\-P\-a\-r\-a\-m\-e\-t\-e\-r\-s\-N\-a\-m\-e\-s\- & \raggedright \textbf{Value:} 
{\tt 'trajectory', 'timeInfo', 'differentiation', 'projection'\texttt{...}}&\\
\cline{1-2}
\raggedright s\-h\-o\-r\-t\-N\-a\-m\-e\- & \raggedright \textbf{Value:} 
{\tt 'ADOS'}&\\
\cline{1-2}
\raggedright c\-a\-n\-B\-e\-E\-s\-t\-i\-m\-a\-t\-e\-d\- & \raggedright \textbf{Value:} 
{\tt True}&\\
\cline{1-2}
\end{longtable}

    \index{nMOLDYN \textit{(package)}!nMOLDYN.Analysis \textit{(package)}!nMOLDYN.Analysis.Dynamics \textit{(module)}!nMOLDYN.Analysis.Dynamics.AngularDensityOfStates \textit{(class)}|)}
    \index{nMOLDYN \textit{(package)}!nMOLDYN.Analysis \textit{(package)}!nMOLDYN.Analysis.Dynamics \textit{(module)}|)}
