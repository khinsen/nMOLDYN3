%
% API Documentation for nMOLDYN
% Module nMOLDYN.GUI.PlotNetCDFVariableDialog
%
% Generated by epydoc 3.0.1
% [Thu Oct  8 17:00:01 2009]
%

%%%%%%%%%%%%%%%%%%%%%%%%%%%%%%%%%%%%%%%%%%%%%%%%%%%%%%%%%%%%%%%%%%%%%%%%%%%
%%                          Module Description                           %%
%%%%%%%%%%%%%%%%%%%%%%%%%%%%%%%%%%%%%%%%%%%%%%%%%%%%%%%%%%%%%%%%%%%%%%%%%%%

    \index{nMOLDYN \textit{(package)}!nMOLDYN.GUI \textit{(package)}!nMOLDYN.GUI.PlotNetCDFVariableDialog \textit{(module)}|(}
\section{Module nMOLDYN.GUI.PlotNetCDFVariableDialog}

    \label{nMOLDYN:GUI:PlotNetCDFVariableDialog}
\begin{alltt}
This modules implements I\{View--{\textgreater}Plot\} dialog.

Classes:
    * SettingsDialog: sets up the settings dialog.
    * ASCIIToNetCDFConversionDialog: creates I\{View--{\textgreater}Plot\} dialog used to plot NetCDF variables.
\end{alltt}


%%%%%%%%%%%%%%%%%%%%%%%%%%%%%%%%%%%%%%%%%%%%%%%%%%%%%%%%%%%%%%%%%%%%%%%%%%%
%%                               Variables                               %%
%%%%%%%%%%%%%%%%%%%%%%%%%%%%%%%%%%%%%%%%%%%%%%%%%%%%%%%%%%%%%%%%%%%%%%%%%%%

  \subsection{Variables}

    \vspace{-1cm}
\hspace{\varindent}\begin{longtable}{|p{\varnamewidth}|p{\vardescrwidth}|l}
\cline{1-2}
\cline{1-2} \centering \textbf{Name} & \centering \textbf{Description}& \\
\cline{1-2}
\endhead\cline{1-2}\multicolumn{3}{r}{\small\textit{continued on next page}}\\\endfoot\cline{1-2}
\endlastfoot\raggedright i\-n\-t\-e\-r\-p\-o\-l\-a\-t\-i\-o\-n\-s\- & \raggedright \textbf{Value:} 
{\tt ['bessel', 'bilinear', 'bicubic', 'blackman', 'catrom', '\texttt{...}}&\\
\cline{1-2}
\raggedright c\-o\-l\-o\-r\-M\-a\-p\-s\- & \raggedright \textbf{Value:} 
{\tt ['autumn', 'bone', 'cool', 'copper', 'flag', 'gray', 'hot\texttt{...}}&\\
\cline{1-2}
\raggedright l\-i\-n\-e\-S\-t\-y\-l\-e\-s\- & \raggedright \textbf{Value:} 
{\tt ['-', '--', '-.', ':', 'None']}&\\
\cline{1-2}
\raggedright m\-a\-r\-k\-e\-r\-S\-t\-y\-l\-e\-s\- & \raggedright \textbf{Value:} 
{\tt ['+', '.', '{\textless}', '{\textgreater}', 'o', 'p', 's', 'v', 'x', '{\textbar}', 'None']}&\\
\cline{1-2}
\raggedright a\-x\-i\-s\-S\-c\-a\-l\-e\-s\- & \raggedright \textbf{Value:} 
{\tt ['linear', 'log']}&\\
\cline{1-2}
\end{longtable}


%%%%%%%%%%%%%%%%%%%%%%%%%%%%%%%%%%%%%%%%%%%%%%%%%%%%%%%%%%%%%%%%%%%%%%%%%%%
%%                           Class Description                           %%
%%%%%%%%%%%%%%%%%%%%%%%%%%%%%%%%%%%%%%%%%%%%%%%%%%%%%%%%%%%%%%%%%%%%%%%%%%%

    \index{nMOLDYN \textit{(package)}!nMOLDYN.GUI \textit{(package)}!nMOLDYN.GUI.PlotNetCDFVariableDialog \textit{(module)}!nMOLDYN.GUI.PlotNetCDFVariableDialog.SettingsDialog \textit{(class)}|(}
\subsection{Class SettingsDialog}

    \label{nMOLDYN:GUI:PlotNetCDFVariableDialog:SettingsDialog}
\begin{tabular}{cccccc}
% Line for nMOLDYN.GUI.Widgets.Toplevel, linespec=[False]
\multicolumn{2}{r}{\settowidth{\BCL}{nMOLDYN.GUI.Widgets.Toplevel}\multirow{2}{\BCL}{nMOLDYN.GUI.Widgets.Toplevel}}
&&
  \\\cline{3-3}
  &&\multicolumn{1}{c|}{}
&&
  \\
&&\multicolumn{2}{l}{\textbf{nMOLDYN.GUI.PlotNetCDFVariableDialog.SettingsDialog}}
\end{tabular}

Sets up a dialog tp perform some settings on the plots.


%%%%%%%%%%%%%%%%%%%%%%%%%%%%%%%%%%%%%%%%%%%%%%%%%%%%%%%%%%%%%%%%%%%%%%%%%%%
%%                                Methods                                %%
%%%%%%%%%%%%%%%%%%%%%%%%%%%%%%%%%%%%%%%%%%%%%%%%%%%%%%%%%%%%%%%%%%%%%%%%%%%

  \subsubsection{Methods}

    \label{nMOLDYN:GUI:PlotNetCDFVariableDialog:SettingsDialog:__init__}
    \index{nMOLDYN \textit{(package)}!nMOLDYN.GUI \textit{(package)}!nMOLDYN.GUI.PlotNetCDFVariableDialog \textit{(module)}!nMOLDYN.GUI.PlotNetCDFVariableDialog.SettingsDialog \textit{(class)}!nMOLDYN.GUI.PlotNetCDFVariableDialog.SettingsDialog.\_\_init\_\_ \textit{(method)}}

    \vspace{0.5ex}

\hspace{.8\funcindent}\begin{boxedminipage}{\funcwidth}

    \raggedright \textbf{\_\_init\_\_}(\textit{self}, \textit{parent})

    \vspace{-1.5ex}

    \rule{\textwidth}{0.5\fboxrule}
\setlength{\parskip}{2ex}
    The constructor.

\setlength{\parskip}{1ex}
      \textbf{Parameters}
      \vspace{-1ex}

      \begin{quote}
        \begin{Ventry}{xxxxxx}

          \item[parent]

          the parent widget.

        \end{Ventry}

      \end{quote}

    \end{boxedminipage}

    \label{nMOLDYN:GUI:PlotNetCDFVariableDialog:SettingsDialog:body}
    \index{nMOLDYN \textit{(package)}!nMOLDYN.GUI \textit{(package)}!nMOLDYN.GUI.PlotNetCDFVariableDialog \textit{(module)}!nMOLDYN.GUI.PlotNetCDFVariableDialog.SettingsDialog \textit{(class)}!nMOLDYN.GUI.PlotNetCDFVariableDialog.SettingsDialog.body \textit{(method)}}

    \vspace{0.5ex}

\hspace{.8\funcindent}\begin{boxedminipage}{\funcwidth}

    \raggedright \textbf{body}(\textit{self}, \textit{master})

    \vspace{-1.5ex}

    \rule{\textwidth}{0.5\fboxrule}
\setlength{\parskip}{2ex}
    Create dialog body. Return widget that should have initial focus.

\setlength{\parskip}{1ex}
    \end{boxedminipage}

    \label{nMOLDYN:GUI:PlotNetCDFVariableDialog:SettingsDialog:buttonbox}
    \index{nMOLDYN \textit{(package)}!nMOLDYN.GUI \textit{(package)}!nMOLDYN.GUI.PlotNetCDFVariableDialog \textit{(module)}!nMOLDYN.GUI.PlotNetCDFVariableDialog.SettingsDialog \textit{(class)}!nMOLDYN.GUI.PlotNetCDFVariableDialog.SettingsDialog.buttonbox \textit{(method)}}

    \vspace{0.5ex}

\hspace{.8\funcindent}\begin{boxedminipage}{\funcwidth}

    \raggedright \textbf{buttonbox}(\textit{self})

    \vspace{-1.5ex}

    \rule{\textwidth}{0.5\fboxrule}
\setlength{\parskip}{2ex}
    Add standard button box.

\setlength{\parskip}{1ex}
    \end{boxedminipage}

    \label{nMOLDYN:GUI:PlotNetCDFVariableDialog:SettingsDialog:ok}
    \index{nMOLDYN \textit{(package)}!nMOLDYN.GUI \textit{(package)}!nMOLDYN.GUI.PlotNetCDFVariableDialog \textit{(module)}!nMOLDYN.GUI.PlotNetCDFVariableDialog.SettingsDialog \textit{(class)}!nMOLDYN.GUI.PlotNetCDFVariableDialog.SettingsDialog.ok \textit{(method)}}

    \vspace{0.5ex}

\hspace{.8\funcindent}\begin{boxedminipage}{\funcwidth}

    \raggedright \textbf{ok}(\textit{self}, \textit{event}={\tt None})

\setlength{\parskip}{2ex}
\setlength{\parskip}{1ex}
    \end{boxedminipage}

    \label{nMOLDYN:GUI:PlotNetCDFVariableDialog:SettingsDialog:cancel}
    \index{nMOLDYN \textit{(package)}!nMOLDYN.GUI \textit{(package)}!nMOLDYN.GUI.PlotNetCDFVariableDialog \textit{(module)}!nMOLDYN.GUI.PlotNetCDFVariableDialog.SettingsDialog \textit{(class)}!nMOLDYN.GUI.PlotNetCDFVariableDialog.SettingsDialog.cancel \textit{(method)}}

    \vspace{0.5ex}

\hspace{.8\funcindent}\begin{boxedminipage}{\funcwidth}

    \raggedright \textbf{cancel}(\textit{self}, \textit{event}={\tt None})

\setlength{\parskip}{2ex}
\setlength{\parskip}{1ex}
    \end{boxedminipage}

    \label{nMOLDYN:GUI:PlotNetCDFVariableDialog:SettingsDialog:validate}
    \index{nMOLDYN \textit{(package)}!nMOLDYN.GUI \textit{(package)}!nMOLDYN.GUI.PlotNetCDFVariableDialog \textit{(module)}!nMOLDYN.GUI.PlotNetCDFVariableDialog.SettingsDialog \textit{(class)}!nMOLDYN.GUI.PlotNetCDFVariableDialog.SettingsDialog.validate \textit{(method)}}

    \vspace{0.5ex}

\hspace{.8\funcindent}\begin{boxedminipage}{\funcwidth}

    \raggedright \textbf{validate}(\textit{self})

\setlength{\parskip}{2ex}
\setlength{\parskip}{1ex}
    \end{boxedminipage}

    \label{nMOLDYN:GUI:PlotNetCDFVariableDialog:SettingsDialog:apply}
    \index{nMOLDYN \textit{(package)}!nMOLDYN.GUI \textit{(package)}!nMOLDYN.GUI.PlotNetCDFVariableDialog \textit{(module)}!nMOLDYN.GUI.PlotNetCDFVariableDialog.SettingsDialog \textit{(class)}!nMOLDYN.GUI.PlotNetCDFVariableDialog.SettingsDialog.apply \textit{(method)}}

    \vspace{0.5ex}

\hspace{.8\funcindent}\begin{boxedminipage}{\funcwidth}

    \raggedright \textbf{apply}(\textit{self})

\setlength{\parskip}{2ex}
\setlength{\parskip}{1ex}
    \end{boxedminipage}

    \label{nMOLDYN:GUI:PlotNetCDFVariableDialog:SettingsDialog:initGlobalSettings}
    \index{nMOLDYN \textit{(package)}!nMOLDYN.GUI \textit{(package)}!nMOLDYN.GUI.PlotNetCDFVariableDialog \textit{(module)}!nMOLDYN.GUI.PlotNetCDFVariableDialog.SettingsDialog \textit{(class)}!nMOLDYN.GUI.PlotNetCDFVariableDialog.SettingsDialog.initGlobalSettings \textit{(method)}}

    \vspace{0.5ex}

\hspace{.8\funcindent}\begin{boxedminipage}{\funcwidth}

    \raggedright \textbf{initGlobalSettings}(\textit{self})

\setlength{\parskip}{2ex}
\setlength{\parskip}{1ex}
    \end{boxedminipage}

    \label{nMOLDYN:GUI:PlotNetCDFVariableDialog:SettingsDialog:initPlotSettings}
    \index{nMOLDYN \textit{(package)}!nMOLDYN.GUI \textit{(package)}!nMOLDYN.GUI.PlotNetCDFVariableDialog \textit{(module)}!nMOLDYN.GUI.PlotNetCDFVariableDialog.SettingsDialog \textit{(class)}!nMOLDYN.GUI.PlotNetCDFVariableDialog.SettingsDialog.initPlotSettings \textit{(method)}}

    \vspace{0.5ex}

\hspace{.8\funcindent}\begin{boxedminipage}{\funcwidth}

    \raggedright \textbf{initPlotSettings}(\textit{self})

\setlength{\parskip}{2ex}
\setlength{\parskip}{1ex}
    \end{boxedminipage}

    \label{nMOLDYN:GUI:PlotNetCDFVariableDialog:SettingsDialog:storeSettings}
    \index{nMOLDYN \textit{(package)}!nMOLDYN.GUI \textit{(package)}!nMOLDYN.GUI.PlotNetCDFVariableDialog \textit{(module)}!nMOLDYN.GUI.PlotNetCDFVariableDialog.SettingsDialog \textit{(class)}!nMOLDYN.GUI.PlotNetCDFVariableDialog.SettingsDialog.storeSettings \textit{(method)}}

    \vspace{0.5ex}

\hspace{.8\funcindent}\begin{boxedminipage}{\funcwidth}

    \raggedright \textbf{storeSettings}(\textit{self})

\setlength{\parskip}{2ex}
\setlength{\parskip}{1ex}
    \end{boxedminipage}

    \label{nMOLDYN:GUI:PlotNetCDFVariableDialog:SettingsDialog:selectColor}
    \index{nMOLDYN \textit{(package)}!nMOLDYN.GUI \textit{(package)}!nMOLDYN.GUI.PlotNetCDFVariableDialog \textit{(module)}!nMOLDYN.GUI.PlotNetCDFVariableDialog.SettingsDialog \textit{(class)}!nMOLDYN.GUI.PlotNetCDFVariableDialog.SettingsDialog.selectColor \textit{(method)}}

    \vspace{0.5ex}

\hspace{.8\funcindent}\begin{boxedminipage}{\funcwidth}

    \raggedright \textbf{selectColor}(\textit{self}, \textit{widget})

\setlength{\parskip}{2ex}
\setlength{\parskip}{1ex}
    \end{boxedminipage}

    \label{nMOLDYN:GUI:PlotNetCDFVariableDialog:SettingsDialog:changeSettings}
    \index{nMOLDYN \textit{(package)}!nMOLDYN.GUI \textit{(package)}!nMOLDYN.GUI.PlotNetCDFVariableDialog \textit{(module)}!nMOLDYN.GUI.PlotNetCDFVariableDialog.SettingsDialog \textit{(class)}!nMOLDYN.GUI.PlotNetCDFVariableDialog.SettingsDialog.changeSettings \textit{(method)}}

    \vspace{0.5ex}

\hspace{.8\funcindent}\begin{boxedminipage}{\funcwidth}

    \raggedright \textbf{changeSettings}(\textit{self}, \textit{event}, \textit{widgetName})

    \vspace{-1.5ex}

    \rule{\textwidth}{0.5\fboxrule}
\setlength{\parskip}{2ex}
    Argument:

    \begin{itemize}
    \setlength{\parskip}{0.6ex}
      \item event: either a Tkinter event, either a Tkinter control variable 
        value that has been traced for changes.

    \end{itemize}

\setlength{\parskip}{1ex}
    \end{boxedminipage}

    \label{nMOLDYN:GUI:PlotNetCDFVariableDialog:SettingsDialog:removePreviousPlotSettings}
    \index{nMOLDYN \textit{(package)}!nMOLDYN.GUI \textit{(package)}!nMOLDYN.GUI.PlotNetCDFVariableDialog \textit{(module)}!nMOLDYN.GUI.PlotNetCDFVariableDialog.SettingsDialog \textit{(class)}!nMOLDYN.GUI.PlotNetCDFVariableDialog.SettingsDialog.removePreviousPlotSettings \textit{(method)}}

    \vspace{0.5ex}

\hspace{.8\funcindent}\begin{boxedminipage}{\funcwidth}

    \raggedright \textbf{removePreviousPlotSettings}(\textit{self})

    \vspace{-1.5ex}

    \rule{\textwidth}{0.5\fboxrule}
\setlength{\parskip}{2ex}
    This method removes the previous plot settings widgets.

\setlength{\parskip}{1ex}
    \end{boxedminipage}

    \label{nMOLDYN:GUI:PlotNetCDFVariableDialog:SettingsDialog:addPlotSettingsWidgets}
    \index{nMOLDYN \textit{(package)}!nMOLDYN.GUI \textit{(package)}!nMOLDYN.GUI.PlotNetCDFVariableDialog \textit{(module)}!nMOLDYN.GUI.PlotNetCDFVariableDialog.SettingsDialog \textit{(class)}!nMOLDYN.GUI.PlotNetCDFVariableDialog.SettingsDialog.addPlotSettingsWidgets \textit{(method)}}

    \vspace{0.5ex}

\hspace{.8\funcindent}\begin{boxedminipage}{\funcwidth}

    \raggedright \textbf{addPlotSettingsWidgets}(\textit{self}, \textit{event}={\tt None})

\setlength{\parskip}{2ex}
\setlength{\parskip}{1ex}
    \end{boxedminipage}

    \index{nMOLDYN \textit{(package)}!nMOLDYN.GUI \textit{(package)}!nMOLDYN.GUI.PlotNetCDFVariableDialog \textit{(module)}!nMOLDYN.GUI.PlotNetCDFVariableDialog.SettingsDialog \textit{(class)}|)}

%%%%%%%%%%%%%%%%%%%%%%%%%%%%%%%%%%%%%%%%%%%%%%%%%%%%%%%%%%%%%%%%%%%%%%%%%%%
%%                           Class Description                           %%
%%%%%%%%%%%%%%%%%%%%%%%%%%%%%%%%%%%%%%%%%%%%%%%%%%%%%%%%%%%%%%%%%%%%%%%%%%%

    \index{nMOLDYN \textit{(package)}!nMOLDYN.GUI \textit{(package)}!nMOLDYN.GUI.PlotNetCDFVariableDialog \textit{(module)}!nMOLDYN.GUI.PlotNetCDFVariableDialog.PlotNetCDFVariableDialog \textit{(class)}|(}
\subsection{Class PlotNetCDFVariableDialog}

    \label{nMOLDYN:GUI:PlotNetCDFVariableDialog:PlotNetCDFVariableDialog}
\begin{tabular}{cccccc}
% Line for nMOLDYN.GUI.Widgets.Toplevel, linespec=[False]
\multicolumn{2}{r}{\settowidth{\BCL}{nMOLDYN.GUI.Widgets.Toplevel}\multirow{2}{\BCL}{nMOLDYN.GUI.Widgets.Toplevel}}
&&
  \\\cline{3-3}
  &&\multicolumn{1}{c|}{}
&&
  \\
&&\multicolumn{2}{l}{\textbf{nMOLDYN.GUI.PlotNetCDFVariableDialog.PlotNetCDFVariableDialog}}
\end{tabular}

Sets up a dialog used to plot variables present in a NetCDF file.


%%%%%%%%%%%%%%%%%%%%%%%%%%%%%%%%%%%%%%%%%%%%%%%%%%%%%%%%%%%%%%%%%%%%%%%%%%%
%%                                Methods                                %%
%%%%%%%%%%%%%%%%%%%%%%%%%%%%%%%%%%%%%%%%%%%%%%%%%%%%%%%%%%%%%%%%%%%%%%%%%%%

  \subsubsection{Methods}

    \label{nMOLDYN:GUI:PlotNetCDFVariableDialog:PlotNetCDFVariableDialog:__init__}
    \index{nMOLDYN \textit{(package)}!nMOLDYN.GUI \textit{(package)}!nMOLDYN.GUI.PlotNetCDFVariableDialog \textit{(module)}!nMOLDYN.GUI.PlotNetCDFVariableDialog.PlotNetCDFVariableDialog \textit{(class)}!nMOLDYN.GUI.PlotNetCDFVariableDialog.PlotNetCDFVariableDialog.\_\_init\_\_ \textit{(method)}}

    \vspace{0.5ex}

\hspace{.8\funcindent}\begin{boxedminipage}{\funcwidth}

    \raggedright \textbf{\_\_init\_\_}(\textit{self}, \textit{parent}, \textit{title}={\tt None}, \textit{netcdf}={\tt None}, \textit{xVar}={\tt None}, \textit{yVar}={\tt None}, \textit{zVar}={\tt None})

    \vspace{-1.5ex}

    \rule{\textwidth}{0.5\fboxrule}
\setlength{\parskip}{2ex}
    The constructor.

\setlength{\parskip}{1ex}
      \textbf{Parameters}
      \vspace{-1ex}

      \begin{quote}
        \begin{Ventry}{xxxxxx}

          \item[parent]

          the parent widget.

          \item[title]

          a string specifying the title of the dialog.

            {\it (type=string)}

          \item[netcdf]

          the name of a NetCDF file to plot (string) or an opened NetCDF 
          trajectory file.

            {\it (type=a string or a Scientific.IO.NetCDF.\_NetCDFFile object)}

          \item[xVar]

          the NetCDF variable name of the X variable to plot.

            {\it (type=string)}

          \item[yVar]

          the NetCDF variable name of the Y variable to plot.

            {\it (type=)}

          \item[zVar]

          the NetCDF variable name of the Z variable to plot.

            {\it (type=)}

        \end{Ventry}

      \end{quote}

    \end{boxedminipage}

    \label{nMOLDYN:GUI:PlotNetCDFVariableDialog:PlotNetCDFVariableDialog:body}
    \index{nMOLDYN \textit{(package)}!nMOLDYN.GUI \textit{(package)}!nMOLDYN.GUI.PlotNetCDFVariableDialog \textit{(module)}!nMOLDYN.GUI.PlotNetCDFVariableDialog.PlotNetCDFVariableDialog \textit{(class)}!nMOLDYN.GUI.PlotNetCDFVariableDialog.PlotNetCDFVariableDialog.body \textit{(method)}}

    \vspace{0.5ex}

\hspace{.8\funcindent}\begin{boxedminipage}{\funcwidth}

    \raggedright \textbf{body}(\textit{self}, \textit{master})

    \vspace{-1.5ex}

    \rule{\textwidth}{0.5\fboxrule}
\setlength{\parskip}{2ex}
    Create dialog body. Return widget that should have initial focus.

\setlength{\parskip}{1ex}
    \end{boxedminipage}

    \label{nMOLDYN:GUI:PlotNetCDFVariableDialog:PlotNetCDFVariableDialog:buttonbox}
    \index{nMOLDYN \textit{(package)}!nMOLDYN.GUI \textit{(package)}!nMOLDYN.GUI.PlotNetCDFVariableDialog \textit{(module)}!nMOLDYN.GUI.PlotNetCDFVariableDialog.PlotNetCDFVariableDialog \textit{(class)}!nMOLDYN.GUI.PlotNetCDFVariableDialog.PlotNetCDFVariableDialog.buttonbox \textit{(method)}}

    \vspace{0.5ex}

\hspace{.8\funcindent}\begin{boxedminipage}{\funcwidth}

    \raggedright \textbf{buttonbox}(\textit{self})

    \vspace{-1.5ex}

    \rule{\textwidth}{0.5\fboxrule}
\setlength{\parskip}{2ex}
    Add standard button box.

\setlength{\parskip}{1ex}
    \end{boxedminipage}

    \label{nMOLDYN:GUI:PlotNetCDFVariableDialog:PlotNetCDFVariableDialog:ok}
    \index{nMOLDYN \textit{(package)}!nMOLDYN.GUI \textit{(package)}!nMOLDYN.GUI.PlotNetCDFVariableDialog \textit{(module)}!nMOLDYN.GUI.PlotNetCDFVariableDialog.PlotNetCDFVariableDialog \textit{(class)}!nMOLDYN.GUI.PlotNetCDFVariableDialog.PlotNetCDFVariableDialog.ok \textit{(method)}}

    \vspace{0.5ex}

\hspace{.8\funcindent}\begin{boxedminipage}{\funcwidth}

    \raggedright \textbf{ok}(\textit{self}, \textit{event}={\tt None})

\setlength{\parskip}{2ex}
\setlength{\parskip}{1ex}
    \end{boxedminipage}

    \label{nMOLDYN:GUI:PlotNetCDFVariableDialog:PlotNetCDFVariableDialog:cancel}
    \index{nMOLDYN \textit{(package)}!nMOLDYN.GUI \textit{(package)}!nMOLDYN.GUI.PlotNetCDFVariableDialog \textit{(module)}!nMOLDYN.GUI.PlotNetCDFVariableDialog.PlotNetCDFVariableDialog \textit{(class)}!nMOLDYN.GUI.PlotNetCDFVariableDialog.PlotNetCDFVariableDialog.cancel \textit{(method)}}

    \vspace{0.5ex}

\hspace{.8\funcindent}\begin{boxedminipage}{\funcwidth}

    \raggedright \textbf{cancel}(\textit{self}, \textit{dialog}, \textit{event}={\tt None})

\setlength{\parskip}{2ex}
\setlength{\parskip}{1ex}
    \end{boxedminipage}

    \label{nMOLDYN:GUI:PlotNetCDFVariableDialog:PlotNetCDFVariableDialog:validate}
    \index{nMOLDYN \textit{(package)}!nMOLDYN.GUI \textit{(package)}!nMOLDYN.GUI.PlotNetCDFVariableDialog \textit{(module)}!nMOLDYN.GUI.PlotNetCDFVariableDialog.PlotNetCDFVariableDialog \textit{(class)}!nMOLDYN.GUI.PlotNetCDFVariableDialog.PlotNetCDFVariableDialog.validate \textit{(method)}}

    \vspace{0.5ex}

\hspace{.8\funcindent}\begin{boxedminipage}{\funcwidth}

    \raggedright \textbf{validate}(\textit{self})

\setlength{\parskip}{2ex}
\setlength{\parskip}{1ex}
    \end{boxedminipage}

    \label{nMOLDYN:GUI:PlotNetCDFVariableDialog:PlotNetCDFVariableDialog:apply}
    \index{nMOLDYN \textit{(package)}!nMOLDYN.GUI \textit{(package)}!nMOLDYN.GUI.PlotNetCDFVariableDialog \textit{(module)}!nMOLDYN.GUI.PlotNetCDFVariableDialog.PlotNetCDFVariableDialog \textit{(class)}!nMOLDYN.GUI.PlotNetCDFVariableDialog.PlotNetCDFVariableDialog.apply \textit{(method)}}

    \vspace{0.5ex}

\hspace{.8\funcindent}\begin{boxedminipage}{\funcwidth}

    \raggedright \textbf{apply}(\textit{self})

\setlength{\parskip}{2ex}
\setlength{\parskip}{1ex}
    \end{boxedminipage}

    \label{nMOLDYN:GUI:PlotNetCDFVariableDialog:PlotNetCDFVariableDialog:openSettingsDialog}
    \index{nMOLDYN \textit{(package)}!nMOLDYN.GUI \textit{(package)}!nMOLDYN.GUI.PlotNetCDFVariableDialog \textit{(module)}!nMOLDYN.GUI.PlotNetCDFVariableDialog.PlotNetCDFVariableDialog \textit{(class)}!nMOLDYN.GUI.PlotNetCDFVariableDialog.PlotNetCDFVariableDialog.openSettingsDialog \textit{(method)}}

    \vspace{0.5ex}

\hspace{.8\funcindent}\begin{boxedminipage}{\funcwidth}

    \raggedright \textbf{openSettingsDialog}(\textit{self})

    \vspace{-1.5ex}

    \rule{\textwidth}{0.5\fboxrule}
\setlength{\parskip}{2ex}
    This method will open the dialog to set up the global settings.

\setlength{\parskip}{1ex}
    \end{boxedminipage}

    \label{nMOLDYN:GUI:PlotNetCDFVariableDialog:PlotNetCDFVariableDialog:resetPlots}
    \index{nMOLDYN \textit{(package)}!nMOLDYN.GUI \textit{(package)}!nMOLDYN.GUI.PlotNetCDFVariableDialog \textit{(module)}!nMOLDYN.GUI.PlotNetCDFVariableDialog.PlotNetCDFVariableDialog \textit{(class)}!nMOLDYN.GUI.PlotNetCDFVariableDialog.PlotNetCDFVariableDialog.resetPlots \textit{(method)}}

    \vspace{0.5ex}

\hspace{.8\funcindent}\begin{boxedminipage}{\funcwidth}

    \raggedright \textbf{resetPlots}(\textit{self})

    \vspace{-1.5ex}

    \rule{\textwidth}{0.5\fboxrule}
\setlength{\parskip}{2ex}
    This method will clear up all the displayed plots.

\setlength{\parskip}{1ex}
    \end{boxedminipage}

    \label{nMOLDYN:GUI:PlotNetCDFVariableDialog:PlotNetCDFVariableDialog:exportPlotDialog}
    \index{nMOLDYN \textit{(package)}!nMOLDYN.GUI \textit{(package)}!nMOLDYN.GUI.PlotNetCDFVariableDialog \textit{(module)}!nMOLDYN.GUI.PlotNetCDFVariableDialog.PlotNetCDFVariableDialog \textit{(class)}!nMOLDYN.GUI.PlotNetCDFVariableDialog.PlotNetCDFVariableDialog.exportPlotDialog \textit{(method)}}

    \vspace{0.5ex}

\hspace{.8\funcindent}\begin{boxedminipage}{\funcwidth}

    \raggedright \textbf{exportPlotDialog}(\textit{self})

    \vspace{-1.5ex}

    \rule{\textwidth}{0.5\fboxrule}
\setlength{\parskip}{2ex}
    This method pops up a dialog from which the plotted datas can be 
    exported to an ASCII file.

\setlength{\parskip}{1ex}
    \end{boxedminipage}

    \label{nMOLDYN:GUI:PlotNetCDFVariableDialog:PlotNetCDFVariableDialog:exportPlot}
    \index{nMOLDYN \textit{(package)}!nMOLDYN.GUI \textit{(package)}!nMOLDYN.GUI.PlotNetCDFVariableDialog \textit{(module)}!nMOLDYN.GUI.PlotNetCDFVariableDialog.PlotNetCDFVariableDialog \textit{(class)}!nMOLDYN.GUI.PlotNetCDFVariableDialog.PlotNetCDFVariableDialog.exportPlot \textit{(method)}}

    \vspace{0.5ex}

\hspace{.8\funcindent}\begin{boxedminipage}{\funcwidth}

    \raggedright \textbf{exportPlot}(\textit{self}, \textit{event}={\tt None})

    \vspace{-1.5ex}

    \rule{\textwidth}{0.5\fboxrule}
\setlength{\parskip}{2ex}
    This method exports plotted datas to an ASCII file.

\setlength{\parskip}{1ex}
    \end{boxedminipage}

    \label{nMOLDYN:GUI:PlotNetCDFVariableDialog:PlotNetCDFVariableDialog:selectXVariable}
    \index{nMOLDYN \textit{(package)}!nMOLDYN.GUI \textit{(package)}!nMOLDYN.GUI.PlotNetCDFVariableDialog \textit{(module)}!nMOLDYN.GUI.PlotNetCDFVariableDialog.PlotNetCDFVariableDialog \textit{(class)}!nMOLDYN.GUI.PlotNetCDFVariableDialog.PlotNetCDFVariableDialog.selectXVariable \textit{(method)}}

    \vspace{0.5ex}

\hspace{.8\funcindent}\begin{boxedminipage}{\funcwidth}

    \raggedright \textbf{selectXVariable}(\textit{self}, \textit{event})

\setlength{\parskip}{2ex}
\setlength{\parskip}{1ex}
    \end{boxedminipage}

    \label{nMOLDYN:GUI:PlotNetCDFVariableDialog:PlotNetCDFVariableDialog:displayVariables}
    \index{nMOLDYN \textit{(package)}!nMOLDYN.GUI \textit{(package)}!nMOLDYN.GUI.PlotNetCDFVariableDialog \textit{(module)}!nMOLDYN.GUI.PlotNetCDFVariableDialog.PlotNetCDFVariableDialog \textit{(class)}!nMOLDYN.GUI.PlotNetCDFVariableDialog.PlotNetCDFVariableDialog.displayVariables \textit{(method)}}

    \vspace{0.5ex}

\hspace{.8\funcindent}\begin{boxedminipage}{\funcwidth}

    \raggedright \textbf{displayVariables}(\textit{self})

    \vspace{-1.5ex}

    \rule{\textwidth}{0.5\fboxrule}
\setlength{\parskip}{2ex}
    This method display the numeric variables found in the NetCDF file into
    their appropriate listbox.

\setlength{\parskip}{1ex}
    \end{boxedminipage}

    \label{nMOLDYN:GUI:PlotNetCDFVariableDialog:PlotNetCDFVariableDialog:openNetCDF}
    \index{nMOLDYN \textit{(package)}!nMOLDYN.GUI \textit{(package)}!nMOLDYN.GUI.PlotNetCDFVariableDialog \textit{(module)}!nMOLDYN.GUI.PlotNetCDFVariableDialog.PlotNetCDFVariableDialog \textit{(class)}!nMOLDYN.GUI.PlotNetCDFVariableDialog.PlotNetCDFVariableDialog.openNetCDF \textit{(method)}}

    \vspace{0.5ex}

\hspace{.8\funcindent}\begin{boxedminipage}{\funcwidth}

    \raggedright \textbf{openNetCDF}(\textit{self}, \textit{event}={\tt None})

    \vspace{-1.5ex}

    \rule{\textwidth}{0.5\fboxrule}
\setlength{\parskip}{2ex}
\begin{alltt}

This method opens a NetCDF file and updates the dialog with the data read from that file.
Arguments:
    -event: Tkinter event.
\end{alltt}

\setlength{\parskip}{1ex}
    \end{boxedminipage}

    \label{nMOLDYN:GUI:PlotNetCDFVariableDialog:PlotNetCDFVariableDialog:plotXY}
    \index{nMOLDYN \textit{(package)}!nMOLDYN.GUI \textit{(package)}!nMOLDYN.GUI.PlotNetCDFVariableDialog \textit{(module)}!nMOLDYN.GUI.PlotNetCDFVariableDialog.PlotNetCDFVariableDialog \textit{(class)}!nMOLDYN.GUI.PlotNetCDFVariableDialog.PlotNetCDFVariableDialog.plotXY \textit{(method)}}

    \vspace{0.5ex}

\hspace{.8\funcindent}\begin{boxedminipage}{\funcwidth}

    \raggedright \textbf{plotXY}(\textit{self})

    \vspace{-1.5ex}

    \rule{\textwidth}{0.5\fboxrule}
\setlength{\parskip}{2ex}
    This method display a 2D plot.

\setlength{\parskip}{1ex}
    \end{boxedminipage}

    \label{nMOLDYN:GUI:PlotNetCDFVariableDialog:PlotNetCDFVariableDialog:displayPlotSlices}
    \index{nMOLDYN \textit{(package)}!nMOLDYN.GUI \textit{(package)}!nMOLDYN.GUI.PlotNetCDFVariableDialog \textit{(module)}!nMOLDYN.GUI.PlotNetCDFVariableDialog.PlotNetCDFVariableDialog \textit{(class)}!nMOLDYN.GUI.PlotNetCDFVariableDialog.PlotNetCDFVariableDialog.displayPlotSlices \textit{(method)}}

    \vspace{0.5ex}

\hspace{.8\funcindent}\begin{boxedminipage}{\funcwidth}

    \raggedright \textbf{displayPlotSlices}(\textit{self}, \textit{event})

    \vspace{-1.5ex}

    \rule{\textwidth}{0.5\fboxrule}
\setlength{\parskip}{2ex}
    This call back plot the orthogonal slices defined by the moving cursor 
    of a 3D plot.

\setlength{\parskip}{1ex}
    \end{boxedminipage}

    \label{nMOLDYN:GUI:PlotNetCDFVariableDialog:PlotNetCDFVariableDialog:plotXYZ}
    \index{nMOLDYN \textit{(package)}!nMOLDYN.GUI \textit{(package)}!nMOLDYN.GUI.PlotNetCDFVariableDialog \textit{(module)}!nMOLDYN.GUI.PlotNetCDFVariableDialog.PlotNetCDFVariableDialog \textit{(class)}!nMOLDYN.GUI.PlotNetCDFVariableDialog.PlotNetCDFVariableDialog.plotXYZ \textit{(method)}}

    \vspace{0.5ex}

\hspace{.8\funcindent}\begin{boxedminipage}{\funcwidth}

    \raggedright \textbf{plotXYZ}(\textit{self})

    \vspace{-1.5ex}

    \rule{\textwidth}{0.5\fboxrule}
\setlength{\parskip}{2ex}
    This method display a 2D plot.

\setlength{\parskip}{1ex}
    \end{boxedminipage}

    \index{nMOLDYN \textit{(package)}!nMOLDYN.GUI \textit{(package)}!nMOLDYN.GUI.PlotNetCDFVariableDialog \textit{(module)}!nMOLDYN.GUI.PlotNetCDFVariableDialog.PlotNetCDFVariableDialog \textit{(class)}|)}
    \index{nMOLDYN \textit{(package)}!nMOLDYN.GUI \textit{(package)}!nMOLDYN.GUI.PlotNetCDFVariableDialog \textit{(module)}|)}
