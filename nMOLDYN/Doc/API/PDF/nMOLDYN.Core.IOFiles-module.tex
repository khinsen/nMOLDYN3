%
% API Documentation for nMOLDYN
% Module nMOLDYN.Core.IOFiles
%
% Generated by epydoc 3.0.1
% [Thu Oct  8 16:59:59 2009]
%

%%%%%%%%%%%%%%%%%%%%%%%%%%%%%%%%%%%%%%%%%%%%%%%%%%%%%%%%%%%%%%%%%%%%%%%%%%%
%%                          Module Description                           %%
%%%%%%%%%%%%%%%%%%%%%%%%%%%%%%%%%%%%%%%%%%%%%%%%%%%%%%%%%%%%%%%%%%%%%%%%%%%

    \index{nMOLDYN \textit{(package)}!nMOLDYN.Core \textit{(package)}!nMOLDYN.Core.IOFiles \textit{(module)}|(}
\section{Module nMOLDYN.Core.IOFiles}

    \label{nMOLDYN:Core:IOFiles}
\begin{alltt}
This module implements IO-related classes, functions and procedures.

Classes:
    * TemporaryFile            : creates a temporary file stroring the evolution of an analysis.
    * EndOfFile                : an empty dummy class used by {\textbar}DCDReader{\textbar}.
    * FortranBinaryFile        : sets up a binary file reader.
    * DCDFile                  : sets up a DCD file reader.
    * AmberNetCDFConverter     : converts a trajectory from Amber {\textgreater} 9 to a MMTK NetCDF trajectory.
    * CHARMMConverter          : converts a trajectory from CHARMM to a MMTK NetCDF trajectory.
    * DL\_POLYConverter         : converts a trajectory from DL\_POLY {\textgreater} 9 to a MMTK NetCDF trajectory.
    * MaterialsStudioConverter : converts a trajectory from MaterialsStudio {\textgreater} 9 to a MMTK NetCDF trajectory.
    * NAMDConverter            : converts a trajectory from NAMD to a MMTK NetCDF trajectory.
    * VASPConverter            : converts a trajectory from VASP {\textgreater} 9 to a MMTK NetCDF trajectory.
        
Procedures:
    * convertNetCDFToASCII: converts a NetCDF file into an ASCII file.
    * convertASCIIToNetCDF: converts an ASCII file into a NetCDF file.
\end{alltt}


%%%%%%%%%%%%%%%%%%%%%%%%%%%%%%%%%%%%%%%%%%%%%%%%%%%%%%%%%%%%%%%%%%%%%%%%%%%
%%                               Functions                               %%
%%%%%%%%%%%%%%%%%%%%%%%%%%%%%%%%%%%%%%%%%%%%%%%%%%%%%%%%%%%%%%%%%%%%%%%%%%%

  \subsection{Functions}

    \label{nMOLDYN:Core:IOFiles:convertNetCDFToASCII}
    \index{nMOLDYN \textit{(package)}!nMOLDYN.Core \textit{(package)}!nMOLDYN.Core.IOFiles \textit{(module)}!nMOLDYN.Core.IOFiles.convertNetCDFToASCII \textit{(function)}}

    \vspace{0.5ex}

\hspace{.8\funcindent}\begin{boxedminipage}{\funcwidth}

    \raggedright \textbf{convertNetCDFToASCII}(\textit{inputFile}, \textit{outputFile}, \textit{variables}, \textit{floatPrecision}={\tt 9}, \textit{doublePrecision}={\tt 17})

    \vspace{-1.5ex}

    \rule{\textwidth}{0.5\fboxrule}
\setlength{\parskip}{2ex}
    Converts a file in NetCDF format to a file in ASCII/CDL format using 
    the ncdump program provided with the netcdf library.

\setlength{\parskip}{1ex}
      \textbf{Parameters}
      \vspace{-1ex}

      \begin{quote}
        \begin{Ventry}{xxxxxxxxxxxxxxx}

          \item[inputFile]

          the name of the NetCDF input file.

            {\it (type=string)}

          \item[outputFile]

          the name of the CDL output file.

            {\it (type=string)}

          \item[variables]

          list of the NetCDF variables names (string) to extract from the 
          NetCDF file for conversion.

            {\it (type=list)}

          \item[floatPrecision]

          the precision on the float numbers.

            {\it (type=integer)}

          \item[doublePrecision]

          the precision on the double numbers.

            {\it (type=integer)}

        \end{Ventry}

      \end{quote}

    \end{boxedminipage}

    \label{nMOLDYN:Core:IOFiles:convertASCIIToNetCDF}
    \index{nMOLDYN \textit{(package)}!nMOLDYN.Core \textit{(package)}!nMOLDYN.Core.IOFiles \textit{(module)}!nMOLDYN.Core.IOFiles.convertASCIIToNetCDF \textit{(function)}}

    \vspace{0.5ex}

\hspace{.8\funcindent}\begin{boxedminipage}{\funcwidth}

    \raggedright \textbf{convertASCIIToNetCDF}(\textit{inputFile}, \textit{outputFile})

    \vspace{-1.5ex}

    \rule{\textwidth}{0.5\fboxrule}
\setlength{\parskip}{2ex}
    Converts a file in ASCII format to a file in NetCDF format using the 
    ncgen program provided with the netcdf library.

\setlength{\parskip}{1ex}
      \textbf{Parameters}
      \vspace{-1ex}

      \begin{quote}
        \begin{Ventry}{xxxxxxxxxx}

          \item[inputFile]

          the name of the NetCDF input file.

            {\it (type=string)}

          \item[outputFile]

          the name of the CDL output file.

            {\it (type=string)}

        \end{Ventry}

      \end{quote}

    \end{boxedminipage}


%%%%%%%%%%%%%%%%%%%%%%%%%%%%%%%%%%%%%%%%%%%%%%%%%%%%%%%%%%%%%%%%%%%%%%%%%%%
%%                           Class Description                           %%
%%%%%%%%%%%%%%%%%%%%%%%%%%%%%%%%%%%%%%%%%%%%%%%%%%%%%%%%%%%%%%%%%%%%%%%%%%%

    \index{nMOLDYN \textit{(package)}!nMOLDYN.Core \textit{(package)}!nMOLDYN.Core.IOFiles \textit{(module)}!nMOLDYN.Core.IOFiles.TemporaryFile \textit{(class)}|(}
\subsection{Class TemporaryFile}

    \label{nMOLDYN:Core:IOFiles:TemporaryFile}
Creates a temporary file used to monitor (progress, start, end  ...) an 
analysis.


%%%%%%%%%%%%%%%%%%%%%%%%%%%%%%%%%%%%%%%%%%%%%%%%%%%%%%%%%%%%%%%%%%%%%%%%%%%
%%                                Methods                                %%
%%%%%%%%%%%%%%%%%%%%%%%%%%%%%%%%%%%%%%%%%%%%%%%%%%%%%%%%%%%%%%%%%%%%%%%%%%%

  \subsubsection{Methods}

    \label{nMOLDYN:Core:IOFiles:TemporaryFile:__init__}
    \index{nMOLDYN \textit{(package)}!nMOLDYN.Core \textit{(package)}!nMOLDYN.Core.IOFiles \textit{(module)}!nMOLDYN.Core.IOFiles.TemporaryFile \textit{(class)}!nMOLDYN.Core.IOFiles.TemporaryFile.\_\_init\_\_ \textit{(method)}}

    \vspace{0.5ex}

\hspace{.8\funcindent}\begin{boxedminipage}{\funcwidth}

    \raggedright \textbf{\_\_init\_\_}(\textit{self}, \textit{module}={\tt ''}, \textit{statusBar}={\tt None})

    \vspace{-1.5ex}

    \rule{\textwidth}{0.5\fboxrule}
\setlength{\parskip}{2ex}
    The constructor.

\setlength{\parskip}{1ex}
      \textbf{Parameters}
      \vspace{-1ex}

      \begin{quote}
        \begin{Ventry}{xxxxxxxxx}

          \item[module]

          the name of the analysis the temporary file will be attached to.

            {\it (type=string.)}

          \item[statusBar]

          if not None, an instance of nMOLDYN.GUI.Widgets.StatusBar. The 
          status bar, attached to the analysis, to update.

            {\it (type=instance of nMOLDYN.GUI.Widgets.StatusBar)}

        \end{Ventry}

      \end{quote}

    \end{boxedminipage}

    \label{nMOLDYN:Core:IOFiles:TemporaryFile:update}
    \index{nMOLDYN \textit{(package)}!nMOLDYN.Core \textit{(package)}!nMOLDYN.Core.IOFiles \textit{(module)}!nMOLDYN.Core.IOFiles.TemporaryFile \textit{(class)}!nMOLDYN.Core.IOFiles.TemporaryFile.update \textit{(method)}}

    \vspace{0.5ex}

\hspace{.8\funcindent}\begin{boxedminipage}{\funcwidth}

    \raggedright \textbf{update}(\textit{self}, \textit{norm})

    \vspace{-1.5ex}

    \rule{\textwidth}{0.5\fboxrule}
\setlength{\parskip}{2ex}
    Updates the temporary file by writing the percentage of the job that 
    has been done.

\setlength{\parskip}{1ex}
      \textbf{Parameters}
      \vspace{-1ex}

      \begin{quote}
        \begin{Ventry}{xxxx}

          \item[norm]

          the number of steps of the outer loop of the analysis to monitor.

            {\it (type=integer.)}

        \end{Ventry}

      \end{quote}

    \end{boxedminipage}

    \label{nMOLDYN:Core:IOFiles:TemporaryFile:close}
    \index{nMOLDYN \textit{(package)}!nMOLDYN.Core \textit{(package)}!nMOLDYN.Core.IOFiles \textit{(module)}!nMOLDYN.Core.IOFiles.TemporaryFile \textit{(class)}!nMOLDYN.Core.IOFiles.TemporaryFile.close \textit{(method)}}

    \vspace{0.5ex}

\hspace{.8\funcindent}\begin{boxedminipage}{\funcwidth}

    \raggedright \textbf{close}(\textit{self})

    \vspace{-1.5ex}

    \rule{\textwidth}{0.5\fboxrule}
\setlength{\parskip}{2ex}
    Closes and removes the temporary file.

\setlength{\parskip}{1ex}
    \end{boxedminipage}

    \index{nMOLDYN \textit{(package)}!nMOLDYN.Core \textit{(package)}!nMOLDYN.Core.IOFiles \textit{(module)}!nMOLDYN.Core.IOFiles.TemporaryFile \textit{(class)}|)}

%%%%%%%%%%%%%%%%%%%%%%%%%%%%%%%%%%%%%%%%%%%%%%%%%%%%%%%%%%%%%%%%%%%%%%%%%%%
%%                           Class Description                           %%
%%%%%%%%%%%%%%%%%%%%%%%%%%%%%%%%%%%%%%%%%%%%%%%%%%%%%%%%%%%%%%%%%%%%%%%%%%%

    \index{nMOLDYN \textit{(package)}!nMOLDYN.Core \textit{(package)}!nMOLDYN.Core.IOFiles \textit{(module)}!nMOLDYN.Core.IOFiles.EndOfFile \textit{(class)}|(}
\subsection{Class EndOfFile}

    \label{nMOLDYN:Core:IOFiles:EndOfFile}
\begin{tabular}{cccccccccc}
% Line for object, linespec=[False, False, False]
\multicolumn{2}{r}{\settowidth{\BCL}{object}\multirow{2}{\BCL}{object}}
&&
&&
&&
  \\\cline{3-3}
  &&\multicolumn{1}{c|}{}
&&
&&
&&
  \\
% Line for exceptions.BaseException, linespec=[False, False]
\multicolumn{4}{r}{\settowidth{\BCL}{exceptions.BaseException}\multirow{2}{\BCL}{exceptions.BaseException}}
&&
&&
  \\\cline{5-5}
  &&&&\multicolumn{1}{c|}{}
&&
&&
  \\
% Line for exceptions.Exception, linespec=[False]
\multicolumn{6}{r}{\settowidth{\BCL}{exceptions.Exception}\multirow{2}{\BCL}{exceptions.Exception}}
&&
  \\\cline{7-7}
  &&&&&&\multicolumn{1}{c|}{}
&&
  \\
&&&&&&\multicolumn{2}{l}{\textbf{nMOLDYN.Core.IOFiles.EndOfFile}}
\end{tabular}


%%%%%%%%%%%%%%%%%%%%%%%%%%%%%%%%%%%%%%%%%%%%%%%%%%%%%%%%%%%%%%%%%%%%%%%%%%%
%%                                Methods                                %%
%%%%%%%%%%%%%%%%%%%%%%%%%%%%%%%%%%%%%%%%%%%%%%%%%%%%%%%%%%%%%%%%%%%%%%%%%%%

  \subsubsection{Methods}


\large{\textbf{\textit{Inherited from exceptions.Exception}}}

\begin{quote}
\_\_init\_\_(), \_\_new\_\_()
\end{quote}

\large{\textbf{\textit{Inherited from exceptions.BaseException}}}

\begin{quote}
\_\_delattr\_\_(), \_\_getattribute\_\_(), \_\_getitem\_\_(), \_\_getslice\_\_(), \_\_reduce\_\_(), \_\_repr\_\_(), \_\_setattr\_\_(), \_\_setstate\_\_(), \_\_str\_\_()
\end{quote}

\large{\textbf{\textit{Inherited from object}}}

\begin{quote}
\_\_hash\_\_(), \_\_reduce\_ex\_\_()
\end{quote}

%%%%%%%%%%%%%%%%%%%%%%%%%%%%%%%%%%%%%%%%%%%%%%%%%%%%%%%%%%%%%%%%%%%%%%%%%%%
%%                              Properties                               %%
%%%%%%%%%%%%%%%%%%%%%%%%%%%%%%%%%%%%%%%%%%%%%%%%%%%%%%%%%%%%%%%%%%%%%%%%%%%

  \subsubsection{Properties}

    \vspace{-1cm}
\hspace{\varindent}\begin{longtable}{|p{\varnamewidth}|p{\vardescrwidth}|l}
\cline{1-2}
\cline{1-2} \centering \textbf{Name} & \centering \textbf{Description}& \\
\cline{1-2}
\endhead\cline{1-2}\multicolumn{3}{r}{\small\textit{continued on next page}}\\\endfoot\cline{1-2}
\endlastfoot\multicolumn{2}{|l|}{\textit{Inherited from exceptions.BaseException}}\\
\multicolumn{2}{|p{\varwidth}|}{\raggedright args, message}\\
\cline{1-2}
\multicolumn{2}{|l|}{\textit{Inherited from object}}\\
\multicolumn{2}{|p{\varwidth}|}{\raggedright \_\_class\_\_}\\
\cline{1-2}
\end{longtable}

    \index{nMOLDYN \textit{(package)}!nMOLDYN.Core \textit{(package)}!nMOLDYN.Core.IOFiles \textit{(module)}!nMOLDYN.Core.IOFiles.EndOfFile \textit{(class)}|)}

%%%%%%%%%%%%%%%%%%%%%%%%%%%%%%%%%%%%%%%%%%%%%%%%%%%%%%%%%%%%%%%%%%%%%%%%%%%
%%                           Class Description                           %%
%%%%%%%%%%%%%%%%%%%%%%%%%%%%%%%%%%%%%%%%%%%%%%%%%%%%%%%%%%%%%%%%%%%%%%%%%%%

    \index{nMOLDYN \textit{(package)}!nMOLDYN.Core \textit{(package)}!nMOLDYN.Core.IOFiles \textit{(module)}!nMOLDYN.Core.IOFiles.FortranBinaryFile \textit{(class)}|(}
\subsection{Class FortranBinaryFile}

    \label{nMOLDYN:Core:IOFiles:FortranBinaryFile}
\begin{tabular}{cccccc}
% Line for object, linespec=[False]
\multicolumn{2}{r}{\settowidth{\BCL}{object}\multirow{2}{\BCL}{object}}
&&
  \\\cline{3-3}
  &&\multicolumn{1}{c|}{}
&&
  \\
&&\multicolumn{2}{l}{\textbf{nMOLDYN.Core.IOFiles.FortranBinaryFile}}
\end{tabular}

\begin{alltt}
Sets up a Fortran binary file reader. 

Comments:
    -written by Konrad Hinsen in the scope of a DCD file reader.
\end{alltt}


%%%%%%%%%%%%%%%%%%%%%%%%%%%%%%%%%%%%%%%%%%%%%%%%%%%%%%%%%%%%%%%%%%%%%%%%%%%
%%                                Methods                                %%
%%%%%%%%%%%%%%%%%%%%%%%%%%%%%%%%%%%%%%%%%%%%%%%%%%%%%%%%%%%%%%%%%%%%%%%%%%%

  \subsubsection{Methods}

    \vspace{0.5ex}

\hspace{.8\funcindent}\begin{boxedminipage}{\funcwidth}

    \raggedright \textbf{\_\_init\_\_}(\textit{self}, \textit{filename}, \textit{byte\_order}={\tt '='})

\setlength{\parskip}{2ex}
    x.\_\_init\_\_(...) initializes x; see x.\_\_class\_\_.\_\_doc\_\_ for 
    signature

\setlength{\parskip}{1ex}
      Overrides: object.\_\_init\_\_ 	extit{(inherited documentation)}

    \end{boxedminipage}

    \label{nMOLDYN:Core:IOFiles:FortranBinaryFile:__iter__}
    \index{nMOLDYN \textit{(package)}!nMOLDYN.Core \textit{(package)}!nMOLDYN.Core.IOFiles \textit{(module)}!nMOLDYN.Core.IOFiles.FortranBinaryFile \textit{(class)}!nMOLDYN.Core.IOFiles.FortranBinaryFile.\_\_iter\_\_ \textit{(method)}}

    \vspace{0.5ex}

\hspace{.8\funcindent}\begin{boxedminipage}{\funcwidth}

    \raggedright \textbf{\_\_iter\_\_}(\textit{self})

\setlength{\parskip}{2ex}
\setlength{\parskip}{1ex}
    \end{boxedminipage}

    \label{nMOLDYN:Core:IOFiles:FortranBinaryFile:next}
    \index{nMOLDYN \textit{(package)}!nMOLDYN.Core \textit{(package)}!nMOLDYN.Core.IOFiles \textit{(module)}!nMOLDYN.Core.IOFiles.FortranBinaryFile \textit{(class)}!nMOLDYN.Core.IOFiles.FortranBinaryFile.next \textit{(method)}}

    \vspace{0.5ex}

\hspace{.8\funcindent}\begin{boxedminipage}{\funcwidth}

    \raggedright \textbf{next}(\textit{self})

\setlength{\parskip}{2ex}
\setlength{\parskip}{1ex}
    \end{boxedminipage}

    \label{nMOLDYN:Core:IOFiles:FortranBinaryFile:skipRecord}
    \index{nMOLDYN \textit{(package)}!nMOLDYN.Core \textit{(package)}!nMOLDYN.Core.IOFiles \textit{(module)}!nMOLDYN.Core.IOFiles.FortranBinaryFile \textit{(class)}!nMOLDYN.Core.IOFiles.FortranBinaryFile.skipRecord \textit{(method)}}

    \vspace{0.5ex}

\hspace{.8\funcindent}\begin{boxedminipage}{\funcwidth}

    \raggedright \textbf{skipRecord}(\textit{self})

\setlength{\parskip}{2ex}
\setlength{\parskip}{1ex}
    \end{boxedminipage}

    \label{nMOLDYN:Core:IOFiles:FortranBinaryFile:getRecord}
    \index{nMOLDYN \textit{(package)}!nMOLDYN.Core \textit{(package)}!nMOLDYN.Core.IOFiles \textit{(module)}!nMOLDYN.Core.IOFiles.FortranBinaryFile \textit{(class)}!nMOLDYN.Core.IOFiles.FortranBinaryFile.getRecord \textit{(method)}}

    \vspace{0.5ex}

\hspace{.8\funcindent}\begin{boxedminipage}{\funcwidth}

    \raggedright \textbf{getRecord}(\textit{self}, \textit{format}, \textit{repeat}={\tt False})

\setlength{\parskip}{2ex}
\setlength{\parskip}{1ex}
    \end{boxedminipage}


\large{\textbf{\textit{Inherited from object}}}

\begin{quote}
\_\_delattr\_\_(), \_\_getattribute\_\_(), \_\_hash\_\_(), \_\_new\_\_(), \_\_reduce\_\_(), \_\_reduce\_ex\_\_(), \_\_repr\_\_(), \_\_setattr\_\_(), \_\_str\_\_()
\end{quote}

%%%%%%%%%%%%%%%%%%%%%%%%%%%%%%%%%%%%%%%%%%%%%%%%%%%%%%%%%%%%%%%%%%%%%%%%%%%
%%                              Properties                               %%
%%%%%%%%%%%%%%%%%%%%%%%%%%%%%%%%%%%%%%%%%%%%%%%%%%%%%%%%%%%%%%%%%%%%%%%%%%%

  \subsubsection{Properties}

    \vspace{-1cm}
\hspace{\varindent}\begin{longtable}{|p{\varnamewidth}|p{\vardescrwidth}|l}
\cline{1-2}
\cline{1-2} \centering \textbf{Name} & \centering \textbf{Description}& \\
\cline{1-2}
\endhead\cline{1-2}\multicolumn{3}{r}{\small\textit{continued on next page}}\\\endfoot\cline{1-2}
\endlastfoot\multicolumn{2}{|l|}{\textit{Inherited from object}}\\
\multicolumn{2}{|p{\varwidth}|}{\raggedright \_\_class\_\_}\\
\cline{1-2}
\end{longtable}

    \index{nMOLDYN \textit{(package)}!nMOLDYN.Core \textit{(package)}!nMOLDYN.Core.IOFiles \textit{(module)}!nMOLDYN.Core.IOFiles.FortranBinaryFile \textit{(class)}|)}

%%%%%%%%%%%%%%%%%%%%%%%%%%%%%%%%%%%%%%%%%%%%%%%%%%%%%%%%%%%%%%%%%%%%%%%%%%%
%%                           Class Description                           %%
%%%%%%%%%%%%%%%%%%%%%%%%%%%%%%%%%%%%%%%%%%%%%%%%%%%%%%%%%%%%%%%%%%%%%%%%%%%

    \index{nMOLDYN \textit{(package)}!nMOLDYN.Core \textit{(package)}!nMOLDYN.Core.IOFiles \textit{(module)}!nMOLDYN.Core.IOFiles.DCDFile \textit{(class)}|(}
\subsection{Class DCDFile}

    \label{nMOLDYN:Core:IOFiles:DCDFile}
\begin{tabular}{cccccc}
% Line for object, linespec=[False]
\multicolumn{2}{r}{\settowidth{\BCL}{object}\multirow{2}{\BCL}{object}}
&&
  \\\cline{3-3}
  &&\multicolumn{1}{c|}{}
&&
  \\
&&\multicolumn{2}{l}{\textbf{nMOLDYN.Core.IOFiles.DCDFile}}
\end{tabular}

Sets up a DCD file reader.


%%%%%%%%%%%%%%%%%%%%%%%%%%%%%%%%%%%%%%%%%%%%%%%%%%%%%%%%%%%%%%%%%%%%%%%%%%%
%%                                Methods                                %%
%%%%%%%%%%%%%%%%%%%%%%%%%%%%%%%%%%%%%%%%%%%%%%%%%%%%%%%%%%%%%%%%%%%%%%%%%%%

  \subsubsection{Methods}

    \vspace{0.5ex}

\hspace{.8\funcindent}\begin{boxedminipage}{\funcwidth}

    \raggedright \textbf{\_\_init\_\_}(\textit{self}, \textit{dcd\_filename})

    \vspace{-1.5ex}

    \rule{\textwidth}{0.5\fboxrule}
\setlength{\parskip}{2ex}
    The constructor.

\setlength{\parskip}{1ex}
      \textbf{Parameters}
      \vspace{-1ex}

      \begin{quote}
        \begin{Ventry}{xxxxxxxxxxxx}

          \item[dcd\_filename]

          the name of the DCD file to read.

            {\it (type=string.)}

        \end{Ventry}

      \end{quote}

      Overrides: object.\_\_init\_\_

    \end{boxedminipage}

    \label{nMOLDYN:Core:IOFiles:DCDFile:readStep}
    \index{nMOLDYN \textit{(package)}!nMOLDYN.Core \textit{(package)}!nMOLDYN.Core.IOFiles \textit{(module)}!nMOLDYN.Core.IOFiles.DCDFile \textit{(class)}!nMOLDYN.Core.IOFiles.DCDFile.readStep \textit{(method)}}

    \vspace{0.5ex}

\hspace{.8\funcindent}\begin{boxedminipage}{\funcwidth}

    \raggedright \textbf{readStep}(\textit{self})

    \vspace{-1.5ex}

    \rule{\textwidth}{0.5\fboxrule}
\setlength{\parskip}{2ex}
    Reads a frame of the DCD file.

\setlength{\parskip}{1ex}
    \end{boxedminipage}

    \label{nMOLDYN:Core:IOFiles:DCDFile:skipStep}
    \index{nMOLDYN \textit{(package)}!nMOLDYN.Core \textit{(package)}!nMOLDYN.Core.IOFiles \textit{(module)}!nMOLDYN.Core.IOFiles.DCDFile \textit{(class)}!nMOLDYN.Core.IOFiles.DCDFile.skipStep \textit{(method)}}

    \vspace{0.5ex}

\hspace{.8\funcindent}\begin{boxedminipage}{\funcwidth}

    \raggedright \textbf{skipStep}(\textit{self})

    \vspace{-1.5ex}

    \rule{\textwidth}{0.5\fboxrule}
\setlength{\parskip}{2ex}
    Skips a frame of the DCD file.

\setlength{\parskip}{1ex}
    \end{boxedminipage}

    \label{nMOLDYN:Core:IOFiles:DCDFile:__iter__}
    \index{nMOLDYN \textit{(package)}!nMOLDYN.Core \textit{(package)}!nMOLDYN.Core.IOFiles \textit{(module)}!nMOLDYN.Core.IOFiles.DCDFile \textit{(class)}!nMOLDYN.Core.IOFiles.DCDFile.\_\_iter\_\_ \textit{(method)}}

    \vspace{0.5ex}

\hspace{.8\funcindent}\begin{boxedminipage}{\funcwidth}

    \raggedright \textbf{\_\_iter\_\_}(\textit{self})

\setlength{\parskip}{2ex}
\setlength{\parskip}{1ex}
    \end{boxedminipage}

    \label{nMOLDYN:Core:IOFiles:DCDFile:next}
    \index{nMOLDYN \textit{(package)}!nMOLDYN.Core \textit{(package)}!nMOLDYN.Core.IOFiles \textit{(module)}!nMOLDYN.Core.IOFiles.DCDFile \textit{(class)}!nMOLDYN.Core.IOFiles.DCDFile.next \textit{(method)}}

    \vspace{0.5ex}

\hspace{.8\funcindent}\begin{boxedminipage}{\funcwidth}

    \raggedright \textbf{next}(\textit{self})

\setlength{\parskip}{2ex}
\setlength{\parskip}{1ex}
    \end{boxedminipage}


\large{\textbf{\textit{Inherited from object}}}

\begin{quote}
\_\_delattr\_\_(), \_\_getattribute\_\_(), \_\_hash\_\_(), \_\_new\_\_(), \_\_reduce\_\_(), \_\_reduce\_ex\_\_(), \_\_repr\_\_(), \_\_setattr\_\_(), \_\_str\_\_()
\end{quote}

%%%%%%%%%%%%%%%%%%%%%%%%%%%%%%%%%%%%%%%%%%%%%%%%%%%%%%%%%%%%%%%%%%%%%%%%%%%
%%                              Properties                               %%
%%%%%%%%%%%%%%%%%%%%%%%%%%%%%%%%%%%%%%%%%%%%%%%%%%%%%%%%%%%%%%%%%%%%%%%%%%%

  \subsubsection{Properties}

    \vspace{-1cm}
\hspace{\varindent}\begin{longtable}{|p{\varnamewidth}|p{\vardescrwidth}|l}
\cline{1-2}
\cline{1-2} \centering \textbf{Name} & \centering \textbf{Description}& \\
\cline{1-2}
\endhead\cline{1-2}\multicolumn{3}{r}{\small\textit{continued on next page}}\\\endfoot\cline{1-2}
\endlastfoot\multicolumn{2}{|l|}{\textit{Inherited from object}}\\
\multicolumn{2}{|p{\varwidth}|}{\raggedright \_\_class\_\_}\\
\cline{1-2}
\end{longtable}

    \index{nMOLDYN \textit{(package)}!nMOLDYN.Core \textit{(package)}!nMOLDYN.Core.IOFiles \textit{(module)}!nMOLDYN.Core.IOFiles.DCDFile \textit{(class)}|)}

%%%%%%%%%%%%%%%%%%%%%%%%%%%%%%%%%%%%%%%%%%%%%%%%%%%%%%%%%%%%%%%%%%%%%%%%%%%
%%                           Class Description                           %%
%%%%%%%%%%%%%%%%%%%%%%%%%%%%%%%%%%%%%%%%%%%%%%%%%%%%%%%%%%%%%%%%%%%%%%%%%%%

    \index{nMOLDYN \textit{(package)}!nMOLDYN.Core \textit{(package)}!nMOLDYN.Core.IOFiles \textit{(module)}!nMOLDYN.Core.IOFiles.AmberNetCDFConverter \textit{(class)}|(}
\subsection{Class AmberNetCDFConverter}

    \label{nMOLDYN:Core:IOFiles:AmberNetCDFConverter}
Converts an Amber NetCDF Trajectory into a MMTK NetCDFFile.

Comments:

\begin{itemize}
\setlength{\parskip}{0.6ex}
  \item this code is an improved version of the original converter written by 
    Paolo Calligari.

\end{itemize}


%%%%%%%%%%%%%%%%%%%%%%%%%%%%%%%%%%%%%%%%%%%%%%%%%%%%%%%%%%%%%%%%%%%%%%%%%%%
%%                                Methods                                %%
%%%%%%%%%%%%%%%%%%%%%%%%%%%%%%%%%%%%%%%%%%%%%%%%%%%%%%%%%%%%%%%%%%%%%%%%%%%

  \subsubsection{Methods}

    \label{nMOLDYN:Core:IOFiles:AmberNetCDFConverter:__init__}
    \index{nMOLDYN \textit{(package)}!nMOLDYN.Core \textit{(package)}!nMOLDYN.Core.IOFiles \textit{(module)}!nMOLDYN.Core.IOFiles.AmberNetCDFConverter \textit{(class)}!nMOLDYN.Core.IOFiles.AmberNetCDFConverter.\_\_init\_\_ \textit{(method)}}

    \vspace{0.5ex}

\hspace{.8\funcindent}\begin{boxedminipage}{\funcwidth}

    \raggedright \textbf{\_\_init\_\_}(\textit{self}, \textit{pdbFile}, \textit{amberNetCDFFile}, \textit{outputFile}, \textit{timeStep}={\tt 1.0})

    \vspace{-1.5ex}

    \rule{\textwidth}{0.5\fboxrule}
\setlength{\parskip}{2ex}
    The constructor. Will do the conversion.

\setlength{\parskip}{1ex}
      \textbf{Parameters}
      \vspace{-1ex}

      \begin{quote}
        \begin{Ventry}{xxxxxxxxxxxxxxx}

          \item[pdbFile]

          the Amber PDB file name of a frame of the trajectory to convert.

            {\it (type=string)}

          \item[amberNetCDFFile]

          the Amber NetCDF file name of the trajectory to convert.

            {\it (type=string)}

          \item[outputFile]

          the name of MMTK NetCDF trajectory output file.

            {\it (type=string)}

          \item[timeStep]

          the timestep that will used when building the MMTK trajectory. 
          Default to 1 ps.

            {\it (type=float.)}

        \end{Ventry}

      \end{quote}

    \end{boxedminipage}

    \index{nMOLDYN \textit{(package)}!nMOLDYN.Core \textit{(package)}!nMOLDYN.Core.IOFiles \textit{(module)}!nMOLDYN.Core.IOFiles.AmberNetCDFConverter \textit{(class)}|)}

%%%%%%%%%%%%%%%%%%%%%%%%%%%%%%%%%%%%%%%%%%%%%%%%%%%%%%%%%%%%%%%%%%%%%%%%%%%
%%                           Class Description                           %%
%%%%%%%%%%%%%%%%%%%%%%%%%%%%%%%%%%%%%%%%%%%%%%%%%%%%%%%%%%%%%%%%%%%%%%%%%%%

    \index{nMOLDYN \textit{(package)}!nMOLDYN.Core \textit{(package)}!nMOLDYN.Core.IOFiles \textit{(module)}!nMOLDYN.Core.IOFiles.CHARMMConverter \textit{(class)}|(}
\subsection{Class CHARMMConverter}

    \label{nMOLDYN:Core:IOFiles:CHARMMConverter}
Converts a CHARMM Trajectory into a MMTK NetCDFFile.

Comments:

\begin{itemize}
\setlength{\parskip}{0.6ex}
  \item this code is based on the original converter written by Konrad Hinsen.

\end{itemize}


%%%%%%%%%%%%%%%%%%%%%%%%%%%%%%%%%%%%%%%%%%%%%%%%%%%%%%%%%%%%%%%%%%%%%%%%%%%
%%                                Methods                                %%
%%%%%%%%%%%%%%%%%%%%%%%%%%%%%%%%%%%%%%%%%%%%%%%%%%%%%%%%%%%%%%%%%%%%%%%%%%%

  \subsubsection{Methods}

    \label{nMOLDYN:Core:IOFiles:CHARMMConverter:__init__}
    \index{nMOLDYN \textit{(package)}!nMOLDYN.Core \textit{(package)}!nMOLDYN.Core.IOFiles \textit{(module)}!nMOLDYN.Core.IOFiles.CHARMMConverter \textit{(class)}!nMOLDYN.Core.IOFiles.CHARMMConverter.\_\_init\_\_ \textit{(method)}}

    \vspace{0.5ex}

\hspace{.8\funcindent}\begin{boxedminipage}{\funcwidth}

    \raggedright \textbf{\_\_init\_\_}(\textit{self}, \textit{pdbFile}, \textit{dcdFile}, \textit{outputFile})

    \vspace{-1.5ex}

    \rule{\textwidth}{0.5\fboxrule}
\setlength{\parskip}{2ex}
    The constructor. Will do the conversion.

\setlength{\parskip}{1ex}
      \textbf{Parameters}
      \vspace{-1ex}

      \begin{quote}
        \begin{Ventry}{xxxxxxxxxx}

          \item[pdbFile]

          the CHARMM PDB file name of a frame of the trajectory to convert.

            {\it (type=string)}

          \item[dcdFile]

          the CHARMM DCD file name of the trajectory to convert.

            {\it (type=string)}

          \item[outputFile]

          the name of MMTK NetCDF trajectory output file.

            {\it (type=string)}

        \end{Ventry}

      \end{quote}

    \end{boxedminipage}

    \index{nMOLDYN \textit{(package)}!nMOLDYN.Core \textit{(package)}!nMOLDYN.Core.IOFiles \textit{(module)}!nMOLDYN.Core.IOFiles.CHARMMConverter \textit{(class)}|)}

%%%%%%%%%%%%%%%%%%%%%%%%%%%%%%%%%%%%%%%%%%%%%%%%%%%%%%%%%%%%%%%%%%%%%%%%%%%
%%                           Class Description                           %%
%%%%%%%%%%%%%%%%%%%%%%%%%%%%%%%%%%%%%%%%%%%%%%%%%%%%%%%%%%%%%%%%%%%%%%%%%%%

    \index{nMOLDYN \textit{(package)}!nMOLDYN.Core \textit{(package)}!nMOLDYN.Core.IOFiles \textit{(module)}!nMOLDYN.Core.IOFiles.DL\_POLYConverter \textit{(class)}|(}
\subsection{Class DL\_POLYConverter}

    \label{nMOLDYN:Core:IOFiles:DL_POLYConverter}
Converts a DL\_POLY Trajectory into a MMTK NetCDFFile.


%%%%%%%%%%%%%%%%%%%%%%%%%%%%%%%%%%%%%%%%%%%%%%%%%%%%%%%%%%%%%%%%%%%%%%%%%%%
%%                                Methods                                %%
%%%%%%%%%%%%%%%%%%%%%%%%%%%%%%%%%%%%%%%%%%%%%%%%%%%%%%%%%%%%%%%%%%%%%%%%%%%

  \subsubsection{Methods}

    \label{nMOLDYN:Core:IOFiles:DL_POLYConverter:__init__}
    \index{nMOLDYN \textit{(package)}!nMOLDYN.Core \textit{(package)}!nMOLDYN.Core.IOFiles \textit{(module)}!nMOLDYN.Core.IOFiles.DL\_POLYConverter \textit{(class)}!nMOLDYN.Core.IOFiles.DL\_POLYConverter.\_\_init\_\_ \textit{(method)}}

    \vspace{0.5ex}

\hspace{.8\funcindent}\begin{boxedminipage}{\funcwidth}

    \raggedright \textbf{\_\_init\_\_}(\textit{self}, \textit{fieldFile}, \textit{historyFile}, \textit{outputFile}, \textit{specialAtoms}={\tt \{\}})

    \vspace{-1.5ex}

    \rule{\textwidth}{0.5\fboxrule}
\setlength{\parskip}{2ex}
    The constructor. Will do the conversion.

\setlength{\parskip}{1ex}
      \textbf{Parameters}
      \vspace{-1ex}

      \begin{quote}
        \begin{Ventry}{xxxxxxxxxxxx}

          \item[fieldFile]

          the DL\_POLY FIELD file name of the trajectory to convert.

            {\it (type=string)}

          \item[historyFile]

          the DL\_POLY HISTORY file name of the trajectory to convert.

            {\it (type=string)}

          \item[outputFile]

          the name of MMTK NetCDF trajectory output file.

            {\it (type=string)}

          \item[specialAtoms]

          dictionnary of the form \{s1 : e1, s2 : e2 ...\} where 's1', 's2'
          ... and 'e1', 'e2' ... are respectively the DL\_POLY name and the
          symbol of atoms 1, 2 ...

            {\it (type=dict)}

        \end{Ventry}

      \end{quote}

    \end{boxedminipage}

    \index{nMOLDYN \textit{(package)}!nMOLDYN.Core \textit{(package)}!nMOLDYN.Core.IOFiles \textit{(module)}!nMOLDYN.Core.IOFiles.DL\_POLYConverter \textit{(class)}|)}

%%%%%%%%%%%%%%%%%%%%%%%%%%%%%%%%%%%%%%%%%%%%%%%%%%%%%%%%%%%%%%%%%%%%%%%%%%%
%%                           Class Description                           %%
%%%%%%%%%%%%%%%%%%%%%%%%%%%%%%%%%%%%%%%%%%%%%%%%%%%%%%%%%%%%%%%%%%%%%%%%%%%

    \index{nMOLDYN \textit{(package)}!nMOLDYN.Core \textit{(package)}!nMOLDYN.Core.IOFiles \textit{(module)}!nMOLDYN.Core.IOFiles.MaterialsStudioConverter \textit{(class)}|(}
\subsection{Class MaterialsStudioConverter}

    \label{nMOLDYN:Core:IOFiles:MaterialsStudioConverter}
Converts a MaterialsStudio Discover or Forcite Trajectory into a MMTK 
NetCDFFile.


%%%%%%%%%%%%%%%%%%%%%%%%%%%%%%%%%%%%%%%%%%%%%%%%%%%%%%%%%%%%%%%%%%%%%%%%%%%
%%                                Methods                                %%
%%%%%%%%%%%%%%%%%%%%%%%%%%%%%%%%%%%%%%%%%%%%%%%%%%%%%%%%%%%%%%%%%%%%%%%%%%%

  \subsubsection{Methods}

    \label{nMOLDYN:Core:IOFiles:MaterialsStudioConverter:__init__}
    \index{nMOLDYN \textit{(package)}!nMOLDYN.Core \textit{(package)}!nMOLDYN.Core.IOFiles \textit{(module)}!nMOLDYN.Core.IOFiles.MaterialsStudioConverter \textit{(class)}!nMOLDYN.Core.IOFiles.MaterialsStudioConverter.\_\_init\_\_ \textit{(method)}}

    \vspace{0.5ex}

\hspace{.8\funcindent}\begin{boxedminipage}{\funcwidth}

    \raggedright \textbf{\_\_init\_\_}(\textit{self}, \textit{module}, \textit{xtdxsdFile}, \textit{histrjFile}, \textit{outputFile}, \textit{subselection}={\tt None})

    \vspace{-1.5ex}

    \rule{\textwidth}{0.5\fboxrule}
\setlength{\parskip}{2ex}
    The constructor. Will do the conversion.

\setlength{\parskip}{1ex}
      \textbf{Parameters}
      \vspace{-1ex}

      \begin{quote}
        \begin{Ventry}{xxxxxxxxxxxx}

          \item[module]

          a string being one of 'Discover' or 'Forcite' specifying which 
          module of MaterialsStudio the trajectory is coming from.

            {\it (type=string)}

          \item[xtdxsdFile]

          the MaterialsStudio XTD or XSD file name of the trajectory to 
          convert.

            {\it (type=string)}

          \item[histrjFile]

          the MaterialsStudio HIS (Discover) or TRJ (Forcite) file name of 
          the trajectory to convert.

            {\it (type=string)}

          \item[outputFile]

          the name of MMTK NetCDF trajectory output file.

            {\it (type=string)}

          \item[subselection]

          if not None, list of the indexes (integer {\textgreater}= 1) of 
          the atoms to select when writing out the MMTK trajectory. The 
          order being the one defined in the XTD/XSD file.

            {\it (type=list)}

        \end{Ventry}

      \end{quote}

    \end{boxedminipage}

    \label{nMOLDYN:Core:IOFiles:MaterialsStudioConverter:createCluster}
    \index{nMOLDYN \textit{(package)}!nMOLDYN.Core \textit{(package)}!nMOLDYN.Core.IOFiles \textit{(module)}!nMOLDYN.Core.IOFiles.MaterialsStudioConverter \textit{(class)}!nMOLDYN.Core.IOFiles.MaterialsStudioConverter.createCluster \textit{(method)}}

    \vspace{0.5ex}

\hspace{.8\funcindent}\begin{boxedminipage}{\funcwidth}

    \raggedright \textbf{createCluster}(\textit{self}, \textit{at}, \textit{clust})

\setlength{\parskip}{2ex}
\setlength{\parskip}{1ex}
    \end{boxedminipage}

    \label{nMOLDYN:Core:IOFiles:MaterialsStudioConverter:readXTDFile}
    \index{nMOLDYN \textit{(package)}!nMOLDYN.Core \textit{(package)}!nMOLDYN.Core.IOFiles \textit{(module)}!nMOLDYN.Core.IOFiles.MaterialsStudioConverter \textit{(class)}!nMOLDYN.Core.IOFiles.MaterialsStudioConverter.readXTDFile \textit{(method)}}

    \vspace{0.5ex}

\hspace{.8\funcindent}\begin{boxedminipage}{\funcwidth}

    \raggedright \textbf{readXTDFile}(\textit{self})

    \vspace{-1.5ex}

    \rule{\textwidth}{0.5\fboxrule}
\setlength{\parskip}{2ex}
    Reads the Materials Studio XTD or  XSD file and set up the universe 
    from which the NetCDF MMTK trajectory will be written.

\setlength{\parskip}{1ex}
\textbf{Note:} the XTD and XSD file are xml file.



    \end{boxedminipage}

    \label{nMOLDYN:Core:IOFiles:MaterialsStudioConverter:readHISFile}
    \index{nMOLDYN \textit{(package)}!nMOLDYN.Core \textit{(package)}!nMOLDYN.Core.IOFiles \textit{(module)}!nMOLDYN.Core.IOFiles.MaterialsStudioConverter \textit{(class)}!nMOLDYN.Core.IOFiles.MaterialsStudioConverter.readHISFile \textit{(method)}}

    \vspace{0.5ex}

\hspace{.8\funcindent}\begin{boxedminipage}{\funcwidth}

    \raggedright \textbf{readHISFile}(\textit{self})

    \vspace{-1.5ex}

    \rule{\textwidth}{0.5\fboxrule}
\setlength{\parskip}{2ex}
    Reads a Materials Studio HIS file and fills up the NetCDF trajectory 
    file.

\setlength{\parskip}{1ex}
    \end{boxedminipage}

    \label{nMOLDYN:Core:IOFiles:MaterialsStudioConverter:readTRJFile}
    \index{nMOLDYN \textit{(package)}!nMOLDYN.Core \textit{(package)}!nMOLDYN.Core.IOFiles \textit{(module)}!nMOLDYN.Core.IOFiles.MaterialsStudioConverter \textit{(class)}!nMOLDYN.Core.IOFiles.MaterialsStudioConverter.readTRJFile \textit{(method)}}

    \vspace{0.5ex}

\hspace{.8\funcindent}\begin{boxedminipage}{\funcwidth}

    \raggedright \textbf{readTRJFile}(\textit{self})

    \vspace{-1.5ex}

    \rule{\textwidth}{0.5\fboxrule}
\setlength{\parskip}{2ex}
    Reads a Materials Studio HIS file and fills up the NetCDF trajectory 
    file.

\setlength{\parskip}{1ex}
    \end{boxedminipage}


%%%%%%%%%%%%%%%%%%%%%%%%%%%%%%%%%%%%%%%%%%%%%%%%%%%%%%%%%%%%%%%%%%%%%%%%%%%
%%                            Class Variables                            %%
%%%%%%%%%%%%%%%%%%%%%%%%%%%%%%%%%%%%%%%%%%%%%%%%%%%%%%%%%%%%%%%%%%%%%%%%%%%

  \subsubsection{Class Variables}

    \vspace{-1cm}
\hspace{\varindent}\begin{longtable}{|p{\varnamewidth}|p{\vardescrwidth}|l}
\cline{1-2}
\cline{1-2} \centering \textbf{Name} & \centering \textbf{Description}& \\
\cline{1-2}
\endhead\cline{1-2}\multicolumn{3}{r}{\small\textit{continued on next page}}\\\endfoot\cline{1-2}
\endlastfoot\raggedright a\-t\-o\-m\-L\-i\-n\-e\-F\-o\-r\-m\-a\-t\- & \raggedright \textbf{Value:} 
{\tt FortranFormat('A5,1X,F14.9,1X,F14.9,1X,F14.9,1X,A4,1X,A7,\texttt{...}}&\\
\cline{1-2}
\end{longtable}

    \index{nMOLDYN \textit{(package)}!nMOLDYN.Core \textit{(package)}!nMOLDYN.Core.IOFiles \textit{(module)}!nMOLDYN.Core.IOFiles.MaterialsStudioConverter \textit{(class)}|)}

%%%%%%%%%%%%%%%%%%%%%%%%%%%%%%%%%%%%%%%%%%%%%%%%%%%%%%%%%%%%%%%%%%%%%%%%%%%
%%                           Class Description                           %%
%%%%%%%%%%%%%%%%%%%%%%%%%%%%%%%%%%%%%%%%%%%%%%%%%%%%%%%%%%%%%%%%%%%%%%%%%%%

    \index{nMOLDYN \textit{(package)}!nMOLDYN.Core \textit{(package)}!nMOLDYN.Core.IOFiles \textit{(module)}!nMOLDYN.Core.IOFiles.NAMDConverter \textit{(class)}|(}
\subsection{Class NAMDConverter}

    \label{nMOLDYN:Core:IOFiles:NAMDConverter}
Converts a NAMD Trajectory into a MMTK NetCDFFile.

Comments:

\begin{itemize}
\setlength{\parskip}{0.6ex}
  \item this code is based on the original converter written by Konrad Hinsen.

\end{itemize}


%%%%%%%%%%%%%%%%%%%%%%%%%%%%%%%%%%%%%%%%%%%%%%%%%%%%%%%%%%%%%%%%%%%%%%%%%%%
%%                                Methods                                %%
%%%%%%%%%%%%%%%%%%%%%%%%%%%%%%%%%%%%%%%%%%%%%%%%%%%%%%%%%%%%%%%%%%%%%%%%%%%

  \subsubsection{Methods}

    \label{nMOLDYN:Core:IOFiles:NAMDConverter:__init__}
    \index{nMOLDYN \textit{(package)}!nMOLDYN.Core \textit{(package)}!nMOLDYN.Core.IOFiles \textit{(module)}!nMOLDYN.Core.IOFiles.NAMDConverter \textit{(class)}!nMOLDYN.Core.IOFiles.NAMDConverter.\_\_init\_\_ \textit{(method)}}

    \vspace{0.5ex}

\hspace{.8\funcindent}\begin{boxedminipage}{\funcwidth}

    \raggedright \textbf{\_\_init\_\_}(\textit{self}, \textit{pdbFile}, \textit{dcdFile}, \textit{xstFile}, \textit{outputFile})

    \vspace{-1.5ex}

    \rule{\textwidth}{0.5\fboxrule}
\setlength{\parskip}{2ex}
    The constructor. Will do the conversion.

\setlength{\parskip}{1ex}
      \textbf{Parameters}
      \vspace{-1ex}

      \begin{quote}
        \begin{Ventry}{xxxxxxxxxx}

          \item[pdbFile]

          the NAMD PDB file name of a frame of the trajectory to convert.

            {\it (type=string)}

          \item[dcdFile]

          the NAMD DCD file name of the trajectory to convert.

            {\it (type=string)}

          \item[xstFile]

          the NAMD XSTfile name of the trajectory to convert.

            {\it (type=string)}

          \item[outputFile]

          the name of MMTK NetCDF trajectory output file.

            {\it (type=string)}

        \end{Ventry}

      \end{quote}

    \end{boxedminipage}

    \index{nMOLDYN \textit{(package)}!nMOLDYN.Core \textit{(package)}!nMOLDYN.Core.IOFiles \textit{(module)}!nMOLDYN.Core.IOFiles.NAMDConverter \textit{(class)}|)}

%%%%%%%%%%%%%%%%%%%%%%%%%%%%%%%%%%%%%%%%%%%%%%%%%%%%%%%%%%%%%%%%%%%%%%%%%%%
%%                           Class Description                           %%
%%%%%%%%%%%%%%%%%%%%%%%%%%%%%%%%%%%%%%%%%%%%%%%%%%%%%%%%%%%%%%%%%%%%%%%%%%%

    \index{nMOLDYN \textit{(package)}!nMOLDYN.Core \textit{(package)}!nMOLDYN.Core.IOFiles \textit{(module)}!nMOLDYN.Core.IOFiles.VASPConverter \textit{(class)}|(}
\subsection{Class VASPConverter}

    \label{nMOLDYN:Core:IOFiles:VASPConverter}
Converts a VASP Trajectory into a MMTK NetCDFFile.


%%%%%%%%%%%%%%%%%%%%%%%%%%%%%%%%%%%%%%%%%%%%%%%%%%%%%%%%%%%%%%%%%%%%%%%%%%%
%%                                Methods                                %%
%%%%%%%%%%%%%%%%%%%%%%%%%%%%%%%%%%%%%%%%%%%%%%%%%%%%%%%%%%%%%%%%%%%%%%%%%%%

  \subsubsection{Methods}

    \label{nMOLDYN:Core:IOFiles:VASPConverter:__init__}
    \index{nMOLDYN \textit{(package)}!nMOLDYN.Core \textit{(package)}!nMOLDYN.Core.IOFiles \textit{(module)}!nMOLDYN.Core.IOFiles.VASPConverter \textit{(class)}!nMOLDYN.Core.IOFiles.VASPConverter.\_\_init\_\_ \textit{(method)}}

    \vspace{0.5ex}

\hspace{.8\funcindent}\begin{boxedminipage}{\funcwidth}

    \raggedright \textbf{\_\_init\_\_}(\textit{self}, \textit{contcarFile}, \textit{xdatcarFile}, \textit{outputFile}, \textit{atomContents})

    \vspace{-1.5ex}

    \rule{\textwidth}{0.5\fboxrule}
\setlength{\parskip}{2ex}
    The constructor. Will do the conversion.

\setlength{\parskip}{1ex}
      \textbf{Parameters}
      \vspace{-1ex}

      \begin{quote}
        \begin{Ventry}{xxxxxxxxxxxx}

          \item[contcarFile]

          the VASP CONTCAR or POSCAR file name of the trajectory to 
          convert.

            {\it (type=string)}

          \item[xdatcarFile]

          the VASP XDATCAR file name of the trajectory to convert.

            {\it (type=string)}

          \item[outputFile]

          the name of MMTK NetCDF trajectory output file.

            {\it (type=string)}

          \item[atomContents]

          List of the element names (string) in the order they appear in 
          the trajectory.

            {\it (type=list)}

        \end{Ventry}

      \end{quote}

    \end{boxedminipage}

    \index{nMOLDYN \textit{(package)}!nMOLDYN.Core \textit{(package)}!nMOLDYN.Core.IOFiles \textit{(module)}!nMOLDYN.Core.IOFiles.VASPConverter \textit{(class)}|)}
    \index{nMOLDYN \textit{(package)}!nMOLDYN.Core \textit{(package)}!nMOLDYN.Core.IOFiles \textit{(module)}|)}
